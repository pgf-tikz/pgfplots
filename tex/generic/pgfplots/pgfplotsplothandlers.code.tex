%--------------------------------------------
%
% Package pgfplots
%
% Provides a user-friendly interface to create function plots (normal
% plots, semi-logplots and double-logplots).
% 
% It is based on Till Tantau's PGF package.
%
% Copyright 2010 by Christian Feuersänger.
%
% This program is free software: you can redistribute it and/or modify
% it under the terms of the GNU General Public License as published by
% the Free Software Foundation, either version 3 of the License, or
% (at your option) any later version.
% 
% This program is distributed in the hope that it will be useful,
% but WITHOUT ANY WARRANTY; without even the implied warranty of
% MERCHANTABILITY or FITNESS FOR A PARTICULAR PURPOSE.  See the
% GNU General Public License for more details.
% 
% You should have received a copy of the GNU General Public License
% along with this program.  If not, see <http://www.gnu.org/licenses/>.
%
%--------------------------------------------

% 1. OVERVIEW
%
% Plot handlers are extended versions of the plot handlers of PGF with
% backwards compatibility.
%
% To remind ourselfes: PGF plot handlers work like this
% \pgfplotstreamstart
% \pgfplotstreampoint{...}
% \pgfplotstreampoint{...}
% \pgfplotstreampoint{...}
% \pgfplotstreampoint{...}
% \pgfplotstreamend
%
% and that's it.
%
% PGFPlots plot handlers also contain these three macros. The tasks
% are (of course) the same. In addition, it supports a set of further
% macros for every plot handler:
%
% \pgfplotsplothandlersurveystart
% \pgfplotsplothandlersurveypoint
% \pgfplotsplothandlersurveyend
%
% and serialization methods
% \pgfplotsplothandlerserializepointto
% \pgfplotsplothandlerdeserializepointfrom
% \pgfplotsplothandlerserializestateto
% \pgfplotsplothandlerdeserializepointfrom
%
% The idea is as follows:
% During \addplot, PGFPLots performs a survey phase. Survey means:
% nothing will be drawn, only stats will be collected. In this phase,
% the \pgfplotsplothandlersurvey* methods will be invoked; followed by
% a serialization.
%
% Then, when every plot has been surveyed, PGFPlots calls
% \pgfplotsplothandlerdeserializestatefrom{<serialized state>}
% \pgfplotstreamstart
% foreach <serizalized point> {
% 	\pgfplotsplothandlerdeserializepointfrom{<serialized point>}
% 	apply data transformations 
% 	handle plot marks
% 	\pgfplotstreampoint{\pgfplotsqpointxy{<x>}{<y}}
% }
% \pgfplotstreamend
%
% See below for details about the survey phase and the visualization
% phase and the API of an axis.
%
% Thus, the PGF basic level streams can (but don't need to) make use
% of the macros assigned by the
% \pgfplotsplothandlerdeserializepointfrom and
% \pgfplotsplothandlerdeserializestatefrom.
%
% Thus, a plot handler *can* redefine any of the survey and/or
% serialization methods to add further functionality. But it doesn't
% need to, it's ok if it only relies on x,y as standard pgf plot
% handlers doredefine any of the survey and/or serialization methods
% to add further functionality. But it doesn't need to, it's ok if it
% only relies on x,y as standard pgf plot handlers do.
%
% If it overwrites something, it should do so in its main method, the
% one in which it also defines the \pgf@plotstreamstart etc. methods
% (see pgfmoduleplot.code.tex for example).
%
% 2. THE SURVEY PHASE:
%   The survey phase works as follows.
%   PRECONDITION:
%   	- we are `inside' of an axis.
%   	- \addplot starts the survey phase.
%
%   The following happens:
%   	- the plot handler is determined and activated such that its
%   	methods are usable ("class is instantiated")
%   	- \pgfplots@PREPARE@COORD@STREAM is the internal, low level
%   	entry point where pgfplots controls the survey phase.
%   	- the coordinate input routine is initialised. It issues a
%   	series of commands (all within the *same* TeX scope):
%   		\pgfplots@coord@stream@start
%   		\pgfplots@coord@stream@coord
%   		\pgfplots@coord@stream@coord
%   		\pgfplots@coord@stream@coord
%   		\pgfplots@coord@stream@coord
%   		\pgfplots@coord@stream@coord
%   		\pgfplots@coord@stream@coord
%   		...
%   		\pgfplots@coord@stream@end
%   	- \pgfplotsplothandlersurveystart is called in
%   	\pgfplots@coord@stream@start.
%   	- \pgfplots@coord@stream@coord calls
%   	\pgfplotsplothandlersurveypoint
%   	- \pgfplots@coord@stream@end calls
%   	\pgfplotsplothandlersurveyend
%   	\pgfplotsplothandlerserializestateto<\macro>
%   	Furthermore, it remembers the <\macro> such that it can be
%   	deserialized later.
%   Then, the survey phase ends. The main point of interest is the
%   \pgfplotsplothandlersurveypoint routine, especially its
%   communication with the axis. It is described below.
%
%   POSTCONDITION:
%   	- the state of the axis is now aware of the new plot (limits,
%   	stacking, ...).
%   	- The plot's survey state is stored using its serialized
%   	representation.
%
% 2.1 The API of an axis
%
% As described above, the coordinate input routine fires a lot of
% \pgfplots@coord@stream@coord commands, which, in turn, invoke
% \pgfplotsplothandlersurveypoint. Somehow this should update the
% axis' state to reflect each point. But the 'data point' is a rather
% abstract thing. Usually, it will contain at least (x,y) (or maybe z)
% coordinates. But it may be more complex.
%
% So, the coordinate input routine provides whatever the user has
% chosen. Let's assume, we are using \addplot table. Then, we can 
% access every cell in the current row (using \thisrow{<colname>} for
% example). The plot handler knows how to extract its information from
% this state. So:
% - the plot handler assembles coordinates.
% - every assembled coordinate should be reported to the axis by
%   defining \pgfplots@current@point@[xyz] to its coordinates and
%   calling
%   \pgfplotsaxispreparecoordinate{}.
%   This will apply coordinate filters, parse the single coordinates
%   and apply high level transformations and any logarithms.
%   It is some sort of advanced coordinate parser which works only for 
%   (x,y) or for (x,y,z), but it doesn't do anything but to return
%   valid coordinates for x,y and z.%
%
%   If necessary, the plot handler calls
%   \pgfplotsaxispreparecoordinate{} multiple times, ones for each
%   encountered coordinate.
%
%   It might happen that a coordinate filter discards a coordinate.
%   This is returned in the \ifpgfplotsaxispreparecoordinateok boolean
%   and has to checked by the plot handler.
%
% - the plot handler knows which of the coordinates contribute to the
%   final plot. It invokes
%   \ifpgfplotsaxispreparecoordinateok
%   	\pgfplotsaxisupdatelimitsforcoordinate
%   \fi
%   for each of these coordinates.
%
% - eventually, the plot handler is satisfied and considers a data
%   point as "readily surveyed". It is allowed if this does *not*
%   happen inside of \pgfplotsplothandlersurveypoint, but it must
%   happen before \pgfplotsplothandlersurveyend is finished.
%   
%   The plot handler invokes \pgfplotsaxisdatapointsurveyed.
%   This tells the axis that it can perform is own surveying tasks
%   (FIXME) and furthermore, that it can serialize the data point.
%   Consequently, it will invoke
%   \pgfplotsplothandlerserializepointto{<\macro>} 
%   and it will remember that \macro internally. This serialization is
%   employed to place plot marks and to apply z buffering techniques
%   (that's why it is done by the axis and not by the plot handler on
%   its own).
%
% - Later, the coordinate input routine invokes
%   \pgfplots@coord@stream@end indicating the end-of-input. This will
%   finalize the survey phase.
%
% A simple example looks like this:
% \pgfplotsplothandlersurveystart: does nothing.
% \pgfplotsplothandlersurveypoint:
%   parses the input format somehow to get (x,y,z) in raw, symbolic format
% 	calls \pgfplotsaxispreparecoordinate
% 	calls \ifpgfplotsaxispreparecoordinateok \pgfplotsaxisupdatelimitsforcoordinate \fi
% 	calls \pgfplotsaxisdatapointsurveyed
% \pgfplotsplothandlersurveyend: does nothing.
%
% In fact, this is more or less the initial configuration used for
% lineto or similar plot handlers.
