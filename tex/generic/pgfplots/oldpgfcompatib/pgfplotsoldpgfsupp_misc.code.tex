
% ======================================================
% compatibility with PGF 2.0
% ======================================================

% from pgfutil-common.tex:

% Solves a linear equation system of size 2x2 using gauss elimination.
%
% It employs TeX register arithmetics to do so.
% #1: should contain 4 sets of braces with matrix entries, 
% 	{<a11>}{<a12>}
% 	{<a21>}{<a22>}
% 	where each entry should be a number without unit.
% #2: should contain 2 sets of braces with the right-hand-side,
% 	{<r1>}{<r2>}
% 	where each entry should be a number without unit.
%
% It will assign \pgfmathresult to contain two sets of braces with the
% result.
%
% Example:
% \pgfutilsolvetwotwoleq{
% 	{0.24}{1}
% 	{-0.97}{0}
% }{
% 	{-7}
% 	{18}
% }
% -> yields \pgfmathresult={−18.55618}{−2.54642}
%
% The algorithm employs column pivotisation.
\def\pgfutilsolvetwotwoleq#1#2{%
	\begingroup
		\dimendef\aa=0
		\dimendef\ab=1
		\dimendef\ba=2
		\dimendef\bb=3
		\dimendef\ra=4
		\dimendef\rb=5
		\dimendef\tmpa=6
		\dimendef\tmpb=7
		\edef\pgf@temp{#1}%
		\expandafter\pgfutilsolvetwotwoleq@A\pgf@temp
		\edef\pgf@temp{#2}%
		\expandafter\pgfutilsolvetwotwoleq@r\pgf@temp
		%
		\pgfutilsolvetwotwoleq@ifislarger\aa\ba{%
			% identity "permutation":
			\def\Pa{a}%
			\def\Pb{b}%
		}{%
			% permutation matrix: switch rows!
			\def\Pa{b}%
			\def\Pb{a}%
		}%
		% \pivot := 1/aa
		\pgfmathreciprocal@
			{\csname m\Pa a\endcsname}%
		\let\pivot=\pgfmathresult
		%
		% \factor := 1/aa * ba 
		\csname \Pb a\endcsname=\pivot\csname \Pb a\endcsname
		\edef\factor{\expandafter\pgf@sys@tonumber\csname \Pb a\endcsname}%
		%
		% bb -= ba/aa * ab
		\tmpa=-\factor\csname \Pa b\endcsname
		\advance\csname \Pb b\endcsname by\tmpa
		%
		% rb -= ba/aa * ra
		\tmpa=-\factor\csname r\Pa\endcsname
		\advance\csname r\Pb\endcsname by\tmpa
		%
		% xb := rb / bb (the modified rb and modified bb!)
		\pgfmathdivide@
			{\expandafter\pgf@sys@tonumber\csname r\Pb\endcsname}
			{\expandafter\pgf@sys@tonumber\csname \Pb b\endcsname}%
		\expandafter\let\csname pgfmathresult\Pb\endcsname=\pgfmathresult
		%
		% ra -= xb * ab
		\tmpa=\csname pgfmathresult\Pb\endcsname\csname \Pa b\endcsname
		\advance\csname r\Pa\endcsname by-\tmpa
		%
		% xa := 1/aa * ra  (the modified ra!)
		\tmpa=\pivot\csname r\Pa\endcsname
		\expandafter\edef\csname pgfmathresult\Pa\endcsname{\pgf@sys@tonumber\tmpa}%
		%
		\edef\pgfmathresult{%
			{\csname pgfmathresult\Pa\endcsname}%
			{\csname pgfmathresult\Pb\endcsname}%
		}%
		\pgfmath@smuggleone\pgfmathresult
	\endgroup
}%
\def\pgfutilsolvetwotwoleq@ifislarger#1#2#3#4{%
	\tmpa=#1
	\ifdim\tmpa<0pt
		\multiply\tmpa by-1
	\fi
	\tmpb=#2
	\ifdim\tmpb<0pt
		\multiply\tmpb by-1
	\fi
	\ifdim\tmpa>\tmpb
		#3%
	\else
		#4%
	\fi
}%
\def\pgfutilsolvetwotwoleq@A#1#2#3#4{%
	\def\maa{#1}\def\mab{#2}%
	\def\mba{#3}\def\mbb{#3}%
	\aa=#1pt \ab=#2pt
	\ba=#3pt \bb=#4pt
}
\def\pgfutilsolvetwotwoleq@r#1#2{%
	\ra=#1pt \rb=#2pt
}%

%%%%%%%
%%%%%%%

% from pgfmoduleshapes.code.tex:
% Invoke an anchor
\def\pgf@sh@reanchor#1#2{%
  \pgfutil@ifundefined{pgf@anchor@#1@#2}%
  {%
    \pgfutil@ifundefined{pgf@anchor@generic@#2}{%
	    \pgfmathsetcounter{pgf@counta}{#2}%
	    \csname pgf@anchor@#1@border\endcsname{\pgfqpointpolar{\c@pgf@counta}{1pt}}%
    }{%
        \csname pgf@anchor@generic@#2\endcsname{#1}%
    }%
  }%
  {\csname pgf@anchor@#1@#2\endcsname}%
}

% Defines a generic anchor, i.e. one which gets the associated shape
% as first argument.
%
% #1: the anchor name.
% #2: the code of the anchor. It may depend upon '##1', the shape's
% name.
%
% The anchor will be defined locally in the current TeX scope.
\def\pgfdeclaregenericanchor#1#2{%
	\expandafter\def\csname pgf@anchor@generic@#1\endcsname##1{#2}%
}%

% from pgfcoretransformations.code.tex :
\def\pgfgettransformentries#1#2#3#4#5#6{%
	\edef#1{\pgf@pt@aa}%
	\edef#2{\pgf@pt@ab}%
	\edef#3{\pgf@pt@ba}%
	\edef#4{\pgf@pt@bb}%
	\edef#5{\the\pgf@pt@x}%
	\edef#6{\the\pgf@pt@y}%
}%
\def\pgfsettransformentries#1#2#3#4#5#6{%
	\pgfsettransform{{#1}{#2}{#3}{#4}{#5}{#6}}%
}%

% pgfutil@loop (from plain.tex)

\def\pgfutil@loop#1\pgfutil@repeat{\def\pgfutil@body{#1}\pgfutil@iterate}
\def\pgfutil@iterate{\pgfutil@body \let\pgfutil@next\pgfutil@iterate \else\let\pgfutil@next\relax\fi \pgfutil@next}
\let\pgfutil@repeat=\fi % this makes \loop...\if...\repeat skippable

\def\pgfqpointxy#1#2{%
  \pgf@x=#1\pgf@xx%
  \advance\pgf@x by #2\pgf@yx%
  \pgf@y=#1\pgf@xy%
  \advance\pgf@y by #2\pgf@yy}
\def\pgfqpointxyz#1#2#3{%
  \pgf@x=#1\pgf@xx%
  \advance\pgf@x by #2\pgf@yx%
  \advance\pgf@x by #3\pgf@zx%
  \pgf@y=#1\pgf@xy%
  \advance\pgf@y by #2\pgf@yy%
  \advance\pgf@y by #3\pgf@zy}
\def\pgfcoordinate#1#2{%
  \edef\pgf@temp{#1}%
  \ifx\pgf@temp\pgfutil@empty% do nothing
  \else%
    \pgf@process{\pgfpointtransformed{#2}}%
    \expandafter\gdef\csname pgf@sh@ns@#1\endcsname{coordinate}%
    \expandafter\xdef\csname pgf@sh@np@#1\endcsname{%
      \noexpand\def\noexpand\centerpoint{\noexpand\pgfqpoint{\the\pgf@x}{\the\pgf@y}}%
    }
    \expandafter\gdef\csname pgf@sh@nt@#1\endcsname{{1}{0}{0}{1}{0pt}{0pt}}%
    \expandafter\global\expandafter\let\csname pgf@sh@ma@#1\endcsname\pgfutil@empty%
    \expandafter\gdef\csname pgf@sh@pi@#1\endcsname{\pgfpictureid}%
  \fi%
}
% A "quick" variant of \pgfpointscale which doesn't invoke the math parser for '#1'.
% #1 must be a number without units, no registers are accepted.
\def\pgfqpointscale#1#2{%
  \pgf@process{#2}%
  \pgf@x=#1\pgf@x%
  \pgf@y=#1\pgf@y%
}
% ======================================================

\def\pgfutilensuremath#1{%
	\ifmmode#1\else$#1$\fi
}

\tikzoption{ybar}[]{\let\tikz@plot@handler=\pgfplothandlerybar}
\tikzoption{xbar}[]{\let\tikz@plot@handler=\pgfplothandlerxbar}
\tikzoption{ybar interval}[]{\let\tikz@plot@handler=\pgfplothandlerybarinterval}
\tikzoption{xbar interval}[]{\let\tikz@plot@handler=\pgfplothandlerxbarinterval}
\tikzoption{const plot}[]{\let\tikz@plot@handler=\pgfplothandlerconstantlineto}
\tikzoption{const plot mark left}[]{\let\tikz@plot@handler=\pgfplothandlerconstantlineto}
\tikzoption{const plot mark right}[]{\let\tikz@plot@handler=\pgfplothandlerconstantlinetomarkright}
\tikzoption{jump mark right}[]{\let\tikz@plot@handler=\pgfplothandlerjumpmarkright}
\tikzoption{jump mark left}[]{\let\tikz@plot@handler=\pgfplothandlerjumpmarkleft}

\def\tikz@plot@samples{25}
\def\tikz@plot@domain{-5:5}
\def\tikz@plot@var{\x}
\def\tikz@plot@samplesat{-5,-4.5833333,...,5}

\tikzoption{mark}{
	\def\tikz@plot@mark{#1}%
	\def\tikz@temp{none}% this check is relatively new
	\ifx\tikz@temp\tikz@plot@mark
		\let\tikz@plot@mark=\pgfutil@empty
	\fi
}
% the 'every mark' style is new:
\tikzset{
	no marks/.style={mark=none},%
	every mark/.style={},
	mark options/.style={%
		every mark/.style={#1}%
	}}
\def\tikz@@@plot{%
    \def\pgfplotlastpoint{\pgfpointorigin}%
    \tikz@plot@handler%
    \tikz@plot@data%
    \global\let\tikz@@@temp=\pgfplotlastpoint%
    \ifx\tikz@plot@mark\pgfutil@empty%
    \else%
      % Marks are drawn after the path.
      \setbox\tikz@figbox=\hbox{%
        \unhbox\tikz@figbox%
        \hbox{{%
          \pgfinterruptpath%
            \pgfscope%
              \let\tikz@options=\pgfutil@empty%
              \let\tikz@transform=\pgfutil@empty%
			  \tikzset{every mark}%
              \tikz@options%
              \ifx\tikz@mark@list\pgfutil@empty%
                \pgfplothandlermark{\tikz@transform\pgfuseplotmark{\tikz@plot@mark}}%
              \else
                \pgfplothandlermarklisted{\tikz@transform\pgfuseplotmark{\tikz@plot@mark}}{\tikz@mark@list}%
              \fi
              \tikz@plot@data%
            \endpgfscope
          \endpgfinterruptpath%
        }}%
      }%
    \fi%
    \global\setbox\tikz@tempbox=\copy\tikz@figbox%
    %\global\let\tikz@after@path@smuggle=\tikz@after@path
  \endgroup%
  \setbox\tikz@figbox=\box\tikz@tempbox%  
  \tikz@make@last@position{\tikz@@@temp}%  
  %\expandafter\tikz@scan@next@command\tikz@after@path@smuggle%
  \tikz@scan@next@command%
}




% ======================================================


\newif\ifpgfmathcomparison

% Assigns \pgfmathresult to 1.0 if #1 ~= #1.
%
% It will also set the boolean \ifpgfmathcomparison accordingly
% (globally).
\def\pgfmathapproxequalto#1#2{%
	\edef\pgfmath@marshal{%
		\noexpand\pgfmathparse{#2}
		\noexpand\let\noexpand\pgfmath@arg\noexpand\pgfmathresult%
		\noexpand\pgfmathparse{#1}%
	}%
	\pgfmath@marshal%
	\pgfmathapproxequalto@{\pgfmathresult}{\pgfmath@arg}}
\def\pgfmathapproxequalto@#1#2{%
	\begingroup%
		\pgfmath@x#1pt%
		\pgfmath@y#2pt%
		\advance\pgfmath@x-\pgfmath@y%
		\ifdim\pgfmath@x<0pt
			\multiply\pgfmath@x by-1
		\fi
		\ifdim\pgfmath@x<0.0001pt\relax%
			\def\pgfmathresult{1.0}%
			\global\pgfmathcomparisontrue
		\else%
			\def\pgfmathresult{0.0}%
			\global\pgfmathcomparisonfalse
		\fi%
	\pgfmath@smuggleone\pgfmathresult
	\endgroup%
}

\newif\ifpgfmarktext@usetikznode
\pgfkeys{
	/pgf/text mark/.initial=p,
	/pgf/text mark style/.initial=,
	/pgf/text mark as node/.is if=pgfmarktext@usetikznode,
	/pgf/text mark as node/.default=true,
	%
	% backw. compat: the extra search path confuses the '.unknown'
	% handlers, so this here is deprecated:
	/pgf/text mark/style/.style={/pgf/text mark style={#1}},%
	/pgf/text mark/as node/.style={/pgf/text mark as node=#1},%
}%

\pgfdeclareplotmark{text}
{
	\pgfkeysgetvalue{/pgf/text mark style}\pgfmarktext@style
	\pgfkeysgetvalue{/pgf/text mark}\pgfmarktext@
	\ifpgfmarktext@usetikznode
		\expandafter\node\expandafter[\pgfmarktext@style]{\pgfmarktext@};
	\else
		\expandafter\pgftext\expandafter[\pgfmarktext@style]{\pgfmarktext@}%
	\fi
}

% A fix for the overlay option and matrices:
\def\pgf@matrix@startcell{%
  %
  % Step 1: Init the list of nodes for this cell
  %
  \let\pgf@nodecallback=\pgf@matrix@nodecallback%
  %
  % Step 2: Setup the bounding box
  %
  \pgfinterruptboundingbox%
  %
  % Step 3: Reset the transformation matrix
  %
  \pgftransformreset%
  %
  % Step 4: Collect everything in a cell box
  %
  \setbox\pgf@matrix@cell=\hbox\bgroup\bgroup%
	% make sure that cell pictures are not affected if matrizes have
	% 'overlay' option on:
	\pgf@relevantforpicturesizetrue
    \pgfsys@beginpicture%
    \normalbaselines%
    % Find out whether the cell is empty:
    \pgfutil@ifnextchar\let%
    {% ok, candidate, check following symbol
      \afterassignment\pgf@matrix@empty@check\let\pgf@matrix@temp=% get rid of \let
    }%
    {% no, not empty
      \pgf@matrix@empty@cell@false%
      \pgfmatrixbegincode%
    }%
}

% From pgfmoduleplot.code.tex:
{
	\catcode`\%=12
	\catcode`\"=12
	\xdef\pgf@gnuplot@head@pgf@two@oo#1{set terminal table; set output "#1.table"; set format "%.5f"}
	\ifx\pgf@gnuplot@head\pgf@gnuplot@head@pgf@two@oo
		\xdef\pgf@gnuplot@head#1{set table "#1.table"; set format "%.5f"}
	\else
		\xdef\pgf@gnuplot@head{set table \noexpand\pgf@plottablefile@quoted; set format "%.5f"}
	\fi
}

