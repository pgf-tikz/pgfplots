%--------------------------------------------
%
% Package pgfplots
%
% Provides a user-friendly interface to create function plots (normal
% plots, semi-logplots and double-logplots).
% 
% It is based on Till Tantau's PGF package.
%
% Copyright 2007-2012 by Christian Feuersänger.
%
% This program is free software: you can redistribute it and/or modify
% it under the terms of the GNU General Public License as published by
% the Free Software Foundation, either version 3 of the License, or
% (at your option) any later version.
% 
% This program is distributed in the hope that it will be useful,
% but WITHOUT ANY WARRANTY; without even the implied warranty of
% MERCHANTABILITY or FITNESS FOR A PARTICULAR PURPOSE.  See the
% GNU General Public License for more details.
% 
% You should have received a copy of the GNU General Public License
% along with this program.  If not, see <http://www.gnu.org/licenses/>.
%
%--------------------------------------------

% PRECONDITION: 
% 	- final axis limits are given in transformed range
% 	-  \pgfplots@set@default@size@options has been invoked before
% POSTCONDITION: 
% 	- the current x,y and z unit vectors are defined properly;
% 	- the fast-access registers are initialised for the axis limits,
%	- the following macros are assigned:
%		\pgfplots@[xyz]@veclength
%		\pgfplots@[xyz]@inverseveclength
%		\pgfplotspointxaxis
%		\pgfplotspointyaxis
%		\pgfplotspointzaxis
%		\pgfplotspointcenter
%		\pgfplotspointminminmin
%
\def\pgfplots@initsizes{%
	% INIT.
	%
	%
	\pgfplots@xmin@reg=\pgfplots@xmin pt %
	\pgfplots@xmax@reg=\pgfplots@xmax pt %
	\pgfplots@ymin@reg=\pgfplots@ymin pt %
	\pgfplots@ymax@reg=\pgfplots@ymax pt %
	\ifpgfplots@threedim
		\pgfplots@zmin@reg=\pgfplots@zmin pt %
		\pgfplots@zmax@reg=\pgfplots@zmax pt %
	\fi
	%
	%-----------------------------------------
	% PROCESS THE 'width' and 'height' options
	%-----------------------------------------
	%
	%
	\pgfkeysgetvalue{/pgfplots/view/az}{\pgfplots@view@az}%
	\pgfkeysgetvalue{/pgfplots/view/el}{\pgfplots@view@el}%
	\ifpgfplots@threedim
	\else
		\let\pgfplots@view@el=\pgfutil@empty
		\let\pgfplots@view@az=\pgfutil@empty
	\fi
	\ifx\pgfplots@view@az\pgfutil@empty
		% Note that in presence of "x,y,z" options, the
		% \pgfplots@set@default@size@options RESETS /pgfplots/view/az.
		%
		%\let\pgfplots@rectangle@width=\pgfutil@empty
		%\let\pgfplots@rectangle@height=\pgfutil@empty
		\pgfplotsmathvectorfromstring{0,0,1}{default}%
		\let\pgfplots@view@dir@threedim=\pgfplotsretval
		%
		\pgfpointdiff
			{\pgfplotsqpointxy{\pgfplots@xmin}{\pgfplots@ymin}}
			{\pgfplotsqpointxy{\pgfplots@xmax}{\pgfplots@ymax}}%
		% only used temporarily in this macro to compute the correct
		% length for unit vectors:
		\edef\pgfplots@initsizes@axisdiag@x{\the\pgf@x}%
		\edef\pgfplots@initsizes@axisdiag@y{\the\pgf@y}%
		%
		\ifx\pgfplots@x\pgfutil@empty
			\ifx\pgfplots@width\pgfutil@empty
				\pgfplots@error{INTERNAL LOGIC ERROR! WIDTH NOT SET}%
			\fi
		\fi
		%
		\ifx\pgfplots@y\pgfutil@empty
			\ifx\pgfplots@height\pgfutil@empty
				\pgfplots@error{INTERNAL LOGIC ERROR! HEIGHT NOT SET}%
			\fi
		\fi
		\ifpgfplots@threedim
			\def\pgfplots@tmp@Zscale{1}%
		\else
			\def\pgfplots@tmp@Zscale{0}%
		\fi
		%
		\pgfplots@initsizes@setunitvector{x}{0}{1}{\pgfplots@tmp@xisaxisparallel}%
		\pgfplots@initsizes@setunitvector{y}{1}{1}{\pgfplots@tmp@yisaxisparallel}%
		\pgfplots@initsizes@setunitvector{z}{2}{\pgfplots@tmp@Zscale}{\pgfplots@loc@TMPc}%
		\ifcase\pgfplots@scale@mode@choice\relax
			% scale mode=auto : check if units have been provided
			% explicitly.
			\ifx\pgfplots@x\pgfutil@empty
			\else
				\def\pgfplots@scale@mode@choice{1}% none
			\fi
			\ifx\pgfplots@y\pgfutil@empty
			\else
				\def\pgfplots@scale@mode@choice{1}% none
			\fi
			\ifx\pgfplots@z\pgfutil@empty
			\else
				\def\pgfplots@scale@mode@choice{1}% none
			\fi
		\fi
		\pgfplots@scale@plotbox@to@widthheight
		\ifpgfplots@threedim
			\pgfplotsgetnormalforcurrentview
		\fi
	\else
		% 3D case by `view':
		\let\pgfplots@x=\pgfutil@empty
		\let\pgfplots@y=\pgfutil@empty
		\let\pgfplots@z=\pgfutil@empty
		% this invokes \pgfplots@scale@plotbox@to@widthheight automatically.
		\pgfplotssetaxesfromazel{\pgfplots@view@az}{\pgfplots@view@el}{\pgfplots@tmp@xisaxisparallel}%
		\if1\pgfplots@tmp@xisaxisparallel%
			\def\pgfplots@tmp@yisaxisparallel{1}%
		\fi
	\fi
%\message{Pgfplots debug: initialised unit vectors to x=(\the\pgf@xx,\the\pgf@xy), y=(\the\pgf@yx,\the\pgf@yy), z=(\the\pgf@zx,\the\pgf@zy), n = (\pgfplots@view@dir@threedim).^^J }%
	%
	\let\pgfplotsmathfloatviewdepthxyz@=\pgfplotsmathfloatviewdepthxyz@infigure
	\let\pgfplotsmathviewdepthxyz@=\pgfplotsmathviewdepthxyz@infigure
	%
	\pgfplotsmath@ifzero{\pgfplots@x@veclength}{\pgfplots@hide@xtrue\pgfplots@shownothingof@xtrue}{}%
	\pgfplotsmath@ifzero{\pgfplots@y@veclength}{\pgfplots@hide@ytrue\pgfplots@shownothingof@ytrue}{}%
	\ifpgfplots@threedim
		\pgfplotsmath@ifzero{\pgfplots@z@veclength}{\pgfplots@hide@ztrue\pgfplots@shownothingof@ztrue}{}%
	\else
		\if1\pgfplots@tmp@xisaxisparallel%
			\if1\pgfplots@tmp@yisaxisparallel%
				% Optimize for axis-parallel case!
				% puh. Did not make any measureable difference!? Ok...
				\let\pgfplotsqpointxy=\pgfplotsqpointxy@orthogonal
			\fi
		\fi
	\fi
	%
	\pgfplots@axis@apply@post@scale x
	\pgfplots@axis@apply@post@scale y
	\ifpgfplots@threedim
		\pgfplots@axis@apply@post@scale z
	\fi
	\pgfplots@rescale@view@dir
}

% Defines \pgfmathresult to be the desired width without axis labels.
%
\def\pgfplots@initsizes@get@width@withoutlabels{%
	\pgfplots@initsizes@handle@label@const{\pgfplots@width}{45pt}{width}%
}

% Defines \pgfmathresult to be the desired height without axis labels.
\def\pgfplots@initsizes@get@height@withoutlabels{%
	\pgfplots@initsizes@handle@label@const{\pgfplots@height}{45pt}{height}%
}

\def\pgfplots@initsizes@handle@label@const#1#2#3{%
	\begingroup
	\pgf@xa=#1\relax
	% EXPECTED WIDTH = X = \pgfplots@width
	% ACTUAL WIDTH = c + x * (xmax-xmin)
	% where c is a CONSTANT (for the axis labels/tick labels).
	% -> \pgfplots@tmpXscale = (X - c) / (x *(xmax-xmin))
	%
	% \pgf@xa := X-c:
	\ifpgfplots@scale@only@axis
	\else
		\advance\pgf@xa by-#2 % FIXME determine 'c' correctly!
	\fi
	\ifdim\pgf@xa<0pt
		\pgfplots@error{Error: Plot #3 `#1' is too small. This cannot be implemented while maintaining constant size for labels. Sorry, label sizes are only approximate. You will need to adjust your #3.}%
		\pgf@xa=0pt
	\fi
	\edef\pgfmathresult{\the\pgf@xa}%
	\pgfmath@smuggleone\pgfmathresult
	\endgroup
}%

% Helper method for initsizes.
%
% It computes a scaling such that \pgfplots@width = SCALE * ACTUAL WIDTH.
% 
% The actual width is 
% 	c + x*(xmax-xmin)
% based on
% - c = estimated, a constant for the axis label/tick labels
%
% Arguments: 
% #1: is expected to be x*(xmax-xmin) measured in pt.
% #2: the output argument for the SCALE.
\def\pgfplots@initsizes@getXscale#1\into#2{%
	\begingroup
	\pgfplots@initsizes@get@width@withoutlabels
	% EXPECTED WIDTH = X = \pgfplots@width
	% ACTUAL WIDTH = c + x * (xmax-xmin)
	% where c is a CONSTANT (for the axis labels/tick labels).
	% -> \pgfplots@tmpXscale = (X - c) / (x *(xmax-xmin))
	%
	% \pgf@xa := X-c:
	% \pgf@xb := x*(xmax-xmin):
	\pgf@xa=\pgfmathresult\relax
	\pgf@xb=#1\relax
	\pgfmathlog@invoke@expanded\pgfmathdivide@{%
		{\pgf@sys@tonumber\pgf@xa}%
		{\pgf@sys@tonumber\pgf@xb}%
	}%
%--------------------------------------------------
% \message{pgfplots.sty: Computing 'x' such that 'width = c + x*(xmax-xmin)';
% 	c=estimated, 
% 	width-c =\the\pgf@xa,  
% 	x*(xmax - xmin)]) = \the\pgf@xb  
% 	-> x-scale =\pgfmathresult }%
%-------------------------------------------------- 
	\pgfmath@smuggleone\pgfmathresult
	\endgroup
	\let#2=\pgfmathresult
}

% The same as \pgfplots@initsizes@getXscale, just for the height.
\def\pgfplots@initsizes@getYscale#1\into#2{%
	\begingroup
	\pgfplots@initsizes@get@height@withoutlabels
	% \pgf@xb := y*(ymax-ymin):
	\pgf@xa=\pgfmathresult\relax
	\pgf@xb=#1%
	\pgfmathlog@invoke@expanded\pgfmathdivide@{%
		{\pgf@sys@tonumber\pgf@xa}%
		{\pgf@sys@tonumber\pgf@xb}%
	}%
%\pgfplots@message{pgfplots.sty: Computing 'y' such that 'height = c + y*(ymax-ymin)';
%	height=\pgfplots@height,
%	c=estimated,
%	height-c =\the\pgf@xa,  
%	y*(ymax[=\the\pgfplots@ycoordmaxTEX] - ymin[=\the\pgfplots@ycoordminTEX)]) = \the\pgf@xb  
%	-> y-scale =#2 }%
	\pgfmath@smuggleone\pgfmathresult
	\endgroup
	\let#2=\pgfmathresult
}

\def\pgfplots@axis@apply@post@scale#1{%
	%
	\pgfkeysgetvalue{/pgfplots/#1 post scale}\pgfmathresult
	\ifx\pgfmathresult\pgfutil@empty
	\else
		\pgfmathparse{\pgfmathresult}%
		\ifdim\pgfmathresult pt=1pt
		\else
			\let\pgfplots@loc@TMPa=\pgfmathresult
			\csname pgf@#1x\endcsname=\pgfmathresult\csname pgf@#1x\endcsname
			\csname pgf@#1y\endcsname=\pgfmathresult\csname pgf@#1y\endcsname
			%
			\pgfmathmultiply@{\csname pgfplots@#1@veclength\endcsname}{\pgfplots@loc@TMPa}%
			\expandafter\let\csname pgfplots@#1@veclength\endcsname=\pgfmathresult
			%
			\pgfmathdivide@{\csname pgfplots@#1@inverseveclength\endcsname}{\pgfplots@loc@TMPa}%
			\expandafter\let\csname pgfplots@#1@inverseveclength\endcsname=\pgfmathresult
		\fi
	\fi
}

% Takes azimuth (horizontal angle) '#1' and elongation (vertical
% angle) '#2' (both in degrees) and computes 
% x,y and z vectors which define the view in the direction
% defined by '#1' and '#2'.
%
% 'azimuth' means a rotation around the viewport's x axis. 'elongation' means
% a rotation around the original coordinate system's z axis.
%
% The method works by computing 
% Az = [ cos(azimuth) -sin(azimuth) 0; ...
%     sin(azimuth) cos(azimuth) 0; ...
%     0 0 1 ];
% 
%
% Ax = [ 1 0 0; ...
%     0 cos(elevation) -sin(elevation) ;...
%     0 sin(elevation) cos(elevation) ];
%
% v= Ax * Az;
% = [ ...
% 	cosaz  -sinaz cosel  sinaz sinel; ...
% 	sinaz  cosaz cosel   -sinel cosaz; ...
% 	0	  sinel         cosel ];
%
% Then, we use the rotated XZ plane as viewport, that means 
%   xvec = v * [1 0 0]' = <first column of v>
%   zvec = v * [0 0 1]' = <third column of v>
% and we define the projection onto the twodimensional surface
% spanned by 'xvec' and 'zvec' as
%   P( q ) = [ q^T xvec,  q^T zvec ]'
% for q in R^3.
% As a consequence, we compute the three unit vectors as
%  x = P( [1 0 0] )
%  	 = [ cosaz,  sinaz sinel ]'
%  y = P( [0 1 0] )
%    = [ sinaz,  -sinel cosaz ]'
%  z = P( [0 0 1] )
%    = [ 0,  cosel]'
%
% Furthermore, the 3D view vector which points into the direction of the view
% is
%   n = v * [0 1 0 ]' = <second column of v> =  [-sinaz cosel,  cosaz cosel,  sinel]' 
% because the normal view point was the XZ plane with y as its normal
% vector.
% The 3D vector n is returned by this routine as well - it is
% necessary for some kind of z buffering (determining what is
% foreground and what is background).
%
% INPUT:
% - #1 : azimuth ("yaw")
% - #2 : elevation ("pitch")
% OUTPUT:
% - #3 : a macro which will be set to '1' if and only if 
%      the viewport is the standard XY axis (i.e. azimuth=0, elevation=90).
% - [xyz] vectors,
%   \pgfplots@[xyz]@veclength,
%   \pgfplots@[xyz]@inverseveclength
%   are set properly
%   \pgfplots@view@dir@threedim will contain the three components
%   of 'n' (without the suffix 'pt', but in units of 'pt') (see
%   \pgfplotsmathvectorfromstring).
\def\pgfplotssetaxesfromazel#1#2#3{%
	\begingroup
	\pgfmathparse{#1}%
	\let\pgfplots@az=\pgfmathresult
	\pgfmathparse{#2}%
	\edef\pgfplots@el{-\pgfmathresult}%
	\pgfmathsin@{\pgfplots@az}%
	\let\sinaz=\pgfmathresult
	\pgfmathcos@{\pgfplots@az}%
	\let\cosaz=\pgfmathresult
	\pgfmathsin@{\pgfplots@el}%
	\let\sinel=\pgfmathresult
	\pgfmathcos@{\pgfplots@el}%
	\let\cosel=\pgfmathresult
	% x:
	\pgfmathmultiply@{\sinaz}{\sinel}%
	\xdef\pgfplots@glob@TMPa{\noexpand\pgfqpoint{\cosaz pt}{\pgfmathresult pt}}%
	% y:
	\pgfmathmultiply@{-\sinel}{\cosaz}%
	\xdef\pgfplots@glob@TMPb{\noexpand\pgfqpoint{\sinaz pt}{\pgfmathresult pt}}%
	% z:
	\xdef\pgfplots@glob@TMPc{\noexpand\pgfqpoint{0pt}{\cosel pt}}%
	%
	\pgfkeysgetvalue{/pgfplots/x dir/value}\pgfplots@loc@dirvalue@x
	\pgfkeysgetvalue{/pgfplots/y dir/value}\pgfplots@loc@dirvalue@y
	\pgfkeysgetvalue{/pgfplots/z dir/value}\pgfplots@loc@dirvalue@z
	\if r\pgfplots@loc@dirvalue@x
		\t@pgfplots@toka=\expandafter{\pgfplots@glob@TMPa}%
		\xdef\pgfplots@glob@TMPa{\noexpand\pgfqpointscale{-1}{\the\t@pgfplots@toka}}%
	\fi
	\if r\pgfplots@loc@dirvalue@y
		\t@pgfplots@toka=\expandafter{\pgfplots@glob@TMPb}%
		\xdef\pgfplots@glob@TMPb{\noexpand\pgfqpointscale{-1}{\the\t@pgfplots@toka}}%
	\fi
	\if r\pgfplots@loc@dirvalue@z
		\t@pgfplots@toka=\expandafter{\pgfplots@glob@TMPc}%
		\xdef\pgfplots@glob@TMPc{\noexpand\pgfqpointscale{-1}{\the\t@pgfplots@toka}}%
	\fi
	%
	% Process 'plot box ratio':
	\def\pgfplots@extract@plot@box@ratio##1##2##3##4\pgfplots@EOI{%
		\pgfmathparse{##1}\let\pgfplots@plotboxratio@x=\pgfmathresult
		\pgfmathparse{##2}\let\pgfplots@plotboxratio@y=\pgfmathresult
		\pgfmathparse{##3}\let\pgfplots@plotboxratio@z=\pgfmathresult
	}%
	\def\pgfplots@extract@plot@box@ratio@spaces##1 ##2 ##3 ##4\pgfplots@EOI{%
		\pgfplots@extract@plot@box@ratio{##1}{##2}{##3}{##4}\pgfplots@EOI
	}%
	\pgfkeysgetvalue{/pgfplots/plot box ratio}\pgfplots@loc@TMPa
	% Auto-determine input format which is either '{x}{y}{z}' or 'x y z'
	\def\pgfplots@loc@TMPb{%
		\pgfutil@ifnextchar\bgroup{%
			\pgfplots@loc@tmptrue
			\pgfplots@gobble@until@EOI
		}{%
			\pgfplots@loc@tmpfalse
			\pgfplots@gobble@until@EOI
		}%
	}%
	\expandafter\pgfplots@loc@TMPb\pgfplots@loc@TMPa\pgfplots@EOI
	\ifpgfplots@loc@tmp
		% Ah- braces format.
		\edef\pgfplots@loc@TMPa{\pgfplots@loc@TMPa{1}{1}{1}}%
		\expandafter\pgfplots@extract@plot@box@ratio\pgfplots@loc@TMPa\pgfplots@EOI
	\else
		% Ah- space-separated
		\edef\pgfplots@loc@TMPa{\pgfplots@loc@TMPa\space 1 1 1}%
		\expandafter\pgfplots@extract@plot@box@ratio@spaces\pgfplots@loc@TMPa\pgfplots@EOI
	\fi
	%
	% process it:
	\ifdim\pgfplots@plotboxratio@x pt=1pt
	\else
		\t@pgfplots@toka=\expandafter{\pgfplots@glob@TMPa}%
		\xdef\pgfplots@glob@TMPa{\noexpand\pgfqpointscale{\pgfplots@plotboxratio@x}{\the\t@pgfplots@toka}}%
	\fi
	\ifdim\pgfplots@plotboxratio@y pt=1pt
	\else
		\t@pgfplots@toka=\expandafter{\pgfplots@glob@TMPb}%
		\xdef\pgfplots@glob@TMPb{\noexpand\pgfqpointscale{\pgfplots@plotboxratio@y}{\the\t@pgfplots@toka}}%
	\fi
	\ifdim\pgfplots@plotboxratio@z pt=1pt
	\else
		\t@pgfplots@toka=\expandafter{\pgfplots@glob@TMPc}%
		\xdef\pgfplots@glob@TMPc{\noexpand\pgfqpointscale{\pgfplots@plotboxratio@z}{\the\t@pgfplots@toka}}%
	\fi
	%
	% n (3D!)
	\pgfmathmultiply@{-\sinaz}{\cosel}%
	\let\pgfmathresultNx=\pgfmathresult
	\pgfmathmultiply@{\cosaz}{\cosel}%
	\xdef\pgfplots@glob@TMPd{{\pgfmathresultNx}{\pgfmathresult}{\sinel}}%
	\endgroup
%\message{Setting x,y and z from {#1}{#2} to^^J x = \meaning\pgfplots@glob@TMPa,^^J y = \meaning\pgfplots@glob@TMPb,^^J z = \meaning\pgfplots@glob@TMPc,^^J n = \pgfplots@glob@TMPd.^^J}%
	\pgfsetxvec{\pgfplots@glob@TMPa}%
	\pgfsetyvec{\pgfplots@glob@TMPb}%
	\pgfsetzvec{\pgfplots@glob@TMPc}%
	\def\pgfplots@loc@TMPa##1##2##3{%
		\pgfplotsmathvectorfromstring{##1,##2,##3}{default}%
		\let\pgfplots@view@dir@threedim=\pgfplotsretval
	}%
	\expandafter\pgfplots@loc@TMPa\pgfplots@glob@TMPd\relax
	\def#3{0}%
	\pgfplots@scale@plotbox@to@widthheight
}%

% Takes the current plot box, defined by the actual PGF x,y and z unit
% vectors, and re-scales it such that it fits into the 
% width and height of the axis (as they have been provided by the
% user).
%
% @see\pgfplots@scaleaxes@to@BB
\def\pgfplots@scale@plotbox@to@widthheight{%
	\pgfplots@initsizes@get@width@withoutlabels
	\let\pgfplots@loc@TMPa=\pgfmathresult
	\pgfplots@initsizes@get@height@withoutlabels
	\edef\pgfplots@loc@TMPa{{\pgfplots@loc@TMPa}{\pgfmathresult}}%
	\expandafter\pgfplots@scaleaxes@to@BB\pgfplots@loc@TMPa
}


% Takes the current PGF x,y and z unit vectors and scales them such
% that the bounding box of the final image has width #1 and height #2.
%
% The relative length of the input vectors is important for the 3D case: it
% will be scaled as-is. 
%
% PRECONDITION
% 	- the x, y and z unit vectors have been set to the proper
% 	DIRECTIONS. Their relative vector lengths are set-up properly
% 	(i.e. y is twice as large as x and half as large as z or so).
%	- \pgfplots@[xyz]@veclength and
%	  \pgfplots@[xyz]@inverseveclength
%	  are set correctly.
%	- the \ifpgfplots@threedim boolean is set.
%	- the data limits have been initialised and transformed according
%	to the data transformation.
% 
% POSTCONDITION
% 	- the unit vectors have been re-scaled such that the final plot
% 	has the desired dimensions.
% 	- the @veclength and @inverseveclength have been re-scaled as
% 	well.
\def\pgfplots@scaleaxes@to@BB#1#2{%
	\if0\pgfplots@scale@mode@choice
		% scale mode=auto
		\def\pgfplots@scale@mode@choice{2}% stretch to fill
	\fi
	\if1\pgfplots@scale@mode@choice
		% scale mode=none
	\else
		\pgfplots@scaleaxes@to@BB@{#1}{#2}%
	\fi
	% FIXME 
	% I suppose I should call \pgfplotsgetnormalforcurrentview now?
	\pgfplots@computeunitvectorlengths
}%

\def\pgfplots@rescale@view@dir{%
	\expandafter\ifx\csname pgfplots@view@dir@threedim\endcsname\relax
	\else
		% set up a scaling vector. Use scale '1' if the axis has length 0:
		\pgfplotsmathvectorfromstring{%
			\ifdim\pgfplots@x@veclength pt=0pt 1\else\pgfplots@x@inverseveclength\fi,%
			\ifdim\pgfplots@y@veclength pt=0pt 1\else\pgfplots@y@inverseveclength\fi,%
			\ifdim\pgfplots@z@veclength pt=0pt 1\else\pgfplots@z@inverseveclength\fi%
		}{default}%
		\pgfplotsmathvectorscaleindividually{\pgfplots@view@dir@threedim}{\pgfplotsretval}{default}%
		\let\pgfplots@view@dir@threedim=\pgfplotsretval
		%
		% normalize. This is not absolutely required -- but it is used
		% to accumulate point depth (for the mesh handler) in pgfmath
		% arithmetics. At least \pgfplotsmathviewdepthxyz should use
		% a properly scaled view dir.
		\pgfplotsmathvectorlength{\pgfplots@view@dir@threedim}{default}%
		\pgfplotscoordmath{default}{op}{reciprocal}{{\pgfplotsretval}}%
		\pgfplotsmathvectorscale{\pgfplots@view@dir@threedim}{\pgfmathresult}{default}%
		\let\pgfplots@view@dir@threedim=\pgfplotsretval
	\fi
}%

\def\pgfplots@BB@for@plotbox{%
	% We just need the
	% relative sizes, i.e.
	%  "0" == lower axis limit
	%  "1" == upper axis limit
	%
	\ifpgfplots@threedim
		\pgfpathmoveto{\pgfqpointxyz000}%
		\pgfpathmoveto{\pgfqpointxyz001}%
		\pgfpathmoveto{\pgfqpointxyz010}%
		\pgfpathmoveto{\pgfqpointxyz011}%
		\pgfpathmoveto{\pgfqpointxyz100}%
		\pgfpathmoveto{\pgfqpointxyz101}%
		\pgfpathmoveto{\pgfqpointxyz110}%
		\pgfpathmoveto{\pgfqpointxyz111}%
	\else
		\pgfpathmoveto{\pgfqpointxy00}%
		\pgfpathmoveto{\pgfqpointxy01}%
		\pgfpathmoveto{\pgfqpointxy10}%
		\pgfpathmoveto{\pgfqpointxy11}%
	\fi
}%

% \pgfplots@BB@for@plotbox@get@unit@scales@for@limits{#1}{#2}{#3}:
% a helper tool which computes individual unit vector scales in order
% to respect the limits.
%
% This method ignores width/height; its purpose is only to make sure
% that [xmin,xmax] fits into the CURRENT plot box. 
%
% In this context, each unit vector is supposed to be scaled such that
% width/height fit if xmin=0 and xmax=1.
%
% #1 [output] a macro name which will contain the scale for x
% #2 [output] a macro name which will contain the scale for y
% #3 [output] a macro name which will contain the scale for z
% 
\def\pgfplots@BB@for@plotbox@get@unit@scales@for@limits#1#2#3{%
	% Consequently, we have to multiply  with 1/(max-min):
	\pgfmathsubtract@{\pgfplots@xmax}{\pgfplots@xmin}%
	% compute 1/(xmax - xmin) in float for more recent versions (see /pgfplots/compat/scaling).
	% I observed that it is much more accurate 
	\pgfplotscoordmath{\pgfplots@compat@scaling@coordmath}{parsenumber}{\pgfmathresult}%
	\pgfplotscoordmath{\pgfplots@compat@scaling@coordmath}{op}{reciprocal}{{\pgfmathresult}}%
	\pgfplotscoordmath{\pgfplots@compat@scaling@coordmath}{tofixed}{\pgfmathresult}%
	\let#1=\pgfmathresult
	%
	\pgfmathsubtract@{\pgfplots@ymax}{\pgfplots@ymin}%
	\pgfplotscoordmath{\pgfplots@compat@scaling@coordmath}{parsenumber}{\pgfmathresult}%
	\pgfplotscoordmath{\pgfplots@compat@scaling@coordmath}{op}{reciprocal}{{\pgfmathresult}}%
	\pgfplotscoordmath{\pgfplots@compat@scaling@coordmath}{tofixed}{\pgfmathresult}%
	\let#2=\pgfmathresult
	%
	\ifpgfplots@threedim
		\pgfmathsubtract@{\pgfplots@zmax}{\pgfplots@zmin}%
		\pgfplotscoordmath{\pgfplots@compat@scaling@coordmath}{parsenumber}{\pgfmathresult}%
		\pgfplotscoordmath{\pgfplots@compat@scaling@coordmath}{op}{reciprocal}{{\pgfmathresult}}%
		\pgfplotscoordmath{\pgfplots@compat@scaling@coordmath}{tofixed}{\pgfmathresult}%
	\else
		\def\pgfmathresult{1}%
	\fi
	\let#3=\pgfmathresult
}%



% Returns width and height of the current plot box
% (the path produced by \pgfplots@BB@for@plotbox).
%
% PRECONDITION: \pgfplots@BB@for@plotbox is defined to produce a path
% for the plot box
%
% POSTCONDITION: \pgfplotsretval contains the with and
% \pgfplotsretvalb contains the height
\def\pgfplots@get@dimension@of@BB{%
	\pgfinterruptboundingbox
	% STEP 1: compute the bounding box for the plot box.
	\pgfplots@BB@for@plotbox
	%
	% TMPa = width
	\pgf@xa=\pgf@pathmaxx
	\advance\pgf@xa by-\pgf@pathminx
	% TMPb = height
	\pgf@xb=\pgf@pathmaxy
	\advance\pgf@xb by-\pgf@pathminy
	\xdef\pgfplots@glob@TMPa{%
		\def\noexpand\pgfplotsretval{\the\pgf@xa}%
		\def\noexpand\pgfplotsretvalb{\the\pgf@xb}%
	}%
	\pgfusepath{discard}%
	\endpgfinterruptboundingbox
	\pgfplots@glob@TMPa
}%

\def\pgfplots@scaleaxes@to@BB@#1#2{%
	\begingroup
	\pgfplots@get@scale@horiz@and@vert
		{#1}%
		{#2}%
		{\xscale}%
		{\yscale}% yscale
		{\pgfplots@target@limitrescale@x}%
		{\pgfplots@target@limitrescale@y}%
		{\pgfplots@target@limitrescale@z}%
\message{Got W/H scale for all x components: \xscale; for all y components: \xscale; ^^J   axis limit componsation scales x=\pgfplots@target@limitrescale@x, y=\pgfplots@target@limitrescale@y, z=\pgfplots@target@limitrescale@z^^J}%
	% Ok, we know the W,H scalings now.
	%
	%
	\def\pgfplots@target@unit@scale@x{0}% the value is irrelevant - will be overwritten.
	\def\pgfplots@target@unit@scale@y{0}%
	\def\pgfplots@target@unit@scale@z{0}%
	%
	% But remember that we have worked with RELATIVE scales up to now.
	% We have to convert them now, by interpreting "1 == upper limit"
	% and "0 == lower limit" for every direction.
	%
	% This here CAN cause anisotropic (different) scaling factors.
	\pgfplots@BB@for@plotbox@get@unit@scales@for@limits
		{\pgfplots@target@unit@scale@x}
		{\pgfplots@target@unit@scale@y}
		{\pgfplots@target@unit@scale@z}%
	%
\message{got scales to fit limits into BB: scalex=\pgfplots@target@unit@scale@x, scaley=\pgfplots@target@unit@scale@y, scalez=\pgfplots@target@unit@scale@z, ^^J   axis limit componsation scales x=\pgfplots@target@limitrescale@x, y=\pgfplots@target@limitrescale@y, z=\pgfplots@target@limitrescale@z^^J}%
	%
	\if3\pgfplots@scale@mode@choice
		% scale mode=scale uniformly
		% 
		% We need to recompensate in case the previous method chose
		% different unit scaling scalings:
		\pgfplots@BB@for@plotbox@get@unit@scales@compensated@axis@limits
			{\pgfplots@target@unit@scale@x}
			{\pgfplots@target@unit@scale@y}
			{\pgfplots@target@unit@scale@z}
			{\pgfplots@target@limitrescale@x@}{\pgfplots@target@limitrescale@y@}{\pgfplots@target@limitrescale@z@}%
		%
		\pgfplots@BB@update@cumulative@limit@compensations
	\fi
\message{adjusted scales for 'scale mode': x=\pgfplots@target@unit@scale@x, y=\pgfplots@target@unit@scale@y, z=\pgfplots@target@unit@scale@z;  ^^J   axis limit componsation scales x=\pgfplots@target@limitrescale@x, y=\pgfplots@target@limitrescale@y, z=\pgfplots@target@limitrescale@z^^J}%
	%
	\pgfplots@apply@unit@ratio
		{\pgfplots@target@unit@scale@x}
		{\pgfplots@target@unit@scale@y}
		{\pgfplots@target@unit@scale@z}
		{\pgfplots@target@limitrescale@x@}{\pgfplots@target@limitrescale@y@}{\pgfplots@target@limitrescale@z@}%
	\pgfplots@BB@update@cumulative@limit@compensations
	%
\message{adjusted scales for 'unit vector ratio': scalex=\pgfplots@target@unit@scale@x, scaley=\pgfplots@target@unit@scale@y, scalez=\pgfplots@target@unit@scale@z;  axis limit componsation scales x=\pgfplots@target@limitrescale@x, y=\pgfplots@target@limitrescale@y, z=\pgfplots@target@limitrescale@z^^J}%
	%
	%
	% and finally, resize limits appropriately and add all cumulative limit compensations:
	\xdef\pgfplots@glob@TMPa{%
		\pgf@xx=\xscale\pgf@xx
		\pgf@xy=\yscale\pgf@xy
		%
		\pgf@yx=\xscale\pgf@yx
		\pgf@yy=\yscale\pgf@yy
		%
		\ifpgfplots@threedim
			\pgf@zx=\xscale\pgf@zx
			\pgf@zy=\yscale\pgf@zy
		\fi
		%
		%
		\pgf@xx=\pgfplots@target@unit@scale@x\pgf@xx
		\pgf@xy=\pgfplots@target@unit@scale@x\pgf@xy
		%
		\pgf@yx=\pgfplots@target@unit@scale@y\pgf@yx
		\pgf@yy=\pgfplots@target@unit@scale@y\pgf@yy
		%
		\ifpgfplots@threedim
			\pgf@zx=\pgfplots@target@unit@scale@z\pgf@zx
			\pgf@zy=\pgfplots@target@unit@scale@z\pgf@zy
		\fi
		%
		\noexpand\pgfplots@apply@unit@vector@rescale@keep@size{x}{\pgfplots@target@limitrescale@x}%
		\noexpand\pgfplots@apply@unit@vector@rescale@keep@size{y}{\pgfplots@target@limitrescale@y}%
		\noexpand\pgfplots@apply@unit@vector@rescale@keep@size{z}{\pgfplots@target@limitrescale@z}%
	}%
	\endgroup
	\pgfplots@glob@TMPa
}%

% Computes the initial scale from a plot box of unit size to the
% desired with and height.
%
% #1 the desired width
% #2 the desired height
% #3 [output] a macro which will contain the horizontal (x) scale
% #4 [output] a macro which will contain the vertical (y) scale
% #5 [output] a macro which will contain a x axis limit compensation scale
% #6 [output] a macro which will contain a y axis limit compensation scale
% #7 [output] a macro which will contain a z axis limit compensation scale
\def\pgfplots@get@scale@horiz@and@vert#1#2#3#4#5#6#7{%
	\begingroup
	\pgfplots@get@dimension@of@BB
	\pgf@xa=\pgfplotsretval\relax
	\pgf@xb=\pgfplotsretvalb\relax
	\pgf@ya=#1\relax
	\pgf@yb=#2\relax
	\edef\w{\pgf@sys@tonumber\pgf@xa}%
	\edef\h{\pgf@sys@tonumber\pgf@xb}%
	\edef\W{\pgf@sys@tonumber\pgf@ya}%
	\edef\H{\pgf@sys@tonumber\pgf@yb}%
\message{PGFPLOTS: the current unit vectors result in a UNIT BB of (\the\pgf@xa,\the\pgf@xb). Scaling it to (\the\pgf@ya,\the\pgf@yb)...^^J}%
	\ifcase\pgfplots@scale@mode@choice
		% scale mode=auto  does not happen here
	\or
		% scale mode=none  does not happen here
	\or
		% scale mode=stretch to fill
		% 
		% This is very simple:
		%
		% Compute individual scaling factors for X and Y
		% such that the UNIT-BB will have size #1,#2. Keep limits.
		\pgfmathdivide@{\W}{\w}%
		\let\scalex=\pgfmathresult
		%
		\pgfmathdivide@{\H}{\h}%
		\let\scaley=\pgfmathresult
		% 
		% no changes to the axis limits - we only rescale units.
		\def\pgfplots@target@limitrescale@x{1}%
		\def\pgfplots@target@limitrescale@y{1}%
		\def\pgfplots@target@limitrescale@z{1}%
	\or
		% scale mode=scale uniformly
		% compute ONE common scale for both, X and Y - and satisfy
		% width/height constraints by adjusting the axis limits.
		%
		% The idea is as follows:
		% we WANT to have width W and height H.
		% The constraint is that each unit vector must get the same
		% scale -- but the axis limits can receive individual
		% compensation scales. But it should "look reasonable well".
		%
		% currently, we have 
		% w = r_x e_xx  + r_y e_yx + rz e_zx  (with e_zx = 0 typically)
		% h = r_x e_xy  + r_y e_yy + rz e_zy
		%
		% where r_x, r_y, r_z are the maximal range of the data in
		% x,y,z respectively. Depending on the context of this method,
		% they are either 1 (relative coords) or
		% (xmax-xmin) (absolute coords).
		%
		% Now, search for a set of real numbers 
		% Rx, Ry, Rz, s
		% such that
		% W = (Rx r_x) (s e_xx)  + (Ry r_y) (s e_yx)  + (Rz r_z) (s e_zx)
		% H = (Rx r_x) (s e_xy)  + (Ry r_y) (s e_yy)  + (Rz r_z) (s e_zy)
		%
		% clearly, the solution is not unique.
		% ONE choice is to employ the fact that e_zx = 0  (or, for 2d
		% plots, e_zx=0, e_zy=0 and e_yx=0):
		%
		% in that case, we can compute s such that the equation for W
		% is satisfied and compensate only the limit r_z, i.e. to
		% choose
		% s := W / w,    (scale to satisfy width constraint)
		% Rx := Ry := 1  (keep limits in X and Y)
		% Rz = ( H - s (w - r_z e_zy) ) / (s r_z e_zy)    (adjust z limit to satisfy height constraint)
		%
		% This approach works well if W < H . If W > H, it will look
		% bad: Rz will be less than 1, causing the limit to become
		% smaller. This, in turn, will clip away parts of the image.
		%
		% 
		%
		% Another solution is to make it the other way: to keep the
		% limit r_z, but to reduce the size and enlarge the other
		% limits to satisfy the size constraints. This solution is
		% considerably more involved; it requires to solve a nonlinear
		% set of equations.
		%
		% Formally, this second solution uses
		% Rz := 1       (no limit componensation scale for z -- keep z limit)
		% R:= Rx := Ry  (same limit componensation scale for both X and Y)
		% R and s still need to be determined from the two equations for W
		% and H.
		%
		% Substituting the given choices into the equations for W and H, we find
		%
		% R = W / (s w)
		%
		% s = H * (R * (h-r_z e_zy) + r_z e_zy)^-1
		%
		% Here, we employed the definition of 'h', see above. The
		% equations are non-linear.
		%
		\if0\pgfplots@scaleuniformly@choice
			% scale uniformly strategy=auto
			\ifdim\W pt>\H pt %
				% -> change horizontal limits:
				\def\pgfplots@scaleuniformly@choice{3}%
			\else
				\ifdim\W pt=\H pt %
					% -> change horizontal limits:
					\def\pgfplots@scaleuniformly@choice{3}%
				\else
					% -> change vertical limits:
					\def\pgfplots@scaleuniformly@choice{2}%
				\fi
			\fi
		\fi
		%
		% ATTENTIN: we assume that the datascaletrafo set params
		% method has been called with THE SAME SCALE IN EACH
		% DIRECTION.
		%
		\ifcase\pgfplots@scaleuniformly@choice\relax
		 	% scale uniformly strategy=auto does not happen here.
		\or
			% scale uniformly strategy=none
			\pgfplots@scaleuniformly@onlyunits
		\or
			% scale uniformly strategy=change vertical limits
			\pgfplots@scaleuniformly@onlyunits
			% ... and change (only) vertical limits:
			\ifdim\pgf@zy=0pt
				\ifdim\pgf@yx=0pt
					\pgfplots@prepare@vertical@rescaling@for@scale@uniformly@in@dir{y}\returninto\pgfplots@target@limitrescale@y
				\else
					\pgfplots@error@scale@uniformly@failed
				\fi
			\else
				\ifdim\pgf@zx=0pt
					\pgfplots@prepare@vertical@rescaling@for@scale@uniformly@in@dir{z}\returninto\pgfplots@target@limitrescale@z
				\else
					\pgfplots@error@scale@uniformly@failed
				\fi
			\fi
		\or
			% scale uniformly strategy=change horizontal limits
			\ifdim\pgf@zy=0pt
				\ifdim\pgf@yx=0pt
					\ifdim\pgf@xy=0pt
						% special 2d routine with explicit solution
						\pgfplots@scaleuniformly@change@horizontal@limits@twodim
							{\scalex}
							{\pgfplots@target@limitrescale@x}
							{\pgfplots@target@limitrescale@y}
							{\pgfplots@target@limitrescale@z}%
					\else
						\pgfplots@error@scale@uniformly@failed
					\fi
				\else
					\pgfplots@error@scale@uniformly@failed
				\fi
			\else
				\ifdim\pgf@zx=0pt
					\pgfplots@scaleuniformly@change@horizontal@limits
						{\scalex}
						{\pgfplots@target@limitrescale@x}
						{\pgfplots@target@limitrescale@y}
						{\pgfplots@target@limitrescale@z}%
				\else
					\pgfplots@error@scale@uniformly@failed
				\fi
			\fi
			\let\scaley=\scalex
		\fi
	\fi
	\xdef\pgfplots@glob@TMPa{%
		\noexpand\def\noexpand#3{\scalex}%
		\noexpand\def\noexpand#4{\scaley}%
		\noexpand\def\noexpand#5{\pgfplots@target@limitrescale@x}%
		\noexpand\def\noexpand#6{\pgfplots@target@limitrescale@y}%
		\noexpand\def\noexpand#7{\pgfplots@target@limitrescale@z}%
	}%
	\endgroup
	\pgfplots@glob@TMPa
}%

\def\pgfplots@scaleuniformly@onlyunits{%
	\def\pgfplots@target@limitrescale@x{1}%
	\def\pgfplots@target@limitrescale@y{1}%
	\def\pgfplots@target@limitrescale@z{1}%
	\pgfmathdivide@{\W}{\w}%
	\let\scalex=\pgfmathresult
	\let\scaley=\scalex % we *need* the same unit scale.
}%

% Computes 'scale uniformly strategy=change horizontal limits'.
% This is a complicated solution, see the documentation in the
% implementation for 
% 'scale mode=scale uniformly'
%
% #1 [output] a macro which will contain the (uniform) scale for the
% unit vectors
% #2 [output] a macro which will contain a x axis limit compensation scale
% #3 [output] a macro which will contain a x axis limit compensation scale
% #4 [output] a macro which will contain a x axis limit compensation scale
\def\pgfplots@scaleuniformly@change@horizontal@limits#1#2#3#4{%
	\begingroup
	%
	\pgfplots@BB@for@plotbox@getunitheight{\pgf@xc}{z}%
	%
	% compute the rest in floating point - intermediate results may
	% become too huge for TeX.
	\pgfplotscoordmath{default}{parsenumber}{\expandafter\pgf@sys@tonumber\csname pgf@xc\endcsname}%
	\let\M=\pgfmathresult
	%
	\pgfplotscoordmath{default}{parsenumber}{\w}%
	\let\w=\pgfmathresult
	\pgfplotscoordmath{default}{parsenumber}{\W}%
	\let\W=\pgfmathresult
	\pgfplotscoordmath{default}{parsenumber}{\h}%
	\let\h=\pgfmathresult
	\pgfplotscoordmath{default}{parsenumber}{\H}%
	\let\H=\pgfmathresult
	\pgfplotscoordmath{default}{op}{divide}{{\W}{\w}}%
	\let\Wwinv=\pgfmathresult
	\pgfplotscoordmath{default}{op}{subtract}{{\h}{\M}}%
	\let\hminusM=\pgfmathresult
	%
	\pgfplotscoordmath{default}{one}%
	\let\S=\pgfmathresult%
	\let\R=\pgfmathresult%
	\let\Rx=\pgfmathresult
	\def\Rz{1}%
	%
	\def\pgfplots@hold@S@get@R{%
		\pgfplotscoordmath{default}{op}{divide}{{\Wwinv}{\S}}%
		\let\R=\pgfmathresult
\message{updated R = \R\space ( S = \S ) ^^J}%
	}%
	\def\pgfplots@hold@R@get@S{%
		\pgfplotscoordmath{default}{op}{multiply}{{\R}{\hminusM}}%
		\pgfplotscoordmath{default}{op}{add}{{\pgfmathresult}{\M}}%
		\pgfplotscoordmath{default}{op}{divide}{{\H}{\pgfmathresult}}%
		\let\S=\pgfmathresult
\message{updated S = \S\space ( R = \R ) ^^J}%
	}%
	%
	% This is the (most stupid) nonlinear method which is at hand:
	% fix point iteration.
	% choose R arbitrarily (R=1 seems adequate), solve for s. 
	% Then, fix s and solve for R. Then, fix R and
	% solve for s until convergence.
	\c@pgf@countc=0
	\pgfplotsloop{%
		\ifnum\c@pgf@countc<\pgfkeysvalueof{/pgfplots/scale uniformly strategy iter} %
			\pgfplotsloopcontinuetrue
		\else
			\pgfplotsloopcontinuefalse
		\fi
	}{%
		\pgfplots@hold@R@get@S \pgfplots@hold@S@get@R
		\advance\c@pgf@countc by1 %
	}%
	%
	\pgfplotscoordmath{default}{tofixed}{\R}\let\R=\pgfmathresult
	\pgfplotscoordmath{default}{tofixed}{\S}\let\S=\pgfmathresult
	\xdef\pgfplots@glob@TMPa{%
		\noexpand\def\noexpand#1{\S}%
		\noexpand\def\noexpand#2{\R}%
		\noexpand\def\noexpand#3{\R}%
		\noexpand\def\noexpand#4{\Rz}%
	}%
	\endgroup
	%
	\pgfplots@glob@TMPa
}%

% Computes 'scale uniformly strategy=change horizontal limits'.
% 
% This is a simplified closed solution assuming that e_xy=0 and e_yx = 0
%
% #1 [output] a macro which will contain the (uniform) scale for the
% unit vectors
% #2 [output] a macro which will contain a x axis limit compensation scale
% #3 [output] a macro which will contain a x axis limit compensation scale
% #4 [output] a macro which will contain a x axis limit compensation scale
\def\pgfplots@scaleuniformly@change@horizontal@limits@twodim#1#2#3#4{%
	\begingroup
	% Assuming that we have a standard 2d axis, i.e. 
	% e_zx = e_zy = 0,  e_xy = 0, and e_yx =0,
	% we can immediately compute a solution.
	%
	% In this case, we have the actual width
	% w = r_x e_xx  + r_y e_yx + rz e_zx
	%   = r_x e_xx
	% and actual height
	% h = r_x e_xy  + r_y e_yy + rz e_zy
	%   = r_y e_yy
	% and, consequently, desired width
	% W = (Rx r_x) (s e_xx)  + (Ry r_y) (s e_yx)  + (Rz r_z) (s e_zx)
	%   = (Rx r_x) (s e_xx)
	% and desired height
	% H = (Rx r_x) (s e_xy)  + (Ry r_y) (s e_yy)  + (Rz r_z) (s e_zy)
	%   = (Ry r_y) (s e_yy).
	% since this strategy changes horizontal limits (only), we have
	% Ry := 1.
	% We find
	% s : = H/h 
	% and
	% Rx : = W/w /s .
	%
	\pgfplotscoordmath{default}{parsenumber}{\w}%
	\let\w=\pgfmathresult
	\pgfplotscoordmath{default}{parsenumber}{\W}%
	\let\W=\pgfmathresult
	\pgfplotscoordmath{default}{parsenumber}{\h}%
	\let\h=\pgfmathresult
	\pgfplotscoordmath{default}{parsenumber}{\H}%
	\let\H=\pgfmathresult
	\pgfplotscoordmath{default}{op}{divide}{{\H}{\h}}%
	\let\S=\pgfmathresult
	\pgfplotscoordmath{default}{op}{divide}{{\W}{\w}}%
	\pgfplotscoordmath{default}{op}{divide}{{\pgfmathresult}{\S}}%
	\let\Rx=\pgfmathresult
	\def\Ry{1}%
	\def\Rz{1}%
	%
	\pgfplotscoordmath{default}{tofixed}{\Rx}\let\Rx=\pgfmathresult
	\pgfplotscoordmath{default}{tofixed}{\S}\let\S=\pgfmathresult
	\xdef\pgfplots@glob@TMPa{%
		\noexpand\def\noexpand#1{\S}%
		\noexpand\def\noexpand#2{\Rx}%
		\noexpand\def\noexpand#3{\Ry}%
		\noexpand\def\noexpand#4{\Rz}%
	}%
	\endgroup
	%
	\pgfplots@glob@TMPa
}%

\def\pgfplots@BB@update@cumulative@limit@compensations{%
	\message{  -> additional limit componensation scales x=\pgfplots@target@limitrescale@x@, y=\pgfplots@target@limitrescale@y@, z=\pgfplots@target@limitrescale@z@^^J}%
	% add limit compensation to what we have from earlier
	% operations:
	\pgfplotscoordmath{pgfbasic}{op}{multiply}{{\pgfplots@target@limitrescale@x@}{\pgfplots@target@limitrescale@x}}%
	\let\pgfplots@target@limitrescale@x=\pgfmathresult
	\pgfplotscoordmath{pgfbasic}{op}{multiply}{{\pgfplots@target@limitrescale@y@}{\pgfplots@target@limitrescale@y}}%
	\let\pgfplots@target@limitrescale@y=\pgfmathresult
	\pgfplotscoordmath{pgfbasic}{op}{multiply}{{\pgfplots@target@limitrescale@z@}{\pgfplots@target@limitrescale@z}}%
	\let\pgfplots@target@limitrescale@z=\pgfmathresult
}%

\def\pgfplots@error@scale@uniformly@failed{%
	\pgfplots@error{Sorry, 'scale uniformly' failed because its actual implementation works only if y_x = 0 (for 2d axes) or if z_x = 0 (for 3d axes). The result will not fill the prescribed dimensions. For 2d axes, you may also want to consider the 'unit vector ratio' key}%
}%

% This is part of the implementation of 'scale mode=scale uniformly'.
%
% Its purpose it to set up the initial scaling such that 
% 1. each unit vector gets the same scale
% 2. the axis limits are resized (enlarged) to keep the plot box ratio
% (as far as possible)
%
% It repairs the outcome of
% \pgfplots@BB@for@plotbox@get@unit@scales@for@limits .
%
% The assumption is that on input #1, #2, and #3 are the factors which
% would be used by stretch-to-fill in order to squeze the axis limits
% into the plot box defined by e_x, e_y, and e_z (the unit vectors).
%
% On output, #1, #2, and #3 will be modified such that *each has the
% same value*. The value will be chosen with care. More precisely, it
% is the *minimum* of {#1,#2,#3}.
%
% Clearly, 'scale mode=scale uniformly' has less freedom than
% strech-to-fill. In order to keep the plot box ratio intact (as far
% as possible), the axis limits will be rescaled to componsate for the
% ignored scaling factors. More precisely, if direction i is not the
% extremal value (as discussed in the last paragraph), the axis limits
% will be rescaled by #i/extremum .
%
%
%
% #1: on input, it contains the x unit scale which would be taken without the
% compensation. On output, it contains the x unit scale which *will* be
% used.
% #2: same as #1, but for y
% #3: same as #1, but for z
% #4: [output] a scale for use as argument of \pgfplots@apply@unit@vector@rescale@keep@size{x}{<arg>}
% #5: [output] a scale for use as argument of \pgfplots@apply@unit@vector@rescale@keep@size{y}{<arg>}
% #6: [output] a scale for use as argument of \pgfplots@apply@unit@vector@rescale@keep@size{z}{<arg>}
%
% The output arguments need to be applied before they take effect.
\def\pgfplots@BB@for@plotbox@get@unit@scales@compensated@axis@limits#1#2#3#4#5#6{%
	\begingroup
	% ATTENTION : this code ASSUMES that the datascaling trafo is
	% initialized with THE SAME SCALE IN EACH DIRECTION.
	% The data scaling also leads to (potentially non-uniform) scaling per component.
	%
	% Note that we could handle the datascaling here -- but we would
	% leave the supported number range easily. That's why that part of
	% the 'scale mode=scale uniformly' implementation has been moved
	% to \pgfplots@set@optimal@datatrafos@allaxes
	%
	% This here handles the limits only.
	\edef\pgfplots@scale@unitx{#1}%
	\edef\pgfplots@scale@unity{#2}%
	\edef\pgfplots@scale@unitz{#3}%
	%
	% compute extreme + arg extreme of these scales:
	\let\pgfplots@extreme@scale=\pgfplots@scale@unitx
	\def\pgfplots@extreme@scale@arg{x}%
	\ifdim\pgfplots@extreme@scale pt>\pgfplots@scale@unity pt
		\let\pgfplots@extreme@scale=\pgfplots@scale@unity
		\def\pgfplots@extreme@scale@arg{y}%
	\fi
	\ifpgfplots@threedim
		\ifdim\pgfplots@extreme@scale pt>\pgfplots@scale@unitz pt
			\let\pgfplots@extreme@scale=\pgfplots@scale@unitz
			\def\pgfplots@extreme@scale@arg{z}%
		\fi
	\fi
	%
	% Now, adjust axis limits to compensate for the effect: we still
	% want to have a plot box which is as close as possible to the
	% target plot box.
	\def\pgfplots@loc@TMPa##1##2{%
		\if\pgfplots@extreme@scale@arg ##1%
			% ok, nothing to do for this direction.
			\pgfplotscoordmath{pgfbasic}{one}%
			\let##2=\pgfmathresult
		\else
			\pgfplotscoordmath{pgfbasic}{op}{divide}{{\csname pgfplots@scale@unit##1\endcsname}{\pgfplots@extreme@scale}}%
			% do not call apply@unit@rescale immediately because the
			% unit vectors may not be in their final state. Postpone until
			% they are final.
			\edef##2{\pgfmathresult}%
		\fi
	}%
	\pgfplots@loc@TMPa{x}{#4}%
	\pgfplots@loc@TMPa{y}{#5}%
	\ifpgfplots@threedim
		\pgfplots@loc@TMPa{z}{#6}%
	\else
		\def#6{1}%
	\fi
	%
	\toks0=\expandafter{#4}%
	\toks1=\expandafter{#5}%
	\toks2=\expandafter{#6}%
	\xdef\pgfplots@glob@TMPa{%
		% same scale in each dir:
		\def\noexpand#1{\pgfplots@extreme@scale}%
		\def\noexpand#2{\pgfplots@extreme@scale}%
		\def\noexpand#3{\pgfplots@extreme@scale}%
		\def\noexpand#4{\the\toks0}%
		\def\noexpand#5{\the\toks1}%
		\def\noexpand#6{\the\toks2}%
	}%
	\endgroup
	\pgfplots@glob@TMPa
}

% #1 : a dimen register
% #2 : x, y, or z
\def\pgfplots@BB@for@plotbox@getunitheight#1#2{%
	#1=\csname pgf@#2y\endcsname
}%

% Modifies the AXIS LIMITS to ensure that a suitable width/height is
% achieved.
%
% This does NOT introduce a further scale to the unit vectors.
%
% #1: a direction (x,y, or z)
% #2: a macro name. It will be assigned globally. It will contain
% EXECUTABLE instructions which will modify the axis limits to fit the
% scaling.
%
% PRECONDITION: 
% - \pgfplots@glob@TMPa contains the already computed
% scaling factor for 'scale uniformly'
% - \pgf@xb is the actual height and \pgf@yb is the desired height
%   (set as in the scaling routine)
%
% POSTCONDITION:
% #2 will contain the argument <arg> for \pgfplots@apply@unit@vector@rescale@keep@size{#1}{<arg>}
\def\pgfplots@prepare@vertical@rescaling@for@scale@uniformly@in@dir#1\returninto#2{%
	% The strategy is as follows:
	% 1. I want to fit the axis into width #1 (\pgf@ya) and
	% height #1 (\pgf@yb).
	% 2. I want to MAINTAIN the unit vector ratio. 
	% 3. I want to MAINTAIN the unit vector directions.
	%
	% I already know the scaling factor to fit the width (it
	% is stored in \pgfplots@glob@TMPa = \pgfplots@glob@TMPb).
	% Let's call it "s".
	%
	% Consequently, a uniform scaling by "s" leads to the image
	% height
	% h = s* (r_x * e_xy + r_y * e_yy + r_z * e_zy)
	% where r_i = (imax - imin). This here is essentially the
	% same as the bounding box computation above (at least for
	% standart orthographic 3D axes).
	%
	% What I want now is to enlarge the limits such that I
	% have BOTH, width #1 AND height #2, without obscuring the
	% unit vector ratio. Recall that width #1 is already
	% given.
	%
	% In my opinion the ONLY way to achieve this goal is to
	% modify axis limits for an axis whose unit vector is
	% parallel to the canvas y axis, i.e. e_i = (0,*). To see
	% this, remember that a rescaling of the other limits will
	% also modify the width.
	%
	% That means I have to introduce a SECOND scale s_z which
	% applies only to the Z unit vector (since e_z = (0,*) ).
	% If H = #2 is the desired height, I find the target
	% equation for s_z,
	%
	% H = s* r_x e_xy + s * r_y e_yy + s_z * s * r_z * e_zy
	% =>
	% s_z = ( H- s*r_x e_xy - s*r_y e_yy) / ( s * r_z * e_zy).
	%
	% Remember that 
	% s = \scalex
	% H = \H
	% h = r_x * e_xy + r_y * e_yy + r_z * e_zy  = \h
	% =>
	% s_z = ( H- s*( h - r_z * e_zy) ) / ( s * r_z * e_zy).
	% 
	\begingroup
	\pgfplots@BB@for@plotbox@getunitheight{\pgf@xc}{#1}%
	%
	% compute the rest in floating point - intermediate results may
	% become too huge for TeX.
	\pgfplotscoordmath{default}{parsenumber}{\expandafter\pgf@sys@tonumber\csname pgf@xc\endcsname}%
	\let\pgfplots@diff=\pgfmathresult
	%
	\pgfplotscoordmath{default}{parsenumber}{\scalex}%
	\let\pgfplots@s=\pgfmathresult
	%
	% this is a precondition of this method:
	\pgfplotscoordmath{default}{parsenumber}{\h}%
	\let\h=\pgfmathresult
	\pgfplotscoordmath{default}{parsenumber}{\H}%
	\let\H=\pgfmathresult
	%
	% compute counter := H - s * (h - (max-min))
	\pgfplotscoordmath{default}{op}{subtract}{{\h}{\pgfplots@diff}}%
	\pgfplotscoordmath{default}{op}{multiply}{{\pgfplots@s}{\pgfmathresult}}%
	\pgfplotscoordmath{default}{op}{subtract}{{\H}{\pgfmathresult}}%
	\let\pgfplots@counter=\pgfmathresult
	%
	% computer denom := s * (max-min)
	\pgfplotscoordmath{default}{op}{multiply}{{\pgfplots@s}{\pgfplots@diff}}%
	\let\pgfplots@denom=\pgfmathresult
	%
	\pgfplotscoordmath{default}{op}{divide}{{\pgfplots@counter}{\pgfplots@denom}}%
	\pgfplotscoordmath{default}{tofixed}{\pgfmathresult}%
	%
	% Now, s_z = \pgfmathresult .
	%
	% Now, adjust the z limits.
	% Note that \pgfplots@apply@unit@vector@rescale@keep@size
	% has a slightly different context; it assumes that the
	% unit vector has been rescaled, not the axis limits.
	% Consequently, the inverse of the scaling factor enters.
	% Since \pgfplots@apply@unit@vector@rescale@keep@size
	% expects the inverse of the scale, we can provide
	% \pgfmathresult:
	\pgfmath@smuggleone\pgfmathresult
	\endgroup
	\let#2=\pgfmathresult
}

\def\pgfplots@computeunitvectorlengths{%
	\pgfplotsutil@edef@invoke\pgfmathveclen@{%
		{\pgf@sys@tonumber\pgf@xx}%
		{\pgf@sys@tonumber\pgf@xy}%
	}%
	\let\pgfplots@x@veclength=\pgfmathresult
	\pgfplotsmath@ifzero{\pgfplots@x@veclength}{%
		\def\pgfmathresult{infty}%
		% this case will be caught in \pgfplots@initsizes
	}{%
		\expandafter\pgfmath@basic@reciprocal@\expandafter{\pgfmathresult}%
	}%
	\let\pgfplots@x@inverseveclength=\pgfmathresult
	%
	\pgfplotsutil@edef@invoke\pgfmathveclen@{%
		{\pgf@sys@tonumber\pgf@yx}%
		{\pgf@sys@tonumber\pgf@yy}%
	}%
	\let\pgfplots@y@veclength=\pgfmathresult
	\pgfplotsmath@ifzero{\pgfplots@y@veclength}{%
		\def\pgfmathresult{infty}%
		% this case will be caught in \pgfplots@initsizes
	}{%
		\expandafter\pgfmath@basic@reciprocal@\expandafter{\pgfmathresult}%
	}%
	\let\pgfplots@y@inverseveclength=\pgfmathresult
	%
	\ifpgfplots@threedim
		\pgfplotsutil@edef@invoke\pgfmathveclen@{%
			{\pgf@sys@tonumber\pgf@zx}%
			{\pgf@sys@tonumber\pgf@zy}%
		}%
		\let\pgfplots@z@veclength=\pgfmathresult
		\pgfplotsmath@ifzero{\pgfplots@z@veclength}{%
			\def\pgfmathresult{infty}%
			% this case will be caught in \pgfplots@initsizes
		}{%
			\expandafter\pgfmath@basic@reciprocal@\expandafter{\pgfmathresult}%
		}%
		\let\pgfplots@z@inverseveclength=\pgfmathresult
	\else
		\def\pgfplots@z@veclength{0}%
		\def\pgfplots@z@inverseveclength{infty}%
	\fi
}%

% Defines \pgfplots@view@dir@threedim according to the actual
% configuration of x,y,z (2d) unit vectors, assuming the associated
% unit vectors form a right-handed-system.
%
% The algorithm works for standard three dimensional axes. It works as
% follows:
%
% First, observe that we have a normal direction N if all its
% multiples are on the mapped onto the same point in 2D canvas
% coordinates. In other words: all 3D coordinates which are mapped
% onto an arbitrary point in 2D canvas coordinates (take, for example,
% the origin (0,0)  ) are on a line in direction of N.
%
% We use this observation to compute the normal axis, i.e. we search
% for all points which are mapped onto the 2D canvas coordinate (0,0):
%  N_x e_xx + N_y e_yx + N_z e_zx = 0
%  N_x e_xy + N_y e_yy + N_z e_zy = 0.
% All solutions make up a linear space of dimension 1 (up to special
% cases). In the general case, we can chose an arbitrary  N_z != 0
% and reduce the linear system to
%  N_x e_xx + N_y e_yx = - N_z e_zx
%  N_x e_xy + N_y e_yy = - N_z e_zy.
% Choosing *any* N_z != 0, say, N_z=-1 (which corresponds to view
% from above) will lead to a vector parallel to the normal direction.
% But it might have the wrong sign.
%
% To find the correct sign for N, I have made several case
% distinctions to identify the cases when we have to multiply with -1.
% The key idea is to assume a right-handed-system of unit vectors;
% this is the condition which allows to determine the sign.
%
% Furthermore, I assume that e_z points to the top, i.e. that e_zy >0.
% Then, there are (mainly) four conditions on the signs of e_x and e_y
% which indicate that we are viewing from below and should switch the
% sign of N (keep in mind that our initial choice was N_z =-1, see above).
%
% The conditions can be identified by drawing a 3D box and
% identifying the corner which represents the lower left 3D limits.
%
% You can visualize these cases using
%--------------------------------------------------
% \pgfplotsset{
% 		separate axis lines,
% 		every outer x axis line/.append style= {-stealth},
% 		every outer y axis line/.append style= {-stealth},
% 		every outer z axis line/.append style= {-stealth},
% 		samples=2,shader=interp,title={view=\h,\v},
% 		domain=0:1,
% 		enlargelimits=false,
% 		view=\h\v,xlabel=x,ylabel=y,
% 		extra description/.code={%
% 			\node[draw,fill=white] at (axis cs:0,0,0) {};
% 		},
% }
% 
% \def\v{30}
% \foreach \h in {30,120,210,300} {
% \message{VIEW={\h}{\v}^^J}
% \begin{tikzpicture}
% 	\begin{axis}
% 	\addplot3[surf] {x};
% 	\end{axis}
% \end{tikzpicture}
% 
% }
% 
% \def\v{-30}
% \foreach \h in {30,120,210,300} {
% \message{VIEW={\h}{\v}^^J}
% \begin{tikzpicture}
% 	\begin{axis}
% 	\addplot3[surf] {x};
% 	\end{axis}
% \end{tikzpicture}
% 
% }
%-------------------------------------------------- 
%  The precise formulas can be found below in the source code.
%
% You can override this function by the /pgfplots/view dir key.
\def\pgfplotsgetnormalforcurrentview{%
	\pgfkeysgetvalue{/pgfplots/view dir}\pgfplots@loc@TMPc
	\ifx\pgfplots@loc@TMPc\pgfutil@empty
	\begingroup	
		% temporarily undo the effects of reversed axes -- we *really*
		% need a right-handed-coordinate system here:
		\if r\pgfkeysvalueof{/pgfplots/x dir/value}%
			\pgf@xx=-\pgf@xx
			\pgf@xy=-\pgf@xy
		\fi
		\if r\pgfkeysvalueof{/pgfplots/y dir/value}%
			\pgf@yx=-\pgf@yx
			\pgf@yy=-\pgf@yy
		\fi
		\if r\pgfkeysvalueof{/pgfplots/z dir/value}%
			\pgf@zx=-\pgf@zx
			\pgf@zy=-\pgf@zy
		\fi
		\def\pgfplots@view@dir@threedim@z{-1}% hold it at some arbitrary value
		\pgf@xa=-\pgfplots@view@dir@threedim@z\pgf@zx
		\pgf@ya=-\pgfplots@view@dir@threedim@z\pgf@zy
		\edef\pgfplots@loc@TMPa{%
			{%
				{\pgf@sys@tonumber\pgf@xx}{\pgf@sys@tonumber\pgf@yx}%
				{\pgf@sys@tonumber\pgf@xy}{\pgf@sys@tonumber\pgf@yy}%
			}%
			{%
				{\pgf@sys@tonumber\pgf@xa}{\pgf@sys@tonumber\pgf@ya}%
			}%
		}%
		\expandafter\pgfutilsolvetwotwoleq\pgfplots@loc@TMPa
		\def\pgfplots@loc@TMPb##1##2{%
			\def\pgfplots@view@dir@threedim@x{##1}%
			\def\pgfplots@view@dir@threedim@y{##2}%
		}%
		\expandafter\pgfplots@loc@TMPb\pgfmathresult
		%
		% Identify if we need to switch the sign.
		% To verify that these cases are useful, I suggest visualizing
		% that stuff using the TeX code from above...
		%
		% I guess it is correct up to collapsing views (as you see, I
		% did not properly identify the cases with "=0" )
		\def\pgfplots@scale{1}%
		\ifdim\pgf@xx>0pt
			\ifdim\pgf@yx>0pt
				%
				\ifdim\pgf@xy>0pt
					\ifdim\pgf@yy<0pt
						\def\pgfplots@scale{-1}%
					\fi
				\fi
				%
			\fi
		\else
			\ifdim\pgf@xx<0pt
				\ifdim\pgf@yx<0pt
					%
					\ifdim\pgf@xy<0pt
						\ifdim\pgf@yy>0pt
							\def\pgfplots@scale{-1}%
						\fi
					\fi
					%
				\fi
			\fi
		\fi
		\ifdim\pgf@xy>0pt
			\ifdim\pgf@yy>0pt
				%
				\ifdim\pgf@xx<0pt
					\ifdim\pgf@yx>0pt
						\def\pgfplots@scale{-1}%
					\fi
				\fi
				%
			\fi
		\else
			\ifdim\pgf@xy<0pt
				\ifdim\pgf@yy<0pt
					%
					\ifdim\pgf@xx>0pt
						\ifdim\pgf@yx<0pt
							\def\pgfplots@scale{-1}%
						\fi
					\fi
					%
				\fi
			\fi
		\fi
		\pgfmathmultiply@{\pgfplots@scale}{\pgfplots@view@dir@threedim@x}%
		\let\pgfplots@view@dir@threedim@x\pgfmathresult
		\pgfmathmultiply@{\pgfplots@scale}{\pgfplots@view@dir@threedim@y}%
		\let\pgfplots@view@dir@threedim@y\pgfmathresult
		\pgfmathmultiply@{\pgfplots@scale}{\pgfplots@view@dir@threedim@z}%
		\let\pgfplots@view@dir@threedim@z\pgfmathresult
		\pgfplotsmathvectorfromstring{\pgfplots@view@dir@threedim@x,\pgfplots@view@dir@threedim@y,\pgfplots@view@dir@threedim@z}{default}%
		\pgfmath@smuggleone\pgfplotsretval
	\endgroup
		\let\pgfplots@view@dir@threedim=\pgfplotsretval
	\else
		\def\pgfplots@loc@TMPb##1##2##3{%
			\pgfplotsmathvectorfromstring{##1,##2,##3}{default}%
			\let\pgfplots@view@dir@threedim=\pgfplotsretval
		}%
		\expandafter\pgfplots@loc@TMPb\pgfplots@loc@TMPc
	\fi
%\message{\string\pgfplotsgetnormalforcurrentview:  got (\pgfplots@view@dir@threedim)^^J}%
}%

% PRECONDITION:
% 	none
% POSTCONDITION:
% 	\pgfplots@default@aspect@ratio is set.
\def\pgfplots@compute@default@aspect@ratio{%
	\expandafter\pgfmath@x\axisdefaultwidth
	\expandafter\pgfmath@y\axisdefaultheight
	\pgfmathlog@invoke@expanded\pgfmathdivide@{%
		{\pgf@sys@tonumber{\pgfmath@x}}%
		{\pgf@sys@tonumber{\pgfmath@y}}%
	}%
	\let\pgfplots@default@aspect@ratio=\pgfmathresult
}

\def\pgfplots@set@default@size@options{%
	% The axes 'x' and 'y' vectors will be scaled such that the total
	% size is (\axisdefaultwidth, \axisdefaultheight).
	%
	% If the user specifies ONE of width OR height, 
	% the plot will be resized; keeping the aspect ratio.
	%
	\let\pgfplots@default@aspect@ratio=\pgfutil@empty
	\pgfkeysgetvalue{/pgfplots/x}{\pgfplots@x}%
	\pgfkeysgetvalue{/pgfplots/y}{\pgfplots@y}%
	\pgfkeysgetvalue{/pgfplots/z}{\pgfplots@z}%
	%\pgfkeysgetvalue{/pgfplots/viewdir}{\pgfplots@viewdir}%
	\pgfkeysgetvalue{/pgfplots/width}{\pgfplots@width}%
	\pgfkeysgetvalue{/pgfplots/height}{\pgfplots@height}%
	\ifx\pgfplots@width\pgfutil@empty
		\def\pgfplots@user@provided@width{0}%
	\else
		\def\pgfplots@user@provided@width{1}%
		\pgfmathparse{\pgfplots@width}%
		\edef\pgfplots@width{\pgfmathresult pt}%
	\fi
	\ifx\pgfplots@height\pgfutil@empty
		\def\pgfplots@user@provided@height{0}%
	\else
		\def\pgfplots@user@provided@height{1}%
		\pgfmathparse{\pgfplots@height}%
		\edef\pgfplots@height{\pgfmathresult pt}%
	\fi
	%
	% CASES:
	% W := 'width' option non-empty
	% H := 'height' option non-empty
	%
	% 		W H
	% 		0 0 -> \axisdefaultwidth 
	% 		0 1 -> determine width out of H and the default aspect ratio
	% 		1 X -> ok, use the user parameter.
	% -> KEEP ASPECT RATIO if just one W, or H is given!
	\ifx\pgfplots@width\pgfutil@empty
		\ifx\pgfplots@height\pgfutil@empty
			% The case W=0 H=0:
			\let\pgfplots@width=\axisdefaultwidth
			\let\pgfplots@height=\axisdefaultheight
		\else
			% The case W=0 H=1:
			\pgfplots@compute@default@aspect@ratio
			\expandafter\pgfmath@y\pgfplots@height
			\pgfmathlog@invoke@expanded\pgfmathmultiply@{%
				{\pgf@sys@tonumber{\pgfmath@y}}%
				{\pgfplots@default@aspect@ratio}%
			}%
			\edef\pgfplots@width{\pgfmathresult pt}%
		\fi
	\else
		\ifx\pgfplots@height\pgfutil@empty
			% The case W=1 H=0:
			\pgfplots@compute@default@aspect@ratio
			\expandafter\pgfmath@x\pgfplots@width
			\pgfmathlog@invoke@expanded\pgfmathdivide@{%
				{\pgf@sys@tonumber{\pgfmath@x}}%
				{\pgfplots@default@aspect@ratio}%
			}%
			\edef\pgfplots@height{\pgfmathresult pt}%
		\else
			% The case W=1 H=1:
		\fi
	\fi
	\pgfkeyslet{/pgfplots/width}{\pgfplots@width}%
	\pgfkeyslet{/pgfplots/height}{\pgfplots@height}%
	%
	\ifpgfplots@threedim
		\pgfplots@set@default@size@options@threedim
	\fi
	%
	\pgfplots@set@scale@mode
}

% This method must be called BEFORE THE DATASCALING is initialized.
\def\pgfplots@set@scale@mode{%
	\pgfkeysgetvalue{/pgfplots/unit vector ratio}\pgfplots@loc@TMPb
	\ifx\pgfplots@loc@TMPb\pgfutil@empty
	\else
		\ifcase\pgfplots@scale@mode@choice
			% 'scale mode'=auto
			\def\pgfplots@scale@mode@choice{3}% set to 'scale uniformly'
		\or
			% scale mode=none: keep it this way.
			\immediate\write-1{PGFPlots: scale mode=none  and unit vector ratio is incompatible. Ignoring unit vector ratio.^^J}%
		\or
			% scale mode=stretch to fill
			\immediate\write-1{PGFPlots: scale mode=stretch to fill  and unit vector ratio might produce unexpected results. Consider using scale mode=auto^^J}%
		\fi
	\fi
	%
}

\def\pgfplots@set@default@size@options@threedim{%
	\pgfplots@loc@tmpfalse
	\ifx\pgfplots@x\pgfutil@empty
	\else
		\pgfplots@loc@tmptrue
	\fi
	\ifx\pgfplots@y\pgfutil@empty
	\else
		\pgfplots@loc@tmptrue
	\fi
	\ifx\pgfplots@z\pgfutil@empty
	\else
		\pgfplots@loc@tmptrue
	\fi
	\ifpgfplots@loc@tmp
		% oh - we have at least one of the [xyz] unit vectors!
		% make sure all of them are there
		\ifx\pgfplots@x\pgfutil@empty
			\pgfplots@set@default@size@options@threedim@{x}{(1pt,0pt)}%
		\fi
		\ifx\pgfplots@y\pgfutil@empty
			\pgfplots@set@default@size@options@threedim@{y}{(0pt,1pt)}%
		\fi
		\ifx\pgfplots@z\pgfutil@empty
			\pgfplots@set@default@size@options@threedim@{z}{(0pt,1pt)}%
		\fi
		\pgfkeyslet{/pgfplots/view/az}\pgfutil@empty
		\pgfkeyslet{/pgfplots/view/el}\pgfutil@empty
	\fi
}
\def\pgfplots@set@default@size@options@threedim@#1#2{%
	\pgfplots@error{Sorry, a 3D axis needs either NONE or ALL of "x,y,z". I found partial information, but (at least) '#1' is lacking... please add '#1'}%
	\expandafter\def\csname pgfplots@#1\endcsname{#2}%
}

% A helper method for \pgfplots@initsizes which
% - applies the data scaling trafo to user arguments
% - sets calls pgfset#1vec
%
% #1: the vector to set (either 'x' or 'y')
% #2: the index of the vector to set (either 0 or 1)
% #3: the already precomputed temporary scale (see pgfplots@initsizes)
% #4: an output argument. It is a macro name which will be defined to
% '1' if and only if the finally set vector is parallel to the #1 axis
% of PGF, that means (x,0) for #1=x and (0,y) for #2=y.
\def\pgfplots@initsizes@setunitvector#1#2#3#4{%
	\pgfkeysgetvalue{/pgfplots/#1 dir/value}\pgfplots@loc@dirvalue
	\expandafter\let\expandafter\pgfplots@loc@TMPb\csname pgfplots@#1\endcsname
	\ifx\pgfplots@loc@TMPb\pgfutil@empty
		\def#4{1}% we have (#1,0) or (0,#1)
		%
%\message{Setting unitvector(#1) to auto-computed multiple of e_#2 ...}%
		\edef\pgfplots@loc@TMPa{#3}%
		\if r\pgfplots@loc@dirvalue
			\edef\pgfplots@loc@TMPa{-#3}%
		\fi
% FIXME : '\pgfplots@loc@TMPa' contains 'pt' leading to missing
% character warning!
		\ifcase#2\relax
			\pgfsetxvec{\pgfqpoint{\pgfplots@loc@TMPa pt}{0pt}}%
		\or
			\pgfsetyvec{\pgfqpoint{0pt}{\pgfplots@loc@TMPa pt}}%
		\or
			\pgfsetzvec{\pgfqpoint{\pgfplots@loc@TMPa pt}{\pgfplots@loc@TMPa pt}}%
		\fi
	\else
		% Ok, we have a user-defined unit vector.
		%
		% That means we also need to apply the scaling trafo!
		%
		% 1. Check whether we have a complete vector of type (x,y):
		\expandafter\pgfutil@in@\expandafter(\expandafter{\pgfplots@loc@TMPb}%
		\ifpgfutil@in@
			% YES: we have (x,y):
			%
			\def#4{0}% we DON'T have (#1,0) or (0,#1). At least I think so.
			%
%\message{Setting unitvector(#1) to non-standard \csname pgfplots@#1\endcsname ...}%
			\def\pgfplots@loc@TMPa(##1,##2){%
				\pgfplotscoordmath{default}{parse}{##1}%
				\pgfplotscoordmath{default}{tofixed}{\pgfmathresult}%
				\let\pgfplots@loc@TMPb=\pgfmathresult
				\pgfplotscoordmath{default}{parse}{##2}%
				\pgfplotscoordmath{default}{tofixed}{\pgfmathresult}%
				\let\pgfplots@loc@TMPc=\pgfmathresult
				%
				\pgfplots@if{pgfplots@apply@datatrafo@#1}{%
					\pgfplotscoordmath{#1}{datascaletrafo noshift inverse to fixed}{\pgfplots@loc@TMPb}%
					\let\pgfplots@loc@TMPb=\pgfmathresult
					\pgfplotscoordmath{#1}{datascaletrafo noshift inverse to fixed}{\pgfplots@loc@TMPc}%
					\let\pgfplots@loc@TMPc=\pgfmathresult
				}{}%
				\csname pgfset#1vec\endcsname{%
					\pgfqpoint
						{\if r\pgfplots@loc@dirvalue -\fi\pgfplots@loc@TMPb pt}
						{\if r\pgfplots@loc@dirvalue -\fi\pgfplots@loc@TMPc pt}}%
			}%
			\expandafter\pgfplots@loc@TMPa\pgfplots@loc@TMPb%
			%
		\else
			% NO we simply have a scalar value.
			\def#4{1}% we have (#1,0) or (0,#1)
%\message{Setting unitvector(#1) to \csname pgfplots@#1\endcsname * e_{#2}...}%
			\pgfplots@if{pgfplots@apply@datatrafo@#1}{%
				\pgfplotscoordmath{default}{parse}{\csname pgfplots@#1\endcsname}%
				\pgfplotscoordmath{default}{tofixed}{\pgfmathresult}%
				\pgfplotscoordmath{#1}{datascaletrafo noshift inverse to fixed}{\pgfmathresult}%
				\edef\pgfplots@loc@TMPb{\pgfmathresult pt}%
			}{\relax}%
			\edef\pgfplots@loc@TMPb{\if r\pgfplots@loc@dirvalue -\fi\pgfplots@loc@TMPb}%
			\begingroup
				\pgf@xa=\pgfplots@loc@TMPb\relax
				\xdef\pgfplots@glob@TMPb{\pgf@sys@tonumber{\pgf@xa}}%
			\endgroup
			\ifcase#2\relax
				\pgfsetxvec{\pgfqpoint{\pgfplots@loc@TMPb}{0pt}}%
			\or
				\pgfsetyvec{\pgfqpoint{0pt}{\pgfplots@loc@TMPb}}%
			\or
				\pgfsetzvec{\pgfqpoint{\pgfplots@loc@TMPb}{\pgfplots@loc@TMPb}}%
			\fi
		\fi
	\fi
%\message{-> got unitvector(#1) = (\the\csname pgf@#1x\endcsname, \the\csname pgf@#1y\endcsname).^^J}%
}%

% Applies the 'axis equal' feature.
%
% PRECONDITION:
% 	- \pgfplots@target@unit@scale@[xyz] contains the current scaling
% 	factors which are to be applied to unit vectors
% 	- \pgfplots@target
% 	- neither unit vectors nor limits are in their final shape
% 	-  \pgfplots@set@default@size@options has been invoked before
%
% POSTCONDITION:
%   - \pgfplots@target@unit@scale@[xyz] 
% 	- unit vectors have been set to equal length,
% 	- axis limits have been rescaled such that the axis dimensions
% 	remain fixed
%
% There is just one algorithmic difficulty: the data scaling
% transformation. All unit vector length above are only meaningful in
% the UNTRANSFORMED range, so we have to mingle with the scaling
% transformation.
\def\pgfplots@apply@unit@ratio#1#2#3#4#5#6{%
	\begingroup
	\edef\pgfplots@x@veclength{#1}%
	\edef\pgfplots@y@veclength{#2}%
	\edef\pgfplots@z@veclength{#3}%
	\def\pgfplots@target@limitrescale@x@{1}%
	\def\pgfplots@target@limitrescale@y@{1}%
	\def\pgfplots@target@limitrescale@z@{1}%
	%
	\pgfkeysgetvalue{/pgfplots/unit vector ratio}\pgfplots@unit@vector@ratio
	\ifx\pgfplots@unit@vector@ratio\pgfutil@empty
	\else
		\edef\pgfplots@unit@vector@ratio{\pgfplots@unit@vector@ratio\space1 1 }%
		%
		\expandafter\pgfplots@unit@vector@ratio@check@nop\pgfplots@unit@vector@ratio\pgfplots@EOI
		\ifpgfplots@loc@tmp
			%
			% Step 1: compute the unit vector which STAYS CONSTANT.
			%
			\pgfkeysgetvalue{/pgfplots/unit vector ratio axis}\pgfplots@apply@unit@ratio@reference
			\ifx\pgfplots@apply@unit@ratio@reference\pgfutil@empty
				\pgfplots@apply@unit@ratio@find@reference%
			\fi
			%
			% FIXME : I could spent some attention here to save work: 
			% both, unit ratios and the resulting scales are computed at
			% least twice (once in \pgfplots@apply@unit@ratio@find@reference and once in the
			% following).
			\expandafter\pgfplots@apply@unit@ratio@prepareratios\pgfplots@unit@vector@ratio\pgfplots@EOI
			%
%\message{USING REFERENCE UNIT VECTOR FROM \pgfplots@apply@unit@ratio@reference; ratio \pgfplots@unit@ratio@x\space \pgfplots@unit@ratio@y\space \pgfplots@unit@ratio@z.^^J}%
			%
			% Step 2: apply the scaling:
			\pgfplots@rescale@unit@vector@reltoreference{x}{\pgfplots@unit@ratio@x}%
			\pgfplots@rescale@unit@vector@reltoreference{y}{\pgfplots@unit@ratio@y}%
			\ifpgfplots@threedim
				\pgfplots@rescale@unit@vector@reltoreference{z}{\pgfplots@unit@ratio@z}%
			\fi
		\else
\message{Skipped application of 'unit vector ratio=\pgfkeysvalueof{/pgfplots/unit vector ratio}': it is already done by 'scale uniformly'.^^J}%
		\fi
	\fi
	\xdef\pgfplots@glob@TMPa{%
		\noexpand\def\noexpand#1{\pgfplots@x@veclength}%
		\noexpand\def\noexpand#2{\pgfplots@y@veclength}%
		\noexpand\def\noexpand#3{\pgfplots@z@veclength}%
		\noexpand\def\noexpand#4{\pgfplots@target@limitrescale@x@}%
		\noexpand\def\noexpand#5{\pgfplots@target@limitrescale@y@}%
		\noexpand\def\noexpand#6{\pgfplots@target@limitrescale@z@}%
	}%
	\endgroup
	\pgfplots@glob@TMPa
}%

% Defines \ifpgfplots@loc@tmp := need to modify scaling factors
\def\pgfplots@unit@vector@ratio@check@nop#1 #2 #3 #4\pgfplots@EOI{%
	\pgfplots@loc@tmptrue
	\if3\pgfplots@scale@mode@choice
		% scale mode=scale uniformly
		\ifpgfplots@threedim
			\ifdim#1pt=#2pt
				\ifdim#1pt=#3pt
					% 'axis equal' is implicitly done by 'scale mode=scale
					% uniformly' anyway
					\pgfplots@loc@tmpfalse
				\fi
			\fi
		\else
			\ifdim#1pt=#2pt
				% 'axis equal' is implicitly done by 'scale mode=scale
				% uniformly' anyway
				\pgfplots@loc@tmpfalse
			\fi
		\fi
	\fi
	% activate the following line to deactivate optimization: [FIXME]
	%\pgfplots@loc@tmpfalse
}%

% This macro determines the reference axis for unit vector rescaling.
% The reference axis remains unscaled (it gets scaling factor 1 if you
% want it this way).
% 
% The other axes are scaled such that the desired unit vector ratios
% are fulfilled.
%
% The idea to select a reference axis is as follows:
% 1. Every unit vector scaling factor s should fulfill  s  <= 1.
% 2. Choose the reference axis such that the minimal amount of scaling
% is performed.
%
% The motivation for (1) is: if all involved scaling factors are at
% most 1, the resulting picture will only become *smaller*.
% Consequently, we can simply enlarge axis limits to restore the
% original width/height!
%
% The motivation for (2) is: a huge amount of scaling might reduce the
% size of the image too much. Of course, the figure will be enlarged
% to fit the original width/height, but most of it will be empty. So,
% use the smallest scaling.
%
% @POSTCONDITION The reference axis is stored in
% \pgfplots@apply@unit@ratio@reference .
%
% @see the key 'unit vector ratio axis=y' which allows to manually
% select the reference axis. This will illustrate what happens here.
\def\pgfplots@apply@unit@ratio@find@reference{%
	%
	\begingroup
	\let\pgfplots@ONE=\pgf@x
	\global\pgfplots@ONE=1.002pt
	%
	\def\pgfplots@optimum@sofar@axis{}%
	\let\pgfplots@optimum@sofar@value=\pgf@y
	\global\pgfplots@optimum@sofar@value=16000pt
	%
	%\pgfplots@apply@unit@ratio@find@reference@checkexplicitlimits
	%
	\ifx\pgfplots@optimum@sofar@axis\pgfutil@empty
		% set \pgfplots@loc@TMPa := 1 if and only if the axis is 3d
		\def\pgfplots@loc@TMPa{0}%
		\ifpgfplots@threedim
			% ah, it IS 3d!
			\ifdim\pgfplots@z@veclength pt=0pt
				% hm. It has view = {0}{90} (i.e. it is 2d)
				\def\pgfplots@loc@TMPa{0}%
			\else
				\def\pgfplots@loc@TMPa{1}%
			\fi
		\fi
		\if1\pgfplots@loc@TMPa
			% 3D is more complicated than 2D:
			% for every fixed reference axis, we have to check *two*
			% scaling factors.
			%
			% Furthermore, the optimality condition (2) needs to be
			% performed on the maximum max{1-s_a, 1-s_b} provided both of
			% these numbers are positive.
			%
			\def\pgfplots@check@##1##2{%
				% PRECONDITION: \pgfplots@apply@unit@ratio@reference is defined.
				%
				% renormalize \pgfplots@unit@[xyz] :
				\expandafter\pgfplots@apply@unit@ratio@prepareratios\pgfplots@unit@vector@ratio\pgfplots@EOI
				%
				% compute s_a :
				\pgfplots@getscale@unit@vector@reltoreference ##1{\csname pgfplots@unit@ratio@##1\endcsname}%
				\let\pgfplots@scale@a=\pgfmathresult
				%
				% compute s_b :
				\pgfplots@getscale@unit@vector@reltoreference ##2{\csname pgfplots@unit@ratio@##2\endcsname}%
				\let\pgfplots@scale@b=\pgfmathresult
				%
				% check if the actual choice of
				% \pgfplots@apply@unit@ratio@reference is FEASIBLE.
				% That is the case if s_a <= 1  && s_b <= 1.
				% We check
				%  (1 - s_a >= 0 )  && ( 1 - s_b >= 0 )
				% instead, since I need the value 
				%   max( 1-s_a, 1-s_b )
				% anyway.
				\def\pgfplots@ref@is@feasible{1}%
				\pgf@xa=\pgfplots@ONE \advance\pgf@xa by-\pgfplots@scale@a pt
				\ifdim\pgf@xa<0sp
					\def\pgfplots@ref@is@feasible{0}%
				\else
					\pgf@xb=\pgfplots@ONE \advance\pgf@xb by-\pgfplots@scale@b pt
					\ifdim\pgf@xb<0sp
						\def\pgfplots@ref@is@feasible{0}%
					\fi
				\fi
				% compute max(1-s_a,1-s_b) into \pgf@xa:
				% pgf@xa= max(pgf@xa,pgf@xb):
				\ifdim\pgf@xb>\pgf@xa	\pgf@xa=\pgf@xb	\fi
				\if1\pgfplots@ref@is@feasible
					\ifdim\pgf@xa<\pgfplots@optimum@sofar@value
						% Ah, ok. The actual choice is BETTER as it
						% involves less scaling.
						%
						% Remember it!
						\let\pgfplots@optimum@sofar@axis=\pgfplots@apply@unit@ratio@reference
						\global\pgfplots@optimum@sofar@value=\pgf@xa
					\fi
				\fi
%\message{^^Junit vector ratio 3D searching reference: checking \pgfplots@apply@unit@ratio@reference. feasable=\pgfplots@ref@is@feasible. \if1\pgfplots@ref@is@feasible max=\the\pgf@xa. \fi Optimum so far: value =\the\pgfplots@optimum@sofar@value\space for axis \pgfplots@optimum@sofar@axis.^^J}%
			}%
			%
			% Check 'x' as reference :
			\def\pgfplots@apply@unit@ratio@reference{x}%
			\pgfplots@check@ yz%
			%
			% Check 'y' as reference :
			\def\pgfplots@apply@unit@ratio@reference{y}%
			\pgfplots@check@ xz%
			%
			% Check 'z' as reference :
			\def\pgfplots@apply@unit@ratio@reference{z}%
			\pgfplots@check@ xy%
			%
		\else
			% 2D is much simpler: find the scale s which fulfills s <= 1.
			% One of them MUST fulfill it.
			% 
			% try 'x' axis as reference:
			\def\pgfplots@apply@unit@ratio@reference{x}%
			%
			% renormalize:
			\expandafter\pgfplots@apply@unit@ratio@prepareratios\pgfplots@unit@vector@ratio\pgfplots@EOI
			%
			% compute scaling factor: 
			\pgfplots@getscale@unit@vector@reltoreference y\pgfplots@unit@ratio@y%
			% 
%\message{^^Junit vector ratio 2D searching reference: checking \pgfplots@apply@unit@ratio@reference. feasable=\pgfmathresult < 1: \ifdim\pgfmathresult pt <\pgfplots@ONE YES-> use x\else NO->use y\fi^^J}%
			% and check (1). The condition (2) is irrelevant; it is met
			% anyway.
			\ifdim\pgfmathresult pt<\pgfplots@ONE
				\def\pgfplots@optimum@sofar@axis{x}%
			\else
				\def\pgfplots@optimum@sofar@axis{y}%
			\fi
		\fi
	\else
%\message{^^Junit vector ratio chose \pgfplots@optimum@sofar@axis\space to fulfill explicitly provided limits (at least partially).^^J}%
	\fi
	%
	\ifx\pgfplots@optimum@sofar@axis\pgfutil@empty
		\ifpgfplots@threedim
			\ifdim\pgfplots@z@veclength pt=0pt
				\def\pgfplots@optimum@sofar@axis{y}%
			\else
				\def\pgfplots@optimum@sofar@axis{z}%
			\fi
		\else
			\def\pgfplots@optimum@sofar@axis{y}%
		\fi
		\pgfplots@warning{The algorithm to implement 'unit vector ratio' failed! It could not determine the axis which shall be scaled and decided to use 'unit vector ratio axis=\pgfplots@optimum@sofar@axis'.}%
	\fi
	\let\pgfplots@apply@unit@ratio@reference=\pgfplots@optimum@sofar@axis
	\pgfmath@smuggleone\pgfplots@apply@unit@ratio@reference
	\endgroup
}%
\def\pgfplots@apply@unit@ratio@find@reference@checkexplicitlimits{%
	\ifpgfplots@autocompute@ymax \else \def\pgfplots@optimum@sofar@axis{y}\fi
	\ifpgfplots@autocompute@ymin \else \def\pgfplots@optimum@sofar@axis{y}\fi
	\ifpgfplots@autocompute@xmax \else \def\pgfplots@optimum@sofar@axis{x}\fi
	\ifpgfplots@autocompute@xmin \else \def\pgfplots@optimum@sofar@axis{x}\fi
	\ifpgfplots@threedim
		\ifpgfplots@autocompute@zmax \else \def\pgfplots@optimum@sofar@axis{z}\fi
		\ifpgfplots@autocompute@zmin \else \def\pgfplots@optimum@sofar@axis{z}\fi
	\fi
}%

\def\pgfplots@apply@unit@ratio@prepareratios#1 #2 #3 #4\pgfplots@EOI{%
	\def\pgfplots@unit@ratio@x{#1}%
	\def\pgfplots@unit@ratio@y{#2}%
	\def\pgfplots@unit@ratio@z{#3}%
	%
	% 'unit vector ratio' is measured relative to the y axis for 2d
	% and relative to the z axis for 3d plots.
	% renormalize such that it is relative to
	% \pgfplots@apply@unit@ratio@reference.
	%
	% Furthermore, renormalize such that
	% unit@ratio@\pgfplots@apply@unit@ratio@reference is 1.
	\pgfmathreciprocal@{\csname pgfplots@unit@ratio@\pgfplots@apply@unit@ratio@reference\endcsname}%
	\let\pgfplots@loc@TMPa=\pgfmathresult
	\ifpgfplots@threedim
		\if z\pgfplots@apply@unit@ratio@reference
		\else
			\pgfmathmultiply@{\pgfplots@loc@TMPa}{\pgfplots@unit@ratio@z}%
			\let\pgfplots@loc@TMPa=\pgfmathresult
		\fi
		%
		\pgfmathmultiply@{\pgfplots@loc@TMPa}{\pgfplots@unit@ratio@x}%
		\let\pgfplots@unit@ratio@x=\pgfmathresult
		%
		\pgfmathmultiply@{\pgfplots@loc@TMPa}{\pgfplots@unit@ratio@y}%
		\let\pgfplots@unit@ratio@y=\pgfmathresult
		%
		\pgfmathmultiply@{\pgfplots@loc@TMPa}{\pgfplots@unit@ratio@z}%
		\let\pgfplots@unit@ratio@z=\pgfmathresult
	\else
		\if y\pgfplots@apply@unit@ratio@reference
		\else
			\pgfmathmultiply@{\pgfplots@loc@TMPa}{\pgfplots@unit@ratio@y}%
			\let\pgfplots@loc@TMPa=\pgfmathresult
		\fi
		%
		\pgfmathmultiply@{\pgfplots@loc@TMPa}{\pgfplots@unit@ratio@x}%
		\let\pgfplots@unit@ratio@x=\pgfmathresult
		%
		\pgfmathmultiply@{\pgfplots@loc@TMPa}{\pgfplots@unit@ratio@y}%
		\let\pgfplots@unit@ratio@y=\pgfmathresult
		%
		\def\pgfplots@unit@ratio@z{<unused>}%
	\fi
	%
}%


% Computes a new unit vector E_#1 for direction #1 such that
% ||E_#1|| = #2 * ||e_reference||.
% Here, #2 is a scaling factor and e_reference is a reference axis.
% The reference axis is stored in
% \pgfplots@apply@unit@ratio@reference, the macro contains one of
% {x,y,z}.
%
% The data limits for '#1' will be enlarged as well (for 'unit rescale
% keep size').
%
% #1 is the axis which should be scaled (i.e. #1 in {x,y,z}).
% It is allowed if #1 = \pgfplots@apply@unit@ratio@reference. In this
% case, you can provide a scale '#2' to rescale the axis.
%
% #2 is a desired scale, relative to
% \pgfplots@apply@unit@ratio@reference. #2 should be a number without
% unit.
%
% The parameter \pgfplots@apply@unit@ratio@reference is also one of
% {x,y,z}.
%
\def\pgfplots@rescale@unit@vector@reltoreference#1#2{%
	\def\pgfplots@loc@TMPa{0}%
	\if#1\pgfplots@apply@unit@ratio@reference
		\pgfplotsmath@ifapproxequal@dim{#2pt}{1pt}{0.0002pt}{%
		}{%
			\def\pgfplots@loc@TMPa{1}%
		}%
	\else
		\def\pgfplots@loc@TMPa{1}%
	\fi
	\ifdim\csname pgfplots@#1@veclength\endcsname pt=0pt
		\def\pgfplots@loc@TMPa{0}%
	\fi
	\if1\pgfplots@loc@TMPa
		%
		\pgfplots@getscale@unit@vector@reltoreference{#1}{#2}%
		\global\let\pgfplots@glob@TMPa=\pgfmathresult
		%
%\message{Rescaling '#1' by \pgfplots@glob@TMPa.^^J}%
		%
		\pgfmathmultiply@{\csname pgfplots@#1@veclength\endcsname}{\pgfplots@glob@TMPa}%
		\expandafter\let\csname pgfplots@#1@veclength\endcsname=\pgfmathresult
		%
		\pgfmathreciprocal@\pgfplots@glob@TMPa
		\expandafter\let\csname pgfplots@target@limitrescale@#1@\endcsname=\pgfmathresult
		%
	\fi
}

% Updates the #1 axis limits such that the axis' dimensions
% stay the same after scaling the #1 unit vector by a scale 's'.
%
% PRECONDITION:
%   - the #1 unit vector has been rescaled by a factor s.
%   For example, e_xnew := e_x * 0.5 .
% 
% POSTCONDITION:
%   - the axis limits are enlarged by a factor 1/s such that 
%   1/s (#1max - #1min) * e_xnew = (#1max- #1min) * e_x.
%
% In other words, the unit vector rescale is componensated by
% modifying the axis limits: we want to add an absolute component 'd'
% to the range:
% 1/s (xmax - xmin ) = xmax - xmin +d 
% =>
% d = (1/s - 1) * (xmax - xmin)
%
% The only remaining thing to do is to distribute 'd' to 'xmax' and
% 'xmin'. Typically, 50% to each will be fine, I guess...
% 
% #1: either x, y or z. It denotes the direction which has been
% modified.
% #2: the INVERSE of the scaling factor, #2 = 1/s .
%
\def\pgfplots@apply@unit@vector@rescale@keep@size#1#2{%
	\ifdim#2pt=1pt
	\else
	\ifpgfplots@unit@vector@rescale@keep@size
		%
%\message{'unit rescale keep size': Resizing data range for #1: from \csname pgfplots@#1min\endcsname:\csname pgfplots@#1max\endcsname\ to}%
		\pgfmathsubtract@{\csname pgfplots@#1max\endcsname}{\csname pgfplots@#1min\endcsname}%
		\begingroup
			\pgf@xa=\pgfmathresult pt
			\pgfmathsubtract@{#2}{1.0}%
			\pgf@xa=\pgfmathresult \pgf@xa% this is 'd'
			%
			% \pgfplots@glob@TMPb : will be subtracted from #1min
			% \pgfplots@glob@TMPc : will be added to #1max
			\pgfplots@if{pgfplots@autocompute@#1min}{%
				\pgfplots@if{pgfplots@autocompute@#1max}{%
					\pgf@xa=0.5 \pgf@xa
					\xdef\pgfplots@glob@TMPb{\pgf@sys@tonumber{\pgf@xa}}%
					\xdef\pgfplots@glob@TMPc{\pgfplots@glob@TMPb}%
				}{%
					\xdef\pgfplots@glob@TMPb{\pgf@sys@tonumber{\pgf@xa}}%
					\xdef\pgfplots@glob@TMPc{0.0}%
				}%
			}{%
				\pgfplots@if{pgfplots@autocompute@#1max}{%
					\xdef\pgfplots@glob@TMPb{0.0}%
					\xdef\pgfplots@glob@TMPc{\pgf@sys@tonumber{\pgf@xa}}%
				}{%
					\xdef\pgfplots@glob@TMPb{0.0}%
					\xdef\pgfplots@glob@TMPc{0.0}%
				}%
			}%
		\endgroup
		\pgfmathsubtract@{\csname pgfplots@#1min\endcsname}{\pgfplots@glob@TMPb}%
		\expandafter\global\expandafter\let\csname pgfplots@#1min\endcsname=\pgfmathresult
		\pgfmathadd@{\csname pgfplots@#1max\endcsname}{\pgfplots@glob@TMPc}%
		\expandafter\global\expandafter\let\csname pgfplots@#1max\endcsname=\pgfmathresult
%\message{\csname pgfplots@#1min\endcsname:\csname pgfplots@#1max\endcsname. [- \pgfplots@glob@TMPb; + \pgfplots@glob@TMPc]^^J}%
		%
		% Update auxiliary data members:
		\pgfplots@visphase@notify@changeofcanvaslimits{#1}%
	\fi
	\fi
}%

% #1: an axis which should be scaled
% #2: the desired final ratio  ||e_#1||/||e_ref||
\def\pgfplots@getscale@unit@vector@reltoreference#1#2{%
	% 
	% If the datascaling transformation is active (which is almost
	% everytime the case here), we have a transformation
	% T^{-1}(x)= 10^scale * x
	% with different scales for every axis.
	%
	% If the datascaling transformation is NOT active, scale is 0
	% and T^{-1} = Identity.
	%
	% Note that the datascaling transformation also has
	% translations (shifts). These are not important here.
	%
	% Goal:
	% compute E_#1 such that
	%   #2* || T^{-1} e_ref || = || T^{-1} E_#1 ||
	% where T^{-1} is the data scaling transformation and e_ref the
	% reference unit vector. Keep in mind that there are
	% *different* data scaling transformations for each axis.
	%
	% We are given e_ref and e_#1 and the desired aspect ratio
	% between e_ref and E_#1, which is available as #2.
	%
	% So: T^{-1} E_#1 :=  s* T^{-1} e_#1 where 
	%  s = #2 * ||T^{-1} e_ref|| / || T^{-1} e_#1 || 
	%    = |10^{scale_ref}| / |10^{scale_#1}| * #2 * || e_ref|| / ||e_#1||.
	% 
	% Then, E_#1 = T ( T^{-1} E_#1 ) = s * e_#1.
	%
	% -> compute 's'! 
	%
	% Part 1: compute
	% #2 * ||e_ref|| / ||e_#1||.
	%
	\def\pgfplots@loc@TMPa{1}%
	\ifdim\csname pgfplots@#1@veclength\endcsname pt=0pt
		\def\pgfplots@loc@TMPa{0}%
	\else
		\ifdim\csname pgfplots@\pgfplots@apply@unit@ratio@reference @veclength\endcsname pt=0pt
			\def\pgfplots@loc@TMPa{0}%
		\fi
	\fi
	\if0\pgfplots@loc@TMPa
		\def\pgfmathresult{16001}%
	\else
		\pgfmathdivide@
			{\csname pgfplots@\pgfplots@apply@unit@ratio@reference @veclength\endcsname}
			{\csname pgfplots@#1@veclength\endcsname}%
		\pgfmathmultiply@
			{\pgfmathresult}%
			{#2}%
		\global\let\pgfplots@glob@TMPa=\pgfmathresult
		% 
		% also compute 1/s, required as temporary value:
		%\pgfmathmultiply@
		%	{\csname pgfplots@\pgfplots@apply@unit@ratio@reference @inverseveclength\endcsname}
		%	{\csname pgfplots@#1@veclength\endcsname}%
		%\ifdim#2pt=1pt
		%\else
		%	\pgfmathdivide@{\pgfmathresult}{#2}%
		%\fi
		%\global\let\pgfplots@glob@TMPb=\pgfmathresult
		%
		% Part 2: handle data scaling trafo scales:
		\begingroup
			\def\pgfplots@tmp@exponentref{0}%
			\def\pgfplots@tmp@exponentK{0}%
			\pgfplots@if{pgfplots@apply@datatrafo@\pgfplots@apply@unit@ratio@reference }{%
				\pgfplots@letcsname{pgfplots@tmp@exponentref}={pgfplots@data@scale@trafo@EXPONENT@\pgfplots@apply@unit@ratio@reference }%
			}{}%
			\pgfplots@if{pgfplots@apply@datatrafo@#1}{%
				\pgfplots@letcsname{pgfplots@tmp@exponentK}={pgfplots@data@scale@trafo@EXPONENT@#1}%
			}{}%
			\c@pgf@counta=\pgfplots@tmp@exponentref\relax
			\advance\c@pgf@counta by-\pgfplots@tmp@exponentK\relax
			\ifnum\c@pgf@counta=0
			\else
				\pgfplotsmathmultiplypowten@{\pgfplots@glob@TMPa}{\c@pgf@counta}%
				\global\let\pgfplots@glob@TMPa=\pgfmathresult
			%	\pgfplotsmathmultiplypowten@{\pgfplots@glob@TMPb}{-\c@pgf@counta}%
			%	\global\let\pgfplots@glob@TMPb=\pgfmathresult
			\fi
			\xdef\pgfplots@glob@TMPc{\the\c@pgf@counta}%
		\endgroup
		\let\pgfmathresult=\pgfplots@glob@TMPa
	\fi
%\message{\string\pgfplots@getscale@unit@vector@reltoreference{#1}{#2} (reference \pgfplots@apply@unit@ratio@reference) = \pgfmathresult.^^J}%
}
