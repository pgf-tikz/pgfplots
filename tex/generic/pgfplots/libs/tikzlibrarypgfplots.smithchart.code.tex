%--------------------------------------------
%
% Package pgfplots, library for smith charts.
%
% Copyright 2010 by Christian Feuersänger.
%
% This program is free software: you can redistribute it and/or modify
% it under the terms of the GNU General Public License as published by
% the Free Software Foundation, either version 3 of the License, or
% (at your option) any later version.
% 
% This program is distributed in the hope that it will be useful,
% but WITHOUT ANY WARRANTY; without even the implied warranty of
% MERCHANTABILITY or FITNESS FOR A PARTICULAR PURPOSE.  See the
% GNU General Public License for more details.
% 
% You should have received a copy of the GNU General Public License
% along with this program.  If not, see <http://www.gnu.org/licenses/>.
%
%
% - CARTESIAN INPUT
% - tick/grid coordinates are from 
%     H := [0,infty] x [-infty,infty]
% - input coordinates can be either from H or (perhaps preferred) from
%   the unit circle.
%   this "preferred" needs to be discussed.
% - the transformed data range is the unit circle (or a sequeezed variant)
% - in order to compute limits etc., I should accept data in H. this
%   should simplify the logic to determine ticks etc considerably.
%   problem: this transformation appears to be quite difficult (?)
%   	-> r(z) = (z-1)/(z+1)
%   in complex arithmetics (but the G-tutorial.pdf says something
%   about these circle equations!?)
%
%   www.amanogawa.com/archive/docs/G-tutorial.pdf
%
%
% Idea: 
%   - work on H
%   - transform whereever necessary
%   - implement all pgfplots wrinkles in analogy to polar axes
%   - provide support for normalized input coords (combined with
%   untransformed limits or something like that)

\pgfplotsdefineaxistype{smithchart}{%
	\pgfplots@smithchartaxis@activate
}%

\newif\ifpgfplotspointisinsmithchartCS

\pgfplotsset{
	/pgfplots/arc limits/.initial=,
	/pgfplots/arc limits start/.initial=0,
	%
	% this boolean may only be used inside of \addplot. It will be
	% ignored otherwise.
	/pgfplots/is smithchart cs/.is if=pgfplotspointisinsmithchartCS,
	/pgfplots/is smithchart cs/.default=true,
	/pgfplots/every smithchart axis/.style={
		grid=major,
		xmin=0,
		xmax=16000,% FIXME : more is not possible because some code uses the \pgfplots@xmin@reg registers... (ticks)
		ymin=-16000,ymax=16000,
		scaled ticks=false,
		xtick pos=left,
		ytick pos=left,
		xtick style={draw=black},
		xtick align=center,
		xtick={0.2,0.5,1,2,5},
		ytick={%
			 0.2, 0.5, 1, 2, 5,
			-0.2,-0.5,-1,-2,-5},
	},
}

\def\pgfplots@smithchartaxis@activate{%
	\let\pgfplotsqpointxy@cart=\pgfplotsqpointxy
	\let\pgfplotsqpointxy=\pgfplotsqpointxy@smithchartaxis
	\let\pgfplotsqpointxy@orthogonal=\pgfplotsqpointxy
	\def\pgfplotsqpointxyz##1##2##3{\pgfplotsqpointxy{##1}{##2}}% FIXME
	\let\pgfplotspointouternormalvectorofaxis@=\pgfplotspointouternormalvectorofaxis@smithchartaxis
	\def\pgfplotspointouternormalvectorofaxis@ifdependson@v##1##2##3{##2}%
	\def\pgfplots@drawticklines@INSTALLCLIP@onorientedsurf##1{}%
	\let\pgfplots@drawgridlines@INSTALLCLIP@onorientedsurf=\pgfplots@drawgridlines@INSTALLCLIP@onorientedsurf@smithchartaxis
	\def\pgfplots@visphase@notify@changeofcanvaslimits##1{}%
	\def\pgfplots@avoid@empty@axis@range@for##1{}%
	\def\pgfplotsaxisifcontainspoint##1##2{##1}%
	\let\pgfplots@clippath@prepare@for@axistype=\pgfplots@clippath@prepare@for@axistype@smithchartaxis
	\let\pgfplots@handle@invalid@range@defaultlimits=\pgfplots@handle@invalid@range@defaultlimits@smithchart%
	\let\pgfplotspointonorientedsurfaceabwithbshift=\pgfplotspointonorientedsurfaceabwithbshift@smithchartaxis
	\let\pgfplots@drawgridlines@onorientedsurf@fromto=\pgfplots@drawgridlines@onorientedsurf@fromto@smithchart
	\let\pgfplots@drawaxis@innerlines@onorientedsurf=\pgfplots@drawaxis@innerlines@onorientedsurf@smithchart
	\let\pgfplots@drawaxis@outerlines@separate@onorientedsurf=\pgfplots@drawaxis@outerlines@separate@onorientedsurf@smithchartaxis
	\let\pgfplotspoint@initialisation@axes=\pgfplotspoint@initialisation@axes@smithchart%
	%\let\pgfplotspoint@initialisation@units=\pgfplotspoint@initialisation@units@smithchart
	\def\axisdefaultheight{\axisdefaultwidth}%
	\let\pgfplots@initsizes=\pgfplots@initsizes@smithchart
	%\let\pgfplots@limits@ready=\pgfplots@limits@ready@smithchart
	%\let\pgfplots@show@ticklabel@=\pgfplots@show@ticklabel@@smithchart
	%\def\pgfplots@xtick@disable@last@tick{0}%
	\let\pgfplots@xtick@check@tickshow=\pgfplots@xtick@check@tickshow@smithchart%
	\let\pgfplots@ytick@check@tickshow=\pgfplots@ytick@check@tickshow@smithchart%
	\let\pgfplots@set@options@sanitize=\relax
	\let\pgfplots@set@options@sanitizemode=\relax
	\let\pgfplotscoordmathnotifydatascalesetfor=\pgfplotscoordmathnotifydatascalesetfor@smithchart
	%
	% this here is set *before* 'every smithchart axis' is invoked.
	\pgfplotsset{
		disabledatascaling,
	}%
	\expandafter\def\expandafter\pgfplots@notify@options@are@set\expandafter{%
		\pgfplots@notify@options@are@set
		\pgfplotsset{%
			separate axis lines,%
			is smithchart cs=false,%
			axis x line*=center,
		}%
		\def\pgfplots@xtickposnum{2}%
	}%
	\def\pgfplots@xticklabel@pos{}%
	\def\pgfplots@yticklabel@pos{}%
	\def\pgfplots@zticklabel@pos{}%
	\def\pgfplots@init@ticklabelaxisspecfor##1##2{}%
	\def\pgfplots@init@ticklabelaxisspec@twodim@for##1##2{}%
	\def\pgfplotspointonorientedsurfaceabmatchaxisline@warn##1{}% clear warning. It works for smith charts.
	\def\pgfplots@xticklabelaxisspec{v20}%
	\def\pgfplots@yticklabelaxisspec{0v0}%
	\def\pgfplots@zticklabelaxisspec{00v}%
	%
	% cartesian cs
	\tikzdeclarecoordinatesystem{cartesian}{\edef\pgfplots@loc@TMPa{##1}\expandafter\pgfplotspointcartesian@\pgfplots@loc@TMPa\pgfplots@coord@end}%
	%
}%

\def\pgfplotspointcartesian@#1,#2\pgfplots@coord@end{%
	\pgfpointxy@orig{#1}{#2}%
}%
\def\pgfplotscoordmathnotifydatascalesetfor@smithchart#1{%
	\def\pgfplotscoordmathnotifydatascalesetfor##1{}%
	\edef\pgfplotscoordmathnotifydatascalesetfor@{#1}%
	\def\pgfplotscoordmathnotifydatascalesetfor@@{x}%
	\ifx\pgfplotscoordmathnotifydatascalesetfor@@\pgfplotscoordmathnotifydatascalesetfor@
		\pgfplotscoordmath{#1}{datascaletrafo set params}{0}{0}%
	\else
		\def\pgfplotscoordmathnotifydatascalesetfor@@{y}%
		\ifx\pgfplotscoordmathnotifydatascalesetfor@@\pgfplotscoordmathnotifydatascalesetfor@
			\pgfplotscoordmath{#1}{datascaletrafo set shift}{0}%
		\fi
	\fi
	\let\pgfplotscoordmathnotifydatascalesetfor=\pgfplotscoordmathnotifydatascalesetfor@smithchart
}%

% #1: the "a" value on the oriented surf
% #2: the "b" value. 
% #3: the shift along the normal.
%
\def\pgfplotspointonorientedsurfaceabwithbshift@smithchartaxis#1#2#3{%
	% implement the shift in "b" direction explicitly:
	\pgfpointadd
		{\pgfplotspointonorientedsurfaceab{#1}{#2}}%
		{%
			\pgfplotspointonorientedsurfaceabtolinespec{v}{0}% FIXME
			\afterassignment\pgfplots@gobble@until@relax
			\pgf@xa=-#3\relax
			\edef\pgfplots@shift@no@unit{\pgf@sys@tonumber\pgf@xa}%
			\pgfqpointscale
				{\pgfplots@shift@no@unit}
				{\expandafter\pgfplotspointouternormalvectorofaxis\expandafter{\pgfplotsretval}}%
		}%
}


% Computes the complex division
% (A + j B) / (C + j D) = (A C + B D  + j (B C - A D) ) / (C^2 + D^2)
% and assigns the result to \pgfmathresult and \pgfmathresultim . Here
% 'j = sqrt{-1}' is the imaginary unit.
%
% #1 : A
% #2 : B
% #3 : C
% #4 : D
%
% The arithmetics is performed in \pgfplotscoordmath{default} (which
% uses the floating point unit in the initial configuration)
%
% Numbers are expected to be already parsed (i.e. you need to invoke 
%  \pgfplotscoordmath{default}{parsenumber}{#1}%
% 	\let\A=\pgfmathresult
% before)
\def\pgfplotscoordmathcomplexdivision#1#2#3#4{%
	\begingroup
	\edef\pgfplots@A{#1}%
	\edef\pgfplots@B{#2}%
	\edef\pgfplots@C{#3}%
	\edef\pgfplots@D{#4}%
	%
	%
	% ok, compute it:
	\pgfplotscoordmath{default}{op}{multiply}{{\pgfplots@A}{\pgfplots@C}}%
	\let\pgfplots@AC=\pgfmathresult
	\pgfplotscoordmath{default}{op}{multiply}{{\pgfplots@A}{\pgfplots@D}}%
	\let\pgfplots@AD=\pgfmathresult
	\pgfplotscoordmath{default}{op}{multiply}{{\pgfplots@B}{\pgfplots@D}}%
	\let\pgfplots@BD=\pgfmathresult
	\pgfplotscoordmath{default}{op}{multiply}{{\pgfplots@B}{\pgfplots@C}}%
	\let\pgfplots@BC=\pgfmathresult
	%
	\pgfplotscoordmath{default}{op}{multiply}{{\pgfplots@C}{\pgfplots@C}}%
	\let\pgfplots@CC=\pgfmathresult
	\pgfplotscoordmath{default}{op}{multiply}{{\pgfplots@D}{\pgfplots@D}}%
	\let\pgfplots@DD=\pgfmathresult
	\pgfplotscoordmath{default}{op}{add}{{\pgfplots@CC}{\pgfplots@DD}}%
	\pgfplotscoordmath{default}{op}{reciprocal}{{\pgfmathresult}}%
	\let\pgfplots@scale=\pgfmathresult
	%
	%
	\pgfplotscoordmath{default}{op}{add}{{\pgfplots@AC}{\pgfplots@BD}}%
	\pgfplotscoordmath{default}{op}{multiply}{{\pgfmathresult}{\pgfplots@scale}}%
	\let\pgfplots@x=\pgfmathresult
	%
	\pgfplotscoordmath{default}{op}{subtract}{{\pgfplots@BC}{\pgfplots@AD}}%
	\pgfplotscoordmath{default}{op}{multiply}{{\pgfmathresult}{\pgfplots@scale}}%
	\let\pgfplots@y=\pgfmathresult
	%
	\xdef\pgfplots@glob@TMPa{%
		\noexpand\def\noexpand\pgfmathresult{\pgfplots@x}%
		\noexpand\def\noexpand\pgfmathresultim{\pgfplots@y}%
	}%
	\endgroup
	\pgfplots@glob@TMPa
}%

\def\pgfplotsqpointxy@smithchartaxis#1#2{%
	\pgf@process{%
		\ifpgfplotspointisinsmithchartCS
			\def\pgfplots@x{#1}%
			\def\pgfplots@y{#2}%
		\else
			% compute rx + j* ry = (#1 + j * #2 -1) / (#1 + j*#2 + 1)
			%
			% I write
			% #1 - 1 + j * #2 = A + j * B
			% 1 + #1 + j * #2 = C + j * D
			%
			% -> rx + j * ry = (A + j B) / (C + j D) = (A C + B D  + j (B C - A D) ) / (C^2 + D^2)
			\pgfplotscoordmath{default}{parsenumber}{#1}%
			\let\pgfplots@x=\pgfmathresult
			%
			\pgfplotscoordmath{default}{parsenumber}{#2}%
			\let\pgfplots@D=\pgfmathresult
			%
			\pgfplotscoordmath{default}{one}%
			\let\pgfplots@one=\pgfmathresult
			%
			\pgfplotscoordmath{default}{op}{add}{{\pgfplots@one}{\pgfplots@x}}%
			\let\pgfplots@C=\pgfmathresult
			%
			\pgfplotscoordmath{default}{op}{subtract}{{\pgfplots@x}{\pgfplots@one}}%
			\let\pgfplots@A=\pgfmathresult
			%
			\let\pgfplots@B=\pgfplots@D
			%
			\pgfplotscoordmathcomplexdivision\pgfplots@A\pgfplots@B\pgfplots@C\pgfplots@D
			\pgfplotscoordmath{default}{tofixed}{\pgfmathresult}%
			\let\pgfplots@x=\pgfmathresult
			\pgfplotscoordmath{default}{tofixed}{\pgfmathresultim}%
			\let\pgfplots@y=\pgfmathresult
		\fi
		%
		\pgfqpointxy@orig\pgfplots@x\pgfplots@y
%\message{pgfplotsqpointxy{#1}{#2} ---> (\pgfplots@x,\pgfplots@y) ---> (\the\pgf@x,\the\pgf@y)}%
	}%
}%
\def\pgfplots@clippath@prepare@for@axistype@smithchartaxis{%
	\def\pgfplots@clippath@install##1{%
		\pgfpathellipse
			{\pgfpointxy@orig{0}{0}}
			{\pgfpointxy@orig{1}{0}}
			{\pgfpointxy@orig{0}{1}}%
		\pgfplots@clippath@use@{##1}%
	}%
}%

\def\pgfplotspointouternormalvectorofaxis@smithchartaxis#1#2#3\relax{%
	\if v#1%
		\pgfqpoint{0pt}{1pt}%
	\else
		\if v#2%
			\pgfplotspointouternormalvectorofaxisgetv{#1#2#3}%
			\ifx\pgfplotsretval\pgfutil@empty
				\def\pgfplotsretval{0}%
			\fi
			\pgfpointdiff
				{\pgfpointxy@orig{0}{0}}%
				{\pgfplotsqpointxy{0}{\pgfplotsretval}}%
			\pgfpointnormalised{}%
		\else
			\pgfqpoint{0pt}{1pt}%
		\fi
	\fi
	\pgf@process{}%
	\endgroup
}%

\def\pgfplotspoint@initialisation@axes@smithchart{%
	\begingroup
	%\pgfplotsqpointxy{\pgfplots@xmin}{\pgfplots@ymin}%
	\gdef\pgfplotspointminminmin{\pgfpointxy@orig{0}{0}}%
	%
	% the "x" axis is the diameter of the circle (for fixed y=0)
	\pgf@x=2\pgf@xx
	\pgf@y=0pt
	\xdef\pgfplotspointxaxis{\noexpand\pgf@x=\the\pgf@x\space\noexpand\pgf@y=\the\pgf@y\space}%
	\pgfmathveclen{\pgf@x}{\pgf@y}%
	\xdef\pgfplotspointxaxislength{\pgfmathresult pt}%
	%
	\pgfplotsqpointxy{\pgfplots@xmax}{\pgfplots@ymax}%
	\xdef\pgfplotspointyaxis{\noexpand\pgf@x=\the\pgf@x\space\noexpand\pgf@y=\the\pgf@y\space}%
	%
	% the length of the "y" axis is 2*pi*r (for fixed x=0, the outer
	% circle).
	% The radius is the length of (0,1) which is (0pt,\pgf@xx1):
	\pgfmath@basic@multiply@{\pgf@sys@tonumber\pgf@xx}{1}%
	\pgfmathmultiply@{\pgfmathresult}{6.28318530717959}% 2*pi * r
	\xdef\pgfplotspointyaxislength{\pgfmathresult pt}%
	%
	\global\let\pgfplotspointzaxis=\pgfpointorigin
	\gdef\pgfplotspointzaxislength{0pt}%
	\endgroup
	%
}
\let\pgfplotspoint@initialisation@units@orig=\pgfplotspoint@initialisation@units
\def\pgfplotspoint@initialisation@units@smithchart{%
	\pgfplotspoint@initialisation@units@orig
	\def\pgfplotspointunitx{%
		\pgfplotspointouternormalvectorofaxisgetv{v10}% angle varying, radius at outer pos
		\ifx\pgfplotsretval\pgfutil@empty
			\def\pgfplotsretval{0}%
		\fi
		\pgfmath@basic@sin@{\pgfplotsretval}%
		\pgf@x=-\pgfmathresult pt
		\pgfmath@basic@cos@{\pgfplotsretval}%
		\pgf@y=\pgfmathresult pt
	}%
	\def\pgfplotsunitxlength{1}%
	\def\pgfplotsunitxinvlength{1}%
}%

\def\pgfplots@drawgridlines@INSTALLCLIP@onorientedsurf@smithchartaxis#1{%
	%\pgfplots@clippath@install{\pgfusepath{clip}}%
	\pgfkeysgetvalue{/pgfplots/arc limits}\pgfplots@arclimits
	\ifx\pgfplots@arclimits\pgfutil@empty
	\else
		\pgfplots@gridlines@init@arc@limits
	\fi
}%

% Initialises the 'arc limits' feature. All it does is to prepare the
% method \pgfplots@get@current@arc@limit. 
%
% The method is quite involved. Please refer to the manual for what it
% is supposed to do, and refer to the code comments below for
% implementational details.
\def\pgfplots@gridlines@init@arc@limits{%
	%
	% we have no "xticknum -> xtickpos" lookup table yet.
	\def\b@pgfplots@xticknum@to@pos@lookup{0}%
	%
	% normalise the argument for 'arc limits': each list element
	% should be of the form '<xtickpos>:<n>' where <n> means that each
	% <n>th arc can pass.
	\pgfplotslistnewempty\pgfplots@arclimits@normalised
	\expandafter\pgfplotsutilforeachcommasep\pgfplots@arclimits\as\entry{%
		\expandafter\pgfplots@gridlines@init@arc@limits@normalise\entry\relax
		\expandafter\pgfplotslistpushback\entry\to\pgfplots@arclimits@normalised
	}%
	%
	% Ok. 
	%
	% Now, the 'arc limits' feature relies *crucially* on grid line
	% indices (for the 'each nth' feature).
	%
	% I sort the arcs according to their absolute magnitude and assign
	% indices into the resulting arrays to normalize that stuff.
	%
	% The array is of the form 
	%   A[i] = entry of \pgfplots@prepared@tick@positions@*
	% and contains *both*, major and minor grid lines.
	\pgfplotsarraynewempty\pgfplots@ygridlines
	\pgfplotscoordmath{default}{zero}%
	\edef\elem{{-1}{\pgfmathresult}}% require 0 to be zero for symmetry even if there is no such tick pos
	\expandafter\pgfplotsarraypushback\elem\to\pgfplots@ygridlines
	\ifpgfplots@ymajorgrids
		% insert all major tick positions, using their absolute value.
		\pgfplotslistforeachungrouped\pgfplots@prepared@tick@positions@major@y\as\elem{%
			\expandafter\pgfplots@prepared@tick@pos@unpack\elem
			\pgfplotscoordmath{default}{parsenumber}{\pgfplots@tick}%
			\pgfplotscoordmath{default}{op}{abs}{{\pgfmathresult}}%
			\edef\elem{{\pgfplots@ticknum}{\pgfmathresult}}%
			\expandafter\pgfplotsarraypushback\elem\to\pgfplots@ygridlines
		}%
	\fi
	\ifpgfplots@yminorgrids
		% now the same for minor grid positions:
		\pgfplotslistforeachungrouped\pgfplots@prepared@tick@positions@minor@y\as\elem{%
			\expandafter\pgfplots@prepared@tick@pos@unpack\elem
			\pgfplotscoordmath{default}{parsenumber}{\pgfplots@tick}%
			\pgfplotscoordmath{default}{op}{abs}{{\pgfmathresult}}%
			\edef\elem{{\pgfplots@ticknum}{\pgfmathresult}}%
			\expandafter\pgfplotsarraypushback\elem\to\pgfplots@ygridlines
		}%
	\fi
	% sort the array!
	\pgfkeysgetvalue{/pgfplots/smithchart@sortlt/.@cmd}\pgfplots@loc@TMPa
	\pgfkeyslet{/pgfplots/iflessthan/.@cmd}\pgfplots@loc@TMPa
	\pgfplotsarraysort\pgfplots@ygridlines
	%
	% ok. Now it is sorted. 
	%
	% I finally need a lookup 
	%   \pgfplots@ticknum --> sort index.
	% If the associated values have the same absolute value, the same
	% sort index should be assigned.
	%
	% For example, the array might be associated to the following tick
	% positions, sorted by absolute value:
	% 0.0, 1.0, -1.0, 3.0, -3.0, 4.0, -4.0, 5.0, -5.0
	% What I want is that 
	% 0.0 gets sort index 0
	% 1.0 and -1.0 get sort index 1
	% 3.0 and -3.0 get sort index 2
	% 4.0 and -4.0 get sort index 3
	% and so on. The array contains only absolute values, so that's
	% not too difficult to check.
	%
	% Since each of the tick positions can be (uniquely) identified by
	% its associated \pgfplots@ticknum value, I map \pgfplots@ticknum
	% to the sort index.
	\countdef\c@sortindex=\c@pgf@counta
	\c@sortindex=\pgfkeysvalueof{/pgfplots/arc limits start} % this is assigned to the '0.0' ygridline (if any)
	\def\pgfplots@lasttickpos{}%
	\pgfplotsarrayforeachungrouped\pgfplots@ygridlines\as\elem{%
		\expandafter\pgfplots@prepared@tick@pos@unpack\elem
		\ifx\pgfplots@lasttickpos\pgfutil@empty
		\else
			\ifx\pgfplots@lasttickpos\pgfplots@tick
			\else
				\advance\c@sortindex by1
			\fi
		\fi
		\expandafter\edef\csname pgfplots@tickpos@to@sortidx@\pgfplots@ticknum\endcsname{\the\c@sortindex}%
%\message{\pgfplots@ticknum\space(abs(tickpos) = \pgfplots@tick) ---> sort index \csname pgfplots@tickpos@to@sortidx@\pgfplots@ticknum\endcsname^^J}%
		\let\pgfplots@lasttickpos=\pgfplots@tick
	}%
}%

\pgfkeysdefargs{/pgfplots/smithchart@sortlt}{#1#2#3#4}{%
	\expandafter\pgfplots@prepared@tick@pos@unpack#1%
	\let\pgfplots@A=\pgfplots@tick
	\expandafter\pgfplots@prepared@tick@pos@unpack#2%
	\let\pgfplots@B=\pgfplots@tick
	\pgfplotscoordmath{default}{if less than}{\pgfplots@A}{\pgfplots@B}{#3}{#4}%
}%

% initialise a lookuptable from ticknumber -> tick position (sort
% of an array)
%
% This is only invoked if it is needed (if 'arc
% limits={[index]4,[index]2}' or something like that is used, see the
% manual).
%
\def\pgfplots@gridlines@init@arc@limits@init@ticknum@lookup{%
	\c@pgf@counta=0
	\ifpgfplots@ymajorgrids
		\pgfplotslistforeachungrouped\pgfplots@prepared@tick@positions@major@x\as\elem{%
			\expandafter\pgfplots@prepared@tick@pos@unpack\elem
			\expandafter\let\csname pgfplots@xtick@num@to@pos@\the\c@pgf@counta\endcsname=\pgfplots@tick
			\advance\c@pgf@counta by1
		}%
	\fi
	\ifpgfplots@yminorgrids
		\pgfplotslistforeachungrouped\pgfplots@prepared@tick@positions@minor@x\as\elem{%
			\expandafter\pgfplots@prepared@tick@pos@unpack\elem
			\expandafter\let\csname pgfplots@xtick@num@to@pos@\the\c@pgf@counta\endcsname=\pgfplots@tick
			\advance\c@pgf@counta by1
		}%
	\fi
	\def\b@pgfplots@xticknum@to@pos@lookup{1}%
}%
\def\pgfplots@gridlines@init@arc@limits@normalise#1\relax{%
	\pgfutil@in@:{#1}%
	\ifpgfutil@in@
	\else
		\edef\entry{\entry:2}%
	\fi
	%
	\pgfutil@in@{[index]}{#1}%
	\ifpgfutil@in@
		\if0\b@pgfplots@xticknum@to@pos@lookup
			\pgfplots@gridlines@init@arc@limits@init@ticknum@lookup
		\fi
		\expandafter\pgfplots@gridlines@arc@limits@unpack\entry\relax
		\pgfutil@ifundefined{pgfplots@xtick@num@to@pos@\pgfplots@arclimit}{%
			\pgfplots@warning{There is no xtick with index '\pgfplots@arclimit'. Skipping it.}%
			\let\entry=\pgfutil@empty
		}{%
			\edef\entry{\csname pgfplots@xtick@num@to@pos@\pgfplots@arclimit\endcsname:\pgfplots@arcskipeachnth}%
		}%
	\fi
}%
\def\pgfplots@gridlines@arc@limits@unpack#1:#2\relax{%
	\def\pgfplots@arclimit{#1}%
	\def\pgfplots@arcskipeachnth{#2}%
}%

% Returns the xtick position which should end the current arc.
%
% Note that arcs correspond to ygrid lines.
%
% #1 is a macro which will be filled with the result. If the result is
% empty, no restriction is imposed. Otherwise, it contains the xtick
% value at which the current arc shall end.
%
% The method relies on the 'arc limits' feature, more specifically the
% stuff prepared by \pgfplots@gridlines@init@arc@limits
\def\pgfplots@get@current@arc@limit#1{%
	\def#1{}%
	\ifx\pgfplots@arclimits\pgfutil@empty
	\else
		% \pgfplots@ticknum is defined in this context here.
		\pgfutil@ifundefined{pgfplots@tickpos@to@sortidx@\pgfplots@ticknum}{%
			\pgfplots@warning{Sorry, I can't get the current arc limit for \pgfplots@ticknum\space (seems like an internal error).}%
		}{%
			% get the sort index for the current tick (which is
			% uniquely identified by its \pgfplots@ticknum)
			\expandafter\let\expandafter\pgfplots@k\csname pgfplots@tickpos@to@sortidx@\pgfplots@ticknum\endcsname
			%
			\pgfplotslistforeachungrouped\pgfplots@arclimits@normalised\as\pgfplots@loc@TMPa{%
				\ifx#1\pgfutil@empty
					% we found no final limit so far. proceed.
					\expandafter\pgfplots@gridlines@arc@limits@unpack\pgfplots@loc@TMPa\relax
					\pgfplotsmathmodint\pgfplots@k\pgfplots@arcskipeachnth
					\ifnum\pgfmathresult=0
						\c@pgf@counta=\pgfplots@k
						\divide\c@pgf@counta by\pgfplots@arcskipeachnth\relax
						\edef\pgfplots@k{\the\c@pgf@counta}%
					\else
						% found the final limit. 
						\let#1=\pgfplots@arclimit
					\fi
				\fi
			}%
		}%
	\fi
}%

\def\pgfplots@smithchart@draw@xcircle#1{%
	\pgfplotscoordmath{default}{one}%
	\let\pgfplots@loc@TMPa=\pgfmathresult
	%
	\pgfplotscoordmath{default}{parsenumber}{#1}%
	\let\pgfplots@loc@TMPb=\pgfmathresult
	%
	\pgfplotscoordmath{default}{op}{add}{{\pgfplots@loc@TMPa}{\pgfplots@loc@TMPb}}%
	\pgfplotscoordmath{default}{op}{reciprocal}{{\pgfmathresult}}%
	\let\pgfplots@radius=\pgfmathresult
	\pgfplotscoordmath{default}{op}{multiply}{{\pgfmathresult}{\pgfplots@loc@TMPb}}%
	\pgfplotscoordmath{default}{tofixed}{\pgfmathresult}%
	\let\pgfplots@center=\pgfmathresult
	%
	\pgfplotscoordmath{default}{tofixed}{\pgfplots@radius}%
	\let\pgfplots@radius=\pgfmathresult
	%
%\message{X grid line \#\csname pgfplots@ticknum\endcsname \space at '#1': center = (\pgfplots@center,0); radius = \pgfplots@radius.^^J}%
	\pgfpathellipse
		{\pgfpointxy@orig{\pgfplots@center}{0}}
		{\pgfpointxy@orig{\pgfplots@radius}{0}}
		{\pgfpointxy@orig{0}{\pgfplots@radius}}%
}

\def\pgfplots@smithchart@draw@yarc#1{%
	\pgfplotsmath@ifapproxequal@dim{#1pt}{0pt}{0.002pt}{%
		\pgfpathmoveto{\pgfpointxy@orig{-1}{0}}%
		\pgfpathlineto{\pgfpointxy@orig{1}{0}}%
	}{%
		\pgfplotscoordmath{default}{parsenumber}{#1}%
		\let\pgfplots@y\pgfmathresult
		\pgfplotscoordmath{default}{op}{reciprocal}{{\pgfmathresult}}%
		\pgfplotscoordmath{default}{tofixed}{\pgfmathresult}%
		\let\pgfplots@signedradius=\pgfmathresult
		\ifdim\pgfplots@signedradius pt<0pt
			\edef\pgfplots@radius{-\pgfplots@signedradius}%
		\else
			\let\pgfplots@radius=\pgfplots@signedradius
		\fi
		% this here is the correct, complete circle -- together
		% with a clip path, you get what you want:
		%\pgfpathellipse
		%	{\pgfpointxy@orig{1}{\pgfplots@signedradius}}
		%	{\pgfpointxy@orig{\pgfplots@signedradius}{0}}
		%	{\pgfpointxy@orig{0}{\pgfplots@signedradius}}%
		% But I only want the arc (probably stopped earlier to
		% improve qualtity of the chart)
		%
		% compute start point for the arc.
		%
		% To do so, we need to compute the intersection between
		% the circle for fixed x=0  and the circle for y=#1.
		%
		% In general, the intersection between the circle for
		% fixed x=A and fixed y=B is given by
		%
		% p + j * q = (A + j * B -1 ) / ( A + j*B +1)
		% see http://www.siart.de/lehre/tutorien.xhtml#smishort
		%
		% inserting A = 0 and B = #1 yields the result
		% p=\pgfplots@start
		% q=\pgfplots@startim
		% as follows:
		\pgfplotscoordmath{default}{one}%
		\let\pgfplots@one=\pgfmathresult
		\pgfplotscoordmath{default}{parsenumber}{-1}%
		\let\pgfplots@mone=\pgfmathresult
		\pgfplotscoordmathcomplexdivision{\pgfplots@mone}{\pgfplots@y}{\pgfplots@one}{\pgfplots@y}%
		\let\pgfplots@start=\pgfmathresult
		\let\pgfplots@startim=\pgfmathresultim
		\pgfplotscoordmath{default}{tofixed}{\pgfplots@start}%
		\let\pgfplots@start=\pgfmathresult
		\pgfplotscoordmath{default}{tofixed}{\pgfplots@startim}%
		\let\pgfplots@startim=\pgfmathresult
		%
		\pgfplots@compute@angle@of@point@in@circle\pgfplots@start\pgfplots@startim{1}{\pgfplots@signedradius}%
		\let\pgfplots@startangle=\pgfmathresult
		%
		%
		% compute end angle.
		\pgfplots@get@current@arc@limit\pgfplots@arc@ends@at@x@circle@value%
		%
		\ifx\pgfplots@arc@ends@at@x@circle@value\pgfutil@empty
			% Ok. There is no specific end point -- simply use the
			% (1,0) point (i.e. draw the full arc).
			%
			% The "0 degree" angle in my circles is in the direction
			% of (1,0) .
			\ifdim\pgfplots@startim pt>0pt
				% ok; this arc belongs to the upper hemisphere.
				\def\pgfplots@endangle{270}%
			\else
				% ok; this arc belongs to the lower hemisphere.
				\def\pgfplots@endangle{90}%
			\fi
		\else
			% Ok. The arc should end before it reaches the (1,0)
			% point. Determine the exact position and the
			% corresponding arc end angle.
			\pgfplotscoordmath{default}{parsenumber}{\pgfplots@arc@ends@at@x@circle@value}%
			\let\pgfplots@arc@ends@at@x@circle@value=\pgfmathresult
			\pgfplotscoordmath{default}{op}{add}{{\pgfmathresult}{\pgfplots@mone}}%
			\let\pgfplots@A@mone=\pgfmathresult
			\pgfplotscoordmath{default}{op}{add}{{\pgfplots@arc@ends@at@x@circle@value}{\pgfplots@one}}%
			\let\pgfplots@A@one=\pgfmathresult
			% oh - we should only draw a partial arc.
			% Well, then compute its end point and the
			% corresponding end angle.
			\pgfplotscoordmathcomplexdivision{\pgfplots@A@mone}{\pgfplots@y}{\pgfplots@A@one}{\pgfplots@y}%
			\let\pgfplots@end=\pgfmathresult
			\let\pgfplots@endim=\pgfmathresultim
			\pgfplotscoordmath{default}{tofixed}{\pgfplots@end}%
			\let\pgfplots@end=\pgfmathresult
			\pgfplotscoordmath{default}{tofixed}{\pgfplots@endim}%
			\let\pgfplots@endim=\pgfmathresult
			%
			\pgfplots@compute@angle@of@point@in@circle\pgfplots@end\pgfplots@endim{1}{\pgfplots@signedradius}%
			\let\pgfplots@endangle=\pgfmathresult
		\fi
		% 
		%
%\message{Y grid line  \#\csname pgfplots@ticknum\endcsname\space at '#1': center = (1,\pgfplots@signedradius); signedradius = \pgfplots@signedradius\space( start angle \pgfplots@startangle, end angle \pgfplots@endangle,  arc limit: \ifx\pgfplots@arc@ends@at@x@circle@value\pgfutil@empty NONE\else \pgfplots@arc@ends@at@x@circle@value\fi)^^J}%
		%
		% Now, compute the arc.
		%
		% first, compute the absolute x/y radii:
		\pgf@xa=\pgfplots@radius\pgf@xx
		\pgf@xb=\pgfplots@radius\pgf@yy
		\pgfpathmoveto{\pgfpointxy@orig{\pgfplots@start}{\pgfplots@startim}}%
		% note that the case startangle > endangle is
		% automatically correct; patharc handles that.
		\edef\pgfplots@loc@TMPa{{\pgfplots@startangle}{\pgfplots@endangle}{\the\pgf@xa\space and \the\pgf@xb}}%
		\expandafter\pgfpatharc\pgfplots@loc@TMPa
	}%
}

\def\pgfplots@drawgridlines@onorientedsurf@fromto@smithchart#1{%
	\if x\pgfplotspointonorientedsurfaceA
		\pgfplots@smithchart@draw@xcircle{#1}%
	\else
		\pgfplots@smithchart@draw@yarc{#1}%
	\fi
}%

% Given a circle with center point (#3,#4), we search for the angle
% of the point (#1,#2). The point is expected to be on the circle.
% The resulting angle is returned in \pgfmathresult
%
% #1  x coordinate of the point for which an angle is searched
% #2  y coordinate of the point for which an angle is searched
% #3  x coordinate of the circle's center point
% #4  y coordinate of the circle's center point
%
% All coordinates are expected in standard TeX precision (numbers
% without unit)
\def\pgfplots@compute@angle@of@point@in@circle#1#2#3#4{%
	%
	% 1. compute diff vector from center=(1,\pgfplots@signedradius) to start:
	\pgfmathsubtract@{#1}{#3}%
	\let\pgfplots@D\pgfmathresult
	\pgfmathsubtract@{#2}{#4}%
	\let\pgfplots@Dim\pgfmathresult
	%
	% 2. compute the start angle.
	% It is related to the angle between the point (1,0) and
	% diff, which, in turn is given by
	%  < (1,0), (D,Dim) > = cos(alpha) ||(D,Dim)||
	%  < (1,0), (D,Dim) > = D
	% Note that PER DEFINITION  D < 0. 
	\pgfmathveclen\pgfplots@D\pgfplots@Dim
	\pgfmathdivide@{\pgfplots@D}{\pgfmathresult}%
	\pgfmathacos@{-\pgfmathresult}% the '-' comes from D<0 .
	\let\pgfplots@tmpangle\pgfmathresult%
	% ok. tmpangle is per definition less than 180; it is the
	% smaller angle between (1,0) and (D,Dim).
	%
	% compute the angle relate to (1,0):
	\ifdim\pgfplots@Dim pt<0pt
		\pgfmathadd@{180}{\pgfplots@tmpangle}%
	\else
		\pgfmathsubtract@{180}{\pgfplots@tmpangle}%
	\fi
}%

\def\pgfplots@drawaxis@innerlines@onorientedsurf@smithchart#1#2#3{%
	\if2\csname pgfplots@#1axislinesnum\endcsname
		\draw[/pgfplots/every inner #1 axis line,%
			decorate,%
			#1discont,%
			decoration={pre length=\csname #1disstart\endcsname, post length=\csname #1disend\endcsname}]
		\pgfextra
		\csname pgfplotspointonorientedsurfaceabsetupforset#3\endcsname{\csname pgfplots@logical@ZERO@#3\endcsname}{2}%
		\if#1x%
			\pgfplotspointonorientedsurfaceabsetupfor{#2}{#1}{\pgfplotspointonorientedsurfaceN}%
			\pgfplots@drawgridlines@onorientedsurf@fromto{0}%
		\else
			\pgfpathmoveto{\pgfplotspointonorientedsurfaceab{\csname pgfplots@#1min\endcsname}{\csname pgfplots@logical@ZERO@#2\endcsname}}%
			\pgfpathlineto{\pgfplotspointonorientedsurfaceab{\csname pgfplots@#1max\endcsname}{\csname pgfplots@logical@ZERO@#2\endcsname}}%
		\fi
		\endpgfextra
		;
	\fi
}%
\def\pgfplots@drawaxis@outerlines@separate@onorientedsurf@smithchartaxis#1#2{%
	\if2\csname pgfplots@#1axislinesnum\endcsname
		% centered axis lines handled elsewhere.
	\else
	\scope[/pgfplots/every outer #1 axis line,
		#1discont,decoration={pre length=\csname #1disstart\endcsname, post length=\csname #1disend\endcsname}]
		\if#1x
			\draw decorate {
				\pgfextra
				% exchange roles of A <-> B axes:
				\pgfplotspointonorientedsurfaceabsetupfor{#2}{#1}{\pgfplotspointonorientedsurfaceN}%
				\pgfplots@drawgridlines@onorientedsurf@fromto{0}%
				\endpgfextra 
				};
		\else
			\pgfplots@ifaxisline@B@onorientedsurf@should@be@drawn{0}{%
				\draw decorate {
					\pgfextra
					% exchange roles of A <-> B axes:
					\pgfplotspointonorientedsurfaceabsetupfor{#2}{#1}{\pgfplotspointonorientedsurfaceN}%
					\pgfplots@drawgridlines@onorientedsurf@fromto{\csname pgfplots@#2min\endcsname}%
					\endpgfextra 
					};
			}{}%
			%--------------------------------------------------
			% \pgfplots@ifaxisline@B@onorientedsurf@should@be@drawn{1}{%
			% 	\draw decorate {
			% 		\pgfextra
			% 		% exchange roles of A <-> B axes:
			% 		\pgfplotspointonorientedsurfaceabsetupfor{#2}{#1}{\pgfplotspointonorientedsurfaceN}%
			% 		\pgfplots@drawgridlines@onorientedsurf@fromto{\csname pgfplots@#2max\endcsname}%
			% 		\endpgfextra 
			% 		};
			% }{}%
			%-------------------------------------------------- 
		\fi
	\endscope
	\fi
}%

\def\pgfplots@initsizes@smithchart{%
	% I copy-pasted most of this code, up to just one position where
	% I introduced the modified scaling for smithchart axes
	%----------------------------------
	% INIT.
	%
	%
	\pgfplots@xmin@reg=\pgfplots@xmin pt %
	\pgfplots@xmax@reg=\pgfplots@xmax pt %
	\pgfplots@ymin@reg=\pgfplots@ymin pt %
	\pgfplots@ymax@reg=\pgfplots@ymax pt %
	\ifpgfplots@threedim
		\pgfplots@zmin@reg=\pgfplots@zmin pt %
		\pgfplots@zmax@reg=\pgfplots@zmax pt %
	\fi
	%
	%
	%-----------------------------------------
	% PROCESS THE 'width' and 'height' options
	%-----------------------------------------
	%
	%
	\pgfkeysgetvalue{/pgfplots/view/az}{\pgfplots@view@az}%
	\pgfkeysgetvalue{/pgfplots/view/el}{\pgfplots@view@el}%
	\ifpgfplots@threedim
		\def\pgfplots@tmpZscale{1pt}%
	\else
		\def\pgfplots@tmpZscale{0pt}%
		\let\pgfplots@view@el=\pgfutil@empty
		\let\pgfplots@view@az=\pgfutil@empty
	\fi
	\ifx\pgfplots@view@az\pgfutil@empty
		%\let\pgfplots@rectangle@width=\pgfutil@empty
		%\let\pgfplots@rectangle@height=\pgfutil@empty
		\def\pgfplots@view@dir@threedim@x{0}%
		\def\pgfplots@view@dir@threedim@y{0}%
		\def\pgfplots@view@dir@threedim@z{1}%
		%
		% FIXME HERE COMES THE smithchart MODIFICATION
		%--------------------------------------------------
		% \pgfpointdiff
		% 	{\pgfplotsqpointxy{\pgfplots@xmin}{\pgfplots@ymin}}
		% 	{\pgfplotsqpointxy{\pgfplots@xmax}{\pgfplots@ymax}}%
		%-------------------------------------------------- 
		% instead of scaling from (min) (max), we only need the UPPER
		% bound -- in both, x and y directions (since we have a cycle.
		% Furthermore, we need it twice since we are interested in the
		% diameter, not the radius.
		\global\pgf@x=2 pt % radius
		\global\pgf@y=\pgf@x            % same for y
		%
		% only used temporarily in this macro to compute the correct
		% length for unit vectors:
		\edef\pgfplots@initsizes@axisdiag@x{\the\pgf@x}%
		\edef\pgfplots@initsizes@axisdiag@y{\the\pgf@y}%
		%
		\ifx\pgfplots@x\pgfutil@empty
			\ifx\pgfplots@width\pgfutil@empty
				\pgfplots@error{INTERNAL LOGIC ERROR! WIDTH NOT SET}%
			\fi
			\pgfplots@initsizes@getXscale\pgfplots@initsizes@axisdiag@x\into\pgfplots@tmpXscale
			%\ifpgfplots@scale@only@axis
			%	\let\pgfplots@rectangle@width=\pgfplots@width
			%\fi
		\else
			\def\pgfplots@tmpXscale{1}%
		\fi
		%
		\ifx\pgfplots@y\pgfutil@empty
			\ifx\pgfplots@height\pgfutil@empty
				\pgfplots@error{INTERNAL LOGIC ERROR! HEIGHT NOT SET}%
			\fi
			\pgfplots@initsizes@getYscale\pgfplots@initsizes@axisdiag@y\into\pgfplots@tmpYscale
			%\ifpgfplots@scale@only@axis
			%	\let\pgfplots@rectangle@height=\pgfplots@height
			%\fi
		\else
			\def\pgfplots@tmpYscale{1}%
		\fi
		%
		\edef\pgfplots@tmpXscale{\pgfplots@tmpXscale pt}%
		\edef\pgfplots@tmpYscale{\pgfplots@tmpYscale pt}%
		%
		\pgfplots@initsizes@setunitvector{x}{0}{\pgfplots@tmpXscale}{\pgfplots@tmp@xisaxisparallel}%
		\pgfplots@initsizes@setunitvector{y}{1}{\pgfplots@tmpYscale}{\pgfplots@tmp@yisaxisparallel}%
		\pgfplots@initsizes@setunitvector{z}{2}{\pgfplots@tmpZscale}{\pgfplots@loc@TMPc}%
	\else
		% 3D case is currently always initialised by `view':
		\let\pgfplots@x=\pgfutil@empty
		\let\pgfplots@y=\pgfutil@empty
		\let\pgfplots@z=\pgfutil@empty
		\pgfplotssetaxesfromazel{\pgfplots@view@az}{\pgfplots@view@el}{\pgfplots@tmp@xisaxisparallel}%
		\if1\pgfplots@tmp@xisaxisparallel%
			\def\pgfplots@tmp@yisaxisparallel{1}%
		\fi
	\fi
%\message{Pgfplots debug: initialised unit vectors to x=(\the\pgf@xx,\the\pgf@xy), y=(\the\pgf@yx,\the\pgf@yy), z=(\the\pgf@zx,\the\pgf@zy). }%
	%
	\let\pgfplotsmathfloatviewdepthxyz@=\pgfplotsmathfloatviewdepthxyz@infigure
	\let\pgfplotsmathviewdepthxyz@=\pgfplotsmathviewdepthxyz@infigure
	%
	\pgfplotsmath@ifzero{\pgfplots@x@veclength}{\pgfplots@hide@xtrue\pgfplots@shownothingof@xtrue}{}%
	\pgfplotsmath@ifzero{\pgfplots@y@veclength}{\pgfplots@hide@ytrue\pgfplots@shownothingof@ytrue}{}%
	\ifpgfplots@threedim
		\pgfplotsmath@ifzero{\pgfplots@z@veclength}{\pgfplots@hide@ztrue\pgfplots@shownothingof@ztrue}{}%
	\else
		\if1\pgfplots@tmp@xisaxisparallel%
			\if1\pgfplots@tmp@yisaxisparallel%
				% Optimize for axis-parallel case!
				% puh. Did not make any measureable difference!? Ok...
				\let\pgfplotsqpointxy=\pgfplotsqpointxy@orthogonal
			\fi
		\fi
	\fi
	% 
	% FIXME : unit vector ratio / axis equal
	%
}

\def\pgfplots@handle@invalid@range@defaultlimits@smithchart{%
	\def\pgfplots@loc@TMPa{0}%
	\ifx\pgfplots@xmin\pgfplots@invalidrange@xmin
		\def\pgfplots@loc@TMPa{1}%
	\fi
	\ifx\pgfplots@xmax\pgfplots@invalidrange@xmax
		\def\pgfplots@loc@TMPa{1}%
	\fi
	\if\pgfplots@loc@TMPa1%
		\pgfplotscoordmath{x}{parsenumber}{0}%
		\global\let\pgfplots@xmin=\pgfmathresult
		\pgfplotscoordmath{x}{parsenumber}{360}%
		\global\let\pgfplots@xmax=\pgfmathresult
		\global\let\pgfplots@data@xmin=\pgfplots@xmin
		\global\let\pgfplots@data@xmax=\pgfplots@xmax
	\fi
	%
	\pgfplotscoordmath{y}{parsenumber}{0}%
	\global\let\pgfplots@ymin=\pgfmathresult
	\pgfplotscoordmath{y}{one}%
	\global\let\pgfplots@ymax=\pgfmathresult
	\global\let\pgfplots@data@ymin=\pgfplots@ymin
	\global\let\pgfplots@data@ymax=\pgfplots@ymax
}

\let\pgfplots@show@ticklabel@@orig=\pgfplots@show@ticklabel@
\def\pgfplots@show@ticklabel@@smithchart#1#2{%
	\def\pgfmathresult{#2}%
	\if#1x%
		\ifdim#2pt>360pt
			\pgfmath@basic@mod@{#2}{360}%
		\fi
	\fi
	\def\pgfplots@loc@TMPa{\pgfplots@show@ticklabel@@orig{#1}}%
	\expandafter\pgfplots@loc@TMPa\expandafter{\pgfmathresult}%
}%

\def\pgfplots@xtick@check@tickshow@smithchart{%
	\pgfplots@tickshowtrue
}
\def\pgfplots@ytick@check@tickshow@smithchart{%
	\pgfplots@tickshowtrue
}

\let\pgfplots@limits@ready@orig=\pgfplots@limits@ready
\def\pgfplots@limits@ready@smithchart{%
	\pgfplots@limits@ready@orig
	%
	% Avoid tick labels at upper *and* lower angle range if both are the
	% same:
	\pgfmath@basic@sin@{\pgfplots@xmin}%
	\let\pgfplots@loc@TMPa=\pgfmathresult
	\pgfmath@basic@sin@{\pgfplots@xmax}%
	\pgfplotsmath@ifapproxequal@dim
		{\pgfmathresult pt}{\pgfplots@loc@TMPa pt}%
		{0.002pt}
		{%
			\def\pgfplots@xtick@disable@last@tick{1}%
		}{%
		}%
}%
\endinput
