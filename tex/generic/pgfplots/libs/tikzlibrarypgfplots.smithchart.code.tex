%--------------------------------------------
%
% Package pgfplots, library for smith charts.
%
% Copyright 2010 by Christian Feuersänger.
%
% This program is free software: you can redistribute it and/or modify
% it under the terms of the GNU General Public License as published by
% the Free Software Foundation, either version 3 of the License, or
% (at your option) any later version.
% 
% This program is distributed in the hope that it will be useful,
% but WITHOUT ANY WARRANTY; without even the implied warranty of
% MERCHANTABILITY or FITNESS FOR A PARTICULAR PURPOSE.  See the
% GNU General Public License for more details.
% 
% You should have received a copy of the GNU General Public License
% along with this program.  If not, see <http://www.gnu.org/licenses/>.
%
%
% - CARTESIAN INPUT
% - tick/grid coordinates are from 
%     H := [0,infty] x [-infty,infty]
% - input coordinates can be either from H or (perhaps preferred) from
%   the unit circle.
%   this "preferred" needs to be discussed.
% - the transformed data range is the unit circle (or a sequeezed variant)
% - in order to compute limits etc., I should accept data in H. this
%   should simplify the logic to determine ticks etc considerably.
%   problem: this transformation appears to be quite difficult (?)
%   	-> r(z) = (z-1)/(z+1)
%   in complex arithmetics (but the G-tutorial.pdf says something
%   about these circle equations!?)
%
%   www.amanogawa.com/archive/docs/G-tutorial.pdf
%
%
% Idea: 
%   - work on H
%   - transform whereever necessary
%   - implement all pgfplots wrinkles in analogy to polar axes
%   - provide support for normalized input coords (combined with
%   untransformed limits or something like that)

\pgfplotsdefineaxistype{smithchart}{%
	\pgfplots@smithchartaxis@activate
}%

\newif\ifpgfplots@smithchartaxis@rellimits

\pgfplotsset{
	/pgfplots/every smithchart axis/.style={
		grid=major,
		xmin=0
	},
}

\def\pgfplots@smithchartaxis@activate{%
	\let\pgfplotsqpointxy@cart=\pgfplotsqpointxy
	\let\pgfplotsqpointxy=\pgfplotsqpointxy@smithchartaxis
	\let\pgfplotsqpointxy@orthogonal=\pgfplotsqpointxy
	\def\pgfplotsqpointxyz##1##2##3{\pgfplotsqpointxy{##1}{##2}}% FIXME
	\let\pgfplotspointouternormalvectorofaxis@=\pgfplotspointouternormalvectorofaxis@smithchartaxis
	\def\pgfplotspointouternormalvectorofaxis@ifdependson@v##1##2##3{##2}%
	\def\pgfplots@drawticklines@INSTALLCLIP@onorientedsurf##1{}%
	\let\pgfplots@drawgridlines@INSTALLCLIP@onorientedsurf=\pgfplots@drawgridlines@INSTALLCLIP@onorientedsurf@smithchartaxis
	\let\pgfplots@clippath@prepare@for@axistype=\pgfplots@clippath@prepare@for@axistype@smithchartaxis
	\let\pgfplots@handle@invalid@range@defaultlimits=\pgfplots@handle@invalid@range@defaultlimits@smithchart%
	\let\pgfplotspointonorientedsurfaceabwithbshift=\pgfplotspointonorientedsurfaceabwithbshift@smithchartaxis
	\let\pgfplots@drawgridlines@onorientedsurf@fromto=\pgfplots@drawgridlines@onorientedsurf@fromto@smithchart
	\let\pgfplots@drawaxis@outerlines@separate@onorientedsurf=\pgfplots@drawaxis@outerlines@separate@onorientedsurf@smithchartaxis
	\let\pgfplotspoint@initialisation@axes=\pgfplotspoint@initialisation@axes@smithchart%
	%\let\pgfplotspoint@initialisation@units=\pgfplotspoint@initialisation@units@smithchart
	\def\axisdefaultheight{\axisdefaultwidth}%
	\let\pgfplots@initsizes=\pgfplots@initsizes@smithchart
	%\let\pgfplots@limits@ready=\pgfplots@limits@ready@smithchart
	%\let\pgfplots@show@ticklabel@=\pgfplots@show@ticklabel@@smithchart
	%\def\pgfplots@xtick@disable@last@tick{0}%
	%\let\pgfplots@xtick@check@tickshow=\pgfplots@xtick@check@tickshow@smithchart%
	\let\pgfplots@set@options@sanitize=\relax
	\let\pgfplots@set@options@sanitizemode=\relax
	\let\pgfplotscoordmathnotifydatascalesetfor=\pgfplotscoordmathnotifydatascalesetfor@smithchart
	%
	% this here is set *before* 'every smithchart axis' is invoked.
	\pgfplotsset{
		disabledatascaling,
	}%
	\expandafter\def\expandafter\pgfplots@notify@options@are@set\expandafter{%
		\pgfplots@notify@options@are@set
		\pgfplotsset{separate axis lines}%
	}%
	\def\pgfplots@xticklabel@pos{}%
	\def\pgfplots@yticklabel@pos{}%
	\def\pgfplots@zticklabel@pos{}%
	\def\pgfplots@init@ticklabelaxisspecfor##1##2{}%
	\def\pgfplots@init@ticklabelaxisspec@twodim@for##1##2{}%
	\def\pgfplots@xticklabelaxisspec{v10}% FIXME this here needs improvement. I need the 'y=0' as axis line (or at least a different y line)!
	\def\pgfplots@yticklabelaxisspec{0v0}%
	\def\pgfplots@zticklabelaxisspec{00v}%
	%
	% cartesian cs
	\tikzdeclarecoordinatesystem{cartesian}{\edef\pgfplots@loc@TMPa{##1}\expandafter\pgfplotspointcartesian@\pgfplots@loc@TMPa\pgfplots@coord@end}%
	%
}%

\def\pgfplotspointcartesian@#1,#2\pgfplots@coord@end{%
	\pgfpointxy@orig{#1}{#2}%
}%
\def\pgfplotscoordmathnotifydatascalesetfor@smithchart#1{%
	\def\pgfplotscoordmathnotifydatascalesetfor##1{}%
	\edef\pgfplotscoordmathnotifydatascalesetfor@{#1}%
	\def\pgfplotscoordmathnotifydatascalesetfor@@{x}%
	\ifx\pgfplotscoordmathnotifydatascalesetfor@@\pgfplotscoordmathnotifydatascalesetfor@
		\pgfplotscoordmath{#1}{datascaletrafo set params}{0}{0}%
	\else
		\def\pgfplotscoordmathnotifydatascalesetfor@@{y}%
		\ifx\pgfplotscoordmathnotifydatascalesetfor@@\pgfplotscoordmathnotifydatascalesetfor@
			\pgfplotscoordmath{#1}{datascaletrafo set shift}{0}%
		\fi
	\fi
	\let\pgfplotscoordmathnotifydatascalesetfor=\pgfplotscoordmathnotifydatascalesetfor@smithchart
}%

% #1: the "a" value on the oriented surf
% #2: the "b" value. 
% #3: the shift along the normal.
%
\def\pgfplotspointonorientedsurfaceabwithbshift@smithchartaxis#1#2#3{%
	% implement the shift in "b" direction explicitly:
	\pgfpointadd
		{\pgfplotspointonorientedsurfaceab{#1}{#2}}%
		{%
			\pgfplotspointonorientedsurfaceabtolinespec{v}{0}% FIXME
			\afterassignment\pgfplots@gobble@until@relax
			\pgf@xa=-#3\relax
			\edef\pgfplots@shift@no@unit{\pgf@sys@tonumber\pgf@xa}%
			\pgfqpointscale
				{\pgfplots@shift@no@unit}
				{\expandafter\pgfplotspointouternormalvectorofaxis\expandafter{\pgfplotsretval}}%
		}%
}


\def\pgfplotsqpointxy@smithchartaxis#1#2{%
	\pgf@process{%
		% compute rx + j* ry = (#1 + j * #2 -1) / (#1 + j*#2 + 1)
		%
		% I write
		% #1 - 1 + j * #2 = A + j * B
		% 1 + #1 + j * #2 = C + j * D
		%
		% -> rx + j * ry = (A C + B D  + j (B C - A D) ) / (C^2 + D^2)
		\pgfplotscoordmath{default}{parsenumber}{#1}%
		\let\pgfplots@x=\pgfmathresult
		%
		\pgfplotscoordmath{default}{parsenumber}{#2}%
		\let\pgfplots@D=\pgfmathresult
		%
		\pgfplotscoordmath{default}{parsenumber}{1}%
		\let\pgfplots@one=\pgfmathresult
		%
		\pgfplotscoordmath{default}{op}{add}{{\pgfplots@one}{\pgfplots@x}}%
		\let\pgfplots@C=\pgfmathresult
		%
		\pgfplotscoordmath{default}{op}{subtract}{{\pgfplots@x}{\pgfplots@one}}%
		\let\pgfplots@A=\pgfmathresult
		%
		\let\pgfplots@B=\pgfplots@D
		%
		% ok, compute it:
		\pgfplotscoordmath{default}{op}{multiply}{{\pgfplots@A}{\pgfplots@C}}%
		\let\pgfplots@AC=\pgfmathresult
		\pgfplotscoordmath{default}{op}{multiply}{{\pgfplots@A}{\pgfplots@D}}%
		\let\pgfplots@AD=\pgfmathresult
		\pgfplotscoordmath{default}{op}{multiply}{{\pgfplots@B}{\pgfplots@D}}%
		\let\pgfplots@BD=\pgfmathresult
		\pgfplotscoordmath{default}{op}{multiply}{{\pgfplots@B}{\pgfplots@C}}%
		\let\pgfplots@BC=\pgfmathresult
		%
		\pgfplotscoordmath{default}{op}{multiply}{{\pgfplots@C}{\pgfplots@C}}%
		\let\pgfplots@CC=\pgfmathresult
		\pgfplotscoordmath{default}{op}{multiply}{{\pgfplots@D}{\pgfplots@D}}%
		\let\pgfplots@DD=\pgfmathresult
		\pgfplotscoordmath{default}{op}{add}{{\pgfplots@CC}{\pgfplots@DD}}%
		\pgfplotscoordmath{default}{op}{reciprocal}{{\pgfmathresult}}%
		\let\pgfplots@scale=\pgfmathresult
		%
		%
		\pgfplotscoordmath{default}{op}{add}{{\pgfplots@AC}{\pgfplots@BD}}%
		\pgfplotscoordmath{default}{op}{multiply}{{\pgfmathresult}{\pgfplots@scale}}%
		\pgfplotscoordmath{default}{tofixed}{\pgfmathresult}%
		\let\pgfplots@x=\pgfmathresult
		%
		\pgfplotscoordmath{default}{op}{subtract}{{\pgfplots@BC}{\pgfplots@AD}}%
		\pgfplotscoordmath{default}{op}{multiply}{{\pgfmathresult}{\pgfplots@scale}}%
		\pgfplotscoordmath{default}{tofixed}{\pgfmathresult}%
		\let\pgfplots@y=\pgfmathresult
		%
		\pgfqpointxy@orig\pgfplots@x\pgfplots@y
%\message{pgfplotsqpointxy{#1}{#2} ---> (\pgfplots@x,\pgfplots@y) ---> (\the\pgf@x,\the\pgf@y)}%
	}%
}%
\def\pgfplots@clippath@prepare@for@axistype@smithchartaxis{%
	\def\pgfplots@clippath@install##1{%
		\pgfpathellipse
			{\pgfpointxy@orig{0}{0}}
			{\pgfpointxy@orig{1}{0}}
			{\pgfpointxy@orig{0}{1}}%
		\pgfplots@clippath@use@{##1}%
	}%
}%

\def\pgfplotspointouternormalvectorofaxis@smithchartaxis#1#2#3\relax{%
	\if v#1%
		\pgfqpoint{0pt}{1pt}%
	\else
		\if v#2%
			\pgfplotspointouternormalvectorofaxisgetv{#1#2#3}%
			\ifx\pgfplotsretval\pgfutil@empty
				\def\pgfplotsretval{0}%
			\fi
			\pgfpointdiff
				{\pgfpointxy@orig{0}{0}}%
				{\pgfplotsqpointxy{0}{\pgfplotsretval}}%
			\pgfpointnormalised{}%
		\else
			\pgfqpoint{0pt}{1pt}%
		\fi
	\fi
	\pgf@process{}%
	\endgroup
}%

\def\pgfplotspoint@initialisation@axes@smithchart{%
	\begingroup
	\pgfplotsqpointxy{\pgfplots@xmin}{\pgfplots@ymin}%
	\xdef\pgfplotspointminminmin{\noexpand\pgf@x=\the\pgf@x\space\noexpand\pgf@y=\the\pgf@y\space}%
	%
	% the "x" axis is the diameter of the circle (for fixed y=0)
	\pgf@x=2\pgf@xx
	\pgf@y=0pt
	\xdef\pgfplotspointxaxis{\noexpand\pgf@x=\the\pgf@x\space\noexpand\pgf@y=\the\pgf@y\space}%
	\pgfmathveclen{\pgf@x}{\pgf@y}%
	\xdef\pgfplotspointxaxislength{\pgfmathresult pt}%
	%
	\pgfplotsqpointxy{\pgfplots@xmax}{\pgfplots@ymax}%
	\xdef\pgfplotspointyaxis{\noexpand\pgf@x=\the\pgf@x\space\noexpand\pgf@y=\the\pgf@y\space}%
	%
	% the length of the "y" axis is 2*pi*r (for fixed x=0, the outer
	% circle).
	% The radius is the length of (0,1) which is (0pt,\pgf@xx1):
	\pgfmath@basic@multiply@{\pgf@sys@tonumber\pgf@xx}{1}%
	\pgfmathmultiply@{\pgfmathresult}{6.28318530717959}% 2*pi * r
	\xdef\pgfplotspointyaxislength{\pgfmathresult pt}%
	%
	\global\let\pgfplotspointzaxis=\pgfpointorigin
	\gdef\pgfplotspointzaxislength{0pt}%
	\endgroup
	%
}
\let\pgfplotspoint@initialisation@units@orig=\pgfplotspoint@initialisation@units
\def\pgfplotspoint@initialisation@units@smithchart{%
	\pgfplotspoint@initialisation@units@orig
	\def\pgfplotspointunitx{%
		\pgfplotspointouternormalvectorofaxisgetv{v10}% angle varying, radius at outer pos
		\ifx\pgfplotsretval\pgfutil@empty
			\def\pgfplotsretval{0}%
		\fi
		\pgfmath@basic@sin@{\pgfplotsretval}%
		\pgf@x=-\pgfmathresult pt
		\pgfmath@basic@cos@{\pgfplotsretval}%
		\pgf@y=\pgfmathresult pt
	}%
	\def\pgfplotsunitxlength{1}%
	\def\pgfplotsunitxinvlength{1}%
}%

\def\pgfplots@drawgridlines@INSTALLCLIP@onorientedsurf@smithchartaxis#1{%
	\pgfplots@clippath@install{\pgfusepath{clip}}%
}%

\def\pgfplots@drawgridlines@onorientedsurf@fromto@smithchart#1{%
	\if x\pgfplotspointonorientedsurfaceA
		\pgfplotscoordmath{default}{parsenumber}{1}%
		\let\pgfplots@loc@TMPa=\pgfmathresult
		%
		\pgfplotscoordmath{default}{parsenumber}{#1}%
		\let\pgfplots@loc@TMPb=\pgfmathresult
		%
		\pgfplotscoordmath{default}{op}{add}{{\pgfplots@loc@TMPa}{\pgfplots@loc@TMPb}}%
		\pgfplotscoordmath{default}{op}{reciprocal}{{\pgfmathresult}}%
		\let\pgfplots@radius=\pgfmathresult
		\pgfplotscoordmath{default}{op}{multiply}{{\pgfmathresult}{\pgfplots@loc@TMPb}}%
		\pgfplotscoordmath{default}{tofixed}{\pgfmathresult}%
		\let\pgfplots@center=\pgfmathresult
		%
		\pgfplotscoordmath{default}{tofixed}{\pgfplots@radius}%
		\let\pgfplots@radius=\pgfmathresult
		%
\message{X grid line at '#1': center = (\pgfplots@center,0); radius = \pgfplots@radius.^^J}%
		\pgfpathellipse
			{\pgfpointxy@orig{\pgfplots@center}{0}}
			{\pgfpointxy@orig{\pgfplots@radius}{0}}
			{\pgfpointxy@orig{0}{\pgfplots@radius}}%
	\else
		\pgfplotsmath@ifapproxequal@dim{#1pt}{0pt}{0.002pt}{%
			\pgfpathmoveto{\pgfpointxy@orig{-1}{0}}%
			\pgfpathlineto{\pgfpointxy@orig{1}{0}}%
		}{%
			\pgfplotscoordmath{default}{parsenumber}{#1}%
			\pgfplotscoordmath{default}{op}{reciprocal}{{\pgfmathresult}}%
			\pgfplotscoordmath{default}{tofixed}{\pgfmathresult}%
			\let\pgfplots@radius=\pgfmathresult
			%
\message{Y grid line at '#1': center = (1,\pgfplots@radius); radius = \pgfplots@radius.^^J}%
			\pgfpathellipse
				{\pgfpointxy@orig{1}{\pgfplots@radius}}
				{\pgfpointxy@orig{\pgfplots@radius}{0}}
				{\pgfpointxy@orig{0}{\pgfplots@radius}}%
		}%
	\fi
}%

\def\pgfplots@drawaxis@outerlines@separate@onorientedsurf@smithchartaxis#1#2{%
	\if2\csname pgfplots@#1axislinesnum\endcsname
		% centered axis lines handled elsewhere.
	\else
	\scope[/pgfplots/every outer #1 axis line,
		#1discont,decoration={pre length=\csname #1disstart\endcsname, post length=\csname #1disend\endcsname}]
		\if#1x
			\draw decorate {
				\pgfextra
				% exchange roles of A <-> B axes:
				\pgfplotspointonorientedsurfaceabsetupfor{#2}{#1}{\pgfplotspointonorientedsurfaceN}%
				\pgfplots@drawgridlines@onorientedsurf@fromto{0}%
				\endpgfextra 
				};
		\else
			\pgfplots@ifaxisline@B@onorientedsurf@should@be@drawn{0}{%
				\draw decorate {
					\pgfextra
					% exchange roles of A <-> B axes:
					\pgfplotspointonorientedsurfaceabsetupfor{#2}{#1}{\pgfplotspointonorientedsurfaceN}%
					\pgfplots@drawgridlines@onorientedsurf@fromto{\csname pgfplots@#2min\endcsname}%
					\endpgfextra 
					};
			}{}%
			%--------------------------------------------------
			% \pgfplots@ifaxisline@B@onorientedsurf@should@be@drawn{1}{%
			% 	\draw decorate {
			% 		\pgfextra
			% 		% exchange roles of A <-> B axes:
			% 		\pgfplotspointonorientedsurfaceabsetupfor{#2}{#1}{\pgfplotspointonorientedsurfaceN}%
			% 		\pgfplots@drawgridlines@onorientedsurf@fromto{\csname pgfplots@#2max\endcsname}%
			% 		\endpgfextra 
			% 		};
			% }{}%
			%-------------------------------------------------- 
		\fi
	\endscope
	\fi
}%

\def\pgfplots@initsizes@smithchart{%
	% I copy-pasted most of this code, up to just one position where
	% I introduced the modified scaling for smithchart axes
	%----------------------------------
	% INIT.
	%
	%
	\pgfplots@xmin@reg=\pgfplots@xmin pt %
	\pgfplots@xmax@reg=\pgfplots@xmax pt %
	\pgfplots@ymin@reg=\pgfplots@ymin pt %
	\pgfplots@ymax@reg=\pgfplots@ymax pt %
	\ifpgfplots@threedim
		\pgfplots@zmin@reg=\pgfplots@zmin pt %
		\pgfplots@zmax@reg=\pgfplots@zmax pt %
	\fi
	%
	%
	%-----------------------------------------
	% PROCESS THE 'width' and 'height' options
	%-----------------------------------------
	%
	%
	\pgfkeysgetvalue{/pgfplots/view/az}{\pgfplots@view@az}%
	\pgfkeysgetvalue{/pgfplots/view/el}{\pgfplots@view@el}%
	\ifpgfplots@threedim
		\def\pgfplots@tmpZscale{1pt}%
	\else
		\def\pgfplots@tmpZscale{0pt}%
		\let\pgfplots@view@el=\pgfutil@empty
		\let\pgfplots@view@az=\pgfutil@empty
	\fi
	\ifx\pgfplots@view@az\pgfutil@empty
		%\let\pgfplots@rectangle@width=\pgfutil@empty
		%\let\pgfplots@rectangle@height=\pgfutil@empty
		\def\pgfplots@view@dir@threedim@x{0}%
		\def\pgfplots@view@dir@threedim@y{0}%
		\def\pgfplots@view@dir@threedim@z{1}%
		%
		% FIXME HERE COMES THE smithchart MODIFICATION
		%--------------------------------------------------
		% \pgfpointdiff
		% 	{\pgfplotsqpointxy{\pgfplots@xmin}{\pgfplots@ymin}}
		% 	{\pgfplotsqpointxy{\pgfplots@xmax}{\pgfplots@ymax}}%
		%-------------------------------------------------- 
		% instead of scaling from (min) (max), we only need the UPPER
		% bound -- in both, x and y directions (since we have a cycle.
		% Furthermore, we need it twice since we are interested in the
		% diameter, not the radius.
		\global\pgf@x=2 pt % radius
		\global\pgf@y=\pgf@x            % same for y
		%
		% only used temporarily in this macro to compute the correct
		% length for unit vectors:
		\edef\pgfplots@initsizes@axisdiag@x{\the\pgf@x}%
		\edef\pgfplots@initsizes@axisdiag@y{\the\pgf@y}%
		%
		\ifx\pgfplots@x\pgfutil@empty
			\ifx\pgfplots@width\pgfutil@empty
				\pgfplots@error{INTERNAL LOGIC ERROR! WIDTH NOT SET}%
			\fi
			\pgfplots@initsizes@getXscale\pgfplots@initsizes@axisdiag@x\into\pgfplots@tmpXscale
			%\ifpgfplots@scale@only@axis
			%	\let\pgfplots@rectangle@width=\pgfplots@width
			%\fi
		\else
			\def\pgfplots@tmpXscale{1}%
		\fi
		%
		\ifx\pgfplots@y\pgfutil@empty
			\ifx\pgfplots@height\pgfutil@empty
				\pgfplots@error{INTERNAL LOGIC ERROR! HEIGHT NOT SET}%
			\fi
			\pgfplots@initsizes@getYscale\pgfplots@initsizes@axisdiag@y\into\pgfplots@tmpYscale
			%\ifpgfplots@scale@only@axis
			%	\let\pgfplots@rectangle@height=\pgfplots@height
			%\fi
		\else
			\def\pgfplots@tmpYscale{1}%
		\fi
		%
		\edef\pgfplots@tmpXscale{\pgfplots@tmpXscale pt}%
		\edef\pgfplots@tmpYscale{\pgfplots@tmpYscale pt}%
		%
		\pgfplots@initsizes@setunitvector{x}{0}{\pgfplots@tmpXscale}{\pgfplots@tmp@xisaxisparallel}%
		\pgfplots@initsizes@setunitvector{y}{1}{\pgfplots@tmpYscale}{\pgfplots@tmp@yisaxisparallel}%
		\pgfplots@initsizes@setunitvector{z}{2}{\pgfplots@tmpZscale}{\pgfplots@loc@TMPc}%
	\else
		% 3D case is currently always initialised by `view':
		\let\pgfplots@x=\pgfutil@empty
		\let\pgfplots@y=\pgfutil@empty
		\let\pgfplots@z=\pgfutil@empty
		\pgfplotssetaxesfromazel{\pgfplots@view@az}{\pgfplots@view@el}{\pgfplots@tmp@xisaxisparallel}%
		\if1\pgfplots@tmp@xisaxisparallel%
			\def\pgfplots@tmp@yisaxisparallel{1}%
		\fi
	\fi
\message{Pgfplots debug: initialised unit vectors to x=(\the\pgf@xx,\the\pgf@xy), y=(\the\pgf@yx,\the\pgf@yy), z=(\the\pgf@zx,\the\pgf@zy). }%
	%
	\let\pgfplotsmathfloatviewdepthxyz@=\pgfplotsmathfloatviewdepthxyz@infigure
	\let\pgfplotsmathviewdepthxyz@=\pgfplotsmathviewdepthxyz@infigure
	%
	\pgfplotsmath@ifzero{\pgfplots@x@veclength}{\pgfplots@hide@xtrue\pgfplots@shownothingof@xtrue}{}%
	\pgfplotsmath@ifzero{\pgfplots@y@veclength}{\pgfplots@hide@ytrue\pgfplots@shownothingof@ytrue}{}%
	\ifpgfplots@threedim
		\pgfplotsmath@ifzero{\pgfplots@z@veclength}{\pgfplots@hide@ztrue\pgfplots@shownothingof@ztrue}{}%
	\else
		\if1\pgfplots@tmp@xisaxisparallel%
			\if1\pgfplots@tmp@yisaxisparallel%
				% Optimize for axis-parallel case!
				% puh. Did not make any measureable difference!? Ok...
				\let\pgfplotsqpointxy=\pgfplotsqpointxy@orthogonal
			\fi
		\fi
	\fi
	% 
	% FIXME : unit vector ratio / axis equal
	%
}

\def\pgfplots@handle@invalid@range@defaultlimits@smithchart{%
	\def\pgfplots@loc@TMPa{0}%
	\ifx\pgfplots@xmin\pgfplots@invalidrange@xmin
		\def\pgfplots@loc@TMPa{1}%
	\fi
	\ifx\pgfplots@xmax\pgfplots@invalidrange@xmax
		\def\pgfplots@loc@TMPa{1}%
	\fi
	\if\pgfplots@loc@TMPa1%
		\pgfplotscoordmath{x}{parsenumber}{0}%
		\global\let\pgfplots@xmin=\pgfmathresult
		\pgfplotscoordmath{x}{parsenumber}{360}%
		\global\let\pgfplots@xmax=\pgfmathresult
		\global\let\pgfplots@data@xmin=\pgfplots@xmin
		\global\let\pgfplots@data@xmax=\pgfplots@xmax
	\fi
	%
	\pgfplotscoordmath{y}{parsenumber}{0}%
	\global\let\pgfplots@ymin=\pgfmathresult
	\pgfplotscoordmath{y}{parsenumber}{1}%
	\global\let\pgfplots@ymax=\pgfmathresult
	\global\let\pgfplots@data@ymin=\pgfplots@ymin
	\global\let\pgfplots@data@ymax=\pgfplots@ymax
}

\let\pgfplots@show@ticklabel@@orig=\pgfplots@show@ticklabel@
\def\pgfplots@show@ticklabel@@smithchart#1#2{%
	\def\pgfmathresult{#2}%
	\if#1x%
		\ifdim#2pt>360pt
			\pgfmath@basic@mod@{#2}{360}%
		\fi
	\fi
	\def\pgfplots@loc@TMPa{\pgfplots@show@ticklabel@@orig{#1}}%
	\expandafter\pgfplots@loc@TMPa\expandafter{\pgfmathresult}%
}%

\let\pgfplots@xtick@check@tickshow@orig=\pgfplots@xtick@check@tickshow{%
\def\pgfplots@xtick@check@tickshow@smithchart{%
	\if1\pgfplots@xtick@disable@last@tick
		\pgfplotsmath@ifapproxequal@dim
			{\pgfplots@tmpa}{\pgfplots@xmax@reg}
			{\pgfplots@loc@tick@placement@tolerance}
			{%
				\pgfplots@tickshowfalse
			}{%
				\pgfplots@xtick@check@tickshow@orig
			}%
	\else
		\pgfplots@xtick@check@tickshow@orig
	\fi
}

\let\pgfplots@limits@ready@orig=\pgfplots@limits@ready
\def\pgfplots@limits@ready@smithchart{%
	\pgfplots@limits@ready@orig
	%
	% Avoid tick labels at upper *and* lower angle range if both are the
	% same:
	\pgfmath@basic@sin@{\pgfplots@xmin}%
	\let\pgfplots@loc@TMPa=\pgfmathresult
	\pgfmath@basic@sin@{\pgfplots@xmax}%
	\pgfplotsmath@ifapproxequal@dim
		{\pgfmathresult pt}{\pgfplots@loc@TMPa pt}%
		{0.002pt}
		{%
			\def\pgfplots@xtick@disable@last@tick{1}%
		}{%
		}%
}%
\endinput
