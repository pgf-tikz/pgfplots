%--------------------------------------------
%
% Package pgfplots
%
% Provides a user-friendly interface to create function plots (normal
% plots, semi-logplots and double-logplots).
% 
% It is based on Till Tantau's PGF package.
%
% Copyright 2007-2015 by Christian Feuersänger.
%
% This program is free software: you can redistribute it and/or modify
% it under the terms of the GNU General Public License as published by
% the Free Software Foundation, either version 3 of the License, or
% (at your option) any later version.
% 
% This program is distributed in the hope that it will be useful,
% but WITHOUT ANY WARRANTY; without even the implied warranty of
% MERCHANTABILITY or FITNESS FOR A PARTICULAR PURPOSE.  See the
% GNU General Public License for more details.
% 
% You should have received a copy of the GNU General Public License
% along with this program.  If not, see <http://www.gnu.org/licenses/>.
%
%--------------------------------------------

% This file essentially belongs to the 'mesh' plot handler.
% It implements the (rather involved) feature 'mesh input=image'.
%
% The 'mesh' plot handler has a "visualization pipeline" as part of
% its visualization:
%
% if it is active, the coordinate stream used for the visualization
% (\pgfplotsplothandlermesh@stream) invokes
% \pgfplotsplothandlermesh@PIPE@DECODE as first item. This, in turn,
% fires other elements of the mesh plot pipeline, but its most
% important task is to assemble patches while it goes. That's where
% 'mesh input=image' hooks in: it simply generates special patches.

% this is part of the implementation of mesh input=image
% the pgf@plotstreampoint routine for the first scan line:
\def\pgfplotsplothandlermesh@PIPE@DECODE@image@fillscanline#1{%
	\pgfplotsplothandlermesh@PIPE@DECODE@matrix@fillscanline@common{#1}%
	\ifnum\c@pgfplots@scanlineindex=\pgfplotsplothandlermesh@scanlinelength\relax
		% second line!
		\let\pgfplotsplothandlermesh@PIPE@DECODE=\pgfplotsplothandlermesh@PIPE@DECODE@image@secondscanline%%
		\c@pgfplots@scanlineindex=0
		\def\c@pgfplotsplothandlermesh@image@nextrownum{2}%
	\fi
}%

% IDEA: 
%
% 1. The first scanline is merely buffered and collected into the deque 'lastscanline' (just as any mesh plot).
% 2. The second scanline 
%  - generates cells for the first scanline _and_
%  - appends the computed intermediate vertices to the deque 'lastinterpscanline'.
%  - during the sweep, a single cell of the first scanline is
%  generated
%  - only the _last_ element of the scanline generates two cells
% 3. All scanlines after the second 
%  - pop the intermediate vertices from the deque 
%  - use them to construct cells for the previous scanline
%  - append their computed intermediate vertices to the deque 'lastinterpscanline'.
%  - during the sweep, a single cell of the first scanline is
%  generated
%  - only the _last_ element of the scanline generates two cells (just as in step 2)
% 4. the last scanline is special in that it generates cells for the
% previous scanline _and_ the last scanline.
%
% The "computed intermediate vertices" are mean values and
% extrapolation points. My idea is to make this special cell-based
% mesh more flexible than a simple imagesc, hoping that this pays off.
\def\pgfplotsplothandlermesh@PIPE@DECODE@image@secondscanline#1{%
	\pgfplotsplothandlermesh@PIPE@DECODE@matrix@collect@and@callback{#1}{%
		%
		% create cell:
		\pgfplotsplothandlermesh@PIPE@DECODE@image@sweep@create@cell{1}%
		%
		\advance\c@pgfplots@scanlineindex by1
		\ifnum\c@pgfplots@scanlineindex=\pgfplotsplothandlermesh@scanlinelength\relax
			% ok, we have to extrapolate the last column.
			\pgfplotsplothandlermesh@PIPE@DECODE@image@sweep@create@cell@lastcolumn{1}%
		\fi
		\advance\c@pgfplots@scanlineindex by-1
	}{%
		% if scanline is completed, switch states:
		\let\pgfplotsplothandlermesh@PIPE@DECODE=\pgfplotsplothandlermesh@PIPE@DECODE@image@after@second%%
		\c@pgfplots@scanlineindex=0
		\pgfplotsutil@advancestringcounter\c@pgfplotsplothandlermesh@image@nextrownum
	}%
}%
\def\pgfplotsplothandlermesh@PIPE@DECODE@image@after@second#1{%
	\pgfplotsplothandlermesh@PIPE@DECODE@matrix@collect@and@callback{#1}{%
		\pgfplotsplothandlermesh@PIPE@DECODE@image@sweep@create@cell{0}%
		%
		\advance\c@pgfplots@scanlineindex by1
		\ifnum\c@pgfplots@scanlineindex=\pgfplotsplothandlermesh@scanlinelength\relax
			% ok, we have to extrapolate the last column.
			\pgfplotsplothandlermesh@PIPE@DECODE@image@sweep@create@cell@lastcolumn{0}%
		\fi
		\advance\c@pgfplots@scanlineindex by-1
	}{%
		% if scanline is completed: advance to next
		\c@pgfplots@scanlineindex=0
		\pgfplotsutil@advancestringcounter\c@pgfplotsplothandlermesh@image@nextrownum
	}%
}%

% To be used within the DECODE "loop" only! It assembles temporary
% variables for the next iteration (and consumes it).
%
% #1: '1' if topmost row is to be extrapolated, '0' if it should be
% acquired from the datastructure 'lastinterpscanline'
\def\pgfplotsplothandlermesh@PIPE@DECODE@image@sweep@create@cell#1{%
		\let\pgfplotsplothandlermesh@patchclass\pgfplotsplothandlermesh@patchclass@input
		%
		% we have to compute the four corners by means of
		% averaging/extrapolation.
		%
		% We place the rectangle around (i-1,j-1), i.e. imjm
		% Notation: the four corners are aa, ab, ba, and bb.
		%  aa -----  ab
		%  |         |
		%  |         |
		%  |   imjm  |          
		%  |         |     
		%  |         |
		%  ba ------ bb
		%
		% bb is the average of the four cell mid points of our
		% scanline, the others are acquired from previous iterations
		% or from extrapolation schemes.
		%
		% compute bb:
		\pgfplotspatchvertexaccumstart
		\expandafter\pgfplotspatchvertexadd\pgfplotsplothandlermesh@im@jm\times{0.25}%
		\expandafter\pgfplotspatchvertexadd\pgfplotsplothandlermesh@im@j\times{0.25}%
		\expandafter\pgfplotspatchvertexadd\pgfplotsplothandlermesh@i@j\times{0.25}%
		\expandafter\pgfplotspatchvertexadd\pgfplotsplothandlermesh@i@jm\times{0.25}%
		\pgfplotspatchvertexfinish\pgfplotsplothandlermesh@bb
		%
		%
		\if1#1%
			% Ah -- this here is the very first row! That means we
			% have to extrapolate the topmost coordinates.
			%
			% Apply extrapolation for ab:
			%
			% compute ab = bb + 2* ( 1/2 (imj + imjm) - bb)
			% which resolves to the following:
			\pgfplotspatchvertexaccumstart
			\expandafter\pgfplotspatchvertexadd\pgfplotsplothandlermesh@bb\times{-1.0}%
			\expandafter\pgfplotspatchvertexadd\pgfplotsplothandlermesh@im@j\times{1.0}%
			\expandafter\pgfplotspatchvertexadd\pgfplotsplothandlermesh@im@jm\times{1.0}%
			\pgfplotspatchvertexfinish\pgfplotsplothandlermesh@ab
		\else
			\ifnum\c@pgfplots@scanlineindex=1
				\pgfplotsdequepopfront{lastinterpscanline}\to\pgfplotsplothandlermesh@aa
			\fi
			\pgfplotsdequepopfront{lastinterpscanline}\to\pgfplotsplothandlermesh@ab
		\fi
		%
		\ifnum\c@pgfplots@scanlineindex=1
			% ah, we have to apply special treatment for the leftmost
			% cell.
			%
			%
			% compute ba = bb + 2* ( 1/2 (imjm + ijm) - bb)
			% which resolves to the following:
			\pgfplotspatchvertexaccumstart
			\expandafter\pgfplotspatchvertexadd\pgfplotsplothandlermesh@bb\times{-1.0}%
			\expandafter\pgfplotspatchvertexadd\pgfplotsplothandlermesh@i@jm\times{1.0}%
			\expandafter\pgfplotspatchvertexadd\pgfplotsplothandlermesh@im@jm\times{1.0}%
			\pgfplotspatchvertexfinish\pgfplotsplothandlermesh@ba
			%
			\if1#1%
				% ok, extrapolate 'aa' as well:
				%
				% compute aa = ba + (ab - bb) 
				% which resolves to the following:
				\pgfplotspatchvertexaccumstart
				\expandafter\pgfplotspatchvertexadd\pgfplotsplothandlermesh@bb\times{-1.0}%
				\expandafter\pgfplotspatchvertexadd\pgfplotsplothandlermesh@ba\times{1.0}%
				\expandafter\pgfplotspatchvertexadd\pgfplotsplothandlermesh@ab\times{1.0}%
				\pgfplotspatchvertexfinish\pgfplotsplothandlermesh@aa
			\else
				% in this case, 'aa' has been fetched from the deque
				% (from the previous scanline)
			\fi
			%
			% as soon as we process the next scanline, we need to know the
			% interpolated values -- they will become the top row of the
			% next scanline's cells.
			\expandafter\pgfplotsdequepushback\pgfplotsplothandlermesh@ba\to{lastinterpscanline}%
		\else
			% ok, we take the left edge (ba and aa) from the previous
			% iteration (see the end of this code where the previous
			% loop iteration prepared them).
		\fi
		%
		\expandafter\pgfplotsplothandlermesh@setnextvertex\expandafter{\pgfplotsplothandlermesh@aa}%
		\expandafter\pgfplotsplothandlermesh@setnextvertex\expandafter{\pgfplotsplothandlermesh@ab}%
		\expandafter\pgfplotsplothandlermesh@setnextvertex\expandafter{\pgfplotsplothandlermesh@bb}%
		\expandafter\pgfplotsplothandlermesh@setnextvertex\expandafter{\pgfplotsplothandlermesh@ba}%
		%
		% ... and communicate this as well to the next scanline (see
		% \pgfplotsplothandlermesh@PIPE@DECODE@image@after@second)
		\expandafter\pgfplotsdequepushback\pgfplotsplothandlermesh@bb\to{lastinterpscanline}%
		%
		\ifnum\c@pgfplotsplothandlermesh@image@nextrownum=\pgfplotsplothandlermesh@numscanlines\relax
			\pgfplotsplothandlermesh@PIPE@DECODE@image@sweep@create@cell@lastrow
		\fi
		%
		% merely shift the current points such that the next loop
		% iteration can use them without modifications:
		\let\pgfplotsplothandlermesh@ba=\pgfplotsplothandlermesh@bb
		\let\pgfplotsplothandlermesh@aa=\pgfplotsplothandlermesh@ab
		%
}%

\def\pgfplotsplothandlermesh@PIPE@DECODE@image@sweep@create@cell@lastrow{%
	%
	\let\pgfplotsplothandlermesh@last@aa=\pgfplotsplothandlermesh@ba
	\let\pgfplotsplothandlermesh@last@ab=\pgfplotsplothandlermesh@bb
	%
	% compute Lbb = bb + 2 * (1/2 ( ijm + ij) - bb ). We assume that
	% we can still access 'bb', the coordinate of the current point.
	% This resolves to the following:
	\pgfplotspatchvertexaccumstart
	\expandafter\pgfplotspatchvertexadd\pgfplotsplothandlermesh@bb\times{-1.0}%
	\expandafter\pgfplotspatchvertexadd\pgfplotsplothandlermesh@i@jm\times{1.0}%
	\expandafter\pgfplotspatchvertexadd\pgfplotsplothandlermesh@i@j\times{1.0}%
	\pgfplotspatchvertexfinish\pgfplotsplothandlermesh@last@bb
	%
	\ifnum\c@pgfplots@scanlineindex=1
		% compute Lba = Lbb + Laa - Lab
		\pgfplotspatchvertexaccumstart
		\expandafter\pgfplotspatchvertexadd\pgfplotsplothandlermesh@last@bb\times{1.0}%
		\expandafter\pgfplotspatchvertexadd\pgfplotsplothandlermesh@last@aa\times{1.0}%
		\expandafter\pgfplotspatchvertexadd\pgfplotsplothandlermesh@last@ab\times{-1.0}%
		\pgfplotspatchvertexfinish\pgfplotsplothandlermesh@last@ba
	\else
		% transported from previous iteration.
	\fi
	%
	\expandafter\pgfplotsplothandlermesh@setnextvertex\expandafter{\pgfplotsplothandlermesh@last@aa}%
	\expandafter\pgfplotsplothandlermesh@setnextvertex\expandafter{\pgfplotsplothandlermesh@last@ab}%
	\expandafter\pgfplotsplothandlermesh@setnextvertex\expandafter{\pgfplotsplothandlermesh@last@bb}%
	\expandafter\pgfplotsplothandlermesh@setnextvertex\expandafter{\pgfplotsplothandlermesh@last@ba}%
	%
	\let\pgfplotsplothandlermesh@last@ba=\pgfplotsplothandlermesh@last@bb
}%

\def\pgfplotsplothandlermesh@PIPE@DECODE@image@sweep@create@cell@lastrow@lastcol{%
	%
	\let\pgfplotsplothandlermesh@last@aa=\pgfplotsplothandlermesh@ba
	\let\pgfplotsplothandlermesh@last@ab=\pgfplotsplothandlermesh@bb
	%
	% I mirrored the formula for lastrow.
	\pgfplotspatchvertexaccumstart
	\expandafter\pgfplotspatchvertexadd\pgfplotsplothandlermesh@last@bb\times{1.0}%
	\expandafter\pgfplotspatchvertexadd\pgfplotsplothandlermesh@last@aa\times{-1.0}%
	\expandafter\pgfplotspatchvertexadd\pgfplotsplothandlermesh@last@ab\times{1.0}%
	\pgfplotspatchvertexfinish\pgfplotsplothandlermesh@last@bb
	%
	\expandafter\pgfplotsplothandlermesh@setnextvertex\expandafter{\pgfplotsplothandlermesh@last@aa}%
	\expandafter\pgfplotsplothandlermesh@setnextvertex\expandafter{\pgfplotsplothandlermesh@last@ab}%
	\expandafter\pgfplotsplothandlermesh@setnextvertex\expandafter{\pgfplotsplothandlermesh@last@bb}%
	\expandafter\pgfplotsplothandlermesh@setnextvertex\expandafter{\pgfplotsplothandlermesh@last@ba}%
}%

\def\pgfplotsplothandlermesh@PIPE@DECODE@image@sweep@create@cell@lastcolumn#1{%
		\let\pgfplotsplothandlermesh@patchclass\pgfplotsplothandlermesh@patchclass@input
		%
		% we have to compute the four corners by means of
		% averaging/extrapolation.
		%
		% We place the rectangle around (i-1,j), i.e. imj.
		% This is in contrast to
		% \pgfplotsplothandlermesh@PIPE@DECODE@image@sweep@create@cell
		% where we placed the rectange around (i-1,j-1).
		%
		% Notation: the four corners are aa, ab, ba, and bb.
		%  aa -----  ab
		%  |         |
		%  |         |
		%  |   imj   |          
		%  |         |     
		%  |         |
		%  ba ------ bb
		%
		% We can simply copy aa and ba from the previous cell (as in
		% the standard case). Note that this has already been prepared
		% for us, i.e. the variables have already been set by the
		% previous iteration.
		%
		% compute ab and bb by means of extrapolation:
		%
		% compute bb:
		\pgfplotspatchvertexaccumstart
		\expandafter\pgfplotspatchvertexadd\pgfplotsplothandlermesh@ba\times{-1.0}%
		\expandafter\pgfplotspatchvertexadd\pgfplotsplothandlermesh@im@j\times{1.0}%
		\expandafter\pgfplotspatchvertexadd\pgfplotsplothandlermesh@i@j\times{1.0}%
		\pgfplotspatchvertexfinish\pgfplotsplothandlermesh@bb
		%
		%
		\if1#1%
			% Ah -- this here is the very first row! That means we
			% have to extrapolate the topmost coordinates.
			%
			% Apply extrapolation for ab:
			%
			% compute ab = bb + (aa - ba)
			\pgfplotspatchvertexaccumstart
			\expandafter\pgfplotspatchvertexadd\pgfplotsplothandlermesh@bb\times{1.0}%
			\expandafter\pgfplotspatchvertexadd\pgfplotsplothandlermesh@aa\times{1.0}%
			\expandafter\pgfplotspatchvertexadd\pgfplotsplothandlermesh@ba\times{-1.0}%
			\pgfplotspatchvertexfinish\pgfplotsplothandlermesh@ab
		\else
			\pgfplotsdequepopfront{lastinterpscanline}\to\pgfplotsplothandlermesh@ab
		\fi
		%
		\expandafter\pgfplotsplothandlermesh@setnextvertex\expandafter{\pgfplotsplothandlermesh@aa}%
		\expandafter\pgfplotsplothandlermesh@setnextvertex\expandafter{\pgfplotsplothandlermesh@ab}%
		\expandafter\pgfplotsplothandlermesh@setnextvertex\expandafter{\pgfplotsplothandlermesh@bb}%
		\expandafter\pgfplotsplothandlermesh@setnextvertex\expandafter{\pgfplotsplothandlermesh@ba}%
		%
		% ... and communicate this as well to the next scanline (see
		% \pgfplotsplothandlermesh@PIPE@DECODE@image@after@second)
		\expandafter\pgfplotsdequepushback\pgfplotsplothandlermesh@bb\to{lastinterpscanline}%
		%
		\ifnum\c@pgfplotsplothandlermesh@image@nextrownum=\pgfplotsplothandlermesh@numscanlines\relax
			\pgfplotsplothandlermesh@PIPE@DECODE@image@sweep@create@cell@lastrow@lastcol
		\fi
}%
