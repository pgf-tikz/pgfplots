%--------------------------------------------
%
% Package pgfplots
%
% Provides a user-friendly interface to create function plots (normal
% plots, semi-logplots and double-logplots).
% 
% It is based on Till Tantau's PGF package.
%
% Copyright 2007-2013 by Christian Feuersänger.
%
% This program is free software: you can redistribute it and/or modify
% it under the terms of the GNU General Public License as published by
% the Free Software Foundation, either version 3 of the License, or
% (at your option) any later version.
% 
% This program is distributed in the hope that it will be useful,
% but WITHOUT ANY WARRANTY; without even the implied warranty of
% MERCHANTABILITY or FITNESS FOR A PARTICULAR PURPOSE.  See the
% GNU General Public License for more details.
% 
% You should have received a copy of the GNU General Public License
% along with this program.  If not, see <http://www.gnu.org/licenses/>.
%
%--------------------------------------------

\pgfkeys{
	/pgfplots/color sequence/colorspace/.initial=auto,
	/pgfplots/color sequence/every input component/.code=,
	/pgfplots/color sequence/context message/.code={\def\pgfplotscolornormalizesequence@context@message{#1}},
	/pgfplots/color sequence/context message=\pgfplotscolornormalizesequence@context@message@default,
}

% the beginning of a normalized color sequence.
%
% It is to be used like
% \pgfplotscolornormalizesequence[colorspace=rgb]
%
% \pgfplotscolornormalizesequencenext{1,1,1}
% -> \pgfplotsretval = {1,1,1} % uses default colorspace
% \pgfplotscolornormalizesequencenext{color=red}
% -> \pgfplotsretval = {1,0,0}
% \pgfplotscolornormalizesequencenext{color=blue}
% -> \pgfplotsretval = {0,0,1}
% \pgfplotscolornormalizesequencenext{rgb=1,0,0}
% -> \pgfplotsretval = {1,0,0}
% \pgfplotscolornormalizesequencenext{cmyk=1,0,0,1}
% -> \pgfplotsretval = {<converted value>}
%
% \endpgfplotscolornormalizesequence
%
% Every color provided in the sequence will be normalized by
% transforming it to the common color space and by computing the
% separate components representing the color in that space.
%
% These components are available for use in interpolation routines.
%
% #1 : keys in the /pgfplots/color sequence path
\def\pgfplotscolornormalizesequence[#1]{%
	\pgfqkeys{/pgfplots/color sequence}{#1}%
	\pgfkeysgetvalue{/pgfplots/color sequence/colorspace}\pgfplots@loc@TMPa
	\pgfkeysgetvalue{/pgfplots/color sequence/every input component/.@cmd}\pgfplotscolornormalizesequencenext@every@component
	\edef\pgfplots@loc@TMPa{\pgfplots@loc@TMPa}%
	\def\pgfplots@loc@TMPb{auto}%
	\ifx\pgfplots@loc@TMPb\pgfplots@loc@TMPa
		% colorspace=auto: same as empty.
		\let\pgfplots@loc@TMPa=\pgfutil@empty
	\fi
	\ifx\pgfplots@loc@TMPa\pgfutil@empty
		% colorspace=auto:
		\def\pgfplotscolornormalizesequence@colspace{}%
	\else
		\def\pgfplots@loc@TMPb{rgb}%
		\ifx\pgfplots@loc@TMPb\pgfplots@loc@TMPa
			% colorspace=rgb:
			\def\pgfplotscolornormalizesequence@colspace{rgb}%
		\else
			\def\pgfplots@loc@TMPb{cmyk}%
			\ifx\pgfplots@loc@TMPb\pgfplots@loc@TMPa
				% colorspace=cmyk:
				\def\pgfplotscolornormalizesequence@colspace{cmyk}%
			\else
				\pgfplots@error{Sorry, the choice 'colormap default colorspace=\pgfplots@loc@TMPa' is unknown to pgfplots, please use one of 'auto,rgb,cmyk'}%
			\fi
		\fi
	\fi
}%

% ends a normalized color sequence and returns the final colorspace
% and its number of components into \pgfplotsretval and
% \pgfplotsretvalb, respectively
\def\endpgfplotscolornormalizesequence{%
	\let\pgfplotsretval=\pgfplotscolornormalizesequence@colspace
	\pgfplotscolor@get@num@components{\pgfplotsretval}{\pgfplotsretvalb}%
}%

% accepts the next color in a sequence of colors. The color will be
% normalized and the normalized result written to \pgfplotsretval.
%
% SEE \pgfplotscolornormalizesequencenext -- it is MUCH more powerful!
%
% #1 the color space of the color.
% #2 the components of the color, separated by comma.
%
% POSTCONDITION: 
%   \pgfplotsretval contains '{<color space>}{<comma-separated-components>}'
%  -> including the curly braces.
%  
%  \pgfplotsretvalb contains the number of input components in
%  \pgfplotsretval
\def\pgfplotscolornormalizesequencenextbycomponents#1#2{%
	\edef\pgfplots@loc@TMPb{#2}%
	\def\pgfplotsretval{}%
	\c@pgf@counta=0 %
	\expandafter\pgfplotscolornormalizesequencenext@prepare@loop\pgfplots@loc@TMPb,\pgfplots@EOI,%
	%
	\edef\pgfplots@loc@TMPa{#1}%
	\ifx\pgfplots@loc@TMPa\pgfplotscolornormalizesequence@colspace
	\else
		\ifx\pgfplotscolornormalizesequence@colspace\pgfutil@empty
			\let\pgfplotscolornormalizesequence@colspace=\pgfplots@loc@TMPa
		\else
			% Oh. We need to convert the color space!
			\edef\pgfplots@loc@TMPa{{\pgfplots@loc@TMPa}{\pgfplotsretval}}%
			\expandafter\pgfutil@convertcolorspec\pgfplots@loc@TMPa
				{\pgfplotscolornormalizesequence@colspace}% target color spaces
				{\pgfplotsretval}% target macro
		\fi
	\fi
	\edef\pgfplotsretval{\pgfplotsretval}%
	\pgfplotscolor@get@num@components{\pgfplotscolornormalizesequence@colspace}{\pgfplotsretvalb}%
}

% assigns the number of color components for the given colorspace to
% \pgfplotsretval
\def\pgfplotscolorspacegetcomponents#1{%
	\pgfplotscolor@get@num@components{#1}{\pgfplotsretval}%
}%

% Defines \pgfplotsretval to contain the number of components in the
% color space.
\def\pgfplotscolornormalizesequencegetnumcomponents{%
	\ifx\pgfplotscolornormalizesequence@colspace\pgfutil@empty
		\pgfplotsthrow{invalid argument}{\pgfplots@loc@TMPa}{Cannot create color with zero components: no colorspace set for the normalized color sequence.}\pgfeov%
	\fi
	\pgfplotscolorspacegetcomponents{\pgfplotscolornormalizesequence@colspace}%
}%

\def\pgfplotscolor@get@num@components#1#2{%
	\pgfutil@ifundefined{pgfplotscolor@get@num@components@#1}{%
		\pgfplotsthrow{invalid argument}{\pgfplots@loc@TMPa}{The input color has an unsupported color space '#1'}\pgfeov%
	}{%
		\edef#2{\csname pgfplotscolor@get@num@components@#1\endcsname}%
	}%
}%

\def\pgfplotscolor@get@num@components@rgb{3}%
\def\pgfplotscolor@get@num@components@cmyk{4}%
\def\pgfplotscolor@colspacename@rgb{rgb}%
\def\pgfplotscolor@colspacename@cmyk{cmyk}%

\def\pgfplotscolornormalizesequencenext@prepare@loop#1,{%
	\def\pgfmathresult{#1}%
	\ifx\pgfmathresult\pgfplots@EOI
	\else
		\begingroup
		\pgfplotscolornormalizesequencenext@every@component{#1}\pgfeov%
		\pgfmath@smuggleone\pgfmathresult
		\endgroup
		\expandafter\pgfplotscolornormalizesequencenextbycomponents@sanitize\expandafter{\pgfmathresult}%
		\ifx\pgfplotsretval\pgfutil@empty
			\edef\pgfplotsretval{\pgfmathresult}%
		\else
			\edef\pgfplotsretval{\pgfplotsretval,\pgfmathresult}%
		\fi
		\advance\c@pgf@counta by1 %
		\expandafter\pgfplotscolornormalizesequencenext@prepare@loop
	\fi
}

% #1 the color component value
\def\pgfplotscolornormalizesequencenextbycomponents@sanitize#1{%
	\pgf@xa=#1pt %
	\ifdim\pgf@xa<-0.001pt %
		\def\pgfmathresult{0}%
		\pgfplotscolor@rangeexception{#1}%
	\fi
	\ifdim\pgf@xa>1.001pt %
		\def\pgfmathresult{1}%
		\pgfplotscolor@rangeexception{#1}%
	\fi
}

\def\pgfplotscolor@rangeexception#1{%
	\pgfplotsthrow{invalid color component}{\pgfmathresult}{Sorry, the color component value #1 (no. \the\c@pgf@counta) is out of range. The allowed range is 0 <= value  <= 1. The error occured near `\pgfplotscolornormalizesequence@context@message'}\pgfeov%
}


% Same as \pgfplotscolornormalizesequencenextbycomponents but with
% more freedom in the input.
%
% #1 some color.
% Accepted formats:
% 'color=<xcolor value>'
% 	<xcolor value> is any valid expression of the xcolor package.
%
% 'rgb=R,G,B'  
% 	R,G,B are numbers in [0,1]
%
% 'rgb255=R,G,B'
% 	R,G,B are numbers in [0,255]
%
% 'cmyk=C,M,Y,K'
% 	C, M, Y, and K are numbers in [0,1]
%
% 'cmyk255=C,M,Y,K'
% 	C, M, Y, and K are numbers in [0,255]
%
% 'gray=G'
%   G is a gray scale number in [0,1]
%
% 'C1,C2,C3'  
% 	any number of Ci can folow, each is interpreted in the default
% 	colorspace (i.e. the one set before at the beginning of the
% 	sequence)
%
%
% Example:
% \pgfplotscolornormalizesequencenext{color=blue}
% \pgfplotscolornormalizesequencenext{rgb=1,0,0}
% \pgfplotscolornormalizesequencenext{cmyk=1,0,0,1}
% \pgfplotscolornormalizesequencenext{1,0,0} (interpreted in the
% colorspace set at \pgfplotscolornormalizesequence)
%
% POSTCONDITION: see \pgfplotscolornormalizesequencenextbycomponents
\def\pgfplotscolornormalizesequencenext#1{%
	\edef\pgfplotscolornormalizesequence@context@message@default{#1}%
	\expandafter\pgfplotscolornormalizesequencenext@\pgfplotscolornormalizesequence@context@message@default\pgfplots@EOI
}
\def\pgfplotscolornormalizesequencenext@{%
	\pgfutil@ifnextchar c{%
		\pgfplotscolornormalizesequencenext@c
	}{%
		\pgfutil@ifnextchar r{%
			\pgfplotscolornormalizesequencenext@rgb
		}{%
			\pgfutil@ifnextchar g{%
				\pgfplotscolornormalizesequencenext@gray
			}{%
				\pgfplotscolornormalizesequencenext@defaultcolspace
			}%
		}%
	}%
}

\def\pgfplotscolornormalizesequencenext@gray gray=#1\pgfplots@EOI{%
	% FIXME : we might be better off with real support for a gray
	% colorspace!
	\pgfplotscolornormalizesequencenextbycomponents{rgb}{#1,#1,#1}%
}

\def\pgfplotscolornormalizesequencenext@defaultcolspace#1\pgfplots@EOI{%
	\edef\pgfplotsretval{#1}%
	\ifx\pgfplotsretval\pgfutil@empty
		% hm. Ok, simply return the empty input.
		%
		% ... and the number of components.
		\pgfplotscolor@get@num@components{\pgfplotscolornormalizesequence@colspace}{\pgfplotsretvalb}%
		%
	\else
		\ifx\pgfplotscolornormalizesequence@colspace\pgfutil@empty
			\pgfplotsthrow{invalid argument}{\pgfplots@loc@TMPa}{The input color #1 cannot be interpreted; please assign a default color space first}\pgfeov%
		\fi
		\pgfplotscolornormalizesequencenextbycomponents{\pgfplotscolornormalizesequence@colspace}{#1}%
	\fi
}
\def\pgfplotscolornormalizesequencenext@rgb rgb{%
	\pgfutil@ifnextchar2{%
		\pgfplotscolornormalizesequencenext@rgb@two@five@five
	}{%
		\pgfplotscolornormalizesequencenext@rgb@plain
	}%
}
\def\pgfplotscolornormalizesequencenext@rgb@plain=#1\pgfplots@EOI{%
	\pgfplotscolornormalizesequencenextbycomponents{rgb}{#1}%
}

\def\pgfplotscolornormalizesequencenext@rgb@two@five@five255=#1,#2,#3\pgfplots@EOI{%
	\pgfplots@createcolormap@grabrgb@two@five@five@rescale{#1}%
	\let\pgfplots@loc@TMPa=\pgfmathresult
	\pgfplots@createcolormap@grabrgb@two@five@five@rescale{#2}%
	\let\pgfplots@loc@TMPb=\pgfmathresult
	\pgfplots@createcolormap@grabrgb@two@five@five@rescale{#3}%
	\edef\pgfplots@loc@TMPc{{rgb}{\pgfplots@loc@TMPa,\pgfplots@loc@TMPb,\pgfmathresult}}%
	\expandafter\pgfplotscolornormalizesequencenextbycomponents\pgfplots@loc@TMPc%
}

\def\pgfplotscolornormalizesequencenext@c c{%
	\pgfutil@ifnextchar m{%
		\pgfplotscolornormalizesequencenext@cmyk
	}{%
		\pgfplotscolornormalizesequencenext@color
	}%
}%

\def\pgfplotscolornormalizesequencenext@cmyk myk{%
	\pgfutil@ifnextchar2{%
		\pgfplotscolornormalizesequencenext@cmyk@two@five@five
	}{%
		\pgfplotscolornormalizesequencenext@cmyk@plain
	}%
}%

\def\pgfplotscolornormalizesequencenext@cmyk@plain=#1\pgfplots@EOI{%
	\pgfplotscolornormalizesequencenextbycomponents{cmyk}{#1}%
}%
\def\pgfplotscolornormalizesequencenext@cmyk@two@five@five255=#1,#2,#3,#4\pgfplots@EOI{%
	\pgfplots@createcolormap@grabcmyk@two@five@five@rescale{#1}%
	\let\pgfplots@loc@TMPa=\pgfmathresult
	\pgfplots@createcolormap@grabcmyk@two@five@five@rescale{#2}%
	\let\pgfplots@loc@TMPb=\pgfmathresult
	\pgfplots@createcolormap@grabcmyk@two@five@five@rescale{#3}%
	\let\pgfplots@loc@TMPc=\pgfmathresult
	\pgfplots@createcolormap@grabcmyk@two@five@five@rescale{#4}%
	\edef\pgfplots@loc@TMPc{{cmyk}{\pgfplots@loc@TMPa,\pgfplots@loc@TMPb,\pgfplots@loc@TMPc,\pgfmathresult}}%
	\expandafter\pgfplotscolornormalizesequencenextbycomponents\pgfplots@loc@TMPc%
}

\def\pgfplotscolornormalizesequencenext@color olor=#1\pgfplots@EOI{%
	\pgfutil@colorlet{pgf@tempcol}{#1}%
	\pgfutil@extractcolorspec{pgf@tempcol}{\pgf@tempcolor}%
	\edef\pgfplots@loc@TMPa{\expandafter\pgfutil@firstoftwo\pgf@tempcolor}%
	\edef\pgfplots@loc@TMPb{\expandafter\pgfutil@secondoftwo\pgf@tempcolor}%
	%
	% I expect that \pgf@tempcolor is of the form
	% {<colorspace>}{<comma-separated components>}
	%
	% unfortunately, the <colorspace> has strange catcodes such that
	% we cannot easily compare it by means of \ifx. I normalize it
	% here:
	\edef\pgf@tempcolor{%
		{\csname pgfplotscolor@colspacename@\pgfplots@loc@TMPa\endcsname}%
		{\pgfplots@loc@TMPb}%
	}%
	\expandafter\pgfplotscolornormalizesequencenextbycomponents\pgf@tempcolor%
}%

% Defines \pgfplotsretval to be a "neutral" color, i.e. one in which
% all components are zero.
\def\pgfplotscolornormalizesequencezero{%
	\pgfplotscolornormalizesequencegetnumcomponents
	\c@pgf@countb=\pgfplotsretval\relax
	%
	\c@pgf@counta=1 %
	\def\pgfplotsretval{0}%
	\pgfutil@loop
	\ifnum\c@pgf@counta<\c@pgf@countb
		\edef\pgfplotsretval{\pgfplotsretval,0}%
		\advance\c@pgf@counta by1 %
	\pgfutil@repeat
}%

% Defines \pgfplotsretval to be the interpolated color #1 + #2*#3.
% #1 the first color (like 0,0,0)
% #2 a scaling factor (like 4)
% #3 the second color (like 1,0,0.5)
%
% POSTCONDITION: \pgfplotsretval contains the result
\def\pgfplotscolornormalizesequenceaddweighted#1#2#3{%
	\begingroup
		\pgfplotscolornormalizesequencegetnumcomponents
		\expandafter\global
		\expandafter\let
		\expandafter\pgfplots@glob@TMPd\csname pgfplotscolornormalizesequencegetnumcomponents@\pgfplotsretval\endcsname
	\endgroup
	\edef\pgfplots@loc@TMPa{#1:#3}%
	\expandafter\pgfplots@glob@TMPd\pgfplots@loc@TMPa\relax{#2}%
}%

\def\pgfplotscolornormalizesequenceaddweighted@single#1#2#3{%
	\pgf@xa=#1pt %
	\pgf@xb=#3pt %
	\advance\pgf@xa by#2\pgf@xb
	\edef\pgfplotsretval{\pgf@sys@tonumber\pgf@xa}%
}%

\expandafter\def\csname pgfplotscolornormalizesequencegetnumcomponents@1\endcsname#1:#2\relax#3{%
	\pgfplotscolornormalizesequenceaddweighted@single{#1}{#3}{#2}%
}%
\expandafter\def\csname pgfplotscolornormalizesequencegetnumcomponents@3\endcsname#1,#2,#3:#4,#5,#6\relax#7{%
	\pgfplotscolornormalizesequenceaddweighted@single{#1}{#7}{#4}%
	\let\pgfplots@loc@TMPa=\pgfplotsretval
	\pgfplotscolornormalizesequenceaddweighted@single{#2}{#7}{#5}%
	\let\pgfplots@loc@TMPb=\pgfplotsretval
	\pgfplotscolornormalizesequenceaddweighted@single{#3}{#7}{#6}%
	\let\pgfplots@loc@TMPc=\pgfplotsretval
	\edef\pgfplotsretval{\pgfplots@loc@TMPa,\pgfplots@loc@TMPb,\pgfplots@loc@TMPc}%
}%
\expandafter\def\csname pgfplotscolornormalizesequencegetnumcomponents@4\endcsname#1,#2,#3,#4:#5,#6,#7,#8\relax#9{%
	\pgfplotscolornormalizesequenceaddweighted@single{#1}{#9}{#5}%
	\let\pgfplots@loc@TMPa=\pgfplotsretval
	\pgfplotscolornormalizesequenceaddweighted@single{#2}{#9}{#6}%
	\let\pgfplots@loc@TMPb=\pgfplotsretval
	\pgfplotscolornormalizesequenceaddweighted@single{#3}{#9}{#7}%
	\let\pgfplots@loc@TMPc=\pgfplotsretval
	\pgfplotscolornormalizesequenceaddweighted@single{#4}{#9}{#8}%
	\let\pgfplots@loc@TMPd=\pgfplotsretval
	\edef\pgfplotsretval{\pgfplots@loc@TMPa,\pgfplots@loc@TMPb,\pgfplots@loc@TMPc,\pgfplots@loc@TMPd}%
}%

\endinput
