\documentclass[a4paper]{article}
\usepackage{german}
\usepackage[utf8]{inputenc} % erlaubt direkte Nutzung von Umlauten

% \usepackage[a4paper,noheadfoot,pdftex,left=0.8cm,right=1cm,top=0.8cm,bottom=0.8cm]{geometry}
% \usepackage{color}
% \usepackage{colortbl}
% \usepackage{longtable}
% \usepackage[a4paper,colorlinks=true]{hyperref} % Querverweise etc. option pdftex,dvipdfm?
% \usepackage{graphicx}
% \usepackage[draft]{graphicx}
% \usepackage{listings} % fuer codefragmente jeder art
\usepackage[intlimits]{amsmath}
\usepackage{amssymb}
\usepackage{amsfonts}
% \usepackage{amsthm}
\usepackage{bibgerm}
%\usepackage{simplelist}
\usepackage{liststructure}


% fuer endvironment 'sidewaysfigure' bspw
% \usepackage{rotating}

%\pdfinfo {
%	/Author	(Christian Feuersaenger)
%}

%%
% My user defined commands ...
%
\usepackage{mymath}
\usepackage{myext}

% von Alex: A4-Groesse, symmetrische Seitenraender.
% \usepackage{mya4sym}


% figure einstellungen:
%\renewcommand{\topfraction}{0.85}
%\renewcommand{\textfraction}{0.06}
%\renewcommand{\floatpagefraction}{0.8}



\author{Christian Feuersänger}
%\title{}

\begin{document}
%\maketitle
{
\tracingcommands=1
%--------------------------------------------------
% \message{TEST toks0:}%
% \toks0={{a}{b}{c}{d}}%
% Der Inhalt davon ist \the\toks0
% 
% \message{TEST 2 toks0:}%
% \def\probe{{a}{b}{c}{d}}%
% \toks0={\probe}%
% Der Inhalt davon ist \the\toks0
% 
% ENDE
% 
% \message{TEST create:}%
% \listcreate{\probe}{{a}{b}{c}{d}}%
% 
% %\show\listcreate
% 
% \message{Nutze probe:}%
% Inhalt: \probe
% 
% \count0=0
% \loop
% \ifnum\count0<5
% \listpopfront\erstes\probe
% Das Erste: \erstes
% 
% Inhalt: \probe
% \vskip2cm
% \advance\count0 by1
% \repeat
% 
% \tracingmacros=2
% \message{Pushback-Test}%
% Inhalt vorher: \probe
% 
% \listpushback\probe{a}%
% Inhalt jetzt: \probe
% 
% \listpushback\probe{[eins]}%
% Inhalt jetzt: \probe
% 
% \listpushback\probe{zwei}%
% Inhalt jetzt: \probe
% 
% \message{lese wieder alles aus:}%
% lese wieder alles aus:%
% \count0=0
% \loop
% \ifnum\count0<5
% \listpopfront\erstes\probe
% Das Erste: \erstes
% 
% Inhalt: \probe
% \vskip2cm
% \advance\count0 by1
% \repeat
% 
% 
% \vskip3cm
% \message{Das hat NICHT geklappt}%
% \listcreate{\probe}{{eins}{drei}}%
% Inhalt: \probe
% 
% \listpopfront\erstes\probe
% Erstes: \erstes
% 
% Inhalt: \probe
% 
% \listpopfront\erstes\probe
% Erstes: \erstes
% 
% Inhalt: \probe
% 
%-------------------------------------------------- 


\tracingmacros=2
\message{HIER GEHTS LOS}%
\let\probe=\empty
\listpushback eins\to\probe
\listpushback [probe]\to\probe
\listpushback [probe2]\to\probe
\listpushback [probe3]\to\probe

Groesse: \listsize\probe\to{\count0}\the\count0

Inhalt:

\listforeach\probe\as\curlistelem{\curlistelem \par}

\listnew\foolist{Eins\\Zwei\\Drei\\}%
\listforeach\foolist\as\foo{Element \foo\par}%

\tracingcommands=2
\message{SELECT TEST:}%
\count1=1
Element \the\count1: \listselect\count1\of\probe\to\elem\elem

Element 3: \listselect3\of\probe\to\elem\elem

Jetzt ist count1=\the\count1
\vskip1cm

\listpopfront\probe\to\first
Erstes: \first

verbleibedne Größe: \listsize\probe\to{\count0}\the\count0

\listpopfront\probe\to\first
Erstes: \first

verbleibedne Größe: \listsize\probe\to{\count0}\the\count0


\message{NEUE LISTE:}%
NEUE LISTE:
\listnew\foolist{First Element\\Second Element\\Third Element\\}
\listnew\foolistX{Eins\\Zwei\\}
{\def\\#1{#1\par}
Inhalt: 

\foolist
}

\listconcat\third=\foolist&\foolistX

{\def\\#1{#1\par}
Ergebnis von concat: 

\third
}
\message{ENDE}%
}



\bibliographystyle{gerabbrv} %gerapali} %gerabbrv} %gerunsrt.bst} %gerabbrv}% gerplain}
% \bibliography{literatur.bib}
\end{document}


