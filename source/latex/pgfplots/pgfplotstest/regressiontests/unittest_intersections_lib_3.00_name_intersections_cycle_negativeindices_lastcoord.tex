\documentclass{article}
\usepackage{tikz,pgfplots}
\pgfplotsset{compat=newest}
\usepgfplotslibrary{fillbetween}
\usetikzlibrary{intersections}
\usepackage{animate}
\pgfmathdeclarefunction{normal}{2}{%
  \pgfmathparse{1/(#2*sqrt(2*pi))*exp(-((x-#1)^2)/(2*#2^2))}%
}
\makeatletter
\pgfmathdeclarefunction{erf}{1}{%
  \begingroup
    \pgfmathparse{#1 > 0 ? 1 : -1}%
    \edef\sign{\pgfmathresult}%
    \pgfmathparse{abs(#1)}%
    \edef\x{\pgfmathresult}%
    \pgfmathparse{1/(1+0.3275911*\x)}%
    \edef\t{\pgfmathresult}%
    \pgfmathparse{%
      1 - (((((1.061405429*\t -1.453152027)*\t) + 1.421413741)*\t 
      -0.284496736)*\t + 0.254829592)*\t*exp(-(\x*\x))}%
    \edef\y{\pgfmathresult}%
    \pgfmathparse{(\sign)*\y}%
    \pgfmath@smuggleone\pgfmathresult%
  \endgroup
}
\makeatother
\pgfmathdeclarefunction{skew}{3}{%
        \pgfmathparse{(exp(-((x-#1)^2)/(2*(#2)^2))*((erf((#3*(x-#1))/(sqrt(2)*#2)))+1))/(sqrt(2*pi)*#2)}%
}
\begin{document}
\begin{tikzpicture}
\begin{axis}[
      no markers, 
      domain=-1:18, 
      samples=20,
      height=5cm, width=12cm,
      xmin = -1, xmax=18,
      clip=false,
	  hide axis,
  ]
  \addplot [name path=normal,very thick,cyan!85!black!50] {normal(10,3.416969)};
  \addplot [name path=skew,very thick,red!85!black!50] {skew(1,4,10)};
  \path [
	draw=green!80,
	fill=green!20,
	% NEGATIVE INDEX:
  	intersection segments={of=skew and normal,sequence={A{-1}[reverse]}}]
	% this REQUIRES the "last point set":
	|- 
	% this REQUIRES the last moveto set:
	cycle
	;
\end{axis}
\end{tikzpicture}
\end{document}
