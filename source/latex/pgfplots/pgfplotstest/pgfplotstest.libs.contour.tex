\testsection{Library: Contour}
\usepgfplotslibrary{contour}
\makeatletter

% Start the process of writing to the file
\newwrite\pgfplotscontourstream
\immediate\openout\pgfplotscontourstream=pgfplotstest.contour.log


Testing the math routines.

\pgfplotsmatrixnewempty\pgfplots@data@matrixX
\pgfplotsmatrixnewempty\pgfplots@data@matrixY
\pgfplotsmatrixnewempty\pgfplots@data@matrixZ
\pgfplotsmatrixresize\pgfplots@data@matrixX{16}{16}
\pgfplotsmatrixresize\pgfplots@data@matrixY{16}{16}
\pgfplotsmatrixresize\pgfplots@data@matrixZ{16}{16}
\foreach \x in {0,...,15} {
    \foreach \y in {0,...,15} {
        \pgfmathparse{\x/15}
        \expandafter\xdef\csname\string\pgfplots@data@matrixX@\x,\y\endcsname{\pgfmathresult}
        \pgfmathparse{\y/15}
        \expandafter\xdef\csname\string\pgfplots@data@matrixY@\x,\y\endcsname{\pgfmathresult}
        \pgfmathparse{\x*\y/225}
        \expandafter\xdef\csname\string\pgfplots@data@matrixZ@\x,\y\endcsname{\pgfmathresult}
    }
}

\scriptsize
\foreach \x in {0,...,15} {
    \foreach \y in {0,...,15} {
        \pgfplotsmatrixvalueofelem{\x},{\y}\of\pgfplots@data@matrixZ,
    }

}
\normalsize
Finding the value of 3,2: \pgfplotsmatrixvalueofelem3,2\of\pgfplots@data@matrixZ


Test the MARK rutines
\pgfplots@contour@mark@runned(2,3)
\pgfplots@contour@read@runned(2,3)\to\pgfplots@contour@tmpA

This should reveal TRUE=\ifx\pgfplots@contour@tmpA\pgfplots@contour@runned@mark TRUE\else FALSE\fi

\pgfplots@contour@read@runned(2,2)\to\pgfplots@contour@tmpA
This should reveal FALSE=\ifx\pgfplots@contour@tmpA\pgfplots@contour@runned@mark TRUE\else FALSE\fi


Test the global of let

\expandafter\let\csname\string\pgfplots@test1,3\endcsname=\relax
\bgroup
\expandafter\global\expandafter\let\csname\string\pgfplots@test1,2\endcsname=\pgfplots@contour@runned@mark
\expandafter\let\csname\string\pgfplots@test1,3\endcsname=\pgfplots@contour@runned@mark
\egroup
\csname\string\pgfplots@test1,3\endcsname=relax
\csname\string\pgfplots@test1,2\endcsname=x


\pgfkeys{/pgfplots/contour/levels={0.1,0.2,0.3,0.4,0.5,0.6,0.7,0.8,0.9,1}}
The levels of the contour: \pgfplots@contour@levels

Test of the interpolation: 
\pgfplots@contour@start

\c@pgf@counta=0
\c@pgf@countb=0
\pgfplotsmatrixsize\pgfplots@contour@matrix@points\to\c@pgf@counta\c@pgf@countb
\advance\c@pgf@counta by-1
\advance\c@pgf@countb by-1

\scriptsize
\foreach \y in {0,...,\the\c@pgf@counta} {
    (\y:0,1) $\rightarrow$ (\pgfplotsmatrixvalueofelem\y,0\of\pgfplots@contour@matrix@points , \pgfplotsmatrixvalueofelem\y,1\of\pgfplots@contour@matrix@points) = \pgfmathparse{\pgfplotsmatrixvalueofelem\y,0\of\pgfplots@contour@matrix@points *\pgfplotsmatrixvalueofelem\y,1\of\pgfplots@contour@matrix@points}\pgfmathresult

}

\pgfplots@contour@reset@runned

\immediate\closeout\pgfplotscontourstream


% preliminary debug output:

\immediate\openout\pgfplotscontourstream=pgfplotstest.contour.dat
	\def\n{^^J}%newline
	\def\t{^^I}%tab
\immediate\write\pgfplotscontourstream{\pgfplotsretval}
\immediate\closeout\pgfplotscontourstream

\begin{tikzpicture}
%\tracingcommands=2\tracingmacros=2
	\begin{axis}
	\addplot3[contour prepared,contour prepared format=matlab] file {pgfplotstest.contour.dat};
	\end{axis}
\end{tikzpicture}

