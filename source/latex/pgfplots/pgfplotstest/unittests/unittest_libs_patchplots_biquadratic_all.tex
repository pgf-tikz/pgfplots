\documentclass{article}
% translate with >> pdflatex -shell-escape <file>

% This file is used as unit test for pgfplots, copyright by Christian Feuersaenger.
% 
% See
%   http://pgfplots.sourceforge.net/pgfplots.pdf
% for pgfplots.
%
% Any required input files (for <plot table> or <plot file> or the table package) can be downloaded
% at
% http://www.ctan.org/tex-archive/graphics/pgf/contrib/pgfplots/doc/latex/
% and
% http://www.ctan.org/tex-archive/graphics/pgf/contrib/pgfplots/doc/latex/plotdata/

\usepackage{pgfplots}
\pgfplotsset{compat=1.4}

\usepgfplotslibrary{patchplots}
\pagestyle{empty}

\begin{document}
\pgfmathsetseed{5000}

\pgfplotsset{
	show nodes/.style={
		nodes near coords=\coordindex,nodes near coords align={center}
	},
	every axis/.style={
		patch,
	},
	patch type=biquadratic,
	point meta=z,
	small,
	show nodes,
}
	 \def\xA{0} \def\yA{0} % 1
	 \def\xB{3} \def\yB{1} % 5
	 \def\xC{6} \def\yC{1} % 2
	 \def\xD{6} \def\yD{3} % 6
	 \def\xE{5} \def\yE{5} % 3
	 \def\xF{2} \def\yF{6} % 7
	 \def\xG{-1}\def\yG{5} % 4
	 \def\xH{0} \def\yH{3} % 8

	 \def\xM{3} \def\yM{3.75} % new middle node
	 \let\xI=\xM \let\yI=\yM
	\newcommand\BIQUADRATIC[9]{%
		\addplot3[] coordinates {
			(\xA,\yA,#1)  
			(\xC,\yC,#2) 
			(\xE,\yE,#3) 
			(\xG,\yG,#4) 
			(\xB,\yB,#5) 
			(\xD,\yD,#6) 
			(\xF,\yF,#7) 
			(\xH,\yH,#8) 
			(\xI,\yI,#9) 
		};
	}
	\newcommand\BIQUADRATICSHAPES[1]{{%
		\noindent
		\pgfplotsset{
			shader=#1,
			title={PGFPlots biquadratic (#1)},
		}
		\noindent
		\begin{tikzpicture}
		\begin{axis}
			\BIQUADRATIC100000000
		\end{axis}
		\end{tikzpicture}


		\noindent
		\begin{tikzpicture}
		\begin{axis}
			\BIQUADRATIC010000000
		\end{axis}
		\end{tikzpicture}

		\begin{tikzpicture}
		\begin{axis}
			\BIQUADRATIC001000000
		\end{axis}
		\end{tikzpicture}


		\noindent
		\begin{tikzpicture}
		\begin{axis}
			\BIQUADRATIC000100000
		\end{axis}
		\end{tikzpicture}

		\pgfplotsset{point meta=rand}
		\begin{tikzpicture}
		\begin{axis}
			\BIQUADRATIC000010000
		\end{axis}
		\end{tikzpicture}


		\noindent
		\begin{tikzpicture}
		\begin{axis}
			\BIQUADRATIC000001000
		\end{axis}
		\end{tikzpicture}

		\begin{tikzpicture}
		\begin{axis}
			\BIQUADRATIC000000100
		\end{axis}
		\end{tikzpicture}


		\noindent
		\begin{tikzpicture}
		\begin{axis}
			\BIQUADRATIC000000010
		\end{axis}
		\end{tikzpicture}

		\begin{tikzpicture}
		\begin{axis}
			\BIQUADRATIC000000001
		\end{axis}
		\end{tikzpicture}

	}}

	\newcommand\BIQUADRATICSHAPESVIEW{{%
		\pgfplotsset{
			shader=interp,
			title={h=\h,v=\v},
			view/h=\h,
			view/v=\v,
	%		point meta=rand,
		}
		\foreach \v in {0,30,90,120,180,210,270,300} {
			\foreach \h in {0,30,90,120,180,210,270,300} {
				\message{view=\h,\v^^J}%
				\begin{tikzpicture}
				\begin{axis}
				\BIQUADRATIC{1}{0.4}{-0.5}{0.1}
					{0.2}{0.5}{-1}{0.3}{0.1}{1}
				\end{axis}
				\end{tikzpicture}
			}
			\par\noindent
		}

	}}




	\BIQUADRATICSHAPES{interp}

	%\BIQUADRATICSHAPES{faceted}

	{
	\BIQUADRATICSHAPES{interp,patch to triangles}
	}

	% this is very expensive:
	%\BIQUADRATICSHAPESVIEW
\end{document}
