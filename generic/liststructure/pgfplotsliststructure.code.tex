%%%%%%%%%%%%%%%%%%%%%%%%%%%%%%%%%%%%%%%%%%%%%%%%%%%%%%%%%%%%%%%%%%%%%%%%%%%%%
%
% This is a helper package with an elementary list datastructure.
%
% Its implementation is based on Knuth's list macros in "The TeXbook".
%
% The following macros are supplied:
%
% \pgfplotslistnewempty
% \pgfplotslistnew
% \pgfplotslistcopy
% \pgfplotslistpopfront
% \pgfplotslistpushback
% \pgfplotslistpushfront
% \pgfplotslistsize
% \pgfplotslistselect
% \pgfplotslistcheckempty
% \pgfplotslistforeach
%
% Copyright 2007/2008 by Christian Feuersänger.
%
% This program is free software: you can redistribute it and/or modify
% it under the terms of the GNU General Public License as published by
% the Free Software Foundation, either version 3 of the License, or
% (at your option) any later version.
% 
% This program is distributed in the hope that it will be useful,
% but WITHOUT ANY WARRANTY; without even the implied warranty of
% MERCHANTABILITY or FITNESS FOR A PARTICULAR PURPOSE.  See the
% GNU General Public License for more details.
% 
% You should have received a copy of the GNU General Public License
% along with this program.  If not, see <http://www.gnu.org/licenses/>.
%
%
%
%%%%%%%%%%%%%%%%%%%%%%%%%%%%%%%%%%%%%%%%%%%%%%%%%%%%%%%%%%%%%%%%%%%%%%%%%%%%%

\newtoks\pgfplotslist@TOK@a
\newtoks\pgfplotslist@TOK@b
\newif\ifpgfplotslistempty


\newif\ifpgfplots@loop@CONTINUE

% Creates a new, empty list.
\long\def\pgfplotslistnewempty#1{\let#1=\pgfutil@empty}

% Creates a new list with an abirtrary number of elements.
% Arguments:
% #1: the list's name (a macro name)
% #2: the elements in the form 
%     first\\second\\third\\ ...\\
%     like in tabular with one column.
%
% Example:
% \pgfplotslistnew\foolist{First Element\\Second Element\\Third Element\\}
%
% WARNING: do NOT forget the final '\\'!
\long\def\pgfplotslistnew#1#2{%
	\pgfplotslistnewempty{#1}%
	\long\def\pgfplotslistnew@impl@rest{#2}%
	\loop
	\ifx\pgfplotslistnew@impl@rest\pgfutil@empty
		\pgfplots@loop@CONTINUEfalse
	\else
		\pgfplots@loop@CONTINUEtrue
	\fi
	\ifpgfplots@loop@CONTINUE
		\expandafter\pgfplotslistnew@impl\pgfplotslistnew@impl@rest\tolist#1\relax
	\repeat
}

% Copies list #1 to list #2.
\def\pgfplotslistcopy#1\to#2{\let#2=#1}


% helper macro for \pgfplotslistnew
\long\def\pgfplotslistnew@impl#1\\#2\tolist#3{%
	\pgfplotslistpushback#1\to#3\relax%
	\def\pgfplotslistnew@impl@rest{#2}%
}


% #1: the item to prepend
% #2: the list as macro name
% Example:
% \pgfplotslistpushfront Next first Element\to\foolist
\long\def\pgfplotslistpushfront#1\to#2{%
	\pgfplotslist@TOK@a={\pgfplotslist@sep{#1}}%
	\pgfplotslist@TOK@b=\expandafter{#2}%
  	\edef#2{\the\pgfplotslist@TOK@a\the\pgfplotslist@TOK@b}%
}

% #1: the item to append
% #2: the list as macro name
% Example:
% \pgfplotslistpushback Next last element\to\foolist
\long\def\pgfplotslistpushback#1\to#2{%
	\pgfplotslist@TOK@a={\pgfplotslist@sep{#1}}%
	\pgfplotslist@TOK@b=\expandafter{#2}%
 	\edef#2{\the\pgfplotslist@TOK@b\the\pgfplotslist@TOK@a}%
}

% Concatenates two lists #2 and #3 into #1
% Example:
% \pgfplotslistconcat\result=\foolist&\bar
\long\def\pgfplotslistconcat#1=#2&#3{%
	\pgfplotslist@TOK@a=\expandafter{#2}%
	\pgfplotslist@TOK@b=\expandafter{#3}%
	\edef#1{\the\pgfplotslist@TOK@a\the\pgfplotslist@TOK@b}%
}

% implements #2 := pop_front(#1)
% Example:
% \pgfplotslistpopfront\foolist\to\poppedfirstelem
\long\def\pgfplotslistpopfront#1\to#2{%
	\pgfplotslistcheckempty#1\relax
	\ifpgfplotslistempty
		\pgfplots@error{ERROR: \string\pgfplotslistpopfront\ from \string#1\ although list is EMPTY}%
	\else
		\expandafter\pgfplotslistpopfront@impl#1\pgfplotslistpopfront@macronames#1#2%
	\fi
}

% implementation helper for listpopfront
\long\def\pgfplotslistpopfront@impl\pgfplotslist@sep#1#2\pgfplotslistpopfront@macronames#3#4{%
	\def#4{#1}%
	\def#3{#2}%
}

% Counts the number of elements in list #1, storing it into the count
% register #2.
% Example:
% \pgfplotslistsize\foo\to{\count0}%
% \the\count0
\long\def\pgfplotslistsize#1\to#2{%
	#2=0%
	\long\def\pgfplotslist@sep##1{\advance#2 by 1 }%
	#1%
}

% Returns the #1th element of list #2 into macro #3
% Arguments:
% #1: a count 0,...,N-1 where N is the list size.
%     You may specify a number of a count.
% #2: a list
% #3: a macro name
% Example:
%   Element 0:
%   \pgfplotslistselect0\of\foo\to\elem
%   \elem
%   Element \count1:
%   \pgfplotslistselect\count1\of\foo\to\elem
\long\def\pgfplotslistselect#1\of#2\to#3{%
	\def#3{\pgfplots@error{The requested item #1 of \string#2 is OUT OF RANGE.}}%
	\begingroup
	\count0=#1\relax
 	\long\def\pgfplotslist@sep##1{%
		\advance\count0-1
		\ifnum\count0=-1
			\global\def\pgfplotslistselect@tmp{\def#3{##1}}%
			% the command \pgfplotslistselect@tmp will be expander AFTER
			% '\endgroup'
			\aftergroup\pgfplotslistselect@tmp%
		\fi%
	}%
	#2%
	\endgroup
}

% Sets the boolean \ifpgfplotslistempty depending on whether list #1 is empty
% or not.
% Example:
%
% \pgfplotslistcheckempty\foolist
% \ifpgfplotslistempty
%    List foolist is empty!
% \else
%    List is not empty.
% \fi
\def\pgfplotslistcheckempty#1{%
	\ifx#1\pgfutil@empty
		\pgfplotslistemptytrue
	\else
		\pgfplotslistemptyfalse
	\fi
}


% Iterates through each list element, names it #2 and calls code #3.
% Example:
% \pgfplotslistnew\foolist{Eins\\Zwei\\Drei\\}%
% \pgfplotslistforeach\foolist\as\foo{Element \foo\par}%
% results in
%  Element Eins
%  Element Zwei
%  Element Drei
\long\def\pgfplotslistforeach#1\as#2#3{{%
	\long\def\pgfplotslist@sep##1{\def#2{##1}#3}%
	#1\relax
}}

% The same but without groups around #3.
\long\def\pgfplotslistforeachungrouped#1\as#2#3{%
	\long\def\pgfplotslist@sep##1{\def#2{##1}#3}%
	#1\relax
}
