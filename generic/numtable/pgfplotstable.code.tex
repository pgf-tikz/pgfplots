%--------------------------------------------
%
% Package numtable
%
% Provides support to read and work with abstact numeric tables of the
% form
%
% COLUMN1	COLUMN2 COLUMN3
% 1 		2		3
% 4			4		552
% 1e124		0.00001	1.2345e-12
% ...
%
% Copyright 2007/2008 by Christian Feuersänger.
%
% This program is free software: you can redistribute it and/or modify
% it under the terms of the GNU General Public License as published by
% the Free Software Foundation, either version 3 of the License, or
% (at your option) any later version.
% 
% This program is distributed in the hope that it will be useful,
% but WITHOUT ANY WARRANTY; without even the implied warranty of
% MERCHANTABILITY or FITNESS FOR A PARTICULAR PURPOSE.  See the
% GNU General Public License for more details.
% 
% You should have received a copy of the GNU General Public License
% along with this program.  If not, see <http://www.gnu.org/licenses/>.
%
%--------------------------------------------

% This file relies on
% - pgfplotsliststructure.code.tex
% - pgfplotsarraystructure.code.tex
% - a (quite!) recent PGF.
%
% and needs some token registers. It does not use more of pgfplots.
%
\newif\ifpgfplotstable@search@header
\newcount\c@pgfplotstable@counta
\newwrite\pgfplotstable@outfile
\newif\ifpgfplotstabletypesetdebug
\newif\ifpgfplotstableuserow

% should always be false; use only in grouped internal macros
\newif\ifpgfplots@table@options@areset

\input pgfplotstable.coltype.code.tex

\pgfkeys{%
	/pgfplots/table/header/.is if=pgfplotstable@search@header,
	/pgfplots/table/header=true,
	/pgfplots/table/x index/.store in=\pgfplots@plot@tbl@xindex,
	/pgfplots/table/x index=0,
	/pgfplots/table/x/.store in=\pgfplots@plot@tbl@x,
	/pgfplots/table/x=,
	/pgfplots/table/y index/.store in=\pgfplots@plot@tbl@yindex,
	/pgfplots/table/y index=1,
	/pgfplots/table/y/.store in=\pgfplots@plot@tbl@y,
	/pgfplots/table/y=,
	/pgfplots/table/x error index/.initial=,
	/pgfplots/table/y error index/.initial=,
	/pgfplots/table/x error/.initial=,
	/pgfplots/table/y error/.initial=,
	/pgfplots/table/row predicate/.code={},
	/pgfplots/table/skip rows between index/.style 2 args={%
		/pgfplots/table/row predicate/.append code={%
			\ifnum##1<#1\relax
			\else
				\ifnum##1<#2\relax
					\pgfplotstableuserowfalse
				\fi
			\fi}
	},
	/pgfplots/table/col sep/.is choice,
	/pgfplots/table/col sep/space/.code		= {\def\pgfplotstableread@COLSEP@CASE{0}},
	/pgfplots/table/col sep/comma/.code		= {\def\pgfplotstableread@COLSEP@CASE{1}},
	/pgfplots/table/col sep/semicolon/.code	= {\def\pgfplotstableread@COLSEP@CASE{2}},
	/pgfplots/table/col sep/colon/.code		= {\def\pgfplotstableread@COLSEP@CASE{3}},
	/pgfplots/table/col sep/braces/.code	= {\def\pgfplotstableread@COLSEP@CASE{4}},
	/pgfplots/table/col sep=space,
	% columns={name1,name2}
	% or
	% columns={[index]2,name2,name3,[index]5}
	/pgfplots/table/columns/.initial=,
	/pgfplots/table/column name/.initial=\pgfkeysnovalue,
	%
	% this thing here allows to MODIFY 'column name'.
	%
	% Argument #1 is the current column name, that means after
	% evaluating 'column name'. If this key changes anything, it
	% should write its result back into 'column name'.
	%
	% That means you can use 'column name' to assign the name as such
	% and 'assign column name' to generate final TeX code (for example
	% to insert \multicolumn{1}{c}{#1} or so).
	% default is empty which means no change.
	%/pgfplots/table/assign column name/.code={
	%	\pgfkeyssetvalue{/pgfplots/table/column name}{#1}%
	%},
	%
	%
	%
	% A style which inserts \multicolumn{1}{#1}{<column name>} for
	% each column name.
	% The column name as such can be set with the 'column name' option.
	/pgfplots/table/multicolumn names/.style={%
		/pgfplots/table/assign column name/.code={%
			\pgfkeyssetvalue{/pgfplots/table/column name}{\multicolumn{1}{#1}{##1}}%
		}%
	},
	/pgfplots/table/multicolumn names/.default=c,
	/pgfplots/table/dec sep align/.style={%
		/pgf/number format/assume math mode,
		/pgf/number format/set decimal separator/.add={&}{},
		/pgfplots/table/assign column name/.code={%
			\pgfkeyssetvalue{/pgfplots/table/column name}{\multicolumn{2}{#1}{##1}}%
		},%
		/pgfplots/table/column type={>{$}r<{$}@{}>{$}l<{$}},
		/pgfplots/table/assign cell content/.code={%
			\pgfmathprintnumberto{##1}\pgfmathresult%
			\expandafter\pgfutil@in@\expandafter&\expandafter{\pgfmathresult}%
			\ifpgfutil@in@
			\else
				\expandafter\def\expandafter\pgfmathresult\expandafter{\pgfmathresult&}%
			\fi
			\pgfkeyslet{/pgfplots/table/@cell content}\pgfmathresult
		},
	},
	/pgfplots/table/dec sep align/.default=c,
	%
	% A style which can be used together with the 'dcolumn' package by
	% David Carlisle.
	% #1: the dcolumn type, defaults to 'D{.}{.}{2}'
	% #2: the column name type, defaults to 'c'
	/pgfplots/table/dcolumn/.style 2 args={%
		/pgf/number format/assume math mode,
		column type={#1},
		multicolumn names=#2,
	},
	/pgfplots/table/dcolumn/.default={D{.}{.}{2}}{c},
	/pgfplots/table/column type/.initial={c},
	/pgfplots/table/every table/.style={},
	/pgfplots/table/every even row/.style={},
	/pgfplots/table/every odd row/.style={},
	/pgfplots/table/every last row/.style={},
	/pgfplots/table/every first row/.style={},
	/pgfplots/table/every head row/.style={},
	/pgfplots/table/every first column/.style={},
	/pgfplots/table/every last column/.style={},
	/pgfplots/table/every even column/.style={},
	/pgfplots/table/every odd column/.style={},
	/pgfplots/table/before row/.initial=,
	/pgfplots/table/after row/.initial=,
	/pgfplots/table/begin table/.initial={\begin{tabular}},
	/pgfplots/table/end table/.initial={\end{tabular}},
	/pgfplots/table/outfile/.initial=,
	/pgfplots/table/debug/.is if=pgfplotstabletypesetdebug,
	%
	% will be redefined by |assign cell content| for every cell:
	/pgfplots/table/@cell content/.initial=,
	%
	% #1: the cells content as it has been found in the input table
	% this command key should somehow fill |cell content|.
	/pgfplots/table/assign cell content/.code={%
		\pgfmathprintnumberto{#1}\pgfmathresult%
		\pgfkeyslet{/pgfplots/table/@cell content}\pgfmathresult
	},
	%
	% this here is the default formatting. It uses
	% \pgfmathprintnumber.
	/pgfplots/table/assign cell content as number/.code={%
		\pgfmathprintnumberto{#1}\pgfmathresult%
		\pgfkeyslet{/pgfplots/table/@cell content}\pgfmathresult
	},
	/pgfplots/table/string type/.style={%
		/pgfplots/table/assign cell content/.style={%
			/pgfplots/table/@cell content={##1}%
		}%
	},%
	/pgfplots/table/font/.initial=,
	/pgfplots/table/.unknown/.code={%
		\pgfqkeys{/pgf/number format}{\pgfkeyscurrentname=#1}%
	}%
}

% \pgfplotstableread[OPTIONS] {FILE} to \name
%
% This method reads a table from FILE to macro \name.
%
% FILE is something like
% G	Basis	dof	L2	A	Lmax	cgiter	maxlevel	eps
% 5	5	5	8.31160034e-02	0.00000000e+00	1.80007647e-01	2	2	-1
% 17	17	17	2.54685628e-02	0.00000000e+00	3.75580565e-02	5	3	-1
% ...
%
% A number format line is also understood:
% G	Basis	dof	L2	A	Lmax	cgiter	maxlevel	eps
% $flags int	int	int	sci:8	sci:8	sci:8	int	int	std:8
% 5	5	5	8.31160034e-02	0.00000000e+00	1.80007647e-01	2	2	-1
%
% or a three-column-gnuplot file with 2 comment headers like
% #Curve 0, 20 points
% #x y type
% 0.00000 0.00000 i
% 0.52632 0.50235 i
%
% The table data is stored columnwise in lists and can be accessed
% with the other methods of this package.
\def\pgfplotstableread{%
	\pgfutil@ifnextchar[{%
		\pgfplotstableread@impl
	}{%
		\pgfplotstableread@impl[]%
	}%
}

% BACKWARDS COMPATIBILITY
\let\pgfnumtableread=\pgfplotstableread

\def\pgfplotstablegetcolumnlist#1\to#2{%
	\let#2=#1
}

\def\pgfplotstablegetcolumnbyname#1\of#2\to#3{%
	\pgfutil@ifundefined{\string#2@#1}{%
		\pgfplots@error{Sorry, could not retrieve column '#1' from table...}%
	}{%
		\expandafter\let\expandafter#3\csname\string#2@#1\endcsname
	}%
}

\def\pgfplotstablegetcolumnnamebyindex#1\of#2\to#3{%
	\pgfplotslistselect#1\of#2\to#3\relax
}%
\def\pgfplotstablegetcolumnbyindex#1\of#2\to#3{%
	\pgfplotslistselect#1\of#2\to#3\relax
	\expandafter\pgfplotstablegetcolumnbyname#3\of#2\to{#3}%
}

\def\pgfplotstablecopy#1\to#2{%
	\let#2=#1%
	\pgfplotslistforeachungrouped#1\as\pgfplotstable@TMP{%
		\def\pgfplotstable@TMPB{%
			\expandafter\let\csname\string#2@\pgfplotstable@TMP\endcsname}%
		\expandafter\pgfplotstable@TMPB\csname\string#1@\pgfplotstable@TMP\endcsname
	}%
}

% Typesets a table.
%
% \pgfplotstabletypeset[<options>]<\tablestructure>
%
% If you do not select any columns, the complete table is drawn.
%
% There are several options and styles which are available in
% <options>, see the declaration above.
%
% ATTENTION: the default implementation employs
% \begin{tabular}...\end{tabular} and is therefor only usable with
% LaTeX!
%
% You will need to reconfigure the tables.
%
% Inside of \pgfplotstabletypeset, the macros \pgfplotstablecol and
% \pgfplotstablerow will expand to the current column index and row
% index.
\def\pgfplotstabletypeset{%
	\pgfutil@ifnextchar[{%	
		\pgfplotstabletypeset@opt
	}{%
		\pgfplotstabletypeset@opt[]%
	}%
}

% Like \pgfplotstabletypeset, but the first argument is a file name.
\def\pgfplotstabletypesetfile{%
	\pgfutil@ifnextchar[{%	
		\pgfplotstabletypesetfile@opt
	}{%
		\pgfplotstabletypesetfile@opt[]%
	}%
}
\def\pgfplotstabletypesetfile@opt[#1]#2{%
	\begingroup
	\ifpgfplots@table@options@areset
	\else
		\pgfplots@table@options@aresettrue
		\pgfplotstableset{/pgfplots/table/every table,#1}%
	\fi
	\pgfplotstableread{#2}\pgfplotstabletypesetfile@opt@@
	\pgfplotstabletypeset\pgfplotstabletypesetfile@opt@@
	\endgroup
}%

%%%%%%%%%%%%%%%%%%%%%%%%%%%%%%%%%%%%%%%%%%%%%%%%%%%%%%%%%%%%%%%%%%%%%%%%%%%
%
% IMPLEMENTATION
%
%%%%%%%%%%%%%%%%%%%%%%%%%%%%%%%%%%%%%%%%%%%%%%%%%%%%%%%%%%%%%%%%%%%%%%%%%%%
\newif\ifpgfplotstableread@curline@contains@colnames
\newif\ifpgfplotstableread@foundcolnames
\newif\ifpgfplotstableread@skipline
\def\pgfplotstableread@impl[#1]#2{%
	\pgfutil@ifnextchar t{%
		\pgfplotstableread@impl@@{#1}{#2}%
	}{%
		\pgfplotstableread@impl@{#1}{#2}%
	}%
}%

% I don't know why; but I started with 
% >> \pgfplotstableread[]{file} to \macro
% That ' to ' is really ugly. This here is for backwards
% compatibility:
\def\pgfplotstableread@impl@@#1#2to #3{%
	\pgfplotstableread@impl@{#1}{#2}{#3}%
}%

\def\pgfplotstableread@impl@#1#2#3{%
	\begingroup
	\ifpgfplots@table@options@areset
	\else
		\pgfplotstableset{/pgfplots/table/every table,#1}%
	\fi
%\pgfplots@message{ATTEMPTING TO READ #2}%
	\openin1=#2\relax
	\def\pgfplotstableread@filename{#2}%
	\let\pgfplotstableread@lineno=\c@pgf@counta
	\let\pgfplotstableread@numcols=\c@pgf@countb
	\let\pgfplotstableread@curcol=\c@pgf@countc
	\let\pgfplotstableread@usablelineno=\c@pgf@countd
	\pgfplotstableread@lineno=0
	\pgfplotstableread@usablelineno=0
	\pgfplotstableread@numcols=0
	\pgfplotslistnewempty\pgfplotstable@colnames
	\ifeof1
		\pgfplotstable@error{Could not read table file '#2'.}%
	\fi
	%
	\pgfplotstableread@loop@over@lines
	%
	\ifpgfplotstableread@foundcolnames
	\else
		\pgfplotstableread@create@column@names@with@numbers
	\fi
	% Well, now write the identified data to #3:
	%
	% make all temporaries global:
	\global\let\pgfplotstable@colnames=\pgfplotstable@colnames
	\pgfplotstableread@curcol=0\relax
	\loop
	\ifnum\pgfplotstableread@curcol<\pgfplotstableread@numcols
%\pgfplots@message{ASSIGNING COLUMN NO \the\pgfplotstableread@curcol / \the\pgfplotstableread@numcols}%
		% numtable@TMP := column list content
		\expandafter\global\expandafter\let\expandafter\pgfplotstable@TMP\csname numtable@col@\the\pgfplotstableread@curcol\endcsname
		% global := numtable@TMP
		\expandafter\global\expandafter\let\csname numtable@col@\the\pgfplotstableread@curcol\endcsname=\pgfplotstable@TMP
		\advance\pgfplotstableread@curcol by1\relax
	\repeat
	\closein1
	\endgroup
	% Now, we can access the global variables!
	% copy them to #3.
	\let#3=\pgfplotstable@colnames
	\c@pgfplotstable@counta=0\relax%
	\pgfplotslistforeachungrouped\pgfplotstable@colnames\as\pgfplotstable@TMP{%
		\def\pgfplotstable@TMPB{%
			\expandafter\let\csname\string#3@\pgfplotstable@TMP\endcsname}%
		\expandafter\pgfplotstable@TMPB\csname numtable@col@\the\c@pgfplotstable@counta\endcsname
%\message{Column '\pgfplotstable@TMP' has entries: \expandafter\meaning\csname numtable@col@\the\c@pgfplotstable@counta\endcsname}%
		\expandafter\global\expandafter\let\csname numtable@col@\the\c@pgfplotstable@counta\endcsname=\pgfutil@empty
		\advance\c@pgfplotstable@counta by1\relax
	}%
	\global\let\pgfplotstable@colnames=\pgfutil@empty
}

\def\pgfplotstableread@loop@over@lines{%
	\ifeof1
%\pgfplots@message{EOF}%
	\else
		\read1 to\pgfplotstable@LINE
		\ifeof1
		\else
		\expandafter\pgfplotstableread@checkspecial@line\pgfplotstable@LINE\pgfplotstable@EOI
		\ifpgfplotstableread@skipline
		\else
			%--------------------------------------------------
			% \ifnum\pgfplotstableread@lineno=0
			% 	\let\pgfplotstable@firstline=\pgfplotstable@LINE
			% \fi
			%-------------------------------------------------- 
%\pgfplots@message{READING LINE \the\pgfplotstableread@lineno: '\meaning\pgfplotstable@LINE'.}%
			\ifnum\pgfplotstableread@numcols=0\relax
				\pgfplotstableread@curcol=0\relax
				\pgfplotstableread@curline@contains@colnamesfalse
				\pgfplotstableread@impl@DO\pgfplotstableread@impl@countcols@and@identifynames@NEXT\pgfplotstable@LINE
				%\expandafter\pgfplotstableread@impl@countcols@and@identifynames@ITERATE\pgfplotstable@LINE\pgfplotstable@EOI
				\pgfplotstableread@numcols=\pgfplotstableread@curcol
				\pgfplotstableread@curcol=0\relax
				% Create empty column lists:
				\pgfplotstableread@create@column@lists
				%
				\ifnum\pgfplotstableread@usablelineno=0\relax
				\ifnum\pgfplotstableread@lineno=2\relax
				\ifnum\pgfplotstableread@numcols=3\relax
					% The file started with
					% #...
					% #...
					% X Y i
					% -> thats a gnuplot file!
					\pgfplotstableread@curline@contains@colnamesfalse
				\fi
				\fi
				\fi
				% Now, read the first line.
				% It contains either
				% - column names,
				% - numerical data,
				% - nothing (comments).
				\ifpgfplotstableread@curline@contains@colnames
					\pgfplotstableread@foundcolnamestrue
					\pgfplotstableread@curcol=0\relax
					\pgfplotstableread@impl@DO\pgfplotstableread@impl@collectcolnames@NEXT\pgfplotstable@LINE
					%\expandafter\pgfplotstableread@impl@collectcolnames@ITERATE\pgfplotstable@LINE\pgfplotstable@EOI
				\else
					\pgfplotstableread@foundcolnamesfalse
					\pgfplotstableread@curcol=0\relax
					% Leave column name lists empty...
					\pgfplotstableread@impl@DO\pgfplotstableread@impl@nextrow@NEXT\pgfplotstable@LINE
					%\expandafter\pgfplotstableread@impl@nextrow@ITERATE\pgfplotstable@LINE\pgfplotstable@EOI
				\fi
%\pgfplots@message{After reading first row: found '\the\pgfplotstableread@numcols' columns; column name list='\meaning\pgfplotstable@colnames'}%
			\else
				\pgfplotstableread@curcol=0\relax
				\pgfplotstableread@impl@DO\pgfplotstableread@impl@nextrow@NEXT\pgfplotstable@LINE
				%\expandafter\pgfplotstableread@impl@nextrow@ITERATE\pgfplotstable@LINE\pgfplotstable@EOI
			\fi
			\ifnum\pgfplotstableread@curcol=\pgfplotstableread@numcols
			\else
				\pgfplotstable@error{ERROR: the input table has an unexpected number of columns in row '\the\pgfplotstableread@lineno'. Expected: '\the\pgfplotstableread@numcols'; got '\the\pgfplotstableread@curcol. Maybe the input table is corrupted?}%
			\fi
			\advance\pgfplotstableread@usablelineno by1\relax
		\fi
		\fi
		\advance\pgfplotstableread@lineno by1\relax
		\pgfplotstableread@loop@over@lines
	\fi
}

\def\pgfplotstableread@checkspecial@line{%
	\pgfutil@ifnextchar##{%
		\pgfplotstableread@skiplinetrue
		\pgfplotstableread@impl@gobble
	}{%
		\pgfutil@ifnextchar${%
			\pgfplotstableread@process@flags@line
		}{%
			\pgfutil@ifnextchar\pgfplotstable@EOI{%
				\pgfplotstableread@skiplinetrue
				\pgfplotstableread@impl@gobble
			}{%
				\pgfutil@ifnextchar\par{%
					\pgfplotstableread@skiplinetrue
					\pgfplotstableread@impl@gobble
				}{%
					\pgfplotstableread@skiplinefalse
					\pgfplotstableread@impl@gobble
				}%
			}%
		}%
	}%
}

\long\def\pgfplotstableread@process@flags@line$flags {%
%\pgfplots@message{Ignoring flags line ...}%
	\pgfplotstableread@skiplinetrue
	\pgfplotstableread@impl@gobble
}

\def\pgfplotstableread@create@column@lists{%
	\loop
	\ifnum\pgfplotstableread@curcol<\pgfplotstableread@numcols
		\expandafter\pgfplotslistnewempty\csname numtable@col@\the\pgfplotstableread@curcol\endcsname
		\advance\pgfplotstableread@curcol by1\relax
	\repeat
}

\def\pgfplotstableread@create@column@names@with@numbers{%
	\pgfplotstableread@curcol=0\relax
	\loop
	\ifnum\pgfplotstableread@curcol<\pgfplotstableread@numcols
		\expandafter\pgfplotslistpushback\the\pgfplotstableread@curcol\to\pgfplotstable@colnames
		\advance\pgfplotstableread@curcol by1\relax
	\repeat
}

\long\def\pgfplotstableread@impl@gobble#1\pgfplotstable@EOI{}%

\def\pgfplotstable@EOI{\pgfplotstable@EOI}%

%%%%%%%%%%%%%%%

% A loop command which processes every single entry in a raw data row #2 
% and invokes the macro #1{<arg>}  for each found column entry.
%
% Columns are separated by the /pgfplots/table/col sep character.
%
% #1: a command which takes precisely one argument. It will be called
% for each found column entry
%
% #2: a macro containing a raw data line with <col sep> separated
% entries.
\def\pgfplotstableread@impl@DO#1#2{%
	\let\pgfplotstableread@impl@ITERATE@NEXT@=#1\relax
	\ifcase\pgfplotstableread@COLSEP@CASE\relax
		% SPACE:
		\expandafter\pgfplotstableread@impl@ITERATE#2\pgfplotstable@EOI
	\or
		% COMMA:
		\let\pgfplotstableread@impl@ITERATE@NEXT=\pgfplotstableread@impl@ITERATE@NEXT@COMMA
		\expandafter\pgfplotstableread@impl@ITERATE#2,\pgfplotstable@EOI
	\or
		% SEMICOLON:
		\let\pgfplotstableread@impl@ITERATE@NEXT=\pgfplotstableread@impl@ITERATE@NEXT@SEMICOLON
		\expandafter\pgfplotstableread@impl@ITERATE#2;\pgfplotstable@EOI
	\or
		% COLON:
		\let\pgfplotstableread@impl@ITERATE@NEXT=\pgfplotstableread@impl@ITERATE@NEXT@COLON
		\expandafter\pgfplotstableread@impl@ITERATE#2:\pgfplotstable@EOI
	\or
		% BRACE:
		\let\pgfplotstableread@impl@ITERATE@NEXT=\pgfplotstableread@impl@ITERATE@NEXT@BRACE
		\expandafter\pgfplotstableread@impl@ITERATE#2\pgfplotstable@EOI
	\fi
}%
\def\pgfplotstableread@impl@ITERATE{%
	\pgfutil@ifnextchar\pgfplotstable@EOI{%
		\pgfplotstableread@impl@gobble
	}{%
		\pgfplotstableread@impl@ITERATE@NEXT
	}%
}%
\def\pgfplotstableread@impl@ITERATE@NEXT#1 {%
	\pgfplotstableread@impl@ITERATE@NEXT@{#1}%
	\pgfplotstableread@impl@ITERATE
}%
\def\pgfplotstableread@impl@ITERATE@NEXT@COMMA#1,{%
	\pgfplotstableread@impl@ITERATE@NEXT@{#1}%
	\pgfplotstableread@impl@ITERATE
}%
\def\pgfplotstableread@impl@ITERATE@NEXT@SEMICOLON#1;{%
	\pgfplotstableread@impl@ITERATE@NEXT@{#1}%
	\pgfplotstableread@impl@ITERATE
}%
\def\pgfplotstableread@impl@ITERATE@NEXT@COLON#1:{%
	\pgfplotstableread@impl@ITERATE@NEXT@{#1}%
	\pgfplotstableread@impl@ITERATE
}%
\def\pgfplotstableread@impl@ITERATE@NEXT@BRACE#1{%
	\pgfplotstableread@impl@ITERATE@NEXT@{#1}%
	\pgfplotstableread@impl@ITERATE
}%
%%%%%%%%%%%%


\long\def\pgfplotstableread@impl@nextrow@NEXT#1{%
%\pgfplots@message{Inserting '#1' at (\the\pgfplotstableread@lineno, \the\pgfplotstableread@curcol).}%
	\def\pgfplotstableread@TMP{\pgfplotslistpushback#1\to}%
	\expandafter\pgfplotstableread@TMP\csname numtable@col@\the\pgfplotstableread@curcol\endcsname
	\advance\pgfplotstableread@curcol by1\relax
}



\long\def\pgfplotstableread@impl@collectcolnames@NEXT#1{%
%\pgfplots@message{Got column name no \the\pgfplotstableread@curcol\ as '#1'}%
	\pgfutil@ifundefined{pgfplotstableread@impl@COLNAME@#1}{%
		\def\pgfplotstable@TMP{#1}%
	}{% generate unique column names
		\pgfplots@warning{Warning: table '\pgfplotstableread@filename' has non-unique column name '#1'. Only the first occurence can be accessed via column names.}%
		\edef\pgfplotstable@TMP{#1--index\the\pgfplotstableread@curcol}%
	}%
	\expandafter\def\csname pgfplotstableread@impl@COLNAME@#1\endcsname{foo}% remember this name.
	\expandafter\pgfplotslistpushback\expandafter{\pgfplotstable@TMP}\to\pgfplotstable@colnames
	\advance\pgfplotstableread@curcol by1\relax
}




\long\def\pgfplotstableread@impl@countcols@and@identifynames@NEXT#1{%
	\advance\pgfplotstableread@curcol by1\relax
	\ifpgfplotstable@search@header
		\ifpgfplotstableread@curline@contains@colnames
		\else
			\pgfplotstableread@isnumber@ITERATE#1\pgfplotstable@EOI
%\ifpgfplotstableread@curline@contains@colnames\pgfplots@message{'#1' is a column name!}\else\pgfplots@message{'#1' is NO column name!}\fi
		\fi
	\fi
}
\def\pgfplotstableread@isnumber@plus{+}
\def\pgfplotstableread@isnumber@minus{-}
\def\pgfplotstableread@isnumber@zero{0}
\def\pgfplotstableread@isnumber@one{1}
\def\pgfplotstableread@isnumber@two{2}
\def\pgfplotstableread@isnumber@three{3}
\def\pgfplotstableread@isnumber@four{4}
\def\pgfplotstableread@isnumber@five{5}
\def\pgfplotstableread@isnumber@six{6}
\def\pgfplotstableread@isnumber@seven{7}
\def\pgfplotstableread@isnumber@eight{8}
\def\pgfplotstableread@isnumber@nine{9}
\def\pgfplotstableread@isnumber@e{e}
\def\pgfplotstableread@isnumber@E{E}
\def\pgfplotstableread@isnumber@period{.}

\def\pgfplotstableread@isnumber@ITERATE#1{%
	\def\pgfplotstableread@CURTOK{#1}%
	\ifx\pgfplotstableread@CURTOK\pgfplotstable@EOI
		\def\pgfplotstableread@NEXT{}%
	\else
		\def\pgfplotstableread@NEXT{\pgfplotstableread@isnumber@ITERATE}%
		\ifx\pgfplotstableread@CURTOK\pgfplotstableread@isnumber@plus
		\else
		\ifx\pgfplotstableread@CURTOK\pgfplotstableread@isnumber@minus
		\else
		\ifx\pgfplotstableread@CURTOK\pgfplotstableread@isnumber@zero
		\else
		\ifx\pgfplotstableread@CURTOK\pgfplotstableread@isnumber@one
		\else
		\ifx\pgfplotstableread@CURTOK\pgfplotstableread@isnumber@two
		\else
		\ifx\pgfplotstableread@CURTOK\pgfplotstableread@isnumber@three
		\else
		\ifx\pgfplotstableread@CURTOK\pgfplotstableread@isnumber@four
		\else
		\ifx\pgfplotstableread@CURTOK\pgfplotstableread@isnumber@five
		\else
		\ifx\pgfplotstableread@CURTOK\pgfplotstableread@isnumber@six
		\else
		\ifx\pgfplotstableread@CURTOK\pgfplotstableread@isnumber@seven
		\else
		\ifx\pgfplotstableread@CURTOK\pgfplotstableread@isnumber@eight
		\else
		\ifx\pgfplotstableread@CURTOK\pgfplotstableread@isnumber@nine
		\else
		\ifx\pgfplotstableread@CURTOK\pgfplotstableread@isnumber@e
		\else
		\ifx\pgfplotstableread@CURTOK\pgfplotstableread@isnumber@E
		\else
		\ifx\pgfplotstableread@CURTOK\pgfplotstableread@isnumber@period
		\else
%\message{NO ITS NOT!  Token: '\meaning\pgfplotstableread@CURTOK'}%
			% it's no number, so it is a column name.
			\pgfplotstableread@curline@contains@colnamestrue
			\def\pgfplotstableread@NEXT{\pgfplotstableread@impl@gobble}%
		\fi\fi\fi\fi\fi\fi\fi\fi\fi\fi\fi\fi\fi\fi\fi
	\fi
	\pgfplotstableread@NEXT
}

\def\pgfplotstable@error#1{\pgfplots@error{#1}}%


\def\pgfplotstableset{%
	\pgfqkeys{/pgfplots/table}%
}%

% Accepts a macro #1 which contains an argument denoting a column
% name.
%
% It checks whether #1 starts with '#', indicating that it is actually
% a column INDEX. If that is the case,
% \ifpgfplotstableread@foundcolnames is set to false and the index is
% returned into #1.
%
% Otherwise, \ifpgfplotstableread@foundcolnames is set to true.
\def\pgfplotstabletypeset@is@colname#1{%
	\expandafter\pgfplotstabletypeset@is@colname@#1\pgfplotstable@EOI
	\ifpgfplotstableread@foundcolnames
	\else
		\let#1=\pgfplotstable@TMP
	\fi
}%
\def\pgfplotstabletypeset@is@colname@{%
	\pgfutil@ifnextchar[{%
		\pgfplotstabletypeset@is@colname@index
	}{%
		\pgfplotstableread@foundcolnamestrue
		\pgfplotstabletypeset@is@colname@name
	}%
}
\def\pgfplotstabletypeset@is@colname@index@@{index}%
\def\pgfplotstabletypeset@is@colname@index[#1]#2\pgfplotstable@EOI{%
	\def\pgfplotstable@TMP{#1}%
	\ifx\pgfplotstable@TMP\pgfplotstabletypeset@is@colname@index@@
		\pgfplotstableread@foundcolnamesfalse
		\def\pgfplotstable@TMP{#2}%
	\else
		\pgfplotstableread@foundcolnamestrue
	\fi
}%
\def\pgfplotstabletypeset@is@colname@name#1\pgfplotstable@EOI{}%
\def\pgfplotstabletypeset@getfinalentry#1#2{%
	\pgfkeysvalueof{/pgfplots/table/assign cell content/.@cmd}#1\pgfeov
	\pgfkeysgetvalue{/pgfplots/table/@cell content}{#2}%
}%

% processes the option 'assign column name'
\def\pgfplotstabletypeset@assign@final@colname#1#2{%
	\pgfkeysifdefined{/pgfplots/table/assign column name/.@cmd}{%
		\pgfkeysvalueof{/pgfplots/table/assign column name/.@cmd}#1\pgfeov
		\pgfkeysgetvalue{/pgfplots/table/column name}{#2}%
	}{}%
}
\def\pgfplotstabletypeset@nocolname{\pgfkeysnovalue}

% checks if #1 contains invalid chars for pgfkeys and sets
% \ifpgfutil@in@ to true if that is the case.
\def\pgfplotstable@checkspecialchars@pgfkeys#1\pgfplotstable@EOI{%
	\pgfutil@in@/{#1}%
	\ifpgfutil@in@
	\else
		\pgfutil@in@={#1}%
		\ifpgfutil@in@
		\else
			\pgfutil@in@,{#1}%
		\fi
	\fi

}%

% TODO
% - replace grouped list foreach by popfront-loop and use arrays
%   directly -> group only the pgfkeys eval
\def\pgfplotstabletypeset@opt[#1]#2{%
	\begingroup
	\def\pgfplotstablecol{\the\c@pgfplotstable@colindex}%
	\def\pgfplotstablerow{\the\c@pgfplotstable@rowindex}%
%	\def\pgfplotstablecols{\the\c@pgfplotstable@numcols}%
%	\def\pgfplotstablerows{\the\c@pgfplotstable@numrows}%
	\ifpgfplots@table@options@areset
	\else
		\pgfplotstableset{/pgfplots/table/every table,#1}%
	\fi
	\pgfkeysgetvalue{/pgfplots/table/columns}{\pgfplotstable@colnames}%
	\ifx\pgfplotstable@colnames\pgfutil@empty
		\pgfplotstablegetcolumnlist#2\to\pgfplotstable@colnames
	\else
		\expandafter\pgfplotslistnew\expandafter\pgfplotstable@colnames\expandafter{\pgfplotstable@colnames}%
	\fi
	\global\pgfplotslistnewempty\pgfplotstabletypeset@final@colnames
	\global\pgfplotslistnewempty\pgfplotstabletypeset@final@coltypes
	\global\pgfplotslistnewempty\pgfplotstabletypeset@final@cols
	\let\c@pgfplotstable@numcols=\c@pgf@counta
	\let\c@pgfplotstable@numrows=\c@pgf@countd
	\let\c@pgfplotstable@rowindex=\c@pgf@countc
	\let\c@pgfplotstable@colindex=\c@pgf@countb
	\pgfplotslistsize\pgfplotstable@colnames\to\c@pgfplotstable@numcols
	\c@pgfplotstable@colindex=0\relax
	\pgfplotslistforeach\pgfplotstable@colnames\as\pgfplotstable@colname{%
		\c@pgfplotstable@rowindex=0\relax
		\pgfplotstabletypeset@is@colname\pgfplotstable@colname
		\ifpgfplotstableread@foundcolnames
		\else
			\pgfplotstablegetcolumnnamebyindex\pgfplotstable@colname\of#2\to\pgfplotstable@colname
		\fi
		\pgfplotstablegetcolumnbyname\pgfplotstable@colname\of#2\to\pgfplotstable@col
		%
		% Set keys for columns!
		\ifodd\c@pgfplotstable@colindex
			\pgfplotslist@TOK@a={every odd column}%
		\else
			\pgfplotslist@TOK@a={every even column}%
		\fi
		\ifnum\c@pgfplotstable@colindex=0\relax
			\pgfplotslist@TOK@a=\expandafter{\the\pgfplotslist@TOK@a,every first column}%
		\fi
		\global\advance\c@pgfplotstable@colindex by1\relax
		\ifnum\c@pgfplotstable@colindex=\c@pgfplotstable@numcols
			\pgfplotslist@TOK@a=\expandafter{\the\pgfplotslist@TOK@a,every last column}%
		\fi
		\pgfplotslist@TOK@b=\expandafter{\pgfplotstable@colname}%
		\expandafter\pgfplotstable@checkspecialchars@pgfkeys\the\pgfplotslist@TOK@b\pgfplotstable@EOI
		\ifpgfutil@in@
			\edef\pgfplotstable@TMP{\the\pgfplotslist@TOK@a,columns/{\the\pgfplotslist@TOK@b}/.try}%
		\else
			\edef\pgfplotstable@TMP{\the\pgfplotslist@TOK@a,columns/\the\pgfplotslist@TOK@b/.try}%
		\fi
		\expandafter\pgfplotstableset\expandafter{\pgfplotstable@TMP}%
		%
		\pgfkeysgetvalue{/pgfplots/table/column name}{\pgfplotstable@colname@out}%
		\ifx\pgfplotstable@colname@out\pgfplotstabletypeset@nocolname
			\let\pgfplotstable@colname@out=\pgfplotstable@colname
		\fi
		\expandafter\pgfplotstabletypeset@assign@final@colname\expandafter{\pgfplotstable@colname@out}\pgfplotstable@colname@out
		{\globaldefs=1
		\expandafter\pgfplotslistpushback\pgfplotstable@colname@out\to\pgfplotstabletypeset@final@colnames
		}%
		\pgfkeysgetvalue{/pgfplots/table/column type}{\pgfplotstable@coltype}%
		{\globaldefs=1
		\expandafter\pgfplotslistpushback\pgfplotstable@coltype\to\pgfplotstabletypeset@final@coltypes
		}%
		%
		\pgfplotslistnewempty\pgfplotstable@col@processed
		\pgfplotslistforeachungrouped\pgfplotstable@col\as\pgfplotstable@entry{%
			\pgfplotstableuserowtrue
			\edef\pgfplotstable@TMP{\noexpand\pgfkeysvalueof{/pgfplots/table/row predicate/.@cmd}\the\c@pgfplotstable@rowindex}%
			\pgfplotstable@TMP\pgfeov
			\ifpgfplotstableuserow
				\expandafter\pgfplotstabletypeset@getfinalentry\expandafter{\pgfplotstable@entry}{\pgfplotstable@entry}%
				\expandafter\pgfplotslistpushback\pgfplotstable@entry\to\pgfplotstable@col@processed
			\fi
			\advance\c@pgfplotstable@rowindex by1\relax
		}%
		{\globaldefs=1
		\expandafter\pgfplotslistpushback\expandafter{\pgfplotstable@col@processed}\to\pgfplotstabletypeset@final@cols
		}%
	}%
	%
	% Ok, I have now everything which will come into the final table.
	%
	% But I have it column-oriented; I need to transpose the storage.
	%
	% The following code assembles a
	% \begin{tabular}{}
	% ...
	% \end{tabular}
	% statement piece after piece.
	%
%\message{I have now \meaning\pgfplotstabletypeset@final@colnames, and \meaning\pgfplotstabletypeset@final@cols.}%
	% Step 1: column names.
	\c@pgfplotstable@colindex=0\relax
	% STEP 1.1: collect column types:
	\def\pgfplotstable@resulttypes{}%
	\pgfplotslistforeachungrouped\pgfplotstabletypeset@final@coltypes\as\pgfplotstable@coltype{%
		\pgfplotslist@TOK@a=\expandafter{\pgfplotstable@resulttypes}%
		\pgfplotslist@TOK@b=\expandafter{\pgfplotstable@coltype}%
		\edef\pgfplotstable@resulttypes{\the\pgfplotslist@TOK@a\the\pgfplotslist@TOK@b}%
	}%
	\pgfkeysgetvalue{/pgfplots/table/begin table}{\pgfplotstable@entry}%
	\pgfplotslist@TOK@a=\expandafter{\pgfplotstable@entry}%
	\ifx\pgfplotstable@resulttypes\pgfutil@empty
		\edef\pgfplotstable@result{\the\pgfplotslist@TOK@a}%
	\else
		\pgfplotslist@TOK@b=\expandafter{\pgfplotstable@resulttypes}%
		\edef\pgfplotstable@result{\the\pgfplotslist@TOK@a{\the\pgfplotslist@TOK@b}}%
	\fi
	%
	\pgfkeysgetvalue{/pgfplots/table/font}{\pgfplotstable@font}%
	\ifx\pgfplotstable@font\pgfutil@empty
	\else
		\pgfplotslist@TOK@a=\expandafter{\pgfplotstable@font}%
		\pgfplotslist@TOK@b=\expandafter{\pgfplotstable@result}%
		\edef\pgfplotstable@result{\noexpand\begingroup\the\pgfplotslist@TOK@a\the\pgfplotslist@TOK@b}%
	\fi
	%

	% Step 1.2: Collect FIRST ROW (column names)
	\begingroup
	\pgfplotstableset{every head row}%
	\pgfkeysgetvalue{/pgfplots/table/before row}{\pgfplotstable@before}%
	\pgfkeysgetvalue{/pgfplots/table/after row}{\pgfplotstable@after}%
	\pgfplotslist@TOK@a=\expandafter{\pgfplotstable@before}%
	\pgfplotslist@TOK@b=\expandafter{\pgfplotstable@after}%
	\xdef\pgfplotstable@TMP{%
		\noexpand\def\noexpand\pgfplotstable@before{\the\pgfplotslist@TOK@a}%
		\noexpand\def\noexpand\pgfplotstable@after{\the\pgfplotslist@TOK@b}%
	}%
	\endgroup
	\pgfplotstable@TMP
	% insert 'before row' here:
	\pgfplotslist@TOK@a=\expandafter{\pgfplotstable@before}%
	\pgfplotslist@TOK@b=\expandafter{\pgfplotstable@result}%
	\edef\pgfplotstable@result{\the\pgfplotslist@TOK@b\the\pgfplotslist@TOK@a}%
	%
	\pgfplotslistforeachungrouped\pgfplotstabletypeset@final@colnames\as\pgfplotstable@colname@out{%
		\advance\c@pgfplotstable@colindex by1\relax
		\pgfplotslist@TOK@a=\expandafter{\pgfplotstable@result}%
		\ifnum\c@pgfplotstable@colindex=\c@pgfplotstable@numcols\relax
			\pgfplotslist@TOK@b=\expandafter{\pgfplotstable@colname@out \\}%
		\else
			\pgfplotslist@TOK@b=\expandafter{\pgfplotstable@colname@out &}%
		\fi
		\edef\pgfplotstable@result{\the\pgfplotslist@TOK@a\the\pgfplotslist@TOK@b}%
	}%
	% insert 'after row' here:
	\pgfplotslist@TOK@a=\expandafter{\pgfplotstable@result}%
	\pgfplotslist@TOK@b=\expandafter{\pgfplotstable@after}%
	\edef\pgfplotstable@result{\the\pgfplotslist@TOK@a\the\pgfplotslist@TOK@b}%
	%
%\message{I have now \meaning\pgfplotstable@result.}%
	% Step 2: column contents.
	% I will first convert \pgfplotstabletypeset@final@cols into an array.
	\c@pgfplotstable@colindex=0\relax
	\pgfplotsarraynewempty\pgfplotstabletypeset@final@cols@array
	\pgfplotslistforeachungrouped\pgfplotstabletypeset@final@cols\as\pgfplotstable@col@processed{%
		\expandafter\pgfplotsarraypushback\expandafter{\pgfplotstable@col@processed}\to\pgfplotstabletypeset@final@cols@array
	}%
	% init numrows:
	\pgfplotsarrayselect\c@pgfplotstable@colindex\of\pgfplotstabletypeset@final@cols@array\to\pgfplotstable@col@processed
	\pgfplotslistsize\pgfplotstable@col@processed\to\c@pgfplotstable@numrows
	%
	% Now, we loop over every column as long as there are still rows
	% left. We assemble rows while we go.
	%
	\c@pgfplotstable@rowindex=0\relax
	\ifnum\c@pgfplotstable@colindex<\c@pgfplotstable@numcols
		\pgfplots@loop@CONTINUEtrue
	\else
		\pgfplots@loop@CONTINUEfalse
	\fi
	\loop
	\ifpgfplots@loop@CONTINUE
		\pgfplotsarrayselect\c@pgfplotstable@colindex\of\pgfplotstabletypeset@final@cols@array\to\pgfplotstable@col@processed
		\pgfplotslistcheckempty\pgfplotstable@col@processed
		\ifpgfplotslistempty
			\pgfplots@loop@CONTINUEfalse
		\else
			\ifnum\c@pgfplotstable@colindex=0\relax
				% Install styles for the next row.
				\begingroup
				\ifodd\c@pgfplotstable@rowindex
					\pgfplotslist@TOK@a={every odd row}%
				\else
					\pgfplotslist@TOK@a={every even row}%
				\fi
				\ifnum\c@pgfplotstable@rowindex=0\relax
					\pgfplotslist@TOK@a=\expandafter{\the\pgfplotslist@TOK@a,every first row}%
				\fi
				% misuse as temporary variable:
				\c@pgfplotstable@colindex=\c@pgfplotstable@rowindex
				\advance\c@pgfplotstable@colindex by1\relax
				\ifnum\c@pgfplotstable@colindex=\c@pgfplotstable@numrows
					\edef\pgfplotstable@TMP{\the\pgfplotslist@TOK@a,every row no \the\c@pgfplotstable@rowindex/.try,every last row}%
				\else
					\edef\pgfplotstable@TMP{\the\pgfplotslist@TOK@a,every row no \the\c@pgfplotstable@rowindex/.try}%
				\fi
				\expandafter\pgfplotstableset\expandafter{\pgfplotstable@TMP}%
				\pgfkeysgetvalue{/pgfplots/table/before row}{\pgfplotstable@before}%
				\pgfkeysgetvalue{/pgfplots/table/after row}{\pgfplotstable@after}%
				\pgfplotslist@TOK@a=\expandafter{\pgfplotstable@before}%
				\pgfplotslist@TOK@b=\expandafter{\pgfplotstable@after}%
				\xdef\pgfplotstable@TMP{%
					\noexpand\def\noexpand\pgfplotstable@before{\the\pgfplotslist@TOK@a}%
					\noexpand\def\noexpand\pgfplotstable@after{\the\pgfplotslist@TOK@b}%
				}%
				\endgroup
				\pgfplotstable@TMP
				% insert 'before row' here:
				\pgfplotslist@TOK@a=\expandafter{\pgfplotstable@before}%
				\pgfplotslist@TOK@b=\expandafter{\pgfplotstable@result}%
				\edef\pgfplotstable@result{\the\pgfplotslist@TOK@b\the\pgfplotslist@TOK@a}%
			\fi
			%
			%
			\pgfplotslistpopfront\pgfplotstable@col@processed\to\pgfplotstable@entry
			\pgfplotsarrayletentry\c@pgfplotstable@colindex\of\pgfplotstabletypeset@final@cols@array=\pgfplotstable@col@processed
			\advance\c@pgfplotstable@colindex by1\relax
			\pgfplotslist@TOK@a=\expandafter{\pgfplotstable@result}%
			\ifnum\c@pgfplotstable@colindex=\c@pgfplotstable@numcols\relax
				\pgfplotslist@TOK@b=\expandafter{\pgfplotstable@entry \\}%
			\else
				\pgfplotslist@TOK@b=\expandafter{\pgfplotstable@entry &}%
			\fi
			\edef\pgfplotstable@result{\the\pgfplotslist@TOK@a\the\pgfplotslist@TOK@b}%
			\ifnum\c@pgfplotstable@colindex=\c@pgfplotstable@numcols\relax
				\c@pgfplotstable@colindex=0\relax
				% insert 'after row' here:
				\pgfplotslist@TOK@a=\expandafter{\pgfplotstable@result}%
				\pgfplotslist@TOK@b=\expandafter{\pgfplotstable@after}%
				\edef\pgfplotstable@result{\the\pgfplotslist@TOK@a\the\pgfplotslist@TOK@b}%
				\advance\c@pgfplotstable@rowindex by1\relax
			\fi
%\message{I have now \meaning\pgfplotstable@result.}%
		\fi
	\repeat
	\pgfplotslist@TOK@a=\expandafter{\pgfplotstable@result}%
	\pgfkeysgetvalue{/pgfplots/table/end table}{\pgfplotstable@entry}%
	\pgfplotslist@TOK@b=\expandafter{\pgfplotstable@entry}%
	\edef\pgfplotstable@result{\the\pgfplotslist@TOK@a\the\pgfplotslist@TOK@b}%
	\ifx\pgfplotstable@font\pgfutil@empty
	\else
		\pgfplotslist@TOK@a=\expandafter{\pgfplotstable@result\endgroup}%
		\edef\pgfplotstable@result{\the\pgfplotslist@TOK@a}%
	\fi
	\ifpgfplotstabletypesetdebug
		\pgfplotslist@TOK@a=\expandafter{\pgfplotstable@result}%
		\immediate\write16{------- PGFPLOTSTABLE DEBUG MODE: --------}%
		\immediate\write16{{\the\pgfplotslist@TOK@a}}%
	\fi
	\pgfkeysgetvalue{/pgfplots/table/outfile}{\pgfplotstable@entry}%
	\ifx\pgfplotstable@entry\pgfutil@empty
	\else
		\begingroup
		\immediate\openout\pgfplotstable@outfile=\pgfplotstable@entry\relax
		\pgfplotslist@TOK@a=\expandafter{\pgfplotstable@result}%
		\immediate\write\pgfplotstable@outfile{\the\pgfplotslist@TOK@a}%
		\immediate\closeout\pgfplotstable@outfile
		\endgroup
	\fi
	\pgfplotstable@result
	\endgroup
}%


\endinput
