%--------------------------------------------
%
% Package pgfplots
%
% Provides a user-friendly interface to create function plots (normal
% plots, semi-logplots and double-logplots).
% 
% It is based on Till Tantau's PGF package.
%
% Copyright 2007/2008 by Christian Feuersänger.
%
% This program is free software: you can redistribute it and/or modify
% it under the terms of the GNU General Public License as published by
% the Free Software Foundation, either version 3 of the License, or
% (at your option) any later version.
% 
% This program is distributed in the hope that it will be useful,
% but WITHOUT ANY WARRANTY; without even the implied warranty of
% MERCHANTABILITY or FITNESS FOR A PARTICULAR PURPOSE.  See the
% GNU General Public License for more details.
% 
% You should have received a copy of the GNU General Public License
% along with this program.  If not, see <http://www.gnu.org/licenses/>.
%
%--------------------------------------------

\input pgfplotsliststructure.code.tex
\input pgfplotsarray.code.tex
\input pgfplotstable.code.tex
\input pgfplotsoldpgfsupp_loader.code.tex
\input pgfplots.stackedplots.code.tex

\usetikzlibrary{decorations,decorations.pathmorphing,decorations.pathreplacing}

% FIXME: reduce number of variables!
\newtoks\pgfplots@toka

\newif\ifpgfplots@xislinear
\newif\ifpgfplots@yislinear
\newcount\pgfplots@numplots
\newdimen\pgfplots@xcoordminTEX
\newdimen\pgfplots@xcoordmaxTEX
\newdimen\pgfplots@ycoordminTEX
\newdimen\pgfplots@ycoordmaxTEX
\newdimen\pgfplots@tmpa
\newif\ifpgfplots@warn@for@filter@discards
\newif\ifpgfplots@isuniformtick
\newif\ifpgfplots@addplot@table@from@macro
\newif\ifpgfplots@clip@limits
\newif\ifpgfplots@enlargelimits
\newif\ifpgfplots@enlargelimits@rel@thresh
\newif\ifpgfplots@enlargelimits@auto
\newif\ifpgfplots@tickshow
\newif\ifpgfplots@xminorticks
\newif\ifpgfplots@xmajorticks
\newif\ifpgfplots@yminorticks
\newif\ifpgfplots@ymajorticks
\newif\ifpgfplots@xminorgrids
\newif\ifpgfplots@xmajorgrids
\newif\ifpgfplots@yminorgrids
\newif\ifpgfplots@ymajorgrids
\newif\ifpgfplots@separate@axis@lines
\newif\ifpgfplots@scaled@ticks
\newif\ifpgfplots@identify@log@minor@tick@pos
\newif\ifpgfplots@disablelogfilter
\newif\ifpgfplots@disabledatascaling
\newif\ifpgfplots@hide@x
\newif\ifpgfplots@hide@y
\newif\ifpgfplots@is@old@list@format
\newif\ifpgfplots@errorbars@enabled
\newif\ifpgfplots@scale@only@axis
\newif\ifpgfplots@xticklabel@interval
\newif\ifpgfplots@yticklabel@interval
\newif\ifpgfplots@stacked@x
\newif\ifpgfplots@stackedmode
\newif\ifpgfplots@stacked@reverse
\newif\ifpgfplots@stacked@plus
\let\pgfplots@TMP=\pgfutil@empty
\let\pgfplots@TMPB=\pgfutil@empty
\let\pgfplots@TMPC=\pgfutil@empty
\let\pgfnodepartimagebox=\pgfnodeparttextbox

\newif\ifpgfplots@collect@firstplot@astick

\def\pgfplots@errorbars@xdirection{0}% pre-init, see below
\def\pgfplots@errorbars@ydirection{0}%

\def\pgfplots@error#1{\PackageError{pgfplots}{#1}{}}%
\def\pgfplots@warning#1{\pgfplots@message{! Package pgfplots Warning: #1}}%
\def\pgfplots@message#1{%
	\immediate\write16{#1}%
}%

\def\axisdefaultwidth{240pt}
\def\axisdefaultheight{207pt}


% Assigns list contents #2 to a list macro #1.
%
% The list contents may be provided in one of two formats:
% a) in the (deprecated) list format 'first\\second\\thirst\\'
% or
% b) in the PGF foreach list format 'first,second,third'.
\def\pgfplots@assign@list#1#2{%
	\def\pgfplots@TMP{#2}%
	\ifx\pgfplots@TMP\pgfutil@empty
		\pgfplotslistnewempty#1%
	\else
		\pgfplots@check@backwards@compatible@list@format #2\\\pgfplots@EOI
		\ifpgfplots@is@old@list@format
			\pgfplotslistnew#1{#2}%
		\else
			\pgfplots@foreach@to@list{#2}\to#1
		\fi
	\fi
}%

\def\pgfplots@EOI{\pgfplots@EOI}%

% Sets the boolean \ifpgfplots@is@old@list@format  to true if and only
% if the input is a list in the format '{first\\second\\}'.
%
% This format is deprecated, but is still provided for backwards
% compatibility.
%
% Usage:
% \pgfplots@check@backwards@compatible@list@format <argument>'\\'\pgfplots@EOI
% you NEED to append '\\\pgfplots@EOI' at the end.
\def\pgfplots@check@backwards@compatible@list@format#1\\#2\pgfplots@EOI{%
	\def\pgfplots@TMP{#2}%
	\ifx\pgfplots@TMP\pgfutil@empty
		\pgfplots@is@old@list@formatfalse
	\else
		\pgfplots@is@old@list@formattrue
	\fi
}%



% Creates a named plot cycle list.
%
% #1:  the name of the final list. Can be used in 'cycle list name'
% #2:  the list entries. You can use either a comma-separated list or
%      a '\\'-terminated list. The latter case also requires '\\'
%      AFTER the last entry.
\def\pgfcreateplotcyclelist#1#2{\pgfplots@assign@list#1{#2}}

\pgfcreateplotcyclelist{\blackwhiteplotspeclist}{%
	mark options={fill=gray},mark=*\\%
	mark options={fill=gray},mark=square*\\%
	mark options={fill=gray},mark=otimes*\\%
	mark=star\\%
	mark options={fill=gray},mark=diamond*\\%
	densely dashed,mark options={solid,fill=gray},mark=*\\%
	densely dashed,mark options={solid,fill=gray},mark=square*\\%
	densely dashed,mark options={solid,fill=gray},mark=otimes*\\%
	densely dashed,mark options={solid},mark=star\\%
	densely dashed,mark options={solid,fill=gray},mark=diamond*\\%
}
\pgfcreateplotcyclelist{\coloredplotspeclist}{%
	blue,mark options={fill=blue!80!black},mark=*\\%
	red,mark options={fill=red!80!black},mark=square*\\%
	brown!60!black,mark options={fill=brown!80!black},mark=otimes*\\%
	black,mark=star\\%
	blue,mark options={fill=blue!80!black},mark=diamond*\\%
	red,densely dashed,mark options={solid,fill=red!80!black},mark=*\\%
	brown!60!black,densely dashed,mark options={solid,fill=brown!80!black},mark=square*\\%
	black,densely dashed,mark options={solid,fill=gray},mark=otimes*\\%
	blue,densely dashed,mark=star,mark options=solid\\%
	red,densely dashed,mark options={solid,fill=red!80!black},mark=diamond*\\%
}

\def\pgfplotsset#{\pgfqkeys{/pgfplots}}

\def\pgfplotsdeprecatedstylecheck#1{%
	\pgfkeysifdefined{#1/.@cmd}{%
		\begingroup
		\edef\pgfkeyscurrentkey{#1}%
		\pgfkeyssplitpath
		\pgfplots@warning{Loading deprecated style option 
			\pgfkeyscurrentpath/\pgfkeyscurrentname.  
			Please replace '\string\tikzstyle{\pgfkeyscurrentname}' 
			with '\string\pgfplotsset{\pgfkeyscurrentname/.style={}}'
			(or '\string\pgfplotsset{\pgfkeyscurrentname/.append style={}}').}%
		\endgroup
		\pgfkeysvalueof{#1/.@cmd}\pgfeov
	}{}%
}%

\pgfkeys{%
	/pgfplots/search path for tikz/.unknown/.code={%
		\let\searchname=\pgfkeyscurrentname%
		\pgfkeysalso{%
			/tikz/\searchname/.try=#1,
			/pgfplots/\searchname/.lastretry=#1
		}%
	},%
	/pgfplots/.is family,
	/pgfplots/scale/.is family,
	/pgfplots/legend/.is family,
	/pgfplots/tick/.is family,
	/pgfplots/axis/.is family,
	/pgfplots/descriptions/.is family,
	/pgfplots/style commands/.is family,
	/pgfplots/naming commands/.is family,
	/pgfplots/error bars/.is family,
	/pgfplots/every axis/.style={},
	/pgfplots/every axis/.append code={\pgfplotsdeprecatedstylecheck{/tikz/every axis}},
	/pgfplots/every axis/.belongs to family=/pgfplots/style commands,
	/pgfplots/every semilogx axis/.style={},
	/pgfplots/every semilogx axis/.append code={\pgfplotsdeprecatedstylecheck{/tikz/every semilogx axis}},
	/pgfplots/every semilogx axis/.belongs to family=/pgfplots/style commands,
	/pgfplots/every semilogy axis/.style={},
	/pgfplots/every semilogy axis/.append code={\pgfplotsdeprecatedstylecheck{/tikz/every semilogy axis}},
	/pgfplots/every semilogy axis/.belongs to family=/pgfplots/style commands,
	/pgfplots/every loglog axis/.style={},
	/pgfplots/every loglog axis/.append code={\pgfplotsdeprecatedstylecheck{/tikz/every loglog axis}},
	/pgfplots/every loglog axis/.belongs to family=/pgfplots/style commands,
	/pgfplots/every linear axis/.style={},
	/pgfplots/every linear axis/.append code={\pgfplotsdeprecatedstylecheck{/tikz/every linear axis}},
	/pgfplots/every linear axis/.belongs to family=/pgfplots/style commands,
	/pgfplots/every axis plot/.style={},
	/pgfplots/every axis plot/.append code={\pgfplotsdeprecatedstylecheck{/tikz/every axis plot}},
	/pgfplots/every axis plot/.belongs to family=/pgfplots/style commands,
	/pgfplots/every axis plot post/.style={},
	/pgfplots/every axis plot post/.append code={\pgfplotsdeprecatedstylecheck{/tikz/every axis plot}},
	/pgfplots/every axis label/.style={},
	/pgfplots/every axis label/.append code={\pgfplotsdeprecatedstylecheck{/tikz/every axis label}},
	/pgfplots/every axis label/.belongs to family=/pgfplots/style commands,
	/pgfplots/every axis x label/.style={at={(0.5,0)},below,yshift=-15pt},
	/pgfplots/every axis x label/.append code={\pgfplotsdeprecatedstylecheck{/tikz/every axis x label}},
	/pgfplots/every axis x label/.belongs to family=/pgfplots/style commands,
	/pgfplots/every axis y label/.style={at={(0,0.5)},xshift=-35pt,rotate=90},
	/pgfplots/every axis y label/.append code={\pgfplotsdeprecatedstylecheck{/tikz/every axis y label}},
	/pgfplots/every axis y label/.belongs to family=/pgfplots/style commands,
	/pgfplots/every axis title/.style={at={(0.5,1)},above,yshift=6pt},
	/pgfplots/every axis title/.append code={\pgfplotsdeprecatedstylecheck{/tikz/every axis title}},
	/pgfplots/every axis title/.belongs to family=/pgfplots/style commands,
	/pgfplots/every tick/.style={very thin,gray},
	/pgfplots/every tick/.append code={\pgfplotsdeprecatedstylecheck{/tikz/every tick}},
	/pgfplots/every tick/.belongs to family=/pgfplots/style commands,
	/pgfplots/every inner x axis line/.style={},
	/pgfplots/every inner y axis line/.style={},
	/pgfplots/every outer x axis line/.style={},
	/pgfplots/every outer y axis line/.style={},
	/pgfplots/x axis line style/.style={
		/pgfplots/every outer x axis line/.append style={#1},
		/pgfplots/every inner x axis line/.append style={#1},
	},
	/pgfplots/y axis line style/.style={
		/pgfplots/every outer y axis line/.append style={#1},
		/pgfplots/every inner y axis line/.append style={#1},
	},
	/pgfplots/outer axis line style/.style={
		/pgfplots/every outer x axis line/.append style={#1},
		/pgfplots/every outer y axis line/.append style={#1}%
	},
	/pgfplots/inner axis line style/.style={
		/pgfplots/every inner x axis line/.append style={#1},
		/pgfplots/every inner y axis line/.append style={#1}%
	},
	/pgfplots/axis line style/.style={
		/pgfplots/inner axis line style={#1},
		/pgfplots/outer axis line style={#1}%
	},
	/pgfplots/separate axis lines/.is if=pgfplots@separate@axis@lines,
	/pgfplots/separate axis lines/.default=true,
	/pgfplots/every minor tick/.style={},
	/pgfplots/every minor tick/.append code={\pgfplotsdeprecatedstylecheck{/tikz/every minor tick}},
	/pgfplots/every minor tick/.belongs to family=/pgfplots/style commands,
	/pgfplots/every major tick/.style={},
	/pgfplots/every major tick/.append code={\pgfplotsdeprecatedstylecheck{/tikz/every major tick}},
	/pgfplots/every major tick/.belongs to family=/pgfplots/style commands,
	/pgfplots/every x tick/.style={},
	/pgfplots/every x tick/.append code={\pgfplotsdeprecatedstylecheck{/tikz/every x tick}},
	/pgfplots/every x tick/.belongs to family=/pgfplots/style commands,
	/pgfplots/every minor x tick/.style={},
	/pgfplots/every minor x tick/.append code={\pgfplotsdeprecatedstylecheck{/tikz/every minor x tick}},
	/pgfplots/every minor x tick/.belongs to family=/pgfplots/style commands,
	/pgfplots/every major x tick/.style={},
	/pgfplots/every major x tick/.append code={\pgfplotsdeprecatedstylecheck{/tikz/every major x tick}},
	/pgfplots/every major x tick/.belongs to family=/pgfplots/style commands,
	/pgfplots/every y tick/.style={},
	/pgfplots/every y tick/.append code={\pgfplotsdeprecatedstylecheck{/tikz/every y tick}},
	/pgfplots/every y tick/.belongs to family=/pgfplots/style commands,
	/pgfplots/every minor y tick/.style={},
	/pgfplots/every minor y tick/.append code={\pgfplotsdeprecatedstylecheck{/tikz/every minor y tick}},
	/pgfplots/every minor y tick/.belongs to family=/pgfplots/style commands,
	/pgfplots/every major y tick/.style={},
	/pgfplots/every major y tick/.append code={\pgfplotsdeprecatedstylecheck{/tikz/every major y tick}},
	/pgfplots/every major y tick/.belongs to family=/pgfplots/style commands,
	/pgfplots/every axis grid/.style={help lines},
	/pgfplots/every axis grid/.append code={\pgfplotsdeprecatedstylecheck{/tikz/every axis grid}},
	/pgfplots/every axis grid/.belongs to family=/pgfplots/style commands,
	/pgfplots/every minor grid/.style={},
	/pgfplots/every minor grid/.append code={\pgfplotsdeprecatedstylecheck{/tikz/every minor grid}},
	/pgfplots/every minor grid/.belongs to family=/pgfplots/style commands,
	/pgfplots/every major grid/.style={},
	/pgfplots/every major grid/.append code={\pgfplotsdeprecatedstylecheck{/tikz/every major grid}},
	/pgfplots/every major grid/.belongs to family=/pgfplots/style commands,
	/pgfplots/every axis x grid/.style={},
	/pgfplots/every axis x grid/.append code={\pgfplotsdeprecatedstylecheck{/tikz/every axis x grid}},
	/pgfplots/every axis x grid/.belongs to family=/pgfplots/style commands,
	/pgfplots/every minor x grid/.style={},
	/pgfplots/every minor x grid/.append code={\pgfplotsdeprecatedstylecheck{/tikz/every minor x grid}},
	/pgfplots/every minor x grid/.belongs to family=/pgfplots/style commands,
	/pgfplots/every major x grid/.style={},
	/pgfplots/every major x grid/.append code={\pgfplotsdeprecatedstylecheck{/tikz/every major x grid}},
	/pgfplots/every major x grid/.belongs to family=/pgfplots/style commands,
	/pgfplots/every axis y grid/.style={},
	/pgfplots/every axis y grid/.append code={\pgfplotsdeprecatedstylecheck{/tikz/every axis y grid}},
	/pgfplots/every axis y grid/.belongs to family=/pgfplots/style commands,
	/pgfplots/every minor y grid/.style={},
	/pgfplots/every minor y grid/.append code={\pgfplotsdeprecatedstylecheck{/tikz/every minor y grid}},
	/pgfplots/every minor y grid/.belongs to family=/pgfplots/style commands,
	/pgfplots/every major y grid/.style={},
	/pgfplots/every major y grid/.append code={\pgfplotsdeprecatedstylecheck{/tikz/every major y grid}},
	/pgfplots/every major y grid/.belongs to family=/pgfplots/style commands,
	/pgfplots/every tick label/.style={},
	/pgfplots/every tick label/.append code={\pgfplotsdeprecatedstylecheck{/tikz/every tick label}},
	/pgfplots/every tick label/.belongs to family=/pgfplots/style commands,
	/pgfplots/every x tick label/.style={},
	/pgfplots/every x tick label/.append code={\pgfplotsdeprecatedstylecheck{/tikz/every x tick label}},
	/pgfplots/every x tick label/.belongs to family=/pgfplots/style commands,
	/pgfplots/every extra x tick/.style={},
	/pgfplots/every extra x tick/.append code={\pgfplotsdeprecatedstylecheck{/tikz/every extra x tick}},
	/pgfplots/every extra x tick/.belongs to family=/pgfplots/style commands,
	/pgfplots/extra x tick style/.belongs to family=/pgfplots/style commands,
	/pgfplots/extra x tick style/.code={%
		\pgfkeysalso{/pgfplots/every extra x tick/.append style={#1}}%
	},
	/pgfplots/every x tick scale label/.style={at={(1,0)},yshift=-2em,left,inner sep=0pt},
	/pgfplots/every x tick scale label/.append code={\pgfplotsdeprecatedstylecheck{/tikz/every x tick scale label}},
	/pgfplots/every x tick scale label/.belongs to family=/pgfplots/style commands,
	/pgfplots/every y tick label/.style={},
	/pgfplots/every y tick label/.append code={\pgfplotsdeprecatedstylecheck{/tikz/every y tick label}},
	/pgfplots/every y tick label/.belongs to family=/pgfplots/style commands,
	/pgfplots/every extra y tick/.style={},
	/pgfplots/every extra y tick/.append code={\pgfplotsdeprecatedstylecheck{/tikz/every extra y tick}},
	/pgfplots/every extra y tick/.belongs to family=/pgfplots/style commands,
	/pgfplots/extra y tick style/.belongs to family=/pgfplots/style commands,
	/pgfplots/extra y tick style/.code={%
		\pgfkeysalso{/pgfplots/every extra y tick/.append style={#1}}%
	},
	/pgfplots/every y tick scale label/.style={at={(0,1)},above right,inner sep=0pt,yshift=0.3em},
	/pgfplots/every y tick scale label/.append code={\pgfplotsdeprecatedstylecheck{/tikz/every y tick scale label}},
	/pgfplots/every y tick scale label/.belongs to family=/pgfplots/style commands,
	/pgfplots/every axis legend/.style={%
		cells={anchor=center},
		inner xsep=3pt,inner ysep=2pt,nodes={inner sep=2pt,text depth=0.15em},
		anchor=north east,%
		shape=rectangle,%
		fill=white,%
		draw=black,
		at={(0.98,0.98)},
	},
	/pgfplots/every axis legend/.append code={\pgfplotsdeprecatedstylecheck{/tikz/every axis legend}},
	/pgfplots/every axis legend/.belongs to family=/pgfplots/style commands,
% tick options:
	/pgfplots/xticklabel/.store in=	\pgfplots@xticklabel,
	/pgfplots/xticklabel/.belongs to family=/pgfplots/tick,
	/pgfplots/xticklabel=,
	/pgfplots/xticklabels/.belongs to family=/pgfplots/tick,
	/pgfplots/xticklabels/.code={%
		\pgfplots@foreach@to@list{#1}\to\pgfplots@xticklabels
		\let\pgfplots@xticklabel=\pgfplots@user@ticklabel@list@x
	},
	/pgfplots/yticklabels/.belongs to family=/pgfplots/tick,
	/pgfplots/yticklabels/.code={%
		\pgfplots@foreach@to@list{#1}\to\pgfplots@yticklabels
		\let\pgfplots@yticklabel=\pgfplots@user@ticklabel@list@y
	},
	/pgfplots/yticklabel/.store in=	\pgfplots@yticklabel,
	/pgfplots/yticklabel/.belongs to family=/pgfplots/tick,
	/pgfplots/yticklabel=,
	/pgfplots/x tick label as interval/.is if=pgfplots@xticklabel@interval,
	/pgfplots/x tick label as interval/.default=true,
	/pgfplots/x tick label as interval/.belongs to family=/pgfplots/tick,
	/pgfplots/y tick label as interval/.is if=pgfplots@yticklabel@interval,
	/pgfplots/y tick label as interval/.default=true,
	/pgfplots/y tick label as interval/.belongs to family=/pgfplots/tick,
	/pgfplots/extra x tick label/.store in=	\pgfplots@extra@xticklabel,
	/pgfplots/extra x tick label/.belongs to family=/pgfplots/tick,
	/pgfplots/extra x tick label=,
	/pgfplots/extra x tick labels/.belongs to family=/pgfplots/tick,
	/pgfplots/extra x tick labels/.code={%
		\pgfplots@foreach@to@list{#1}\to\pgfplots@extra@xticklabels
		\let\pgfplots@extra@xticklabel=\pgfplots@user@extra@ticklabel@list@x
	},
	/pgfplots/extra y tick labels/.code={%
		\pgfplots@foreach@to@list{#1}\to\pgfplots@extra@yticklabels
		\let\pgfplots@extra@yticklabel=\pgfplots@user@extra@ticklabel@list@y
	},
	/pgfplots/xtick/.store in=			\pgfplots@xtick,
	/pgfplots/xtick/.belongs to family=/pgfplots/tick,
	/pgfplots/xtick=,
	/pgfplots/extra x ticks/.store in=\pgfplots@extra@xtick,
	/pgfplots/extra x ticks/.belongs to family=/pgfplots/tick,
	/pgfplots/extra x ticks=,
	/pgfplots/xtickten/.store in=		\pgfplots@xtickten,
	/pgfplots/xtickten/.belongs to family=/pgfplots/tick,
	/pgfplots/xtickten=,
	/pgfplots/extra y tick label/.store in=	\pgfplots@extra@yticklabel,
	/pgfplots/extra y tick label/.belongs to family=/pgfplots/tick,
	/pgfplots/extra y tick label=,
	/pgfplots/ytick/.store in=			\pgfplots@ytick,
	/pgfplots/ytick/.belongs to family=/pgfplots/tick,
	/pgfplots/ytick=,
	/pgfplots/extra y ticks/.store in=\pgfplots@extra@ytick,
	/pgfplots/extra y ticks/.belongs to family=/pgfplots/tick,
	/pgfplots/extra y ticks=,
	/pgfplots/ytickten/.store in=		\pgfplots@ytickten,
	/pgfplots/ytickten/.belongs to family=/pgfplots/tick,
	/pgfplots/ytickten=,
	/pgfplots/tick scale label code/.code={$\cdot 10^{#1}$},
	/pgfplots/tick scale label code/.belongs to family=/pgfplots/tick,
	/pgfplots/scaled ticks/.is if=pgfplots@scaled@ticks,
	/pgfplots/scaled ticks/.default=false,
	/pgfplots/scaled ticks/.belongs to family=/pgfplots/tick,
	/pgfplots/scaled ticks=true,
	/pgfplots/scale ticks above exponent/.store in=	\pgfplots@scale@ticks@above@exponent,
	/pgfplots/scale ticks above exponent/.belongs to family=/pgfplots/tick,
	/pgfplots/scale ticks above exponent=3,
	/pgfplots/scale ticks below exponent/.store in=	\pgfplots@scale@ticks@below@exponent,
	/pgfplots/scale ticks below exponent/.belongs to family=/pgfplots/tick,
	/pgfplots/scale ticks below exponent=-1,
	/pgfplots/subtickwidth/.store in=	\pgfplots@subtickwidth,
	/pgfplots/subtickwidth/.belongs to family=/pgfplots/tick,
	/pgfplots/subtickwidth=0.1cm,
	/pgfplots/tickwidth/.store in=		\pgfplots@tickwidth,
	/pgfplots/tickwidth/.belongs to family=/pgfplots/tick,
	/pgfplots/tickwidth=0.15cm,
	/pgfplots/minor x tick num/.initial=0,
	/pgfplots/minor x tick num/.belongs to family=/pgfplots/tick,
	/pgfplots/minor y tick num/.initial=0,
	/pgfplots/minor y tick num/.belongs to family=/pgfplots/tick,
	/pgfplots/minor tick num/.style={/pgfplots/minor x tick num=#1,/pgfplots/minor y tick num=#1},
	/pgfplots/minor tick num/.belongs to family=/pgfplots/tick,
	/pgfplots/minor tick length/.estore in=\pgfplots@subtickwidth,
	/pgfplots/minor tick length/.belongs to family=/pgfplots/tick,
	/pgfplots/major tick length/.estore in=\pgfplots@tickwidth,
	/pgfplots/major tick length/.belongs to family=/pgfplots/tick,
	/pgfplots/max space between ticks/.estore in=\axisdefaulttickwidth,
	/pgfplots/max space between ticks/.belongs to family=/pgfplots/tick,
	/pgfplots/max space between ticks=35,% the maximum space between adjacent ticks (in pt, but don't specify the unit 'pt')
	/pgfplots/try min ticks/.estore in=			\axisdefaulttryminticks,
	/pgfplots/try min ticks/.belongs to family=/pgfplots/tick,
	/pgfplots/try min ticks=4,
	/pgfplots/try min ticks log/.estore in=			\pgfplots@default@try@minticks@log,
	/pgfplots/try min ticks log/.belongs to family=/pgfplots/tick,
	/pgfplots/try min ticks log=3,
	/pgfplots/log plot exponent style/.style={/pgf/number format/fixed,/pgf/number format/precision=2},
	/pgfplots/log plot exponent style/.belongs to family=/pgfplots/tick,
	/pgfplots/log identify minor tick positions/.is if=pgfplots@identify@log@minor@tick@pos,
	/pgfplots/log identify minor tick positions/.belongs to family=/pgfplots/tick,
	/pgfplots/log identify minor tick positions=true,
	/pgfplots/log number format code/.code={{%
		\pgfmathlogtologten@{#1}%
		\ifpgfplots@identify@log@minor@tick@pos
			\expandafter\pgfplots@is@log@tick@a@minor@tick@pos\pgfmathresult\relax%
		\else
			\pgfplots@log@tick@isminor@tick@posfalse
		\fi
		\ifpgfplots@log@tick@isminor@tick@pos
			\pgfmathprintnumber[sci]{\pgfmathresult}%
		\else
			\pgfkeysalso{/pgfplots/log plot exponent style,/pgfplots/log base 10 number format code=\pgfmathresult}%
		\fi
	}},
	/pgfplots/log number format code/.belongs to family=/pgfplots/tick,
	/pgfplots/log base 10 number format code/.code={$10^{\pgfmathprintnumber{#1}}$},
	/pgfplots/log base 10 number format code/.belongs to family=/pgfplots/tick,
% sets \pgfplots@tickposnum to
% left=0
% right=1
% both=2
	/pgfplots/tickpos/.is choice,
	/pgfplots/tickpos/.belongs to family=/pgfplots/tick,
	/pgfplots/tickpos/left/.code	={\def\pgfplots@xtickposnum{1}\def\pgfplots@ytickposnum{1}},
	/pgfplots/tickpos/left/.belongs to family=/pgfplots/tick,
	/pgfplots/tickpos/right/.code	={\def\pgfplots@xtickposnum{3}\def\pgfplots@ytickposnum{3}},
	/pgfplots/tickpos/right/.belongs to family=/pgfplots/tick,
	/pgfplots/tickpos/both/.code	={\def\pgfplots@xtickposnum{0}\def\pgfplots@ytickposnum{0}},
	/pgfplots/tickpos/both/.belongs to family=/pgfplots/tick,
	/pgfplots/tickpos=both,
% sets \pgfplots@{x,y}tickalignnum to
% inside=0
% outside=1
% center=2
	/pgfplots/xtick align/.is choice,
	/pgfplots/xtick align/.belongs to family=/pgfplots/tick,
	/pgfplots/xtick align/inside/.code	={\def\pgfplots@xtickalignnum{0}},
	/pgfplots/xtick align/inside/.belongs to family=/pgfplots/tick,
	/pgfplots/xtick align/outside/.code	={\def\pgfplots@xtickalignnum{1}},
	/pgfplots/xtick align/outside/.belongs to family=/pgfplots/tick,
	/pgfplots/xtick align/center/.code	={\def\pgfplots@xtickalignnum{2}},
	/pgfplots/xtick align/center/.belongs to family=/pgfplots/tick,
	/pgfplots/xtick align=inside,
	/pgfplots/ytick align/.is choice,
	/pgfplots/ytick align/.belongs to family=/pgfplots/tick,
	/pgfplots/ytick align/inside/.code	={\def\pgfplots@ytickalignnum{0}},
	/pgfplots/ytick align/inside/.belongs to family=/pgfplots/tick,
	/pgfplots/ytick align/outside/.code	={\def\pgfplots@ytickalignnum{1}},
	/pgfplots/ytick align/outside/.belongs to family=/pgfplots/tick,
	/pgfplots/ytick align/center/.code	={\def\pgfplots@ytickalignnum{2}},
	/pgfplots/ytick align/center/.belongs to family=/pgfplots/tick,
	/pgfplots/ytick align=inside,
	/pgfplots/tick align/.belongs to family=/pgfplots/tick,
	/pgfplots/tick align/.style={%
		/pgfplots/xtick align=#1,
		/pgfplots/ytick align=#1,
	},%
% 'axis' options:
	/pgfplots/anchor/.belongs to family=/pgfplots,
	/pgfplots/anchor/.store in=			\pgfplots@anchorname,
	/pgfplots/anchor=south west,
% tick options:
	/pgfplots/ticks/.is choice,
	/pgfplots/ticks/.belongs to family=/pgfplots/tick,
	/pgfplots/ticks/none/.belongs to family=/pgfplots/tick,
	/pgfplots/ticks/none/.code={%
		\pgfplots@xminorticksfalse
		\pgfplots@yminorticksfalse
		\pgfplots@xmajorticksfalse
		\pgfplots@ymajorticksfalse
	},
	/pgfplots/ticks/major/.belongs to family=/pgfplots/tick,
	/pgfplots/ticks/major/.code={%
		\pgfplots@xminorticksfalse
		\pgfplots@yminorticksfalse
		\pgfplots@xmajortickstrue
		\pgfplots@ymajortickstrue
	},
	/pgfplots/ticks/minor/.belongs to family=/pgfplots/tick,
	/pgfplots/ticks/minor/.code={%
		\pgfplots@xminortickstrue
		\pgfplots@yminortickstrue
		\pgfplots@xmajorticksfalse
		\pgfplots@ymajorticksfalse
	},
	/pgfplots/ticks/both/.belongs to family=/pgfplots/tick,
	/pgfplots/ticks/both/.code={%
		\pgfplots@xminortickstrue
		\pgfplots@yminortickstrue
		\pgfplots@xmajortickstrue
		\pgfplots@ymajortickstrue
	},
	/pgfplots/ticks=both,
	/pgfplots/grid/.is choice,
	/pgfplots/grid/.belongs to family=/pgfplots/tick,
	/pgfplots/grid/none/.belongs to family=/pgfplots/tick,
	/pgfplots/grid/none/.code={%
		\pgfplots@xminorgridsfalse
		\pgfplots@yminorgridsfalse
		\pgfplots@xmajorgridsfalse
		\pgfplots@ymajorgridsfalse
	},
	/pgfplots/grid/major/.belongs to family=/pgfplots/tick,
	/pgfplots/grid/major/.code={%
		\pgfplots@xminorgridsfalse
		\pgfplots@yminorgridsfalse
		\pgfplots@xmajorgridstrue
		\pgfplots@ymajorgridstrue
	},
	/pgfplots/grid/minor/.belongs to family=/pgfplots/tick,
	/pgfplots/grid/minor/.code={%
		\pgfplots@xminorgridstrue
		\pgfplots@yminorgridstrue
		\pgfplots@xmajorgridsfalse
		\pgfplots@ymajorgridsfalse
	},
	/pgfplots/grid/both/.belongs to family=/pgfplots/tick,
	/pgfplots/grid/both/.code={%
		\pgfplots@xminorgridstrue
		\pgfplots@yminorgridstrue
		\pgfplots@xmajorgridstrue
		\pgfplots@ymajorgridstrue
	},
	/pgfplots/grid=none,
	/pgfplots/grid/.default=major,
	/pgfplots/xminorticks/.is if=pgfplots@xminorticks,
	/pgfplots/xminorticks/.default=true,
	/pgfplots/xminorticks/.belongs to family=/pgfplots/tick,
	/pgfplots/xmajorticks/.is if=pgfplots@xmajorticks,
	/pgfplots/xmajorticks/.default=true,
	/pgfplots/xmajorticks/.belongs to family=/pgfplots/tick,
	/pgfplots/yminorticks/.is if=pgfplots@yminorticks,
	/pgfplots/yminorticks/.default=true,
	/pgfplots/yminorticks/.belongs to family=/pgfplots/tick,
	/pgfplots/ymajorticks/.is if=pgfplots@ymajorticks,
	/pgfplots/ymajorticks/.default=true,
	/pgfplots/ymajorticks/.belongs to family=/pgfplots/tick,
	/pgfplots/xminorgrids/.is if=pgfplots@xminorgrids,
	/pgfplots/xminorgrids/.default=true,
	/pgfplots/xminorgrids/.belongs to family=/pgfplots/tick,
	/pgfplots/xmajorgrids/.is if=pgfplots@xmajorgrids,
	/pgfplots/xmajorgrids/.default=true,
	/pgfplots/xmajorgrids/.belongs to family=/pgfplots/tick,
	/pgfplots/yminorgrids/.is if=pgfplots@yminorgrids,
	/pgfplots/yminorgrids/.default=true,
	/pgfplots/yminorgrids/.belongs to family=/pgfplots/tick,
	/pgfplots/ymajorgrids/.is if=pgfplots@ymajorgrids,
	/pgfplots/ymajorgrids/.default=true,
	/pgfplots/ymajorgrids/.belongs to family=/pgfplots/tick,
% legend options:
	/pgfplots/legend entries/.initial={},
	/pgfplots/legend entries/.belongs to family=/pgfplots/legend,
	/pgfplots/legend columns/.store in=\pgfplots@legend@columns,
	/pgfplots/legend columns/.belongs to family=/pgfplots/legend,
	/pgfplots/legend columns=1,
	/pgfplots/legend plot pos/.is choice,
	/pgfplots/legend plot pos/.belongs to family=/pgfplots/legend,
	/pgfplots/legend plot pos/left/.code=	{\def\pgfplots@legend@plot@pos{0}},
	/pgfplots/legend plot pos/left/.belongs to family=/pgfplots/legend,
	/pgfplots/legend plot pos/right/.code=	{\def\pgfplots@legend@plot@pos{1}},
	/pgfplots/legend plot pos/right/.belongs to family=/pgfplots/legend,
	/pgfplots/legend plot pos/none/.code=	{\def\pgfplots@legend@plot@pos{2}},
	/pgfplots/legend plot pos/none/.belongs to family=/pgfplots/legend,
	/pgfplots/legend plot pos=left,
	/pgfplots/legend image code/.code={%
		\draw[#1,mark repeat=2,mark phase=2] 
			plot coordinates {
				(0cm,0cm) 
				(0.3cm,0cm)
				(0.6cm,0cm)%
			};%
	},
	/pgfplots/legend image code/.belongs to family=/pgfplots/legend,
% axis description options:
	/pgfplots/title/.store in=		\pgfplots@title,
	/pgfplots/title/.belongs to family=/pgfplots/descriptions,
	/pgfplots/title=,
	/pgfplots/xlabel/.store in=	\pgfplots@xlabel,
	/pgfplots/xlabel/.belongs to family=/pgfplots/descriptions,
	/pgfplots/xlabel=,
	/pgfplots/ylabel/.store in=	\pgfplots@ylabel,
	/pgfplots/ylabel/.belongs to family=/pgfplots/descriptions,
	/pgfplots/ylabel=,
	/pgfplots/before end axis/.code=,
	/pgfplots/after end axis/.code=,
	/pgfplots/extra description/.code=,
	/pgfplots/extra description/.belongs to family=/pgfplots/descriptions,
% axis options:
	/pgfplots/at/.code={\tikz@scan@one@point\pgfplots@set@at#1},
	/pgfplots/at/.belongs to family=/pgfplots,
	/pgfplots/clip limits/.is if=pgfplots@clip@limits,
	/pgfplots/clip limits/.default=true,
	/pgfplots/clip limits=true,
	/pgfplots/clip limits/.belongs to family=/pgfplots,
	/pgfplots/xmin/.belongs to family=/pgfplots,
	/pgfplots/xmin/.initial=,
	/pgfplots/xmax/.belongs to family=/pgfplots,
	/pgfplots/xmax/.initial=,
	/pgfplots/ymin/.belongs to family=/pgfplots,
	/pgfplots/ymin/.initial=,
	/pgfplots/ymax/.belongs to family=/pgfplots,
	/pgfplots/ymax/.initial=,
	/pgfplots/stack plots/.is choice,
	/pgfplots/stack plots/.belongs to family=/pgfplots,
	/pgfplots/stack plots/x/.code={\pgfplots@stacked@xtrue\pgfplots@stackedmodetrue},
	/pgfplots/stack plots/x/.belongs to family=/pgfplots,
	/pgfplots/stack plots/y/.code={\pgfplots@stacked@xfalse\pgfplots@stackedmodetrue},
	/pgfplots/stack plots/y/.belongs to family=/pgfplots,
	/pgfplots/stack plots/false/.code={\pgfplots@stackedmodefalse},
	/pgfplots/stack plots/false/.belongs to family=/pgfplots,
	/pgfplots/stack plots=false,
	/pgfplots/reverse stacked plots/.is if=pgfplots@stacked@reverse,
	/pgfplots/reverse stacked plots/.belongs to family=/pgfplots,
	/pgfplots/reverse stacked plots/.default=true,
	/pgfplots/reverse stacked plots=true,
	/pgfplots/stack dir/.is choice,
	/pgfplots/stack dir/.belongs to family=/pgfplots,
	/pgfplots/stack dir/plus/.code={\pgfplots@stacked@plustrue},
	/pgfplots/stack dir/plus/.belongs to family=/pgfplots,
	/pgfplots/stack dir/minus/.code={\pgfplots@stacked@plusfalse},
	/pgfplots/stack dir/minus/.belongs to family=/pgfplots,
	/pgfplots/stack dir=plus,
	/pgfplots/filter discard warning/.is if=pgfplots@warn@for@filter@discards,
	/pgfplots/filter discard warning=true,
	/pgfplots/x filter/.code={},
	/pgfplots/x filter/.belongs to family=/pgfplots,
	/pgfplots/y filter/.code={},
	/pgfplots/y filter/.belongs to family=/pgfplots,
	/pgfplots/skip coords between index/.style 2 args={%
		/pgfplots/x filter/.append code={%
			\ifnum\coordindex<#1\relax
			\else
				\ifnum\coordindex<#2\relax
					\let\pgfmathresult=\pgfutil@empty
				\fi
			\fi}
	},
	/pgfplots/xfilter/.initial=,% DEPRECATED
	/pgfplots/xfilter/.belongs to family=/pgfplots,
	/pgfplots/yfilter/.initial=,% DEPRECATED
	/pgfplots/yfilter/.belongs to family=/pgfplots,
	/pgfplots/width/.store in=		\pgfplots@width,
	/pgfplots/width/.belongs to family=/pgfplots,
	/pgfplots/width=,
	/pgfplots/height/.store in=	\pgfplots@height,
	/pgfplots/height/.belongs to family=/pgfplots,
	/pgfplots/height=,
	/pgfplots/execute at begin plot/.store in=\pgfplots@execute@at@begin@plot,
	/pgfplots/execute at begin plot/.belongs to family=/pgfplots,
	/pgfplots/execute at begin plot=,
	/pgfplots/execute at end plot/.store in=		\pgfplots@execute@at@end@plot,
	/pgfplots/execute at end plot/.belongs to family=/pgfplots,
	/pgfplots/execute at end plot=,
	/pgfplots/enlarge x limits/.initial=auto,
	/pgfplots/enlarge x limits/.default=true,
	/pgfplots/enlarge y limits/.initial=auto,
	/pgfplots/enlarge y limits/.default=true,
	/pgfplots/enlargelimits/.style={%
		/pgfplots/enlarge x limits=#1,%
		/pgfplots/enlarge y limits=#1,%
	},%
	/pgfplots/enlargelimits/.default=true,
	/pgfplots/x/.store in=		\pgfplots@x,
	/pgfplots/x/.belongs to family=/pgfplots,
	/pgfplots/x=,% is implicitly set by 'width' and/or '\axisdefaultwidth'
	/pgfplots/y/.store in=		\pgfplots@y,
	/pgfplots/y/.belongs to family=/pgfplots,
	/pgfplots/y=,% is implicitly set by 'width' and/or '\axisdefaultwidth'
	/pgfplots/cycle list/.code={\pgfplots@assign@list\autoplotspeclist{#1}},
	/pgfplots/cycle list/.belongs to family=/pgfplots,
	/pgfplots/cycle list name/.code={\let\autoplotspeclist=#1\relax},
	/pgfplots/cycle list name/.belongs to family=/pgfplots,
	/pgfplots/cycle list name=\coloredplotspeclist,
	/pgfplots/legend style/.belongs to family=/pgfplots/style commands,
	/pgfplots/legend style/.code={%
		\pgfkeysalso{/pgfplots/every axis legend/.append style={#1}}%
	},
	/pgfplots/label style/.belongs to family=/pgfplots/style commands,
	/pgfplots/label style/.code={%
		\pgfkeysalso{/pgfplots/every axis label/.append style={#1}}%
	},%
	/pgfplots/x label style/.belongs to family=/pgfplots/style commands,
	/pgfplots/x label style/.code={%
		\pgfkeysalso{/pgfplots/every axis x label/.append style={#1}}%
	},
	/pgfplots/y label style/.belongs to family=/pgfplots/style commands,
	/pgfplots/y label style/.code={%
		\pgfkeysalso{/pgfplots/every axis y label/.append style={#1}}%
	},
	/pgfplots/title style/.belongs to family=/pgfplots/style commands,
	/pgfplots/title style/.code={%
		\pgfkeysalso{/pgfplots/every axis title/.append style={#1}}%
	},
	/pgfplots/tick label style/.belongs to family=/pgfplots/style commands,
	/pgfplots/tick label style/.code={%
		\pgfkeysalso{/pgfplots/every tick label/.append style={#1}}%
	},
	/pgfplots/x tick label style/.belongs to family=/pgfplots/style commands,
	/pgfplots/x tick label style/.code={%
		\pgfkeysalso{/pgfplots/every x tick label/.append style={#1}}%
	},
	/pgfplots/y tick label style/.belongs to family=/pgfplots/style commands,
	/pgfplots/y tick label style/.code={%
		\pgfkeysalso{/pgfplots/every y tick label/.append style={#1}}%
	},
	/pgfplots/x tick scale label style/.belongs to family=/pgfplots/style commands,
	/pgfplots/x tick scale label style/.code={%
		\pgfkeysalso{/pgfplots/every x scale tick label/.append style={#1}}%
	},
	/pgfplots/y tick scale label style/.belongs to family=/pgfplots/style commands,
	/pgfplots/y tick scale label style/.code={%
		\pgfkeysalso{/pgfplots/every y scale tick label/.append style={#1}}%
	},
	/pgfplots/tick style/.belongs to family=/pgfplots/style commands,
	/pgfplots/tick style/.code={%
		\pgfkeysalso{/pgfplots/every tick/.append style={#1}}%
	},
	/pgfplots/minor tick style/.belongs to family=/pgfplots/style commands,
	/pgfplots/minor tick style/.code={%
		\pgfkeysalso{/pgfplots/every minor tick/.append style={#1}}%
	},
	/pgfplots/major tick style/.belongs to family=/pgfplots/style commands,
	/pgfplots/major tick style/.code={%
		\pgfkeysalso{/pgfplots/every major tick/.append style={#1}}%
	},
	/pgfplots/x tick style/.belongs to family=/pgfplots/style commands,
	/pgfplots/x tick style/.code={%
		\pgfkeysalso{/pgfplots/every x tick/.append style={#1}}%
	},
	/pgfplots/minor x tick style/.belongs to family=/pgfplots/style commands,
	/pgfplots/minor x tick style/.code={%
		\pgfkeysalso{/pgfplots/every minor x tick/.append style={#1}}%
	},
	/pgfplots/major x tick style/.belongs to family=/pgfplots/style commands,
	/pgfplots/major x tick style/.code={%
		\pgfkeysalso{/pgfplots/every major x tick/.append style={#1}}%
	},
	/pgfplots/y tick style/.belongs to family=/pgfplots/style commands,
	/pgfplots/y tick style/.code={%
		\pgfkeysalso{/pgfplots/every y tick/.append style={#1}}%
	},
	/pgfplots/minor y tick style/.belongs to family=/pgfplots/style commands,
	/pgfplots/minor y tick style/.code={%
		\pgfkeysalso{/pgfplots/every minor y tick/.append style={#1}}%
	},
	/pgfplots/major y tick style/.belongs to family=/pgfplots/style commands,
	/pgfplots/major y tick style/.code={%
		\pgfkeysalso{/pgfplots/every major y tick/.append style={#1}}%
	},
	/pgfplots/grid style/.belongs to family=/pgfplots/style commands,
	/pgfplots/grid style/.code={%
		\pgfkeysalso{/pgfplots/every axis grid/.append style={#1}}%
	},
	/pgfplots/minor grid style/.belongs to family=/pgfplots/style commands,
	/pgfplots/minor grid style/.code={%
		\pgfkeysalso{/pgfplots/every minor grid/.append style={#1}}%
	},
	/pgfplots/major grid style/.belongs to family=/pgfplots/style commands,
	/pgfplots/major grid style/.code={%
		\pgfkeysalso{/pgfplots/every major grid/.append style={#1}}%
	},
	/pgfplots/x grid style/.belongs to family=/pgfplots/style commands,
	/pgfplots/x grid style/.code={%
		\pgfkeysalso{/pgfplots/every axis x grid/.append style={#1}}%
	},
	/pgfplots/minor x grid style/.belongs to family=/pgfplots/style commands,
	/pgfplots/minor x grid style/.code={%
		\pgfkeysalso{/pgfplots/every minor x grid/.append style={#1}}%
	},
	/pgfplots/major x grid style/.belongs to family=/pgfplots/style commands,
	/pgfplots/major x grid style/.code={%
		\pgfkeysalso{/pgfplots/every major x grid/.append style={#1}}%
	},
	/pgfplots/y grid style/.belongs to family=/pgfplots/style commands,
	/pgfplots/y grid style/.code={%
		\pgfkeysalso{/pgfplots/every axis y grid/.append style={#1}}%
	},
	/pgfplots/minor y grid style/.belongs to family=/pgfplots/style commands,
	/pgfplots/minor y grid style/.code={%
		\pgfkeysalso{/pgfplots/every minor y grid/.append style={#1}}%
	},
	/pgfplots/major y grid style/.belongs to family=/pgfplots/style commands,
	/pgfplots/major y grid style/.code={%
		\pgfkeysalso{/pgfplots/every major y grid/.append style={#1}}%
	},
	/pgfplots/disablelogfilter/.is if=pgfplots@disablelogfilter,
	/pgfplots/disablelogfilter/.default=true,
	/pgfplots/disablelogfilter/.belongs to family=/pgfplots,
	/pgfplots/disablelogfilter=false,
	/pgfplots/disabledatascaling/.is if=pgfplots@disabledatascaling,
	/pgfplots/disabledatascaling/.default=true,
	/pgfplots/disabledatascaling/.belongs to family=/pgfplots,
	/pgfplots/disabledatascaling=false,
	/pgfplots/hide x axis/.is if=pgfplots@hide@x,
	/pgfplots/hide x axis/.default=true,
	/pgfplots/hide x axis=false,
	/pgfplots/hide y axis/.is if=pgfplots@hide@y,
	/pgfplots/hide y axis/.default=true,
	/pgfplots/hide y axis=false,
	/pgfplots/hide axis/.style={%
		/pgfplots/hide x axis=#1,
		/pgfplots/hide y axis=#1,
	},
	/pgfplots/hide axis/.default=true,
%	/pgfplots/hide axis/.belongs to family=/pgfplots,
% sets \pgfplots@xaxislinesnum to
% box=0
% bottom=1
% middle=2 ( aliased with center )
% top=3
	/pgfplots/axis x line/.is choice,
	/pgfplots/axis x line/.belongs to family=/pgfplots, %/axis,
	/pgfplots/axis x line/box/.code	={\def\pgfplots@xaxislinesnum{0}\def\pgfplots@xtickposnum{0}},
	/pgfplots/axis x line/box/.belongs to family=/pgfplots, %/axis,
	/pgfplots/axis x line/bottom/.code	={\def\pgfplots@xaxislinesnum{1}\def\pgfplots@xtickposnum{1}},
	/pgfplots/axis x line/bottom/.belongs to family=/pgfplots, %/axis,
	/pgfplots/axis x line/middle/.code	={\def\pgfplots@xaxislinesnum{2}\def\pgfplots@xtickposnum{2}},
	/pgfplots/axis x line/middle/.belongs to family=/pgfplots, %/axis,
	/pgfplots/axis x line/center/.code	={\def\pgfplots@xaxislinesnum{2}\def\pgfplots@xtickposnum{2}},
	/pgfplots/axis x line/center/.belongs to family=/pgfplots, %/axis,
	/pgfplots/axis x line/top/.code	={\def\pgfplots@xaxislinesnum{3}\def\pgfplots@xtickposnum{3}},
	/pgfplots/axis x line/top/.belongs to family=/pgfplots, %/axis,
	/pgfplots/axis x line/none/.code	={\def\pgfplots@xaxislinesnum{4}\def\pgfplots@xtickposnum{4}},
	/pgfplots/axis x line/none/.belongs to family=/pgfplots, %/axis,
	/pgfplots/axis x line=box,
% sets \pgfplots@yaxislinesnum to
% box=0
% left=1
% center=2 ( aliased with middle )
% right=3
	/pgfplots/axis y line/.is choice,
	/pgfplots/axis y line/.belongs to family=/pgfplots, %/axis,
	/pgfplots/axis y line/box/.code	={\def\pgfplots@yaxislinesnum{0}\def\pgfplots@ytickposnum{0}},
	/pgfplots/axis y line/box/.belongs to family=/pgfplots, %/axis,
	/pgfplots/axis y line/left/.code	={\def\pgfplots@yaxislinesnum{1}\def\pgfplots@ytickposnum{1}},
	/pgfplots/axis y line/left/.belongs to family=/pgfplots, %/axis,
	/pgfplots/axis y line/center/.code	={\def\pgfplots@yaxislinesnum{2}\def\pgfplots@ytickposnum{2}},
	/pgfplots/axis y line/center/.belongs to family=/pgfplots, %/axis,
	/pgfplots/axis y line/middle/.code	={\def\pgfplots@yaxislinesnum{2}\def\pgfplots@ytickposnum{2}},
	/pgfplots/axis y line/middle/.belongs to family=/pgfplots, %/axis,
	/pgfplots/axis y line/right/.code	={\def\pgfplots@yaxislinesnum{3}\def\pgfplots@ytickposnum{3}},
	/pgfplots/axis y line/right/.belongs to family=/pgfplots, %/axis,
	/pgfplots/axis y line/none/.code	={\def\pgfplots@yaxislinesnum{4}\def\pgfplots@ytickposnum{4}},
	/pgfplots/axis y line/none/.belongs to family=/pgfplots, %/axis,
	/pgfplots/axis y line=box,
% set \pgfplots@xaxisdiscontnum
% none = 0
% crunch = 1
% open = 2
	/pgfplots/axis x discontinuity/.is choice,
	/pgfplots/axis x discontinuity/.belongs to family=/pgfplots, %/axis,
	/pgfplots/axis x discontinuity/none/.code	={\def\pgfplots@xaxisdiscontnum{0}},
	/pgfplots/axis x discontinuity/none/.belongs to family=/pgfplots, %/axis,
	/pgfplots/axis x discontinuity/crunch/.code	={\def\pgfplots@xaxisdiscontnum{1}},
	/pgfplots/axis x discontinuity/crunch/.belongs to family=/pgfplots, %/axis,
	/pgfplots/axis x discontinuity/parallel/.code	={\def\pgfplots@xaxisdiscontnum{2}},
	/pgfplots/axis x discontinuity/parallel/.belongs to family=/pgfplots, %/axis,
	/pgfplots/axis x discontinuity=none,
% set \pgfplots@yaxisdiscontnum
% none = 0
% crunch = 1
% open = 2
	/pgfplots/axis y discontinuity/.is choice,
	/pgfplots/axis y discontinuity/.belongs to family=/pgfplots, %/axis,
	/pgfplots/axis y discontinuity/none/.code	={\def\pgfplots@yaxisdiscontnum{0}},
	/pgfplots/axis y discontinuity/none/.belongs to family=/pgfplots, %/axis,
	/pgfplots/axis y discontinuity/crunch/.code	={\def\pgfplots@yaxisdiscontnum{1}},
	/pgfplots/axis y discontinuity/crunch/.belongs to family=/pgfplots, %/axis,
	/pgfplots/axis y discontinuity/parallel/.code	={\def\pgfplots@yaxisdiscontnum{2}},
	/pgfplots/axis y discontinuity/parallel/.belongs to family=/pgfplots, %/axis,
	/pgfplots/axis y discontinuity=none,
	/pgfplots/scale only axis/.is if=pgfplots@scale@only@axis,
	/pgfplots/scale only axis/.default=true,
	/pgfplots/scale only axis/.belongs to family=/pgfplots,
	/pgfplots/scale only axis=false,
% sets \pgfplots@xislinear to
% normal=true
% log=false
	/pgfplots/xmode/.is choice,
	/pgfplots/xmode/.belongs to family=/pgfplots/scale,
	/pgfplots/xmode/normal/.code={\pgfplots@xislineartrue},
	/pgfplots/xmode/normal/.belongs to family=/pgfplots/scale,
	/pgfplots/xmode/linear/.code={\pgfplots@xislineartrue},
	/pgfplots/xmode/linear/.belongs to family=/pgfplots/scale,
	/pgfplots/xmode/log/.code={\pgfplots@xislinearfalse},
	/pgfplots/xmode/log/.belongs to family=/pgfplots/scale,
	/pgfplots/xmode=linear,
	/pgfplots/ymode/.is choice,
	/pgfplots/ymode/.belongs to family=/pgfplots/scale,
	/pgfplots/ymode/normal/.code={\pgfplots@yislineartrue},
	/pgfplots/ymode/normal/.belongs to family=/pgfplots/scale,
	/pgfplots/ymode/linear/.code={\pgfplots@yislineartrue},
	/pgfplots/ymode/linear/.belongs to family=/pgfplots/scale,
	/pgfplots/ymode/log/.code={\pgfplots@yislinearfalse},
	/pgfplots/ymode/log/.belongs to family=/pgfplots/scale,
	/pgfplots/ymode=linear,
	/pgfplots/error bars/x fixed/.code=				\def\pgfplots@errorbars@xfixed{#1}\def\pgfplots@errorbars@xmode{0},
	/pgfplots/error bars/x fixed relative/.code=		\def\pgfplots@errorbars@xrel{#1}\def\pgfplots@errorbars@xmode{1},
	/pgfplots/error bars/x explicit/.code=			\def\pgfplots@errorbars@xmode{2},
	/pgfplots/error bars/x explicit relative/.code=	\def\pgfplots@errorbars@xmode{3},
	/pgfplots/error bars/x fixed relative=0,
	/pgfplots/error bars/x fixed=0,
	/pgfplots/error bars/y fixed/.code=				\def\pgfplots@errorbars@yfixed{#1}\def\pgfplots@errorbars@ymode{0},
	/pgfplots/error bars/y fixed relative/.code=		\def\pgfplots@errorbars@yrel{#1}\def\pgfplots@errorbars@ymode{1},
	/pgfplots/error bars/y explicit/.code=			\def\pgfplots@errorbars@ymode{2},
	/pgfplots/error bars/y explicit relative/.code=	\def\pgfplots@errorbars@ymode{3},
	/pgfplots/error bars/y fixed relative=0,
	/pgfplots/error bars/y fixed=0,
	/pgfplots/error bars/x dir/.is choice,
	/pgfplots/error bars/x dir/none/.code={%
		\def\pgfplots@errorbars@xdirection{0}%
		\ifnum\pgfplots@errorbars@ydirection=0
			\pgfplots@errorbars@enabledfalse
		\fi
	},
	/pgfplots/error bars/x dir/plus/.code=				\def\pgfplots@errorbars@xdirection{1}\pgfplots@errorbars@enabledtrue,
	/pgfplots/error bars/x dir/minus/.code=				\def\pgfplots@errorbars@xdirection{2}\pgfplots@errorbars@enabledtrue,
	/pgfplots/error bars/x dir/both/.code=				\def\pgfplots@errorbars@xdirection{3}\pgfplots@errorbars@enabledtrue,
	/pgfplots/error bars/x dir=none,
	/pgfplots/error bars/y dir/.is choice,
	/pgfplots/error bars/y dir/none/.code={%
		\def\pgfplots@errorbars@ydirection{0}%
		\ifnum\pgfplots@errorbars@xdirection=0
			\pgfplots@errorbars@enabledfalse
		\fi
	},
	/pgfplots/error bars/y dir/plus/.code=				\def\pgfplots@errorbars@ydirection{1}\pgfplots@errorbars@enabledtrue,
	/pgfplots/error bars/y dir/minus/.code=				\def\pgfplots@errorbars@ydirection{2}\pgfplots@errorbars@enabledtrue,
	/pgfplots/error bars/y dir/both/.code=				\def\pgfplots@errorbars@ydirection{3}\pgfplots@errorbars@enabledtrue,
	/pgfplots/error bars/y dir=none,
	/pgfplots/error bars/error mark/.initial={-},
	/pgfplots/error bars/error mark options/.initial={rotate=90},
	/pgfplots/error bars/error bar style/.code={%
		\pgfkeysalso{/pgfplots/every error bar/.append style={#1}}%
	},
	/pgfplots/every error bar/.style={thin},
	/pgfplots/every error bar/.append code={\pgfplotsdeprecatedstylecheck{/tikz/every error bar}},
	/pgfplots/error bars/draw error bar/.code 2 args={%
%\message{/pgfplots/error bars/draw error bar:  working with '#1' -- '#2'.}%
		\pgfkeysgetvalue{/pgfplots/error bars/error mark}{\pgfplotserrorbarsmark}%
		\pgfkeysgetvalue{/pgfplots/error bars/error mark options}{\pgfplotserrorbarsmarkopts}%
		\draw #1 -- #2 node[pos=1,sloped,allow upside down] {%
			\expandafter\tikz\expandafter[\pgfplotserrorbarsmarkopts]{%
				\expandafter\pgfuseplotmark\expandafter{\pgfplotserrorbarsmark}%
				\pgfusepath{stroke}}%
		};
	},
	/pgfplots/bar cycle list/.style={/pgfplots/cycle list={%
		{blue,fill=blue!30!white,mark=none},%
		{red,fill=red!30!white,mark=none},%
		{brown!60!black,fill=brown!30!white,mark=none},%
		{black,fill=gray,mark=none},%
		}
	},
	/pgfplots/area cycle list/.style={bar cycle list},
	/pgfplots/ybar/.style={
		bar cycle list,
		tick align=outside,
		/pgfplots/legend image code/.code={\draw[##1,bar width=3pt,yshift=-0.2em,bar shift=0pt] plot coordinates {(0cm,0.8em) (2*\pgfplotbarwidth,0.6em)};},
		/pgf/bar shift={%
				% total width = n*w + (n-1)*skip
				% -> subtract half for centering
				-0.5*(\numplots*\pgfplotbarwidth + (\numplots-1)*#1)  + 
				% the '0.5*w' is for centering
				(.5+\plotnum)*\pgfplotbarwidth + \plotnum*#1},%
		/tikz/ybar,
	},
	/pgfplots/ybar/.default=2pt,
	/pgfplots/ybar/.belongs to family=/pgfplots,
	/pgfplots/xbar/.style={
		bar cycle list,
		tick align=outside,
		/pgfplots/legend image code/.code={\draw[##1,bar width=3pt,yshift=-0.2em,bar shift=0pt] plot coordinates {(0cm,0.8em) (2*\pgfplotbarwidth,0.6em)};},
		/pgf/bar shift={%
				% total width = n*w + (n-1)*skip
				% -> subtract half for centering
				-0.5*(\numplots*\pgfplotbarwidth + (\numplots-1)*#1)  + 
				% the '0.5*w' is for centering
				(.5+\plotnum)*\pgfplotbarwidth + \plotnum*#1},%
		/tikz/xbar,
	},
	/pgfplots/xbar/.default=2pt,
	/pgfplots/xbar/.belongs to family=/pgfplots,
	/pgfplots/ybar interval/.style={%
		bar cycle list,
		x tick label as interval,
		xmajorgrids,
	%	xtick=data,
		/pgfplots/legend image code/.code={\draw[##1,yshift=-0.2em,bar interval width=0.7,bar interval shift=0.5] plot coordinates {(0cm,0.8em) (5pt,0.6em) (10pt,0.6em)};},
		bar interval width={#1/\numplots},
		bar interval shift={(\plotnum+0.5)/\numplots},
		/tikz/ybar interval,
	},
	/pgfplots/ybar interval/.default=1,
	/pgfplots/ybar interval/.belongs to family=/pgfplots,
	/pgfplots/xbar interval/.style={%
		bar cycle list,
		y tick label as interval,
	%	ytick=data,
		ymajorgrids,
		/pgfplots/legend image code/.code={\draw[##1,yshift=-0.2em,bar interval width=0.7,bar interval shift=0.5] plot coordinates {(0cm,0.8em) (5pt,0.6em) (10pt,0.6em)};},
		bar interval width={#1/\numplots},
		bar interval shift={(\plotnum+0.5)/\numplots},
		/tikz/xbar interval,
	},
	/pgfplots/xbar interval/.default=1,
	/pgfplots/xbar interval/.belongs to family=/pgfplots,
	/pgfplots/xbar stacked/.style={
		bar cycle list,
		stack plots=x,
		stack dir=#1,
		/tikz/xbar,
	},
	/pgfplots/xbar stacked/.default=plus,
	/pgfplots/xbar stacked/.belongs to family=/pgfplots,
	/pgfplots/ybar stacked/.style={
		bar cycle list,
		stack plots=y,
		stack dir=#1,
		/tikz/ybar,
	},
	/pgfplots/ybar stacked/.default=plus,
	/pgfplots/ybar stacked/.belongs to family=/pgfplots,
	/pgfplots/xbar interval stacked/.style={
		bar cycle list,
		stack plots=x,
		stack dir=#1,
		/tikz/xbar interval,
	},
	/pgfplots/xbar interval stacked/.default=plus,
	/pgfplots/xbar interval stacked/.belongs to family=/pgfplots,
	/pgfplots/ybar interval stacked/.style={
		bar cycle list,
		stack plots=y,
		stack dir=#1,
		/tikz/ybar interval,
	},
	/pgfplots/ybar interval stacked/.default=plus,
	/pgfplots/ybar interval stacked/.belongs to family=/pgfplots,
	/pgfplots/yticklabel interval boundaries/.style={%
		y tick label as interval,
		yticklabel={$\pgfmathprintnumber{\tick}$ -- $\pgfmathprintnumber{\nexttick}$}
	},
	/pgfplots/xticklabel interval boundaries/.style={%
		x tick label as interval,
		xticklabel={$\pgfmathprintnumber{\tick}$ -- $\pgfmathprintnumber{\nexttick}$}
	},
}

\newif\ifpgfplots@log@tick@isminor@tick@pos
% Checks whether the tick position given as #1.#2=log10(T) belongs to
% T=i*10^j with an integer i>1.
%
% If T=i*10^j,  \ifpgfplots@log@tick@isminor@tick@pos will be set to true and
% \pgfmathresult will contain T.
%
% Otherwise, \ifpgfplots@log@tick@isminor@tick@pos will be set to false and
% pgfmathresult to #1.#2
%
% Arguments:
% #1.#2  the value log10(T)
%
% Implementation:  
% if T = i*10^j,  log10(T) = log10(i) + j.
% That means if log10(T) in \Z,  we have T = 10^j. If not, we need to
% check wether i is an integer. Please note that log10(i) < 1.
%
% Further note: log(T) < 0 <=>  j<0.
% In case j<0, we have 
%   #1.#2 = j + log(i) 
%         = - ( -j - log(i) ) 
%         = - ( -j - 1  + (1-log(i)) )
%         = #1 '.' #2 [ up to the '0.'
% that means #1 = j-1  and #2 = 1-log(i).
\def\pgfplots@is@log@tick@a@minor@tick@pos#1.#2\relax{%
	\pgfmathapproxequalto@{#1.#2}{#1.0}%
	\ifpgfmathcomparison
		% in MOST cases, this here will be true:
		\pgfplots@log@tick@isminor@tick@posfalse
		\def\pgfmathresult{#1.#2}%
	\else
		% I guess this won't happen too often. In fact, it's a very
		% special case.
		\begingroup
		\c@pgf@counta=#1\relax
		\ifnum\c@pgf@counta<0
			\advance\c@pgf@counta by-1
			\pgfmathsubtract@{1}{0.#2}%
			\expandafter\pgfplots@is@log@tick@a@minor@tick@pos@IDENTIFY@LOGi\pgfmathresult\relax
			\ifpgfplots@log@tick@isminor@tick@pos
				\aftergroup\pgfplots@log@tick@isminor@tick@postrue
				\edef\pgfmathresult{\pgfmathresult e\the\c@pgf@counta}%
			\else
			\aftergroup\pgfplots@log@tick@isminor@tick@posfalse
				\def\pgfmathresult{#1.#2}%
			\fi
		\else
			\pgfplots@is@log@tick@a@minor@tick@pos@IDENTIFY@LOGi0.#2\relax
			\ifpgfplots@log@tick@isminor@tick@pos
				\aftergroup\pgfplots@log@tick@isminor@tick@postrue
				\edef\pgfmathresult{\pgfmathresult e\the\c@pgf@counta}%
			\else
				\aftergroup\pgfplots@log@tick@isminor@tick@posfalse
				\def\pgfmathresult{#1.#2}%
			\fi
		\fi
		\pgfmath@smuggleone\pgfmathresult
		\endgroup
	\fi
}

% expects a positive number.
\def\pgfplots@is@log@tick@a@minor@tick@pos@IDENTIFY@LOGi0.#1\relax{%
	\pgfplots@log@tick@isminor@tick@postrue
	\pgfmathapproxequalto@{0.#1}{0.3010299956639}%
	\ifpgfmathcomparison
		\def\pgfmathresult{2}%
	\else
		\pgfmathapproxequalto@{0.#1}{0.4771212547196}%
		\ifpgfmathcomparison
			\def\pgfmathresult{3}%
		\else
			\pgfmathapproxequalto@{0.#1}{0.6020599913279}%
			\ifpgfmathcomparison
				\def\pgfmathresult{4}%
			\else
				\pgfmathapproxequalto@{0.#1}{0.698970004}%
				\ifpgfmathcomparison
					\def\pgfmathresult{5}%
				\else
					\pgfmathapproxequalto@{0.#1}{0.7781512503}%
					\ifpgfmathcomparison
						\def\pgfmathresult{6}%
					\else
						\pgfmathapproxequalto@{0.#1}{0.8450980400}%
						\ifpgfmathcomparison
							\def\pgfmathresult{7}%
						\else
							\pgfmathapproxequalto@{0.#1}{0.9030899869}%
							\ifpgfmathcomparison
								\def\pgfmathresult{8}%
							\else
								\pgfmathapproxequalto@{0.#1}{0.954242509439}%
								\ifpgfmathcomparison
									\def\pgfmathresult{9}%
								\else
									\pgfplots@log@tick@isminor@tick@posfalse
								\fi
							\fi
						\fi
					\fi
				\fi
			\fi
		\fi
	\fi
}


\def\pgfplots@set@at#1{\def\pgfplots@at{#1}}%



% converts a comma-separated list (PGF foreach)  to my internal list
% structure.
\long\def\pgfplots@foreach@to@list#1\to#2{%
	\pgfplotslistnewempty\pgfplots@TMP
	\begingroup
	\foreach \i in {#1} {%
		\globaldefs=1
		\expandafter\pgfplotslistpushback\i\to\pgfplots@TMP
		\globaldefs=0
	}%
	\endgroup
	\pgfplotslistcopy\pgfplots@TMP\to#2\relax
}

% Several tick options:
\long\def\axispreset#1{%
	\pgfplotsset{every axis/.append style={#1}}%
}
\long\def\legendpreset#1{%
	\pgfplots@error{Sorry, legendpreset is now deprecated, along with the legend options text width and font. Legends are now TikZ-matrizes which provide better alignment and can be placed horizontally. See the manual for details.}%
}

% Checks whether we need to create a separate 'tick scale label',
% a node with ' * 10^3' on the side of the axis:
%
% PRECONDITION:
%    Axis limits for #1 are given. I need their values before any data
%    scale transformation has been applied.
%    If  
%    	\pgfplots@#1min@unscaled@as@float 
%    and
%    	\pgfplots@#1max@unscaled@as@float 
%    exist; I will use these macros.
%    Otherwise, I will use \pgfplots@#1min and \pgfplots@#1max;
%    assuming that no data scale transformation is active.
%    FIXME : does that need further attention?
\def\pgfplots@init@scaled@tick@for#1{%
	\begingroup
	% the \pgfplots@xmin@unscaled@as@float  is set just before the data
	% scale transformation is initialised.
	%
	% The variables are empty if there is no datascale transformation.
	\expandafter\let\expandafter\pgfplots@cur@min@unscaled\csname pgfplots@#1min@unscaled@as@float\endcsname
	\expandafter\let\expandafter\pgfplots@cur@max@unscaled\csname pgfplots@#1max@unscaled@as@float\endcsname
	%
	\ifx\pgfplots@cur@min@unscaled\pgfutil@empty
		\xdef\pgfplots@TMP{\csname pgfplots@#1min\endcsname}%
		\expandafter\pgfmathfloatparsenumber\expandafter{\pgfplots@TMP}%
		\let\pgfplots@cur@min@unscaled=\pgfmathresult
		\xdef\pgfplots@TMP{\csname pgfplots@#1max\endcsname}%
		\expandafter\pgfmathfloatparsenumber\expandafter{\pgfplots@TMP}%
		\let\pgfplots@cur@max@unscaled=\pgfmathresult
	\fi
	%
	\expandafter\pgfmathfloat@decompose@E\pgfplots@cur@min@unscaled\relax\pgfmathfloat@a@E
	\expandafter\pgfmathfloat@decompose@E\pgfplots@cur@max@unscaled\relax\pgfmathfloat@b@E
	\ifnum\pgfmathfloat@b@E<\pgfmathfloat@a@E
		\pgfmathfloat@b@E=\pgfmathfloat@a@E
	\fi
	\xdef\pgfplots@TMP{\pgfplots@scale@ticks@above@exponent}%
	\expandafter\ifnum\pgfplots@TMP<\pgfmathfloat@b@E
		% ok, scale it:
		\multiply\pgfmathfloat@b@E by-1
		\xdef\pgfplots@TMP{\the\pgfmathfloat@b@E}%
	\else
		\xdef\pgfplots@TMP{\pgfplots@scale@ticks@below@exponent}%
		\expandafter\ifnum\pgfplots@TMP>\pgfmathfloat@b@E
			% ok, scale it:
			\multiply\pgfmathfloat@b@E by-1
			\xdef\pgfplots@TMP{\the\pgfmathfloat@b@E}%
		\else
			% no scaling necessary:
			\xdef\pgfplots@TMP{0}%
		\fi
	\fi
	\endgroup
	\expandafter\let\csname pgfplots@tick@scale@#1\endcsname=\pgfplots@TMP%
}

% x-axis tick labels for #1th tick
% #1: the axis (x or y)
% #2: the value
% #3,#4: coordinates for tikz
% #5: ticknumber
\def\pgfplots@show@ticklabel#1#2(#3,#4)#5{{%
	\csname ifpgfplots@#1ticklabel@interval\endcsname
		\pgfmathparse{#3}%
		\edef\pgfplots@show@ticklabel@coord@x@new{\pgfmathresult pt}%
		\pgfmathparse{#4}%
		\edef\pgfplots@show@ticklabel@coord@y@new{\pgfmathresult pt}%
		%
		\pgfplots@show@ticklabel@{#1}{#2}%
		\let\nexttick=\tick
		\ifx\pgfplots@show@ticklabel@LASTTICK\pgfutil@empty
			% its the first call. Simply remember arguments and wait
			% for interval boundary before proceeding.
		\else
			% acquire options of first interval boundary:
			\pgfplots@show@ticklabel@LASTTICK
			% compute new node position:
			\pgfmathparse{0.5*(\csname pgfplots@show@ticklabel@coord@#1\endcsname + \csname pgfplots@show@ticklabel@coord@#1@new\endcsname)}%
			\expandafter\edef\csname pgfplots@show@ticklabel@coord@#1\endcsname{\pgfmathresult pt}%
			\let\ticknum=\pgfplots@show@ticklabel@num\relax%
			\let\tick=\pgfplots@show@ticklabel@tick%
			\xdef\pgfplots@TMP{at (\pgfplots@show@ticklabel@coord@x,\pgfplots@show@ticklabel@coord@y)}%
			\expandafter\node\pgfplots@TMP
				{\csname pgfplots@#1ticklabel\endcsname};%
		\fi
		\xdef\pgfplots@show@ticklabel@LASTTICK{%
			\noexpand\def\noexpand\pgfplots@show@ticklabel@tick{\nexttick}%
			\noexpand\def\noexpand\pgfplots@show@ticklabel@coord@x{\pgfplots@show@ticklabel@coord@x@new}%
			\noexpand\def\noexpand\pgfplots@show@ticklabel@coord@y{\pgfplots@show@ticklabel@coord@y@new}%
			\noexpand\edef\noexpand\pgfplots@show@ticklabel@num{#5}%
		}%
	\else
		\let\ticknum=#5\relax%
		\pgfplots@show@ticklabel@{#1}{#2}%
		\node at (#3,#4) {\csname pgfplots@#1ticklabel\endcsname};%
	\fi
}}

% Defines \tick by applying any necessary math to the (possibly
% transformed) tick value #2.
%
% #1: axis (x or y)
% #2: tick value.
\def\pgfplots@show@ticklabel@#1#2{%
	\csname ifpgfplots@apply@datatrafo@#1\endcsname
		\csname pgfplots@inverse@datascaletrafo@#1\endcsname{#2}%
		% FIXME: do that for any linear axis, also if no datascale is
		% applied!
		\ifpgfplots@scaled@ticks
			\expandafter\pgfmathfloatshift@\expandafter{\pgfmathresult}{\csname pgfplots@tick@scale@#1\endcsname}%
		\fi
		% .. and this here provides \tick as fixed point repr:
		\expandafter\pgfmathfloattofixed\expandafter{\pgfmathresult}%
		\let\tick=\pgfmathresult
	\else
		\edef\tick{#2}%
	\fi
}%

\def\pgfplots@user@ticklabel@list@x{%
	\pgfplotslistselectorempty\ticknum\of\pgfplots@xticklabels\to\tick
	\tick
}
\def\pgfplots@user@ticklabel@list@y{%
	\pgfplotslistselectorempty\ticknum\of\pgfplots@yticklabels\to\tick
	\tick
}
\def\pgfplots@user@extra@ticklabel@list@x{%
	\pgfplotslistselectorempty\ticknum\of\pgfplots@extra@xticklabels\to\tick
	\tick
}
\def\pgfplots@user@extra@ticklabel@list@y{%
	\pgfplotslistselectorempty\ticknum\of\pgfplots@extra@yticklabels\to\tick
	\tick
}

\def\axisdefaultticklabel{%
	$\pgfmathprintnumber{\tick}$%
}

\def\axisdefaultticklabellog{%
	\pgfkeysgetvalue{/pgfplots/log number format code/.@cmd}\pgfplots@log@label@style
	\expandafter\pgfplots@log@label@style\tick\pgfeov
}
	

\def\pgfplots@show@label#1{%
	\node 
		[/pgfplots/every axis label,%
		/pgfplots/every axis #1 label]
	{\csname pgfplots@#1label\endcsname};
}

\def\pgfplots@show@title{%
	\node%
		[/pgfplots/every axis title]
		{\pgfplots@title};
}

% Check if a label does not cross the x-axis
\def\pgfplots@ytick@check@tickshow{%
	\pgfplots@tickshowtrue
	\ifnum\pgfplots@yaxislinesnum=2\relax
		\ifcase\pgfplots@xaxislinesnum\relax
			\ifdim\pgfplots@tmpa=\pgfplots@ycoordminTEX\relax
				\pgfplots@tickshowfalse
			\fi
			\ifdim\pgfplots@tmpa=\pgfplots@ycoordmaxTEX\relax
				\pgfplots@tickshowfalse
			\fi
		\or
			\ifdim\pgfplots@tmpa=\pgfplots@ycoordminTEX\relax
				\pgfplots@tickshowfalse
			\fi
		\or
			\ifdim\pgfplots@tmpa=\pgfplots@ZERO@y\relax
				\pgfplots@tickshowfalse
			\fi
		\or
			\ifdim\pgfplots@tmpa=\pgfplots@ycoordmaxTEX\relax
				\pgfplots@tickshowfalse
			\fi
		\fi
	\fi
}

% Fills the macros
%   \pgfplots@tick@beg@a \pgfplots@tick@end@a
%   \pgfplots@tick@beg@b \pgfplots@tick@end@b
% with coordinates such that 
%   (\pgfplots@tick@beg@a,\pgfplots@tmpa) -- (\pgfplots@tick@end@a,\pgfplots@tmpa)
% produces a correct tick line. 
%
% The '@b' variant is only use in case of \pgfplots@ytickposnum = 0
%
% #1 : the current axis (x or y).
% #2 : the axis which is NOT fixed. For x tick lines, this is the y
% axis and for y tick lines, this is the x axis (#1 = x or y)
% #3 : the current tick width
\def\pgfplots@prepare@tick@offsets@for@#1#2#3{%
	\ifcase\csname pgfplots@#1tickposnum\endcsname\relax
		%(\pgfplots@xcoordminTEX-\pgfplots@tick@offset, \pgfplots@tmpa) -- ++( #3, 0pt)
		\edef\pgfplots@tick@beg@a{\pgf@sys@tonumber{\csname pgfplots@#2coordminTEX\endcsname}}%
		\pgfmathsubtract@{\pgfplots@tick@beg@a}{\pgfplots@tick@offset}%
		\let\pgfplots@tick@beg@a=\pgfmathresult
		\pgfmathadd@{\pgfplots@tick@beg@a}{#3}%
		\edef\pgfplots@tick@beg@a{\pgfplots@tick@beg@a pt}%
		\edef\pgfplots@tick@end@a{\pgfmathresult pt}%
		%
		%(\pgfplots@xcoordmaxTEX+\pgfplots@tick@offset, \pgfplots@tmpa)	-- ++(-#3, 0pt)
		\edef\pgfplots@tick@beg@b{\pgf@sys@tonumber{\csname pgfplots@#2coordmaxTEX\endcsname}}%
		\pgfmathadd@{\pgfplots@tick@beg@b}{\pgfplots@tick@offset}%
		\let\pgfplots@tick@beg@b=\pgfmathresult
		\pgfmathsubtract@{\pgfplots@tick@beg@b}{#3}%
		\edef\pgfplots@tick@beg@b{\pgfplots@tick@beg@b pt}%
		\edef\pgfplots@tick@end@b{\pgfmathresult pt}%
	\or
		% (\pgfplots@xcoordminTEX-\pgfplots@tick@offset, \pgfplots@tmpa) -- ++( #3, 0pt);
		\edef\pgfplots@tick@beg@a{\pgf@sys@tonumber{\csname pgfplots@#2coordminTEX\endcsname}}%
		\pgfmathsubtract@{\pgfplots@tick@beg@a}{\pgfplots@tick@offset}%
		\let\pgfplots@tick@beg@a=\pgfmathresult
		\pgfmathadd@{\pgfplots@tick@beg@a}{#3}%
		\edef\pgfplots@tick@beg@a{\pgfplots@tick@beg@a pt}%
		\edef\pgfplots@tick@end@a{\pgfmathresult pt}%
	\or
		% (\pgfplots@ZERO@x      -\pgfplots@tick@offset, \pgfplots@tmpa) -- ++( #3, 0pt);
		\pgfmathsubtract@{\csname pgfplots@ZERO@#2\endcsname}{\pgfplots@tick@offset}%
		\let\pgfplots@tick@beg@a=\pgfmathresult
		\pgfmathadd@{\pgfplots@tick@beg@a}{#3}%
		\edef\pgfplots@tick@beg@a{\pgfplots@tick@beg@a pt}%
		\edef\pgfplots@tick@end@a{\pgfmathresult pt}%
	\or
		% (\pgfplots@xcoordmaxTEX+\pgfplots@tick@offset, \pgfplots@tmpa)	-- ++(-#3, 0pt);
		\edef\pgfplots@tick@beg@a{\pgf@sys@tonumber{\csname pgfplots@#2coordmaxTEX\endcsname}}%
		\pgfmathadd@{\pgfplots@tick@beg@a}{\pgfplots@tick@offset}%
		\let\pgfplots@tick@beg@a=\pgfmathresult
		\pgfmathsubtract@{\pgfplots@tick@beg@a}{#3}%
		\edef\pgfplots@tick@beg@a{\pgfplots@tick@beg@a pt}%
		\edef\pgfplots@tick@end@a{\pgfmathresult pt}%
	\fi
}%

\def\pgfplots@drawyaxis@placecomputedtick{%
	\pgfpathmoveto{\pgfqpoint{\pgfplots@tick@beg@a}{\pgfplots@tmpa}}%
	\pgfpathlineto{\pgfqpoint{\pgfplots@tick@end@a}{\pgfplots@tmpa}}%
	\ifnum\pgfplots@ytickposnum=0\relax
		\pgfpathmoveto{\pgfqpoint{\pgfplots@tick@beg@b}{\pgfplots@tmpa}}%
		\pgfpathlineto{\pgfqpoint{\pgfplots@tick@end@b}{\pgfplots@tmpa}}%
	\fi
}%
\def\pgfplots@drawyaxis@placecomputedgridline{%
	\pgfpathmoveto{\pgfqpoint{\pgfplots@xcoordminTEX}{\pgfplots@tmpa}}%
	\pgfpathlineto{\pgfqpoint{\pgfplots@xcoordmaxTEX}{\pgfplots@tmpa}}%
}%

\newif\ifpgfplots@needsminorloop

\def\pgfplots@draw@tick@scale@label@for#1{%
	\csname ifpgfplots@#1islinear\endcsname
		\ifpgfplots@scaled@ticks
			\begingroup
				\xdef\pgfplots@TMP{\csname pgfplots@tick@scale@#1\endcsname}%
				\expandafter\c@pgf@counta\pgfplots@TMP\relax
				\multiply\c@pgf@counta by-1
				\ifnum\c@pgf@counta=0\relax
					\global\let\pgfplots@TMP=\pgfutil@empty
				\else
					\xdef\pgfplots@TMP{\the\c@pgf@counta}%
				\fi
			\endgroup
			\ifx\pgfplots@TMP\pgfutil@empty
			\else
				\edef\pgfplots@tick@scale@labels{\noexpand\pgfplots@invoke@pgfkeyscode{/pgfplots/tick scale label code/.@cmd}{\pgfplots@TMP}}%
				\node 
					[%
					xshift=\pgfplots@xcoordminTEX,%
					yshift=\pgfplots@ycoordminTEX,%
					x=\pgfplots@xcoordmaxTEX-\pgfplots@xcoordminTEX,%
					y=\pgfplots@ycoordmaxTEX-\pgfplots@ycoordminTEX,%
					/pgfplots/every tick label,%
					/pgfplots/every #1 tick label,%
					/pgfplots/every #1 tick scale label]
				{\pgfplots@tick@scale@labels};
			\fi
		\fi
	\fi
}

% Simply invokes the code of PGF key #1 with value #2, that means
% #1#2\pgfeov
\def\pgfplots@invoke@pgfkeyscode#1#2{%
	\pgfkeysgetvalue{#1}\pgfplots@invoke@pgfkeyscode@CODE
	\pgfplots@invoke@pgfkeyscode@CODE#2\pgfeov
}

% Check if the current tick position, stored in \pgfplots@tmpa, 
% does not cross the y-axis.
%
% This is just a special case for centered axis lines.
\def\pgfplots@xtick@check@tickshow{%
	\pgfplots@tickshowtrue
	\ifnum\pgfplots@xaxislinesnum=2\relax
		\ifcase\pgfplots@yaxislinesnum\relax
			\ifdim\pgfplots@tmpa=\pgfplots@xcoordminTEX\relax
				\pgfplots@tickshowfalse
			\fi
			\ifdim\pgfplots@tmpa=\pgfplots@xcoordmaxTEX\relax
				\pgfplots@tickshowfalse
			\fi
		\or
			\ifdim\pgfplots@tmpa=\pgfplots@xcoordminTEX\relax
				\pgfplots@tickshowfalse
			\fi
		\or
			\ifdim\pgfplots@tmpa=\pgfplots@ZERO@x\relax
				\pgfplots@tickshowfalse
			\fi
		\or
			\ifdim\pgfplots@tmpa=\pgfplots@xcoordmaxTEX\relax
				\pgfplots@tickshowfalse
			\fi
		\fi
	\fi
}

\def\pgfplots@drawxaxis@placecomputedtick{%
	\pgfpathmoveto{\pgfqpoint{\pgfplots@tmpa}{\pgfplots@tick@beg@a}}%
	\pgfpathlineto{\pgfqpoint{\pgfplots@tmpa}{\pgfplots@tick@end@a}}%
	\ifnum\pgfplots@xtickposnum=0\relax
		\pgfpathmoveto{\pgfqpoint{\pgfplots@tmpa}{\pgfplots@tick@beg@b}}%
		\pgfpathlineto{\pgfqpoint{\pgfplots@tmpa}{\pgfplots@tick@end@b}}%
	\fi
}%
\def\pgfplots@drawxaxis@placecomputedgridline{%
	\pgfpathmoveto{\pgfqpoint{\pgfplots@tmpa}{\pgfplots@ycoordminTEX}}%
	\pgfpathlineto{\pgfqpoint{\pgfplots@tmpa}{\pgfplots@ycoordmaxTEX}}%
}%

% #1: axis (x or y)
% #2: axis index (0 or 1)
% #3: tick position list
\def\pgfplots@draw@extra@ticks@for#1#2#3{%
	\begingroup
	\pgfplots@scaled@ticksfalse
	\csname pgfplots@#1minorticksfalse\endcsname
	\csname pgfplots@#1minorgridsfalse\endcsname
	\expandafter\let\expandafter\axis@TMP\csname pgfplots@extra@#1ticklabel\endcsname
	\expandafter\let\csname pgfplots@#1ticklabel\endcsname=\axis@TMP
	% FIXME: that search path - thing is far from perfect.
	% In fact, it doesn't even work for choice keys... with
	% unexpected results.
	% I have fixed that in PGF CVS - but PGF 2.0 does not yet work
	% here!
	\pgfqkeys{/pgfplots/search path for tikz}{/pgfplots/every extra #1 tick}%
	\pgfplots@prepare@ticks@for{#3}{#1}{#2}%
	\pgfplots@drawgridlines@for{#1}{#2}%
	\pgfplots@drawticklines@for{#1}{#2}%
	\pgfplots@drawticklabels@for{#1}{#2}%
	\endgroup
}

\def\pgfplots@drawaxis@innerlines{%
	\ifpgfplots@hide@y\else
		\ifnum\pgfplots@yaxislinesnum=2
			\draw[/pgfplots/every inner y axis line,%
				decorate,%
				ydiscont,%
				decoration={pre length=\ydisstart, post length=\ydisend}]
				(\pgfplots@ZERO@x,	\pgfplots@ycoordminTEX) -- (\pgfplots@ZERO@x,	\pgfplots@ycoordmaxTEX);
		\fi
	\fi
	\ifpgfplots@hide@x\else
		\ifnum\pgfplots@xaxislinesnum=2
			\draw[/pgfplots/every inner x axis line,%
				decorate,%
				xdiscont,%
				decoration={pre length=\xdisstart, post length=\xdisend}]
				(\pgfplots@xcoordminTEX, \pgfplots@ZERO@y) -- (\pgfplots@xcoordmaxTEX, \pgfplots@ZERO@y);
		\fi
	\fi
}%

% Ok, we don't mind whether edges with thick lines look ugly. We just
% draw separate lines. This here is necessary if we want arrow heads.
\def\pgfplots@drawaxis@outerlines@separate{%
	\ifpgfplots@hide@x
	\else
		\begin{scope}[/pgfplots/every outer x axis line,
			xdiscont,decoration={pre length=\xdisstart, post length=\xdisend}]
		\ifcase\pgfplots@xaxislinesnum
			\draw decorate { 
					(\pgfplots@xcoordminTEX,\pgfplots@ycoordminTEX) 
				--	(\pgfplots@xcoordmaxTEX,\pgfplots@ycoordminTEX) };
			\draw decorate {
					(\pgfplots@xcoordminTEX,\pgfplots@ycoordmaxTEX)
				--	(\pgfplots@xcoordmaxTEX,\pgfplots@ycoordmaxTEX) };

		\or
			\draw decorate { 
					(\pgfplots@xcoordminTEX,\pgfplots@ycoordminTEX) 
				--	(\pgfplots@xcoordmaxTEX,\pgfplots@ycoordminTEX) };
		\or 
		\or
			\draw decorate {
					(\pgfplots@xcoordminTEX,\pgfplots@ycoordmaxTEX)
				--	(\pgfplots@xcoordmaxTEX,\pgfplots@ycoordmaxTEX) };

		\fi
		\end{scope}
	\fi
	\ifpgfplots@hide@y
	\else
		\begin{scope}[/pgfplots/every outer y axis line,
			ydiscont,decoration={pre length=\ydisstart, post length=\ydisend}]
		\ifcase\pgfplots@yaxislinesnum
			\draw decorate { 
					(\pgfplots@xcoordminTEX,\pgfplots@ycoordminTEX) 
				--	(\pgfplots@xcoordminTEX,\pgfplots@ycoordmaxTEX) };
			\draw decorate {
					(\pgfplots@xcoordmaxTEX,\pgfplots@ycoordminTEX)
				--	(\pgfplots@xcoordmaxTEX,\pgfplots@ycoordmaxTEX) };

		\or
			\draw decorate { 
					(\pgfplots@xcoordminTEX,\pgfplots@ycoordminTEX) 
				--	(\pgfplots@xcoordminTEX,\pgfplots@ycoordmaxTEX) };
		\or 
		\or
			\draw decorate {
					(\pgfplots@xcoordmaxTEX,\pgfplots@ycoordminTEX)
				--	(\pgfplots@xcoordmaxTEX,\pgfplots@ycoordmaxTEX) };

		\fi
		\end{scope}
	\fi
}%

% This here is complicated: we try to create good edges and draw a
% SINGLE path for the partial or complete rectangle
%
%  -----
%  |   |
%  |   |
%  -----
\def\pgfplots@drawaxis@outerlines@cycledpath{%
	\draw[
		/pgfplots/every outer x axis line, % FIXME! these outer styles need much more attention :-(
		/pgfplots/every outer y axis line]
	(\pgfplots@xcoordminTEX,	\pgfplots@ycoordminTEX)
\ifpgfplots@hide@y
	{ (\pgfplots@xcoordminTEX,	\pgfplots@ycoordmaxTEX) }
\else
 	\ifcase\pgfplots@yaxislinesnum
	    decorate [ydiscont,decoration={pre length=\ydisstart, post length=\ydisend}] { -- (\pgfplots@xcoordminTEX,	\pgfplots@ycoordmaxTEX) }
	\or decorate [ydiscont,decoration={pre length=\ydisstart, post length=\ydisend}] { -- (\pgfplots@xcoordminTEX,	\pgfplots@ycoordmaxTEX) }
	\or { (\pgfplots@xcoordminTEX,	\pgfplots@ycoordmaxTEX) }
	\or { (\pgfplots@xcoordminTEX,	\pgfplots@ycoordmaxTEX) }
	\fi
\fi
\ifpgfplots@hide@x
	{ (\pgfplots@xcoordmaxTEX,	\pgfplots@ycoordmaxTEX) }
\else
	\ifcase\pgfplots@xaxislinesnum
	    decorate [xdiscont,decoration={pre length=\xdisstart, post length=\xdisend}] { -- (\pgfplots@xcoordmaxTEX,	\pgfplots@ycoordmaxTEX) }
	\or { (\pgfplots@xcoordmaxTEX,	\pgfplots@ycoordmaxTEX) }
	\or { (\pgfplots@xcoordmaxTEX,	\pgfplots@ycoordmaxTEX) }
	\or decorate [xdiscont,decoration={pre length=\xdisstart, post length=\xdisend}] { -- (\pgfplots@xcoordmaxTEX,	\pgfplots@ycoordmaxTEX) }
	\fi
\fi
\ifpgfplots@hide@y
	{ (\pgfplots@xcoordmaxTEX,	\pgfplots@ycoordminTEX) }
\else
	\ifcase\pgfplots@yaxislinesnum
	    decorate [ydiscont,decoration={pre length=\ydisend, post length=\ydisstart}] { -- (\pgfplots@xcoordmaxTEX,	\pgfplots@ycoordminTEX) }
	\or { (\pgfplots@xcoordmaxTEX,	\pgfplots@ycoordminTEX) }
	\or { (\pgfplots@xcoordmaxTEX,	\pgfplots@ycoordminTEX) }
	\or decorate [ydiscont,decoration={pre length=\ydisend, post length=\ydisstart}] { -- (\pgfplots@xcoordmaxTEX,	\pgfplots@ycoordminTEX) }
	\fi
\fi
\ifpgfplots@hide@x
	{ (\pgfplots@xcoordminTEX,	\pgfplots@ycoordminTEX) }
\else
	\ifcase\pgfplots@xaxislinesnum
	    decorate [xdiscont,decoration={pre length=\xdisend, post length=\xdisstart}] { -- (\pgfplots@xcoordminTEX,	\pgfplots@ycoordminTEX) }
	\or decorate [xdiscont,decoration={pre length=\xdisend, post length=\xdisstart}] { -- (\pgfplots@xcoordminTEX,	\pgfplots@ycoordminTEX) }
	\or { (\pgfplots@xcoordminTEX,	\pgfplots@ycoordminTEX) }
	\or { (\pgfplots@xcoordminTEX,	\pgfplots@ycoordminTEX) }
	\fi% can't use cycle here!
\fi
	; %	-- cycle;
}%

\def\pgfplots@drawaxis@outerlines{%
	\ifpgfplots@separate@axis@lines
		\pgfplots@drawaxis@outerlines@separate
	\else
		\pgfplots@drawaxis@outerlines@cycledpath
	\fi
}%

\def\pgfplots@drawaxis@lines{%
	\begingroup
	\begingroup
	% this group employs several temporary dimension registers
	% and is therefor scoped:
	\let\ydisstart=\pgf@xa
	\let\ydisend=\pgf@xb
	\let\xdisstart=\pgf@ya
	\let\xdisend=\pgf@yb
	\ydisend=\pgfplots@ycoordmaxTEX
	\advance \ydisend by -\pgfplots@ycoordminTEX
	\ifcase\pgfplots@yaxisdiscontnum\relax
		\let\ydiscontstyle=\pgfutil@empty
	\or
		\def\ydiscontstyle{decoration={zigzag,segment length=12pt, amplitude=4pt}}%
		\advance \ydisend by -16pt
	\or
		\def\ydiscontstyle{decoration={ticks,segment length=4pt, amplitude=8pt}}%
		\advance \ydisend by -8pt
	\fi
	\ifpgfplots@apply@datatrafo@y
		\advance \pgfplots@ycoordmaxTEX by\pgfplots@data@scale@trafo@SHIFT@y pt
	\fi
	\ifdim\pgfplots@ycoordmaxTEX<0pt
		\ydisstart=\ydisend
		\ydisend=4pt
	\else
		\ydisstart=4pt
	\fi
	\ifpgfplots@apply@datatrafo@y
		\advance \pgfplots@ycoordmaxTEX by-\pgfplots@data@scale@trafo@SHIFT@y pt
	\fi
	\xdisend=\pgfplots@xcoordmaxTEX
	\advance \xdisend by -\pgfplots@xcoordminTEX
	\ifcase\pgfplots@xaxisdiscontnum\relax
		\let\xdiscontstyle=\pgfutil@empty
	\or
		\def\xdiscontstyle{decoration={zigzag,segment length=12pt, amplitude=4pt}}%
		\advance \xdisend by -16pt
	\or
		\def\xdiscontstyle{decoration={ticks,segment length=4pt, amplitude=8pt}}%
		\advance \xdisend by -8pt
	\fi
	\ifpgfplots@apply@datatrafo@x
		\advance \pgfplots@xcoordmaxTEX by\pgfplots@data@scale@trafo@SHIFT@x pt
	\fi
	\ifdim\pgfplots@xcoordmaxTEX<0pt
		\xdisstart=\xdisend
		\xdisend=4pt
	\else
		\xdisstart=4pt
	\fi
	\ifpgfplots@apply@datatrafo@x
		\advance \pgfplots@xcoordmaxTEX by-\pgfplots@data@scale@trafo@SHIFT@x pt
	\fi
	% carry local computations outside of group:
	\xdef\pgfplots@TMP{%
		\noexpand\def\noexpand\xdisstart{\the\xdisstart}%
		\noexpand\def\noexpand\xdisend{\the\xdisend}%
		\noexpand\def\noexpand\ydisstart{\the\ydisstart}%
		\noexpand\def\noexpand\ydisend{\the\ydisend}%
		\noexpand\pgfkeys{%
			/tikz/xdiscont/.style={\xdiscontstyle},%
			/tikz/ydiscont/.style={\ydiscontstyle}}%
	}%
	\endgroup
	\pgfplots@TMP
	\pgfplots@drawaxis@outerlines%
	\pgfplots@drawaxis@innerlines%
	\endgroup
}%

% Computes final major and minor tick positions into global lists
% \pgfplots@prepared@tick@positions@major@x
% and
% \pgfplots@prepared@tick@positions@minor@x.
%
% The major tick list contains tuples (canvas position,logical position)
% while the minor tick list contains just the canvas position.
%
% #1: the tick list.
% #2: the axis name as character, i.e. 'x' or 'y'
% #3: the axis name as integer, i.e. 0 or 1
\def\pgfplots@prepare@ticks@for#1#2#3{%
	\begingroup
	\expandafter\let\expandafter\ifpgfplots@islinear\csname ifpgfplots@#2islinear\endcsname
	\expandafter\let\expandafter\ifpgfplots@minorticks\csname ifpgfplots@#2minorticks\endcsname
	\expandafter\let\expandafter\ifpgfplots@minorgrids\csname ifpgfplots@#2minorgrids\endcsname
	% these lists need to be global such that I can fill them inside
	% of \foreach statements. And, yes: I have also added a TeX group
	% on my own (but that's not the problem).
	\global\pgfplotslistnewempty\pgfplots@prepared@tick@positions@major
	\global\pgfplotslistnewempty\pgfplots@prepared@tick@positions@minor
	\edef\pgfplots@TMP{#1}%
	\ifx\pgfplots@TMP\pgfutil@empty
	\else
		\ifpgfplots@minorticks
			\pgfplots@needsminorlooptrue
		\else
			\ifpgfplots@minorgrids
				\pgfplots@needsminorlooptrue
			\else
				\pgfplots@needsminorloopfalse
			\fi
		\fi
		\ifpgfplots@needsminorloop
			\ifpgfplots@islinear
				\expandafter\let\expandafter\pgfplots@minor@tick@num\csname pgfplots@minor@#2tick@num\endcsname
				\begingroup
				\c@pgf@counta=\pgfplots@minor@tick@num\relax
				\advance\c@pgf@counta by1\relax
				\pgfplots@tmpa=\csname pgfplots@tick@distance@#2\endcsname pt
				\divide\pgfplots@tmpa by\c@pgf@counta
				\edef\pgfmathresult{\pgf@sys@tonumber{\pgfplots@tmpa}}%
				\pgfmath@smuggleone\pgfmathresult
				\endgroup
				\let\pgfplots@minor@tick@dist=\pgfmathresult
			\else
				\def\pgfplots@minor@tick@num{9}%
			\fi
		\fi
		\foreach \x in {#1} {	
			\ifcase#3\relax
				\pgfextractx{\pgfplots@tmpa}{\pgfpointxy{\x}{0}}%
			\or
				\pgfextracty{\pgfplots@tmpa}{\pgfpointxy{0}{\x}}%
			\fi
			\csname pgfplots@#2tick@check@tickshow\endcsname
			%
			\ifpgfplots@tickshow
				\ifdim\pgfplots@tmpa<\csname pgfplots@#2coordminTEX\endcsname
				\else
					\ifdim\pgfplots@tmpa>\csname pgfplots@#2coordmaxTEX\endcsname
					\else
						{\globaldefs=1
						\xdef\pgfplots@TMP{{\the\pgfplots@tmpa}{\x}}%
						\expandafter\pgfplotslistpushback\pgfplots@TMP\to\pgfplots@prepared@tick@positions@major
						}%
					\fi
				\fi
			\fi
			% X-Axis ticks bottom and top
			% in log:
			%  log( i*10^k ) = log\i + k\log10 -> draw ticks for i=1..9
			\ifpgfplots@needsminorloop
				\foreach \i in {1,...,\pgfplots@minor@tick@num} {
					\begingroup
					\ifcase#3\relax
						\ifpgfplots@islinear
							\pgfextractx{\pgfplots@tmpa}{\pgfpointxy{\i*\pgfplots@minor@tick@dist}{0}}%
						\else
							\pgfextractx{\pgfplots@tmpa}{\pgfpointxy{\logi\i}{0}}%
						\fi
					\or
						\ifpgfplots@islinear
							\pgfextracty{\pgfplots@tmpa}{\pgfpointxy{0}{\i*\pgfplots@minor@tick@dist}}%
						\else
							\pgfextracty{\pgfplots@tmpa}{\pgfpointxy{0}{\logi\i}}%
						\fi
					\fi
					\edef\pgfmathresult{\the\pgfplots@tmpa}%
					\pgfmath@smuggleone\pgfmathresult
					\endgroup
					\advance\pgfplots@tmpa by\pgfmathresult\relax
					\ifdim\pgfplots@tmpa<\csname pgfplots@#2coordminTEX\endcsname
					\else
						\ifdim\pgfplots@tmpa>\csname pgfplots@#2coordmaxTEX\endcsname
						\else
							{\globaldefs=1
							\expandafter\pgfplotslistpushback\the\pgfplots@tmpa\to\pgfplots@prepared@tick@positions@minor
							}%
						\fi
					\fi
				}%
			\fi
		}%
	\fi
	\endgroup
	\expandafter\let\csname pgfplots@prepared@tick@positions@minor@#2\endcsname=\pgfplots@prepared@tick@positions@minor
	\expandafter\let\csname pgfplots@prepared@tick@positions@major@#2\endcsname=\pgfplots@prepared@tick@positions@major
	\global\let\pgfplots@prepared@tick@positions@major=\relax
	\global\let\pgfplots@prepared@tick@positions@minor=\relax
}%

% Converts the tuples in major tick lists into two separate variables.
%
% Usage: 
% \pgfplotslistforeach\pgfplots@prepared@tick@positions@major@x\as\curelement{%
%	 \expandafter\pgfplots@parse@prepared@major@tickpos\curelement\canvas\logical
% }
%
% #1#2 : the tuples as they are found in the prepared major tick list,
% #3   : a dimen register which will be filled with #1
% #4   : a macro which will be assigned with #2
\def\pgfplots@parse@prepared@major@tickpos#1#2#3#4{%
	#3=#1\relax%
	\def#4{#2}%
}%
	

% #1 : the verbatim axis name (either 'x' or 'y')
% #2 : the index of the axis  (either 0 or 1)
\def\pgfplots@drawgridlines@for#1#2{%
	\begingroup
	\expandafter\let\expandafter\ifpgfplots@major\csname ifpgfplots@#1majorgrids\endcsname
	\expandafter\let\expandafter\ifpgfplots@minor\csname ifpgfplots@#1minorgrids\endcsname
	\expandafter\let\expandafter\pgfplots@prepared@tick@positions@major@\csname pgfplots@prepared@tick@positions@major@#1\endcsname
	\expandafter\let\expandafter\pgfplots@prepared@tick@positions@minor@\csname pgfplots@prepared@tick@positions@minor@#1\endcsname
	\pgfplots@loop@CONTINUEfalse
	\ifpgfplots@major
		\pgfplots@loop@CONTINUEtrue
	\fi
	\ifpgfplots@minor
		\pgfplots@loop@CONTINUEtrue
	\fi
	\ifpgfplots@loop@CONTINUE
		\scope
		%\clip 			(\pgfplots@xcoordminTEX\pgfplots@leftover,	\pgfplots@ycoordminTEX\pgfplots@leftover) 
		%	rectangle	(\pgfplots@xcoordmaxTEX\pgfplots@rightover,	\pgfplots@ycoordmaxTEX\pgfplots@rightover);
		\pgfinterruptboundingbox%
		\ifcase#2\relax
			\clip 			(\pgfplots@xcoordminTEX,	\pgfplots@ycoordminTEX-5cm) 
				rectangle	(\pgfplots@xcoordmaxTEX,	\pgfplots@ycoordmaxTEX+5cm);
		\or
			\clip 			(\pgfplots@xcoordminTEX-5cm,	\pgfplots@ycoordminTEX) 
				rectangle	(\pgfplots@xcoordmaxTEX+5cm,	\pgfplots@ycoordmaxTEX);
		\fi
		\endpgfinterruptboundingbox%
		\ifpgfplots@major
			\scope[%
				/pgfplots/every axis grid,
				/pgfplots/every major grid,
				/pgfplots/every axis #1 grid,
				/pgfplots/every major #1 grid]%
			\pgfplotslistforeach\pgfplots@prepared@tick@positions@major@\as\pgfplots@curgridpos{%
				\expandafter\pgfplots@parse@prepared@major@tickpos\pgfplots@curgridpos\pgfplots@tmpa\pgfplots@curgridpos
				\csname pgfplots@draw#1axis@placecomputedgridline\endcsname
			}%
			\pgfusepath{stroke}%
			\endscope
		\fi
		%
		\ifpgfplots@minor
			\scope[%
				/pgfplots/every axis grid,
				/pgfplots/every minor grid,
				/pgfplots/every axis #1 grid,
				/pgfplots/every minor #1 grid]%
			\pgfplotslistforeach\pgfplots@prepared@tick@positions@minor@\as\pgfplots@curgridpos{%
				\pgfplots@tmpa=\pgfplots@curgridpos
				\csname pgfplots@draw#1axis@placecomputedgridline\endcsname
			}%
			\pgfusepath{stroke}%
			\endscope
		\fi
		\endscope
	\fi
	\endgroup
}

% #1 : the verbatim axis name (either 'x' or 'y')
% #2 : the index of the axis  (either 0 or 1)
\def\pgfplots@drawticklines@for#1#2{%
	\scope
	\expandafter\let\expandafter\ifpgfplots@major\csname ifpgfplots@#1majorticks\endcsname
	\expandafter\let\expandafter\ifpgfplots@minor\csname ifpgfplots@#1minorticks\endcsname
	\expandafter\let\expandafter\pgfplots@prepared@tick@positions@major@\csname pgfplots@prepared@tick@positions@major@#1\endcsname
	\expandafter\let\expandafter\pgfplots@prepared@tick@positions@minor@\csname pgfplots@prepared@tick@positions@minor@#1\endcsname
	%\clip 			(\pgfplots@xcoordminTEX\pgfplots@leftover,	\pgfplots@ycoordminTEX\pgfplots@leftover) 
	%	rectangle	(\pgfplots@xcoordmaxTEX\pgfplots@rightover,	\pgfplots@ycoordmaxTEX\pgfplots@rightover);
	\pgfinterruptboundingbox%
	\ifcase#2\relax
		\clip 			(\pgfplots@xcoordminTEX,	\pgfplots@ycoordminTEX-5cm) 
			rectangle	(\pgfplots@xcoordmaxTEX,	\pgfplots@ycoordmaxTEX+5cm);
	\or
		\clip 			(\pgfplots@xcoordminTEX-5cm,	\pgfplots@ycoordminTEX) 
			rectangle	(\pgfplots@xcoordmaxTEX+5cm,	\pgfplots@ycoordmaxTEX);
	\fi
	\endpgfinterruptboundingbox%
	\ifpgfplots@major
		\scope[%
			/pgfplots/every tick,
			/pgfplots/every major tick,
			/pgfplots/every #1 tick,
			/pgfplots/every major #1 tick]%
		\pgfmathparse{\pgfplots@tickwidth}%
		\let\pgfplots@tickwidth@=\pgfmathresult
		\edef\pgfplots@tickwidth{\pgfplots@tickwidth@ pt}%
		\ifcase\csname pgfplots@#1tickalignnum\endcsname\relax
			\def\pgfplots@tick@offset{0}
		\or
			\let\pgfplots@tick@offset=\pgfplots@tickwidth@%
		\or
			\pgfmathmultiply@{0.5}{\pgfplots@tickwidth@}%
			\let\pgfplots@tick@offset=\pgfmathresult%
		\fi
		\ifcase#2\relax
			\pgfplots@prepare@tick@offsets@for@{#1}{y}{\pgfplots@tickwidth@}%
		\else
			\pgfplots@prepare@tick@offsets@for@{#1}{x}{\pgfplots@tickwidth@}%
		\fi
		\pgfplotslistforeach\pgfplots@prepared@tick@positions@major@\as\pgfplots@curtickpos{%
			\expandafter\pgfplots@parse@prepared@major@tickpos\pgfplots@curtickpos\pgfplots@tmpa\pgfplots@curtickpos
			\csname pgfplots@draw#1axis@placecomputedtick\endcsname
		}%
		\pgfusepath{stroke}%
		\endscope
	\fi
	%
	\ifpgfplots@minor
		\scope[%
			/pgfplots/every tick,
			/pgfplots/every minor tick,
			/pgfplots/every #1 tick,
			/pgfplots/every minor #1 tick]%
		\pgfmathparse{\pgfplots@subtickwidth}%
		\let\pgfplots@subtickwidth@=\pgfmathresult
		\edef\pgfplots@subtickwidth{\pgfplots@subtickwidth@ pt}%
		\ifcase\csname pgfplots@#1tickalignnum\endcsname\relax
			\def\pgfplots@tick@offset{0}%
		\or
			\let\pgfplots@tick@offset=\pgfplots@subtickwidth@%
		\or
			\pgfmathmultiply@{0.5}{\pgfplots@subtickwidth@}%
			\let\pgfplots@tick@offset=\pgfmathresult%
		\fi
		\ifcase#2\relax
			\pgfplots@prepare@tick@offsets@for@{#1}{y}{\pgfplots@subtickwidth@}%
		\else
			\pgfplots@prepare@tick@offsets@for@{#1}{x}{\pgfplots@subtickwidth@}%
		\fi
		\pgfplotslistforeach\pgfplots@prepared@tick@positions@minor@\as\pgfplots@curtickpos{%
			\pgfplots@tmpa=\pgfplots@curtickpos
			\csname pgfplots@draw#1axis@placecomputedtick\endcsname
		}%
		\pgfusepath{stroke}%
		\endscope
	\fi
	\endscope
}

% #1 : the verbatim axis name (either 'x' or 'y')
% #2 : the index of the axis  (either 0 or 1)
\def\pgfplots@drawticklabels@for#1#2{%
	\begingroup
	\expandafter\let\expandafter\ifpgfplots@major\csname ifpgfplots@#1majorticks\endcsname
	\expandafter\let\expandafter\ifpgfplots@islinear\csname ifpgfplots@#1islinear\endcsname
	\expandafter\let\expandafter\pgfplots@prepared@tick@positions@major@\csname pgfplots@prepared@tick@positions@major@#1\endcsname
	\ifpgfplots@major
		\ifpgfplots@islinear
			\ifpgfplots@scaled@ticks
				\pgfplots@init@scaled@tick@for{#1}%
			\fi
		\fi
		\begingroup
		\pgfkeys{/tikz/every node/.append style={/pgfplots/every tick label,/pgfplots/every #1 tick label}}%
		\ifcase\csname pgfplots@#1tickalignnum\endcsname\relax
			\def\pgfmathresult{0}%
		\or
			\pgfmathparse{\pgfplots@tickwidth}%
		\or
			\pgfmathmultiply{0.5}{\pgfplots@tickwidth}%
		\fi
		\let\pgfplots@tick@offset=\pgfmathresult
		\ifcase#2\relax
			\ifcase\csname pgfplots@#1tickposnum\endcsname\relax
				\edef\pgfplots@tick@origin{\pgf@sys@tonumber{\pgfplots@ycoordminTEX}}%
			\or
				\edef\pgfplots@tick@origin{\pgf@sys@tonumber{\pgfplots@ycoordminTEX}}%
			\or
				\let\pgfplots@tick@origin=\pgfplots@ZERO@y%
			\or
				\edef\pgfplots@tick@origin{\pgf@sys@tonumber{\pgfplots@ycoordmaxTEX}}%
			\fi
			\def\pgfplots@tickposchoicea{\tikzset{above}}%
			\def\pgfplots@tickposchoiceb{\tikzset{below}}%
		\or
			\ifcase\csname pgfplots@#1tickposnum\endcsname\relax
				\edef\pgfplots@tick@origin{\pgf@sys@tonumber{\pgfplots@xcoordminTEX}}%
			\or
				\edef\pgfplots@tick@origin{\pgf@sys@tonumber{\pgfplots@xcoordminTEX}}%
			\or
				\let\pgfplots@tick@origin=\pgfplots@ZERO@x%
			\or
				\edef\pgfplots@tick@origin{\pgf@sys@tonumber{\pgfplots@xcoordmaxTEX}}%
			\fi
			\def\pgfplots@tickposchoicea{\tikzset{right}}%
			\def\pgfplots@tickposchoiceb{\tikzset{left}}%
		\fi
		\ifnum\csname pgfplots@#1tickposnum\endcsname=3
			\pgfplots@tickposchoicea
			\pgfmathadd@{\pgfplots@tick@origin}{\pgfplots@tick@offset}%
		\else
			\pgfplots@tickposchoiceb
			\pgfmathsubtract@{\pgfplots@tick@origin}{\pgfplots@tick@offset}%
		\fi
		\edef\pgfplots@tick@origin{\pgfmathresult pt}%
		\def\pgfplots@ticknum{0}%
		\xdef\pgfplots@show@ticklabel@LASTTICK{}%
		\pgfplotslistforeachungrouped\pgfplots@prepared@tick@positions@major@\as\pgfplots@curtickpos{%
			\expandafter\pgfplots@parse@prepared@major@tickpos\pgfplots@curtickpos\pgfplots@tmpa\pgfplots@curtickpos
			\ifcase#2\relax
				\pgfplots@show@ticklabel
					{#1}{\pgfplots@curtickpos}(\pgfplots@tmpa,\pgfplots@tick@origin)%
					{\pgfplots@ticknum}%
			\or
				\pgfplots@show@ticklabel
					{#1}{\pgfplots@curtickpos}(\pgfplots@tick@origin,\pgfplots@tmpa)%
					{\pgfplots@ticknum}%
			\fi
			\begingroup
			\c@pgf@counta=\pgfplots@ticknum\relax
			\advance\c@pgf@counta by1
			\edef\pgfplots@ticknum{\the\c@pgf@counta}%
			\pgfmath@smuggleone\pgfplots@ticknum
			\endgroup
		}%
		\endgroup
		\pgfplots@draw@tick@scale@label@for #1%
	\fi
	\endgroup
}

\def\pgfplots@rememberplotspec#1{%
	\begingroup
	\globaldefs=1
	\pgfplotslistpushback{#1}\to\pgfplots@plotspeclist
	\endgroup
}

\def\pgfplots@getautoplotspec into#1{%
	\pgfplotslistsize\autoplotspeclist\to\c@pgf@counta
	\ifnum\c@pgf@counta=0
		\let#1=\pgfutil@empty
	\else
		\c@pgf@countb=\pgfplots@numplots
		% offset modulo size:
		\loop
		\ifnum\c@pgf@countb<\c@pgf@counta
			\pgfplots@loop@CONTINUEfalse
		\else
			\pgfplots@loop@CONTINUEtrue
		\fi
		\ifpgfplots@loop@CONTINUE
			\advance\c@pgf@countb by-\c@pgf@counta
		\repeat
		\pgfplotslistselect\c@pgf@countb\of\autoplotspeclist\to#1
%\pgfplots@message{pgfplots.sty: using \string\autoplotspeclist\ specification no\#\the\c@pgf@countb (of \the\c@pgf@counta): #1}%
	\fi
}

% The main interface to draw a plot into an axis.
%
% Usage:
% \addplot 
% 	plot coordinates {
% 		(0,0)
% 		(1,1)
% 	};
% 
% or
%
% \addplot[color=blue,mark=*]
% 	plot coordinates {
% 		(0,0)
% 		(1,1)
% 	};
% 
% The first syntax will use the next plot specification in the list
% \autoplotspeclist
% and the first will use blue color and * markers. 
%
% The linespec. will be used in the legend.
%
% Low-level implementation:
%
% \pgfplots@addplot 
% \pgfplots@addplotimpl
% \pgfplots@start@plot@with@behavioroptions <--- \begingroup
% ...
% ... remember options GLOBALLY
% ... update limits GLOBALLY
% ... \pgfplots@addplot@enqueue@coords GLOBALLY
% ...
% \pgfplots@end@plot <--- \endgroup
\long\def\pgfplots@addplot{%
	\pgfutil@ifnextchar+{%
		\pgfplots@getautoplotspec into\nextplotspec
		\pgfplots@addplotimplAPPEND
	}{%
		\pgfutil@ifnextchar[{%
			\pgfplots@addplotimpl%
		}{%
			\pgfplots@getautoplotspec into\nextplotspec
			% the space after ']' is required here:
			% FIXME: 
			% - \addplot[]plot coordinates is NOT allowed!?
			\expandafter\pgfplots@addplotimpl\expandafter[\nextplotspec]
		}%
	}%
}

\long\def\pgfplots@addplotimplAPPEND+[{%
	\expandafter\pgfplots@addplotimpl\expandafter[\nextplotspec,
}

\long\def\pgfplots@addplotimpl[#1]{%
	\pgfutil@ifnextchar p{%
		\pgfplots@addplotimpl@plot{#1}%
	}{%
		\pgfplots@addplotimpl@plot{#1}plot
	}%
}

\def\pgfplots@addplotimpl@plot#1plot{%
	\pgfutil@ifnextchar[{%
		\pgfplots@addplotimpl@plot@withoptions{#1}%
	}{%
		\pgfplots@addplotimpl@plot@withoptions{#1}[]%
	}%
}

\def\pgfplots@addplotimpl@plot@withoptions#1[#2]{
	\pgfplots@start@plot@with@behavioroptions{#2}%
	\pgfutil@ifnextchar c{%
		\pgfplots@addplotimpl@coordinates{#1}plot 
	}{%
		\pgfutil@ifnextchar f{%
			\pgfplots@addplotimpl@f{#1}%
		}{%
			\pgfutil@ifnextchar t{%
				\pgfplots@addplotimpl@table{#1}%
			}{%
				\pgfutil@ifnextchar ({%
					\pgfplots@addplotimpl@expression{#1}%
				}{%
					\pgfplots@error{Sorry, the supplied plot command is unknown or unsupported by pgfplots! Ignoring it.}%
					\pgfplots@gobble@until@semicolon
				}%
			}%
		}%
	}%
}

\def\pgfplots@gobble@until@semicolon#1;{}

% Currently, plot expression is really inefficient:
%
% 1. it invokes the math parser to get all coordinates. Ok, that's
% what one expects.
%
% 2. It collects the result into one large list instead of calling
% pgfplots' stream methods. That's because \foreach encapsulates its
% code in at least one TeX-group.
%
% 3. It processes the collected list; applying any floating point
% parser routines - it does NOT know that any numbers are already in
% TeX-precision!
%
% FIXME : configure pgfplots to finish the data scaling trafo! It's
% perfectly ok if it is simply the identity!
\long\def\pgfplots@addplotimpl@expression#1(#2,#3)#4;{%
	\pgfplots@PREPARE@COORD@STREAM{#1}{#4}%
	\edef\pgfplots@plot@data{\noexpand\foreach\expandafter\noexpand\tikz@plot@var in {\tikz@plot@samplesat}}%
	\global\let\pgfplots@TMP=\pgfutil@empty
	\pgfplots@plot@data{%
		\pgfmathparse{#2}%
		\let\pgfplots@current@point@x=\pgfmathresult
		\pgfmathparse{#3}%
		\let\pgfplots@current@point@y=\pgfmathresult
		\pgfplots@toka=\expandafter{\pgfplots@TMP}%
		\xdef\pgfplots@TMP{\the\pgfplots@toka(\pgfplots@current@point@x,\pgfplots@current@point@y)}%
	}%
	\expandafter\pgfplots@coord@stream@foreach\expandafter{\pgfplots@TMP}%
}%

\def\pgfplots@addplotimpl@f#1f{%
	\pgfutil@ifnextchar i{\pgfplots@addplotimpl@file{#1}}{\pgfplots@addplotimpl@function{#1}}%
}%

\let\pgfplots@backupof@pgfplotxyfile=\pgfplotxyfile

% the following code 
% results finally in
%
% set format "%.7e";; set samples <...>; plot ...
%
% The windows port of gnuplot doesn't run without the second semicolon
% - for whatever reason.
{
  \catcode`\%=12
  \catcode`\"=12
  \xdef\pgfplots@gnuplot@format{set format "%.7e";}
}

% #1: args of \addplot[...]
% #2: file name
% #3: trailing path commands until ';'
\def\pgfplots@addplotimpl@function#1unction#2#3;{%
	\def\tikz@plot@filename{\tikz@plot@prefix\tikz@plot@id}%  
	\iftikz@plot@raw@gnuplot%
		\def\tikz@plot@data{\pgfplotgnuplot[\tikz@plot@filename]{#2}}%
	\else%
		\iftikz@plot@parametric%   
			\def\tikz@plot@data{\pgfplotgnuplot[\tikz@plot@filename]{%
				\pgfplots@gnuplot@format;
				set samples \tikz@plot@samples;
				set parametric;
				plot [t=\tikz@plot@domain] #2}}%
		\else%
			\def\tikz@plot@data{\pgfplotgnuplot[\tikz@plot@filename]{%
				\pgfplots@gnuplot@format;
				set samples \tikz@plot@samples;
				plot [x=\tikz@plot@domain] #2}}%
		\fi%
	\fi%
	\def\pgfplotxyfile{\pgfplots@addplotimpl@gnuplotresult{#1}{#3}}%
	\tikz@plot@data
	\let\pgfplotxyfile=\pgfplots@backupof@pgfplotxyfile
}%

\def\pgfplots@addplotimpl@gnuplotresult#1#2#3{%
	\begingroup
	\openin1=#3
	\ifeof1
		\pgfplots@error{Sorry, the gnuplot-result file '#3' could not be found. Maybe you need to enable the shell-escape feature? For pdflatex, this is '>> pdflatex -shell-escape'. You can also invoke '>> gnuplot #3' manually.}%
		\aftergroup\pgfplots@loop@CONTINUEfalse
	\else
		\aftergroup\pgfplots@loop@CONTINUEtrue
	\fi
	\closein1
	\endgroup
	\ifpgfplots@loop@CONTINUE
		\pgfplots@addplotimpl@file{#1}ile{#3}#2;%
	\fi
}

\def\pgfplots@addplotimpl@file#1ile{%
	\pgfplots@addplotimpl@table{#1}table[x index=0,y index=1,header=false]%
}%

\def\pgfplots@addplotimpl@table#1table{%
	\pgfutil@ifnextchar[{%
		\pgfplots@addplotimpl@table@getopts{#1}%
	}{%
		\pgfplots@addplotimpl@table@getopts{#1}[x index=0,y index=1]%
	}%
}%

\def\pgfplots@addplotimpl@table@f#1#2from{%
	\pgfplots@addplot@table@from@macrotrue
	\pgfplots@addplotimpl@table@START{#1}{#2}%
}

\def\pgfplots@addplotimpl@table@getopts#1[#2]{%
	\pgfutil@ifnextchar f{%
		\pgfplots@addplotimpl@table@f{#1}{#2}%
	}{%
		\pgfplots@addplot@table@from@macrofalse
		\pgfplots@addplotimpl@table@START{#1}{#2}%
	}%
}

% #1: arguments to \addplot[...]
% #2: arguments to table[...]
% #3: the argument of plot table{...}
% #4: trailing path arguments after plot table{...}#4;
\long\def\pgfplots@addplotimpl@table@START#1#2#3#4;{%
	\begingroup
	\pgfplotstableset{#2}%
	\ifpgfplots@addplot@table@from@macro
		\pgfplotstablecopy#3\to\pgfplots@table
	\else
		\pgfplotstableread{#3} to \pgfplots@table
	\fi
	\ifx\pgfplots@plot@tbl@x\pgfutil@empty
		\pgfplotstablegetcolumnbyindex\pgfplots@plot@tbl@xindex\of\pgfplots@table\to\addplot@tbl@x
	\else
		\pgfplotstablegetcolumnbyname\pgfplots@plot@tbl@x\of\pgfplots@table\to\addplot@tbl@x
	\fi
	\ifx\pgfplots@plot@tbl@y\pgfutil@empty
		\pgfplotstablegetcolumnbyindex\pgfplots@plot@tbl@yindex\of\pgfplots@table\to\addplot@tbl@y
	\else
		\pgfplotstablegetcolumnbyname\pgfplots@plot@tbl@y\of\pgfplots@table\to\addplot@tbl@y
	\fi
	%
	\ifpgfplots@errorbars@enabled
		\let\addplot@tbl@error@x=\pgfutil@empty
		\let\addplot@tbl@error@y=\pgfutil@empty
		\pgfkeysgetvalue{/pgfplots/table/x error index}\pgfplots@plot@tbl@error@xindex
		\pgfkeysgetvalue{/pgfplots/table/x error}\pgfplots@plot@tbl@error@x
		\pgfkeysgetvalue{/pgfplots/table/y error index}\pgfplots@plot@tbl@error@yindex
		\pgfkeysgetvalue{/pgfplots/table/y error}\pgfplots@plot@tbl@error@y
		\ifx\pgfplots@plot@tbl@error@x\pgfutil@empty
			\ifx\pgfplots@plot@tbl@error@xindex\pgfutil@empty
			\else
				\pgfplotstablegetcolumnbyindex\pgfplots@plot@tbl@error@xindex\of\pgfplots@table\to\addplot@tbl@error@x
			\fi
		\else
			\pgfplotstablegetcolumnbyname\pgfplots@plot@tbl@error@x\of\pgfplots@table\to\addplot@tbl@error@x
		\fi
		\ifx\pgfplots@plot@tbl@error@y\pgfutil@empty
			\ifx\pgfplots@plot@tbl@error@yindex\pgfutil@empty
			\else
				\pgfplotstablegetcolumnbyindex\pgfplots@plot@tbl@error@yindex\of\pgfplots@table\to\addplot@tbl@error@y
			\fi
		\else
			\pgfplotstablegetcolumnbyname\pgfplots@plot@tbl@error@y\of\pgfplots@table\to\addplot@tbl@error@y
		\fi
	\fi
	%
	\pgfplots@PREPARE@COORD@STREAM{#1}{#4}%
	\pgfplots@coord@stream@start
	\loop
	\pgfplotslistcheckempty\addplot@tbl@x
	\ifpgfplotslistempty
		\pgfplots@loop@CONTINUEfalse
	\else
		% This here is just for sanity checking: if the 'y' column is 
		% - for whatever reasons - invalid; provide good error
		%   recovery.
		\pgfplotslistcheckempty\addplot@tbl@y
		\ifpgfplotslistempty
			\pgfplots@loop@CONTINUEfalse
		\else
			\pgfplots@loop@CONTINUEtrue
		\fi
	\fi
	\ifpgfplots@loop@CONTINUE
		\pgfplotslistpopfront\addplot@tbl@x\to\addplot@tbl@cur@x
		\pgfplotslistpopfront\addplot@tbl@y\to\addplot@tbl@cur@y
		\ifpgfplots@errorbars@enabled
			\ifx\addplot@tbl@error@x\pgfutil@empty
				\let\addplot@tbl@error@cur@x=\pgfutil@empty
			\else
				\pgfplotslistpopfront\addplot@tbl@error@x\to\addplot@tbl@error@cur@x
			\fi
			\ifx\addplot@tbl@error@y\pgfutil@empty
				\let\addplot@tbl@error@cur@y=\pgfutil@empty
			\else
				\pgfplotslistpopfront\addplot@tbl@error@y\to\addplot@tbl@error@cur@y
			\fi
			\edef\pgfplots@TMP{{\addplot@tbl@cur@x}{\addplot@tbl@cur@y}{\addplot@tbl@error@cur@x}{\addplot@tbl@error@cur@y}}%
			\expandafter\pgfplots@coord@stream@coord\pgfplots@TMP
		\else
			\edef\pgfplots@TMP{{\addplot@tbl@cur@x}{\addplot@tbl@cur@y}{}{}}%
			\expandafter\pgfplots@coord@stream@coord\pgfplots@TMP
		\fi
	\repeat
	\pgfplots@coord@stream@end
	\endgroup
}

% #1:  arguments to \addplot plot[#1]
%   -> these are called 'behavior' options in the manual; they are set
%   immediately.
\def\pgfplots@start@plot@with@behavioroptions#1{%
	\begingroup
	\def\pgfplots@current@point@coordindex{0}% can be used inside of coordinate filters.
	\def\coordindex{\pgfplots@current@point@coordindex}% valid inside of \addplot
	\ifx\pgfplots@execute@at@begin@plot\pgfutil@empty
	\else
		\pgfplots@execute@at@begin@plot
	\fi
	\ifpgfplots@stackedmode
		\pgfplots@stacked@beginplot
	\fi
	\def\pgfplots@addplot@nonlegend@options{#1}%
	\ifx\pgfplots@addplot@nonlegend@options\pgfutil@empty
	\else
		\pgfqkeys{/tikz}{#1}%
	\fi
}

% Initialises 
% \pgfplots@coord@stream@start
% \pgfplots@coord@stream@coord
% \pgfplots@coord@stream@end
% such that a following coordinate stream is processed properly. The
% following coordinate stream may come from different input methods.
%
% Arguments:
% #1:  all options of \addplot[...] (the plot style)
% #2:  any trailing path commands after the 'plot' command as such,
%      for example \addplot plot coordinates {...} -- (0,0);
%      would yield #2 =' -- (0,0)'
%
\long\def\pgfplots@PREPARE@COORD@STREAM#1#2{%
	\ifpgfplots@errorbars@enabled
		\pgfplots@streamerrorbar@recordto{\pgfplots@recordederrorbar}%
		\pgfplots@streamerrorbarstart
	\else
		\let\pgfplots@recordederrorbar=\pgfutil@empty
	\fi
	\pgfplots@coord@stream@INIT\pgfplots@coord@stream@recorded
	%
	% make sure that \pgfplots@coord@stream@end@@ is called at the end
	% of the lately prepare end-stream method.
	\expandafter\def\expandafter\pgfplots@coord@stream@end@\expandafter{%
		\pgfplots@coord@stream@end@
		\pgfplots@coord@stream@end@@
	}%
	\def\pgfplots@coord@stream@end@@{%
		\ifpgfplots@errorbars@enabled
			\pgfplots@streamerrorbarend
		\fi
		\ifpgfplots@coord@stream@isfirst
			\pgfplots@warning{Empty plot silently dropped.}%
		\else
			% Idea: use
			%   \scope[plot specification]
			%   <any paths for error bars>
			%   \endscope
			%   \draw plot coordinates {...};
			% to share plot specifications between error bars and plot
			% coordinates. Unfortunately, it is NOT sufficient to use
			% \tikzset{#1}
			\edef\pgfplots@addplot@preoptionsTMP{/pgfplots/every axis plot,/pgfplots/every axis plot no \the\pgfplots@numplots/.try}%
			\expandafter\pgfplots@rememberplotspec\expandafter{\pgfplots@addplot@preoptionsTMP,#1,/pgfplots/every axis plot post}%
			% warning: rememberplotspec calls list macros which
			% overwrite \pgfplotslist@TOK@a
			\pgfplotslist@TOK@a=\expandafter{\pgfplots@addplot@preoptionsTMP,#1,/pgfplots/every axis plot post}%
			% ATTENTION: do NOT call list macros from here on!
			%
			% Assemble a
			% \pgfplots@addplot@enqueue@coords{}{}...{}
			% statement with correct arguments:
			\ifx\pgfplots@recordederrorbar\pgfutil@empty
				\let\pgfplots@TMPB=\pgfutil@empty%
			\else
				\pgfplotslist@TOK@b=\expandafter{\pgfplots@recordederrorbar}%
				\def\pgfplots@TMPB{%
					\noexpand\pgfplots@errorbars@finishwithstyleoptions[\the\pgfplotslist@TOK@a]{\the\pgfplotslist@TOK@b}%
				}%
			\fi
			\ifpgfplots@datascaletrafo@initialised
				\pgfplots@addplot@get@named@startendpoints@command\pgfplots@TMP
				\pgfplots@toka=\expandafter{\pgfplots@TMP}%
			\else
				\pgfplots@toka={}%
			\fi
			\edef\pgfplots@TMP{%
				\noexpand\pgfplots@addplot@enqueue@coords
				{
					\noexpand\def\noexpand\plotnum{\the\pgfplots@numplots}%
					\noexpand\pgfplots@initzerolevelhandler
					\the\pgfplots@toka% named start/end points
					\noexpand\tikzstyle{current plot style}=[\the\pgfplotslist@TOK@a]%
					\pgfplots@TMPB% error bar commands
				}%
				{%
					\noexpand\draw[current plot style]%
				}%
			}%
			\expandafter\pgfplots@TMP\expandafter{\pgfplots@coord@stream@recorded #2;}%
				{}%
			\pgfplots@end@plot
		\fi
	}%
}%

\def\pgfplots@end@plot{%
	\global\advance\pgfplots@numplots by1\relax%
	\ifpgfplots@stackedmode
		\pgfplots@stacked@endplot
	\fi
	\ifx\pgfplots@execute@at@end@plot\pgfutil@empty
	\else
		\pgfplots@execute@at@end@plot
	\fi
	\global\let\pgfplots@TMP=\relax
	\ifpgfplots@collect@firstplot@astick
		\ifnum\pgfplots@numplots=1\relax
			\pgfplotslist@TOK@a=\expandafter{\pgfplots@firstplot@coords@x}%
			\pgfplotslist@TOK@b=\expandafter{\pgfplots@firstplot@coords@y}%
			\xdef\pgfplots@TMP{%
				\noexpand\def\noexpand\pgfplots@firstplot@coords@x{\the\pgfplotslist@TOK@a}%
				\noexpand\def\noexpand\pgfplots@firstplot@coords@y{\the\pgfplotslist@TOK@b}%
			}%
		\fi
	\fi
	\endgroup
	\pgfplots@TMP% see above
}

% #1: arguments to \addplot[...]
% #2: the plot coordinates
% #3: any trailing path command before the final ';'. It will be used as-is.
\long\def\pgfplots@addplotimpl@coordinates#1plot coordinates#2#3;{%
%\tracingmacros=2\tracingcommands=2
%\pgfplots@message{processing plots coords with trailing path '#3'}%
	\pgfplots@PREPARE@COORD@STREAM{#1}{#3}%
	\pgfplots@coord@stream@foreach{#2}%
}%

\newif\ifpgfplots@update@limits@for@one@point@ISCLIPPED
\def\pgfplots@math@ONE{1.0}%

\def\pgfplots@streamerrorbarstart{%
}%
\def\pgfplots@streamerrorbarend{%
}%
\def\pgfplots@streamerrorbarcoords#1#2{%
}%

\def\pgfplots@streamerrorbar@recordto#1{%
	\def\pgfplots@streamerrorbarstart{\let#1=\pgfutil@empty}%
	\def\pgfplots@streamerrorbarend{}%
	\def\pgfplots@streamerrorbarcoords##1##2{%
		\expandafter\def\expandafter#1\expandafter{#1%
			\pgfplots@errorbar@draw{##1}{##2}%
		}%
	}%
}
\def\pgfplots@streamerrorbar@directdraw{%
	\def\pgfplots@streamerrorbarstart{}%
	\def\pgfplots@streamerrorbarend{}%
	\def\pgfplots@streamerrorbarcoords##1##2{%
		\pgfplots@errorbar@draw{##1}{##2}%
	}%
}
	


% Updates the current x and y limits for point (#1,#2).
%
% The point coordinates may be given in floating point format, see
% below.
%
% Please note that if user specified limits are given, automatic
% limits are only applied to points which fall into the user specified
% clipping region.
%
% PRECONDITIONS:
%   - 'floating point numerics active' 
%   		=> coordinate filtering is active
%   		=> #1 and/or #2 are in floating point format
%   - no coordinate filters active 
%   		=> #1 and #2 MAY be macros, but they must be valid TeX
%   		length (without unit).
% #3= error for x coord (or empty)
% #4= error for y coord (or empty)
\long\def\pgfplots@update@limits@for@one@point#1#2#3#4{%
%\tracingmacros=2\tracingcommands=2
%\pgfplots@message{Updating limits for (#1,#2) ...}%
	\pgfplots@update@limits@for@one@point@ISCLIPPEDfalse
	\ifpgfplots@clip@limits
		\ifpgfplots@autocompute@xlim
		\else
			\ifpgfplots@float@numerics@mode@x
				\pgfmathfloatlessthan@{#1}{\pgfplots@xmin}%
				\ifpgfmathfloatcomparison
					\pgfplots@update@limits@for@one@point@ISCLIPPEDtrue
				\fi
				\pgfmathfloatlessthan@{\pgfplots@xmax}{#1}%
				\ifpgfmathfloatcomparison
					\pgfplots@update@limits@for@one@point@ISCLIPPEDtrue
				\fi
			\else
				\pgfmathlessthan@{#1}{\pgfplots@xmin}%
				\ifx\pgfmathresult\pgfplots@math@ONE
					\pgfplots@update@limits@for@one@point@ISCLIPPEDtrue
				\fi
				\pgfmathlessthan@{\pgfplots@xmax}{#1}%
				\ifx\pgfmathresult\pgfplots@math@ONE
					\pgfplots@update@limits@for@one@point@ISCLIPPEDtrue
				\fi
			\fi
		\fi
		\ifpgfplots@autocompute@ylim
		\else
			\ifpgfplots@float@numerics@mode@y
				\pgfmathfloatlessthan@{#2}{\pgfplots@ymin}%
				\ifpgfmathfloatcomparison
					\pgfplots@update@limits@for@one@point@ISCLIPPEDtrue
				\fi
				\pgfmathfloatlessthan@{\pgfplots@ymax}{#2}%
				\ifpgfmathfloatcomparison
					\pgfplots@update@limits@for@one@point@ISCLIPPEDtrue
				\fi
			\else
				\pgfmathlessthan@{#2}{\pgfplots@ymin}%
				\ifx\pgfmathresult\pgfplots@math@ONE
					\pgfplots@update@limits@for@one@point@ISCLIPPEDtrue
				\fi
				\pgfmathlessthan@{\pgfplots@ymax}{#2}%
				\ifx\pgfmathresult\pgfplots@math@ONE
					\pgfplots@update@limits@for@one@point@ISCLIPPEDtrue
				\fi
			\fi
		\fi
	\fi
	\ifpgfplots@update@limits@for@one@point@ISCLIPPED
	\else
		\ifpgfplots@autocompute@xlim
			\ifpgfplots@float@numerics@mode@x
				\pgfmathfloatmin@{\pgfplots@xmin}{#1}%
				\global\let\pgfplots@xmin=\pgfmathresult
				\pgfmathfloatmax@{\pgfplots@xmax}{#1}%
				\global\let\pgfplots@xmax=\pgfmathresult
			\else
				\pgfmathmin@{\pgfplots@xmin}{#1}%
				\global\let\pgfplots@xmin=\pgfmathresult
				\pgfmathmax@{\pgfplots@xmax}{#1}%
				\global\let\pgfplots@xmax=\pgfmathresult
			\fi
		\fi
		\ifpgfplots@autocompute@ylim
			\ifpgfplots@float@numerics@mode@y
				\pgfmathfloatmin@{\pgfplots@ymin}{#2}%
				\global\let\pgfplots@ymin=\pgfmathresult
				\pgfmathfloatmax@{\pgfplots@ymax}{#2}%
				\global\let\pgfplots@ymax=\pgfmathresult
			\else
				\pgfmathmin@{\pgfplots@ymin}{#2}%
				\global\let\pgfplots@ymin=\pgfmathresult
				\pgfmathmax@{\pgfplots@ymax}{#2}%
				\global\let\pgfplots@ymax=\pgfmathresult
			\fi
		\fi
		\global\pgfplots@limits@are@computedtrue
	\fi
%\pgfplots@message{Updated limits: (\pgfplots@xmin,\pgfplots@ymin) rectangle  (\pgfplots@xmax,\pgfplots@ymax).}%
%\tracingmacros=0\tracingcommands=0
}

\def\pgfplots@invoke@filter#1#2{%
	\pgfkeysvalueof{/pgfplots/#2 filter/.@cmd}#1\pgfeov%
}%

% #3= error for x coord (or empty)
% #4= error for y coord (or empty)
\def\pgfplots@process@one@point#1#2#3#4{%
	\pgfplots@prepare@xcoord{#1}%
	\expandafter\pgfplots@invoke@filter\expandafter{\pgfmathresult}{x}%
	\let\pgfplots@current@point@x=\pgfmathresult
	%
	\pgfplots@prepare@ycoord{#2}%
	\expandafter\pgfplots@invoke@filter\expandafter{\pgfmathresult}{y}%
	\let\pgfplots@current@point@y=\pgfmathresult
	%
	\ifx\pgfplots@current@point@x\pgfutil@empty
		\ifpgfplots@warn@for@filter@discards
			\pgfplots@message{NOTE: coordinate (#1,#2) has been dropped because of the x-coordinate filter.}%
		\fi
	\else
		\ifx\pgfplots@current@point@y\pgfutil@empty
			\ifpgfplots@warn@for@filter@discards
				\pgfplots@message{NOTE: coordinate (#1,#2) has been dropped because of the y-coordinate filter.}%
			\fi
		\else
			\ifpgfplots@apply@datatrafo
				\ifpgfplots@apply@datatrafo@x
					\pgfmathfloatparsenumber{\pgfplots@current@point@x}%
					\let\pgfplots@current@point@x=\pgfmathresult
				\fi
				\ifpgfplots@apply@datatrafo@y
					\pgfmathfloatparsenumber{\pgfplots@current@point@y}%
					\let\pgfplots@current@point@y=\pgfmathresult
				\fi
				\ifpgfplots@datascaletrafo@initialised
					% apply data transformation directly.
					\ifpgfplots@apply@datatrafo@x
						\pgfplots@datascaletrafo@x\pgfplots@current@point@x
						\let\pgfplots@current@point@x=\pgfmathresult
					\fi
					\ifpgfplots@apply@datatrafo@y
						\pgfplots@datascaletrafo@y\pgfplots@current@point@y
						\let\pgfplots@current@point@y=\pgfmathresult
					\fi
				\fi
			\fi
			% All following routines (limit updating/stacking/error
			% bars) will use float numerics if necessary (controlled
			% by ifs).
			\ifpgfplots@stackedmode
				\pgfplots@stacked@preparepoint@inmacro{\pgfplots@current@point@x}{\pgfplots@current@point@y}%
				\ifpgfplots@datascaletrafo@initialised% is also true if there is no scale trafo.
					\pgfplots@stacked@finishpoint{\pgfplots@current@point@x}{\pgfplots@current@point@y}%
				\fi
			\fi
			\ifpgfplots@autocomputelimits
				% update also axis limits:
				\pgfplots@update@limits@for@one@point{\pgfplots@current@point@x}{\pgfplots@current@point@y}{}{}%
			\fi
			\ifpgfplots@errorbars@enabled
				\pgfplots@process@errorbar@for{\pgfplots@current@point@x}{\pgfplots@current@point@y}{#3}{#4}{#1}{#2}%
			\fi
			\pgfplots@toka=\expandafter{\pgfplots@coord@stream@recorded}%
			\edef\pgfplots@coord@stream@recorded{\the\pgfplots@toka (\pgfplots@current@point@x,\pgfplots@current@point@y)}%
			\ifpgfplots@collect@firstplot@astick
				\ifnum\pgfplots@numplots=0
					\ifx\pgfplots@firstplot@coords@x\pgfutil@empty
						\pgfplots@toka=\expandafter{}%
					\else
						\pgfplots@toka=\expandafter{\pgfplots@firstplot@coords@x,}%
					\fi
					\edef\pgfplots@firstplot@coords@x{\the\pgfplots@toka\pgfplots@current@point@x}%
					\ifx\pgfplots@firstplot@coords@y\pgfutil@empty
						\pgfplots@toka=\expandafter{}%
					\else
						\pgfplots@toka=\expandafter{\pgfplots@firstplot@coords@y,}%
					\fi
					\edef\pgfplots@firstplot@coords@y{\the\pgfplots@toka\pgfplots@current@point@y}%
				\fi
			\fi
%\pgfplots@message{FILTERING: appended \pgfplots@current@point@x,\pgfplots@current@point@y}%  -> results in \pgfplots@coord@stream@recorded}%
		\fi
	\fi
	%
	% increase \pgfplots@current@point@coordindex:
	\begingroup
	\c@pgf@counta=\pgfplots@current@point@coordindex
	\advance\c@pgf@counta by1\relax
	\edef\pgfmathresult{\the\c@pgf@counta}%
	\pgfmath@smuggleone\pgfmathresult
	\endgroup
	\let\pgfplots@current@point@coordindex=\pgfmathresult
}

% Internal stream methods.
%
% Please overwrite \pgfplots@coord@stream@start@,
% \pgfplots@coord@stream@end@ and \pgfplots@coord@stream@coord@
% if you implement streams.
%
% FIXME: REPLACE THIS HERE WITH METHODS OF THE VISUALIZATION FRAMEWORK
\newif\ifpgfplots@coord@stream@isfirst
\def\pgfplots@coord@stream@start{%
	\pgfplots@coord@stream@isfirsttrue
	\pgfplots@coord@stream@start@}%
\def\pgfplots@coord@stream@end{\pgfplots@coord@stream@end@}

% Will be invoked for every point coordinate.
%
% It invokes \pgfplots@coord@stream@coord@.
% It is expected to assign \pgfplots@current@point@x and
% \pgfplots@current@point@y
%
% It is expected to
% (#1,#2) : x, y coordinates
% (#3,#4) : any x, y errors (or empty)
\def\pgfplots@coord@stream@coord#1#2#3#4{%
	\pgfplots@coord@stream@coord@{#1}{#2}{#3}{#4}%
	\ifpgfplots@coord@stream@isfirst
		\let\pgfplots@currentplot@firstcoord@x=\pgfplots@current@point@x
		\let\pgfplots@currentplot@firstcoord@y=\pgfplots@current@point@y
		\ifx\pgfplots@currentplot@firstcoord@x\pgfutil@empty
		\else
			\ifx\pgfplots@currentplot@firstcoord@y\pgfutil@empty
			\else
				\pgfplots@coord@stream@isfirstfalse
			\fi
		\fi
	\fi
	\ifx\pgfplots@current@point@x\pgfutil@empty
	\else
		\let\pgfplots@currentplot@lastcoord@x=\pgfplots@current@point@x
	\fi
	\ifx\pgfplots@current@point@y\pgfutil@empty
	\else
		\let\pgfplots@currentplot@lastcoord@y=\pgfplots@current@point@y
	\fi
}%

% A looping method which applies
% \pgfplots@coord@stream@start
% for each coordinate '(x,y)'  or '(x,y) +- (ex,ey)', call
%   \pgfplots@coord@stream@coord{x}{y}{ex}{ey}
% \pgfplots@coord@stream@end
%
% #1 a sequence of coordinates of the form 
%   '(x,y)'
%   or
%   '(x,y) +- (ex,ey)'
%   separated by white-space.
\long\def\pgfplots@coord@stream@foreach#1{%
	\pgfplots@coord@stream@start
	\pgfplots@foreach@plot@coord@do%
		\pgfplots@coord@stream@coord
	\for
	plot coordinates {#1};%
	\pgfplots@EOI
	\pgfplots@coord@stream@end
}%

% Processes coordinates and writes a new, modified 'plot coordinates'
% command to macro #1.
% #1:  macro name for processed coordinates.
%
% Please note that it may be necessary to process the same coordinates
% once more at the end of an axis - see
% pgfplots@coord@stream@finalize@storedcoords@START
\def\pgfplots@coord@stream@INIT#1{%
	\def\pgfplots@coord@stream@start@{%
		\let\pgfplots@coord@stream@recorded=\pgfutil@empty
	}%
	\def\pgfplots@coord@stream@end@{%
		\pgfplots@toka=\expandafter{\pgfplots@coord@stream@recorded}%
		\edef#1{plot coordinates {\the\pgfplots@toka}}%
	}%
	\let\pgfplots@coord@stream@coord@=\pgfplots@process@one@point
}

% Takes a sequence of PREPARED coordinates which are given in floating
% point representation and applies the data scaling trafo.
%
% This stream is designed to be done at the end of an axis.
% See \pgfplots@coord@stream@finalize@storedcoords@START
\def\pgfplots@coord@stream@INIT@finalize@storedcoords#1{%
	\def\pgfplots@coord@stream@start@{%
		\let\pgfplots@data@scaletrafo@result=\pgfutil@empty
	}%
	\def\pgfplots@coord@stream@end@{%
		\let#1=\pgfplots@data@scaletrafo@result
	}%
	\let\pgfplots@coord@stream@coord@=\pgfplots@apply@data@scaletrafo@to@one@point
}

\def\pgfplots@addplot@get@named@startendpoints@command#1{%
	\edef#1{%
		\noexpand\path 
			(\pgfplots@currentplot@firstcoord@x,\pgfplots@currentplot@firstcoord@y) coordinate (current plot begin)
			(\pgfplots@currentplot@lastcoord@x,\pgfplots@currentplot@lastcoord@y) coordinate (current plot end)
		;
	}%
}%

% INPUT: 
%   a sequence of points (SMeE,SMeE) (terminated with ';') 
%   where S=sign, M=mantisse, E = exponent.
%   Such input is generated in the case pgfplots@apply@datatrafotrue
%   by the coordinate filters.
% OUTPUT: 
%   proper sequence of fixed point coordinates which are the result of the
%   data scaling transformation (\pgfplots@datascaletrafo@x).
%
% The output will be written into the macro #3
\long\def\pgfplots@coord@stream@finalize@storedcoords@START plot coordinates #1#2;\to#3{%
	\pgfplots@coord@stream@finalize@storedcoords@START@ plot coordinates {#1}#2;\to#3
	\ifpgfplots@errorbars@enabled
		\pgfplots@errorbars@finishwithstyleoptions[current plot style]{\pgfplots@recordederrorbar}%
	\fi
	\pgfplots@addplot@get@named@startendpoints@command\pgfplots@TMP
	\pgfplots@TMP
}%
\long\def\pgfplots@coord@stream@finalize@storedcoords@START@ plot coordinates #1#2;\to#3{%
	\ifpgfplots@stackedmode
		\pgfplots@stacked@beginplot
	\fi
	\pgfplots@coord@stream@INIT@finalize@storedcoords#3
	\ifpgfplots@errorbars@enabled
		\pgfplots@streamerrorbar@recordto{\pgfplots@recordederrorbar}%
		\pgfplots@streamerrorbarstart
	\fi
	\pgfplots@coord@stream@foreach{#1}%
	\ifpgfplots@stackedmode
		\pgfplots@stacked@endplot
	\fi
	\pgfplots@toka=\expandafter{#3}%
	\pgfplotslist@TOK@a={#2}%
	\edef#3{plot coordinates {\the\pgfplots@toka}\the\pgfplotslist@TOK@a;}%
	\ifpgfplots@errorbars@enabled
		\pgfplots@streamerrorbarend
	\fi
}%

% does the same job, but it does not execute any drawing commands.
%
% Instead, it generates a macro #4 which - if used as "precommand"
% befor a draw operation - restores all required options and performs
% pre-drawing.
\long\def\pgfplots@coord@stream@finalize@storedcoords@START@dryrun plot coordinates #1#2;\to#3#4{%
	\pgfplots@coord@stream@finalize@storedcoords@START@ plot coordinates {#1}#2;\to#3
	\ifpgfplots@stackedmode
		\pgfplots@stacked@savestateto\pgfplots@TMP
		\pgfplotslist@TOK@a=\expandafter{\pgfplots@TMP}%
	\else
		\pgfplotslist@TOK@a={}%
	\fi
	\pgfplots@addplot@get@named@startendpoints@command\pgfplots@TMP
	\pgfplots@toka=\expandafter{\pgfplots@TMP}%
	\ifpgfplots@errorbars@enabled
		\pgfplotslist@TOK@b=\expandafter{\pgfplots@recordederrorbar}%
		\edef#4{%
			\the\pgfplots@toka
			\the\pgfplotslist@TOK@a
			\noexpand\pgfplots@errorbars@finishwithstyleoptions[current plot style]{\the\pgfplotslist@TOK@b}%
		}%
	\else
		\edef#4{%
			\the\pgfplots@toka
			\the\pgfplotslist@TOK@a
		}%
	\fi
}%

% A looping command to loop through plot coordinates.
% For every point, #1{X}{Y} will be invoked.
%
% No scoping is used during this operation, so you can access outer
% variables.
\long\def\pgfplots@foreach@plot@coord@do#1\for plot coordinates #2;\pgfplots@EOI{%
	\def\pgfplots@foreach@plot@coord@do@CMD{#1}%
	\pgfplots@foreach@plot@coord@ITERATE#2\pgfplots@EOI%
}

\def\pgfplots@foreach@plot@coord@ITERATE{%
	\pgfutil@ifnextchar\pgfplots@EOI{%
		\pgfplots@foreach@plot@coord@FINISH%
	}{%
		\pgfplots@foreach@plot@coord@NEXT%
	}%
}

\def\pgfplots@foreach@plot@coord@NEXT(#1,#2){%
	\pgfutil@ifnextchar+{%
		\pgfplots@foreach@plot@coord@NEXT@WITH@ERRORRANGE{#1}{#2}%
	}{%
		\pgfplots@foreach@plot@coord@do@CMD{#1}{#2}{}{}%
		\pgfplots@foreach@plot@coord@ITERATE
	}%
}

% processing something like '(x,y) +- (error_x,error_y)'
\def\pgfplots@foreach@plot@coord@NEXT@WITH@ERRORRANGE#1#2+-#3({%
	\pgfplots@foreach@plot@coord@NEXT@WITH@ERRORRANGE@{#1}{#2}%
}
\def\pgfplots@foreach@plot@coord@NEXT@WITH@ERRORRANGE@#1#2#3,#4){%
	\pgfplots@foreach@plot@coord@do@CMD{#1}{#2}{#3}{#4}%
	\pgfplots@foreach@plot@coord@ITERATE
}

\def\pgfplots@foreach@plot@coord@FINISH\pgfplots@EOI{}


% for use in \ifx:
\def\pgfplots@EOI{\pgfplots@EOI}%
	
% #1= x coord
% #2= y coord
% #3= error for x coord (or empty)
% #4= error for y coord (or empty)
\def\pgfplots@apply@data@scaletrafo@to@one@point#1#2#3#4{%
	\ifpgfplots@apply@datatrafo@x
		\pgfplots@datascaletrafo@x{#1}%
		\let\pgfplots@current@point@x=\pgfmathresult
	\else
		\def\pgfplots@current@point@x{#1}%
	\fi
	\ifpgfplots@apply@datatrafo@y
		\pgfplots@datascaletrafo@y{#2}%
		\let\pgfplots@current@point@y=\pgfmathresult
	\else
		\def\pgfplots@current@point@y{#2}%
	\fi
	\ifpgfplots@stackedmode
		% all these calls work with pgfmath; no more floating point
		% arithmetics are applied.
		\pgfplots@stacked@getnextzerolevelpoint
		\pgfplots@stacked@finishpoint{\pgfplots@current@point@x}{\pgfplots@current@point@y}%
		\pgfplots@stacked@rememberzerolevelpoint@for@next@plot{(\pgfplots@current@point@x,\pgfplots@current@point@y)}%
	\fi
	\pgfplots@toka=\expandafter{\pgfplots@data@scaletrafo@result}%
	\edef\pgfplots@data@scaletrafo@result{\the\pgfplots@toka(\pgfplots@current@point@x,\pgfplots@current@point@y)}%
}

% This thing here shall draw all error bar commands listed in '#2'.
%
% It will be invoked when any plotting commands take effect (that
% means all limits are computed; the axis has been drawn,
% transformations are set up...)
\def\pgfplots@errorbars@finishwithstyleoptions[#1]#2{%
	\scope[%/pgfplots/search path for tikz/.cd,
		#1,/pgfplots/every error bar]%
	#2%
	\endscope
}

\def\pgfplots@errorbar@draw@float(#1,#2)(#3,#4){%
	\ifpgfplots@apply@datatrafo@x
		\pgfplots@datascaletrafo@x{#1}%
		\let\pgfplots@xarg=\pgfmathresult%
		\pgfplots@datascaletrafo@x{#3}%
		\let\pgfplots@error@xarg=\pgfmathresult%
	\else
		\def\pgfplots@xarg{#1}%
		\def\pgfplots@error@xarg{#3}%
	\fi
	\ifpgfplots@apply@datatrafo@y
		\pgfplots@datascaletrafo@y{#2}%
		\let\pgfplots@yarg=\pgfmathresult%
		\pgfplots@datascaletrafo@y{#4}%
		\let\pgfplots@error@yarg=\pgfmathresult%
	\else
		\def\pgfplots@yarg{#2}%
		\def\pgfplots@error@yarg{#4}%
	\fi
	\edef\pgfplots@TMP{{(\pgfplots@xarg,\pgfplots@yarg)}{(\pgfplots@error@xarg,\pgfplots@error@yarg)}}%
	\def\pgfplots@TMPB{\pgfkeysvalueof{/pgfplots/error bars/draw error bar/.@cmd}}%
	\expandafter\pgfplots@TMPB\pgfplots@TMP\pgfeov
}

\def\pgfplots@errorbar@draw#1#2{%
	\begingroup
	\ifpgfplots@apply@datatrafo
		\pgfplots@errorbar@draw@float#1#2
	\else
		\pgfkeysvalueof{/pgfplots/error bars/draw error bar/.@cmd}{#1}{#2}\pgfeov%
	\fi
	\endgroup
}%

% Also provides UNFILTERED arguments x (#5) and y (#6). These are use
% in case of logplots, because we may need to compute log( x + e_x )
% or log( y + e_y ).
\def\pgfplots@process@errorbar@for#1#2#3#4#5#6{%
%	\begingroup
	\edef\pgfplots@xarg{#1}%
	\edef\pgfplots@yarg{#2}%
	\def\pgfplots@xarg@unfiltered{#5}%
	\def\pgfplots@yarg@unfiltered{#6}%
	\edef\pgfplots@error@xarg{#3}%
	\edef\pgfplots@error@yarg{#4}%
	\def\pgfplots@TMP{0}%
	\ifx\pgfplots@TMP\pgfplots@error@xarg
		\let\pgfplots@error@xarg=\pgfutil@empty
	\fi
	\ifx\pgfplots@TMP\pgfplots@error@yarg
		\let\pgfplots@error@yarg=\pgfutil@empty
	\fi
	%  FIXME : INEFFICIENT! This code here does every computation 
	%  multiple times!
	\ifcase\pgfplots@errorbars@xdirection
	\or
		\pgfplots@process@errorbar@@for{x}{+}1%
	\or
		\pgfplots@process@errorbar@@for{x}{-}2%
	\or
		\pgfplots@process@errorbar@@for{x}{+}1%
		\pgfplots@process@errorbar@@for{x}{-}2%
	\fi
	\ifcase\pgfplots@errorbars@ydirection
	\or
		\pgfplots@process@errorbar@@for{y}{+}1%
	\or
		\pgfplots@process@errorbar@@for{y}{-}2%
	\or
		\pgfplots@process@errorbar@@for{y}{+}1%
		\pgfplots@process@errorbar@@for{y}{-}2%
	\fi
%	\endgroup
}

% #1: either 'x' or 'y'
% #2: either '+' or '-'
% #3: an integer representing the argument #2. It is '1' if #2='+'
%     and '2' is #2 = '-'.
\def\pgfplots@process@errorbar@@for#1#2#3{%
	\expandafter\let\expandafter\ifpgfplots@is@datascaled\csname ifpgfplots@float@numerics@mode@#1\endcsname
	\csname ifpgfplots@#1islinear\endcsname
		\edef\pgfplots@error@src{\csname pgfplots@#1arg\endcsname}%
		\ifcase\csname pgfplots@errorbars@#1mode\endcsname
			% fixed absolute error.
			\ifpgfplots@is@datascaled
				\edef\pgfplots@error@coord{\csname pgfplots@errorbars@#1fixed\endcsname}%
				\pgfmathfloatparsenumber{\pgfplots@error@coord}%
				\let\pgfplots@error@coord=\pgfmathresult
				\ifnum#3=1
					\pgfmathfloatadd@{\pgfplots@error@src}{\pgfplots@error@coord}%
				\else
					\pgfmathfloatsubtract@{\pgfplots@error@src}{\pgfplots@error@coord}%
				\fi
			\else
				\edef\pgfplots@error@coord{\csname pgfplots@errorbars@#1fixed\endcsname}%
				\pgfmathparse{\pgfplots@error@src#2 \pgfplots@error@coord}%
			\fi
			\let\pgfplots@error@coord=\pgfmathresult
		\or% fixed relative error:
			\ifpgfplots@is@datascaled
				\pgfmathparse{(1#2\csname pgfplots@errorbars@#1rel\endcsname)}%
				\let\pgfplots@error@coord=\pgfmathresult
				\pgfmathfloatmultiplyfixed@{\pgfplots@error@src}{\pgfplots@error@coord}%
			\else
				\pgfmathparse{\pgfplots@error@src*(1#2\csname pgfplots@errorbars@#1rel\endcsname)}%
			\fi
			\let\pgfplots@error@coord=\pgfmathresult
		\or% explicit absolute:
			\edef\pgfplots@error@coord{\csname pgfplots@error@#1arg\endcsname}%
			\ifx\pgfplots@error@coord\pgfutil@empty
			\else
				\ifpgfplots@is@datascaled
					\pgfmathfloatparsenumber{\pgfplots@error@coord}%
					\let\pgfplots@error@coord=\pgfmathresult
					\ifnum#3=1
						\pgfmathfloatadd@{\pgfplots@error@src}{\pgfplots@error@coord}%
					\else
						\pgfmathfloatsubtract@{\pgfplots@error@src}{\pgfplots@error@coord}%
					\fi
				\else
					\pgfmathparse{\pgfplots@error@src#2 \pgfplots@error@coord}%
				\fi
				\let\pgfplots@error@coord=\pgfmathresult
			\fi
		\or% explicit relative:
			\edef\pgfplots@error@coord{\csname pgfplots@error@#1arg\endcsname}%
			\ifx\pgfplots@error@coord\pgfutil@empty
			\else
				\ifpgfplots@is@datascaled
					\pgfmathparse{(1#2\pgfplots@error@coord)}%
					\let\pgfplots@error@coord=\pgfmathresult
					\pgfmathfloatmultiplyfixed@{\pgfplots@error@src}{\pgfplots@error@coord}%
				\else
					\pgfmathparse{\pgfplots@error@src*(1#2\pgfplots@error@coord)}%
				\fi
				\let\pgfplots@error@coord=\pgfmathresult
			\fi
		\fi
	\else
		% LOGARITHMIC scaling. All errors are interpreted as 
		%   log(x +- e_x)
		% or
		%   log( x*(1+-e_x) )
		\ifcase\csname pgfplots@errorbars@#1mode\endcsname
			% fixed absolute, log( x +- e_x )
			\edef\pgfplots@error@src{\csname pgfplots@#1arg@unfiltered\endcsname}%
			\edef\pgfplots@error@coord{\csname pgfplots@errorbars@#1fixed\endcsname}%
			\pgfmathfloatparsenumber{\pgfplots@error@coord}%
			\let\pgfplots@error@coord=\pgfmathresult
			\pgfmathfloatparsenumber{\pgfplots@error@src}%
			\let\pgfplots@error@src=\pgfmathresult
			\ifnum#3=1
				\pgfmathfloatadd@{\pgfplots@error@src}{\pgfplots@error@coord}%
			\else
				\pgfmathfloatsubtract@{\pgfplots@error@src}{\pgfplots@error@coord}%
			\fi
			\pgfmathlog@float{\pgfmathresult}%
			\let\pgfplots@error@coord=\pgfmathresult
		\or% fixed relative, log( x ( 1+-e_x ) ) = log(x) + log(1+-e_x)
			\edef\pgfplots@error@src{\csname pgfplots@#1arg\endcsname}%
			\pgfmathparse{\pgfplots@error@src + ln(1#2\csname pgfplots@errorbars@#1rel\endcsname)}%
			\let\pgfplots@error@coord=\pgfmathresult%
		\or% explicit absolute
			\edef\pgfplots@error@src{\csname pgfplots@#1arg@unfiltered\endcsname}%
			\edef\pgfplots@error@coord{\csname pgfplots@error@#1arg\endcsname}%
			\ifx\pgfplots@error@coord\pgfutil@empty
			\else
				\pgfmathfloatparsenumber{\pgfplots@error@coord}%
				\let\pgfplots@error@coord=\pgfmathresult
				\pgfmathfloatparsenumber{\pgfplots@error@src}%
				\let\pgfplots@error@src=\pgfmathresult
				\ifnum#3=1
					\pgfmathfloatadd@{\pgfplots@error@src}{\pgfplots@error@coord}%
				\else
					\pgfmathfloatsubtract@{\pgfplots@error@src}{\pgfplots@error@coord}%
				\fi
				\pgfmathlog@float{\pgfmathresult}%
				\let\pgfplots@error@coord=\pgfmathresult
			\fi
		\or% explicit relative:
			\edef\pgfplots@error@src{\csname pgfplots@#1arg\endcsname}%
			\edef\pgfplots@error@coord{\csname pgfplots@error@#1arg\endcsname}%
			\ifx\pgfplots@error@coord\pgfutil@empty
			\else
				\pgfmathparse{\pgfplots@error@src + ln(1#2 \pgfplots@error@coord)}%
				\let\pgfplots@error@coord=\pgfmathresult%
			\fi
		\fi
	\fi
	\ifx\pgfplots@error@coord\pgfutil@empty
	\else
		\def\pgfplots@TMP{#1}%
		\def\pgfplots@TMPB{x}%
		\ifx\pgfplots@TMP\pgfplots@TMPB
			\ifpgfplots@autocomputelimits
				\pgfplots@update@limits@for@one@point{\pgfplots@error@coord}{\pgfplots@yarg}{}{}%
			\fi
			\edef\pgfplots@TMP{%
				{(\pgfplots@xarg,\pgfplots@yarg)}%
				{(\pgfplots@error@coord,\pgfplots@yarg)}%
			}%
		\else
			\ifpgfplots@autocomputelimits
				\pgfplots@update@limits@for@one@point{\pgfplots@xarg}{\pgfplots@error@coord}{}{}%
			\fi
			\edef\pgfplots@TMP{%
				{(\pgfplots@xarg,\pgfplots@yarg)}%
				{(\pgfplots@xarg,\pgfplots@error@coord)}%
			}%
		\fi
		\expandafter\pgfplots@streamerrorbarcoords\pgfplots@TMP
	\fi
}

\long\def\pgfplots@path#1;{%
	\pgfplots@path@enqueue{#1;}%
}

% This thing here shall be used to replace any '\path' where \axispath
% shall be used.
\long\def\pgfplots@replacement@for@tikz@path#1;{%
	\axispath\path#1;%
}


% Will be available as \closedcycle command inside of an axis.
%
% It closes the current plot by drawing lines to the last "zero
% level".
%
% That means the current plot is connected orthogonally with the
% x-axis, allowing fill commands.
%
% For stacked plots, \closedcycle is special (it connects with the
% previous \addplot command).
%
% Example:
% \addplot coordinates {(3,0.5) (4,2) (5,1)} \closedcycle;
\def\pgfplots@path@closed@cycle{%
	\ifpgfplots@stackedmode
		\pgfplots@stacked@path@closed@cycle
	\else
		\pgfplots@path@closed@cycle@std
	\fi
}
\def\pgfplots@path@closed@cycle@std{%
	|- (perpendicular cs: 
		vertical line through={(current plot begin)}, 
		horizontal line through={(\pgfplots@ZERO@x,\pgfplots@ZERO@y)})
	-- cycle
}%

% Remembers the plotting command #2 and '#3=plot coordinates {...} ...';
% for later postprocessing of the coordinates.
%
% #1: commands which should be executed before issueing the plotting
%        command #2 #3.
% #2:
%    the command which is responsable for drawing.
%    - If #2 is NOT '\pgfutil@empty', we expect #3 to contain only
%      EXPANDABLE DATA.
%      That's important for postponed floating point arithmetics in #3.
%    - If #2='\pgfutil@empty', we don't make any assumption about #3
%      and process it as-is.
%
% #3: see above.
%
% #4: commands which should be executed after '#2 #3'.
%
\long\def\pgfplots@path@enqueue@coords#1#2#3#4{%
	\ifpgfplots@draw@at@end
		\begingroup
		\globaldefs=1
		\pgfplotslistpushback{#1}{#2}{#3}{#4}\to\pgfplots@stored@plotlist
		\endgroup
	\else
		\begingroup
		#1#2#3#4%
		\endgroup
	\fi
}

% The same as \pgfplots@path@enqueue@coords, but this here also stores
% \pgfplots@addplot@nonlegend@options
\long\def\pgfplots@addplot@enqueue@coords#1#2#3#4{%
	\ifpgfplots@draw@at@end
		\ifx\pgfplots@addplot@nonlegend@options\pgfutil@empty
			\pgfplots@path@enqueue@coords{#1}{#2}{#3}{#4}%
		\else
			\expandafter
			\pgfplots@path@enqueue@coords
				\expandafter{%
				\expandafter\tikzset\expandafter{\pgfplots@addplot@nonlegend@options}%
				#1}{#2}{#3}{#4}%
		\fi
	\else
		\pgfplots@path@enqueue@coords{#1}{#2}{#3}{#4}%
	\fi
}


% Remembers the plotting command #1.
\long\def\pgfplots@path@enqueue#1{%
	\pgfplots@path@enqueue@coords{}{}{#1}{}%
}


% Assigns a legend.
% Syntax:
% \legend{entry 1\\entry2\\entry3}
\def\pgfplots@command@legend{
	\pgfutil@ifnextchar[{%
		\pgfplots@error{Sorry, legend options are now deprecated. Legends are now TikZ-matrizes which provide better alignment and can be placed horizontally. See the manual for details.}%
		\pgfplots@command@legend@impl
	}{%
		\pgfplots@command@legend@impl
	}%
}

\def\pgfplots@command@legend@impl#1{%
	\pgfplots@assign@list\pgfplots@TMPC{#1}%
	\global\let\pgfplots@legend=\pgfplots@TMPC
}


\def\pgfplots@addlegendentry{%
	\pgfutil@ifnextchar[{%
		\pgfplots@addlegendentry@opts
	}{%
		\pgfplots@addlegendentry@opts[]%
	}%
}
\def\pgfplots@addlegendentry@opts[#1]#2{%
	{%
	\globaldefs=1% this will ONLY store the options and the legend into global variables
	\pgfplotslistpushback[#1]#2\to\pgfplots@legend
	}%
}

\def\pgfplots@pop@next@legend{{%
	\globaldefs=1
	\pgfplotslistcheckempty\pgfplots@plotspeclist
	\ifpgfplotslistempty
		\let\pgfplots@curplotlist=\pgfutil@empty
	\else
		\pgfplotslistpopfront\pgfplots@plotspeclist\to\pgfplots@curplotlist
	\fi
	%
	\pgfplotslistcheckempty\pgfplots@legend
	\ifpgfplotslistempty
		\let\pgfplots@curlegend=\pgfutil@empty
	\else
		\pgfplotslistpopfront\pgfplots@legend\to\pgfplots@curlegend
	\fi
	\advance\pgfplots@numplots by-1
}}

% This routine is called at the begin of every plot.
% It initialised a zero level stream.
%
% The default is to use '0' as zero level streams.
%
% This method is called as "precommand"; before any Tikz drawing
% commands have been started.
\def\pgfplots@initzerolevelhandler{%
	\ifpgfplots@stackedmode
		% ATTENTION: this thing here says:
		%    "draw zero level coordinates from list XYZ."
		% But at the time of this initialisation, the list will be EMPTY!
		%
		% It will be filled later. That's ok, because 
		% \pgfplots@initzerolevelhandler will be
		% used as 'precommand', that means before Tikz sees any
		% coordinates.
		\pgfplots@stacked@initzerolevelhandler
	\else
		\pgfplotxzerolevelstreamconstant{\pgfplots@ZERO@x}%
		\pgfplotyzerolevelstreamconstant{\pgfplots@ZERO@y}%
	\fi
}

\def\pgfplots@try@mirror@plot@handler{%
	\pgfutil@ifundefined{tikz@plot@handler}{}{%
		\ifx\tikz@plot@handler\pgfplothandlerconstantlineto
			\let\tikz@plot@handler=\pgfplothandlerconstantlinetomarkright
		\else
			\ifx\tikz@plot@handler\pgfplothandlerconstantlinetomarkright
				\let\tikz@plot@handler=\pgfplothandlerconstantlineto
			\fi
		\fi
	}%
}

\long\def\pgfplots@show@small@legendplots#1#2{
	\begingroup
	\def\pgfplots@TMPB{#1}%
	\ifx\pgfplots@TMPB\pgfutil@empty
	\else
		\pgfqkeys{/pgfplots/search path for tikz}{#1}%
	\fi
	\pgfkeysvalueof{/pgfplots/legend image code/.@cmd}#2\pgfeov
	\endgroup
}

\def\pgfplots@split@opts{%
	\pgfutil@ifnextchar[{%
		\pgfplots@split@opts@opts
	}{%
		\pgfplots@split@opts@opts[]%
	}%
}
\def\pgfplots@split@opts@opts[#1]#2\pgfplots@result@to#3#4{%
	\def#3{#2}%
	\def#4{#1}%
}

% Typesets a legend node.
%
% It will either typeset a previously computed legend (which needs to be
% stored in the macro \pgfplots@already@computed@legend@node)
%
% or it creates a legend, stores the commands into the macro named
% above and typesets it.
\def\pgfplots@createlegend{%
	\ifx\pgfplots@already@computed@legend@node\pgfutil@empty
		\pgfplotslistcheckempty\pgfplots@legend
		\ifpgfplotslistempty
			% No legend commands appeared in the document. So,
			% consider the key:
			\pgfkeysgetvalue{/pgfplots/legend entries}\pgfplots@legend
			\expandafter\pgfplots@assign@list\expandafter\pgfplots@legend\expandafter{\pgfplots@legend}%
			\pgfplotslistcheckempty\pgfplots@legend
		\fi
		\ifpgfplotslistempty
		\else
		%
		% 
		\begingroup
			% assemble a 
			% \matrix {
			% 	small plot  & legend1\\
			% 	small plot  & legend2\\
			% 	...
			% };
			% command [and using the 'legend columns' option]
			%
			% \pgfplotslist@TOK@a={
			% 	small plot  & legend1\\
			% 	small plot  & legend2\\
			% 	...
			% }
			% ( I have allocated the token registers in my
			% liststructure.sty)
			% 
			% \global\def\pgfplots@TMP{
			% 	\matrix {
			% 		\TOKL@TA
			% 	};
			% }
			% -> finally, \pgfplots@TMP will contain the complete command.
			\pgfplotslist@TOK@a={}%
			\let\curcolumnNum=\c@pgf@counta
			\let\maxcolumnCount=\c@pgf@countb
			\let\legendplotpos=\c@pgf@countc
			\legendplotpos\expandafter=\pgfplots@legend@plot@pos
			\curcolumnNum=0
			\maxcolumnCount=\pgfplots@legend@columns\relax
			%
			\loop
			\ifnum0<\pgfplots@numplots\relax
				\pgfplots@pop@next@legend
				\ifx\pgfplots@curlegend\pgfutil@empty
				\else
					\advance\curcolumnNum by1
					\begingroup
					\expandafter\pgfplots@split@opts\pgfplots@curlegend\pgfplots@result@to{\pgfplots@curlegend}{\pgfplots@curlegend@opts}%
					\ifcase\legendplotpos
						% legend plot pos=left 
						\pgfplotslist@TOK@b=\expandafter{%
								\expandafter\pgfplots@show@small@legendplots
								\expandafter{\pgfplots@curlegend@opts}}%
						\pgfplotslist@TOK@b=\expandafter\expandafter\expandafter{\expandafter\the\expandafter\pgfplotslist@TOK@b\expandafter{\pgfplots@curplotlist}%
								\pgfmatrixnextcell\node}%
						\pgfplotslist@TOK@b=\expandafter\expandafter\expandafter{%
							\expandafter\the\expandafter\pgfplotslist@TOK@b
							\expandafter{\pgfplots@curlegend};}%
					\or
						% legend plot pos=right
						\pgfplotslist@TOK@b=\expandafter{%
							\expandafter\node
							\expandafter{\pgfplots@curlegend};%
							\pgfmatrixnextcell
							\pgfplots@show@small@legendplots}%
						\pgfplotslist@TOK@b=\expandafter\expandafter\expandafter{\expandafter\the\expandafter\pgfplotslist@TOK@b\expandafter{\pgfplots@curlegend@opts}}%
						\pgfplotslist@TOK@b=\expandafter\expandafter\expandafter{\expandafter\the\expandafter\pgfplotslist@TOK@b\expandafter{\pgfplots@curplotlist}}%
					\or
						% legend plot pos=none
						\pgfplotslist@TOK@b=\expandafter{\expandafter\node\expandafter{\pgfplots@curlegend};}%
					\fi
					\ifnum\curcolumnNum=\maxcolumnCount
						\pgfplotslist@TOK@b=\expandafter{\the\pgfplotslist@TOK@b\\}%
					\else
						\ifnum\pgfplots@numplots=0
							\pgfplotslist@TOK@b=\expandafter{\the\pgfplotslist@TOK@b\\}%
						\else
							\pgfplotslist@TOK@b=\expandafter{\the\pgfplotslist@TOK@b\pgfmatrixnextcell}%
						\fi
					\fi
					\xdef\pgfplots@TMP{%
						\the\pgfplotslist@TOK@a
						\the\pgfplotslist@TOK@b
					}%
					\endgroup
					\ifnum\curcolumnNum=\maxcolumnCount
						\curcolumnNum=0
					\fi
					\expandafter\pgfplotslist@TOK@a\expandafter{\pgfplots@TMP}%
				\fi
			\repeat
			\pgfplotslist@TOK@b={\matrix[/pgfplots/every axis legend]}%
			\xdef\pgfplots@TMP{%
				\noexpand\def\noexpand\plotnum{0}%
				\the\pgfplotslist@TOK@b {%
					\the\pgfplotslist@TOK@a
				};%
			}%
		\endgroup
		\let\pgfplots@already@computed@legend@node=\pgfplots@TMP
		\fi
	\fi
%\pgfplots@message{Vor legende: \meaning\pgfplots@already@computed@legend@node}%
	\pgfplots@already@computed@legend@node
}

\newif\ifpgfplots@checkuniform@isfirst
% Checks whether the argument to xtick or ytick is a UNIFORM tick
% sequence.
%
% A uniform tick sequence is 0,...,10 and 3,4,5 and -5,-4,-2 but 
% NOT 0,2,4 or 4,10.
%
% Furthermore, any NON-integer tick arguments are also assumed to be
% NOT uniform.
%
% INPUT:
% #1: a tick argument (i.e. something which can be put to 
%     \foreach \x in {#1})
%
% OUTPUT:
%    \pgfplots@isuniformticktrue
%    or
%    \pgfplots@isuniformtickfalse
% depending on the check.
% This variable will be set globally.
\def\pgfplots@checkisuniformLOGtick#1{%
	\begingroup
	\global\pgfplots@isuniformticktrue
	\pgfplots@checkuniform@isfirsttrue
	\foreach \x in {#1}{%
		\pgfmathmultiply@\x\reciproclogten
		\let\cur=\pgfmathresult
		% check whether
		%  \cur - last == 1 (last = \pgfplots@TMPB)
		\ifpgfplots@checkuniform@isfirst
			\global\pgfplots@checkuniform@isfirstfalse
		\else
			\pgfmathsubtract@\cur\pgfplots@TMPB%
			\pgfmathapproxequalto@\pgfmathresult{1.0}%
			\ifpgfmathcomparison
			\else
				\global\pgfplots@isuniformtickfalse
				\breakforeach
			\fi
		\fi
		\global\let\pgfplots@TMPB=\cur
	}%
	\endgroup
}

% Checks whether the linear tick sequence #1 is a uniform tick.
%
% It also assigns pgfplots@tick@distance@#1  as the distance.
%
% see \pgfplots@checkisuniformLOGtick for details.
\def\pgfplots@checkisuniformLINEARtick#1{%
	\begingroup
	\global\pgfplots@isuniformticktrue
	\pgfplots@checkuniform@isfirsttrue
	\global\let\pgfplots@TMPB=\pgfutil@empty
	\global\let\pgfplots@TMP=\relax
	\foreach \x in {#1}{%
		\ifx\pgfplots@TMPB\pgfutil@empty
		\else
			\pgfmathsubtract@\x\pgfplots@TMPB
			\ifpgfplots@checkuniform@isfirst
				% remember first distance h = x_1 - x_0
				\global\let\pgfplots@TMP=\pgfmathresult
				\global\pgfplots@checkuniform@isfirstfalse
			\else
				% check whether x_i - x_{i-1} = h
				\pgfmathapproxequalto@\pgfmathresult\pgfplots@TMP%
				\ifpgfmathcomparison
				\else
					\global\pgfplots@isuniformtickfalse
					\breakforeach
				\fi
			\fi
		\fi
		\global\let\pgfplots@TMPB=\x%
	}%
	\endgroup
	\expandafter\let\csname pgfplots@tick@distance@#1\endcsname=\pgfplots@TMP
}

\def\skipsuffixzero#1.#2|{
	{%
	\def\pgfplots@TMP{#2}%
	\def\pgfplots@TMPB{0}%
	\ifx\pgfplots@TMP\pgfplots@TMPB
		\global\def\pgfmathresult{#1}%
	\else
		\global\def\pgfmathresult{#1.#2}%
	\fi
	}%
}

\def\pgfmathlogtologten#1{%
	\pgfmathparse{#1}%
	\expandafter\pgfmathlogtologten@\expandafter{\pgfmathresult}%
}

% Simply divides #1 by log(10).
\def\pgfmathlogtologten@#1{%
	\pgfmathmultiply@{#1}\reciproclogten%
}%

% DEPRECATED.
\def\logtologtentomacro#1#2{%
	\pgfmathmultiply@{#1}\reciproclogten%
	\expandafter\skipsuffixzero\pgfmathresult|%
	\let#2=\pgfmathresult
}

% DEPRECATED.
\def\logtologten#1{%
	\pgfmathmultiply@{#1}\reciproclogten%
	\expandafter\skipsuffixzero\pgfmathresult|%
	\pgfmathresult
}

% helper method which computes log10*\x foreach \x in {#1}.
% The result will be \xdef'ed into #2.
\def\pgfplots@compute@tick@times@logten#1\to#2{%
	\global\let#2=\pgfutil@empty
	\foreach \pgfplots@TMPB in {#1} {%
		\pgfmathmultiply@\pgfplots@TMPB\logten%
		\ifx#2\pgfutil@empty
			\xdef#2{\pgfmathresult}%
		\else
			\xdef#2{#2,\pgfmathresult}%
		\fi
	}%
}

\def\pgfplots@enlarge@limit@ifconfigured#1{%
	\pgfplots@enlargelimits@autofalse
	\pgfplots@enlargelimitsfalse
	\pgfplots@enlargelimits@rel@threshfalse
	\pgfkeysgetvalue{/pgfplots/enlarge #1 limits}{\pgfplots@TMP}%
	%
	\def\pgfplots@TMPB{true}%
	\ifx\pgfplots@TMP\pgfplots@TMPB
		\pgfplots@enlargelimitstrue
	\else
		\def\pgfplots@TMPB{false}%
		\ifx\pgfplots@TMP\pgfplots@TMPB
		\else
			\def\pgfplots@TMPB{auto}%
			\ifx\pgfplots@TMP\pgfplots@TMPB
				\pgfplots@enlargelimits@autotrue
			\else
				\begingroup
				% try to read it as number:
				\expandafter\pgf@xa\pgfplots@TMP pt\relax
				\endgroup
				\pgfplots@enlargelimitstrue
				\pgfplots@enlargelimits@rel@threshtrue
			\fi
		\fi
	\fi
	\expandafter\let\expandafter\ifpgfplots@autocompute@@@tmp\csname ifpgfplots@autocompute@#1lim\endcsname
	\ifpgfplots@enlargelimits
		% relax the sizes.
		%
		% Idea: if the user chose his xmin,xmax tight to his data,
		% this here will look better.
		\pgfplots@enlarge@limit@for{#1}%
	\else
		\ifpgfplots@enlargelimits@auto
			%\ifpgfplots@hide@x
				% there is no axis, so skip this enlargement (unless
				% the user explizitly requests it)
				% WHY!?
			%\else
				% FIXME : this here should be user-configurable!
				\ifpgfplots@autocompute@@@tmp
					\pgfplots@enlarge@limit@for{#1}%
				\fi
			%\fi
		\fi
	\fi
}

\def\pgfplots@enlarge@limit@for#1{%
	\begingroup
	\expandafter\let\expandafter\min\expandafter=\csname pgfplots@#1min\endcsname
	\expandafter\let\expandafter\max\expandafter=\csname pgfplots@#1max\endcsname
	\ifpgfplots@enlargelimits@rel@thresh
		\pgfkeysgetvalue{/pgfplots/enlarge #1 limits}{\enlargepercent}%
	\else
		\def\enlargepercent{0.1}% FIXME : pack 10% as default into option 'enlargelimits' or so
	\fi
	\pgfmathsubtract@\max\min%
	\pgf@xa\pgfmathresult pt
	\pgf@xb\enlargepercent\pgf@xa
	\ifdim\pgf@xb>0.001pt
		% the case with 
		%   enlargeabsolute ~= 0
		% means that \min ~= \max.
		% It is handled in another method.
		\edef\enlargeabsolute{\pgf@sys@tonumber{\pgf@xb}}%
		%
		% compute xmin := xmin - enlargeabsolute
		\pgfmathsubtract@\min\enlargeabsolute%
		\let\min=\pgfmathresult
		\pgfmathadd@\max\enlargeabsolute%
		\let\max=\pgfmathresult
	\fi
	\xdef\pgfplots@TMP{\min}%
	\xdef\pgfplots@TMPB{\max}%
	\endgroup
	\expandafter\global\expandafter\let\csname pgfplots@#1min\endcsname=\pgfplots@TMP
	\expandafter\global\expandafter\let\csname pgfplots@#1max\endcsname=\pgfplots@TMPB
}

% Computes tick positions using the current axis limits.
%
% Parameters:
% /pgfplots/max space between ticks
%    Determines the maximum space which is not filled by at least one
%    tick label (approximate, there is some rounding internally)
% /pgfplots/try min ticks
%    see manual
%
% Idea:
% We want ticks at each 
%    { i*H, i in \Z }.
% Of course, there shouldn't be TOO MUCH ticks. 
%
% Our heuristics is to set
%    desirednumticks = round(ACTUAL WIDTH / (max space between ticks) )
% and generate H = (axis range) / (desirednumticks).
%
% Since not all step sizes H look well, restrict H to a set of allowed 
% step sizes such as 
%   { 1, 1/2, 1/5, 1/10 },
% or, to be more precise:
%   { 1*10^e, 2*10^e, 5*10^e }
% -> round to the nearest matching number!
%
% For log-plots, 
% 	H in { j*log(10), j=1,2,3,... }
% where the usual case should be j = 1.
%
% Then, the resulting tick is
% TICK={MIN,MIN+H,...,MAX}
% where
%    MIN = I*H
% is chosen such that 
%    axis minimum limit = I*H + rest; |rest| < H.
%
% Again, log plots follow a slightly different approach: here,
%   MIN = I * log(10)
% is chosen such that
%    axis minimum limit = I*log(10) + rest; |rest| < log(10)
% while H = j*log(10), j>=1.
%
%
% PRECONDITION:
% - limits are correct
% - axis width/height is set correctly
%
% POSTCONDITION:
% - Tick for axis #1 is assigned
% - \ifpgfplots@determinedefaultvalues@needs@check@uniformtick is set
%
% REMARKS:
% - this algorithms works also if the data range has been transformed
%   with a LINEAR transformation.
%   ATTENTION: as of 2008-05-15, the scaling trafo is AFFINE LINEAR.
%   That means we have to eliminate the 'affine' shifting before the
%   algorithms works correctly.
\def\pgfplots@assign@default@tick@foraxis#1{%
	\begingroup
	% Shortcut-names:
	\expandafter\let\expandafter\ifpgfplots@is@datascaled\csname ifpgfplots@apply@datatrafo@#1\endcsname
	% Attention here: use UNSHIFTET scalings, see remark above
	\expandafter\let\expandafter\pgfplots@data@scale@trafo\csname pgfplots@datascaletrafo@#1@noshift\endcsname
	\expandafter\let\expandafter\pgfplots@data@scale@inverse@trafo\csname pgfplots@inverse@datascaletrafo@#1@noshift\endcsname
	\expandafter\let\expandafter\ifpgfplots@cur@is@linear\csname ifpgfplots@#1islinear\endcsname
	%
	\let\desirednumticks=\c@pgf@countd
	\let\Wr=\pgf@xc
	\Wr=\csname pgfplots@#1coordmaxTEX\endcsname
	\advance\Wr by-\csname pgfplots@#1coordminTEX\endcsname
	% r = max place without ticks in pt -> choose desirednumticks >= W/r
	\expandafter\expandafter\divide\Wr\axisdefaulttickwidth
	\pgfmathsetcount{\desirednumticks}{\Wr}%
	\advance\desirednumticks by1
	\csname ifpgfplots@#1islinear\endcsname
		\expandafter\ifnum\axisdefaulttryminticks>\desirednumticks
			\expandafter\desirednumticks\axisdefaulttryminticks
		\fi
	\else
		\expandafter\ifnum\pgfplots@default@try@minticks@log>\desirednumticks
			\expandafter\desirednumticks\pgfplots@default@try@minticks@log\relax
		\fi
		\expandafter\ifx\csname pgfplots@#1tickten\endcsname\pgfutil@empty
		\else
			% log plot and tickten-option: provide special processing.
			\edef\pgfplots@TMP{\csname pgfplots@#1tickten\endcsname}%
			\expandafter\pgfplots@compute@tick@times@logten\pgfplots@TMP\to\pgfplots@TMP
			\expandafter\let\csname pgfplots@#1tick\endcsname=\pgfplots@TMP
			\aftergroup\pgfplots@determinedefaultvalues@needs@check@uniformticktrue
		\fi
	\fi
	%
	\expandafter\ifx\csname pgfplots@#1tick\endcsname\pgfutil@empty
		% Ok, we have either log or linear axis and need default
		% ticks MIN,MIN+H,...,MAX.
		\let\MINH=\pgf@xa
		\let\H=\pgf@xb
		\let\MAX=\pgf@ya
		\let\MIN=\pgf@yb
		% compute step size 'H':
		\expandafter\MAX\csname pgfplots@#1max\endcsname pt
		\advance\MAX by0.001pt% avoid round errors
		%\expandafter\MIN\the\c@pgf@counta pt
		\expandafter\MIN\csname pgfplots@#1min\endcsname pt
		\H=\MAX
		\advance\H by-\MIN
%\message{Axis limit #1: [\the\MIN:\the\MAX], diff = \the\H.}%
		\c@pgf@counta=\desirednumticks
		\advance\c@pgf@counta by-1
		\divide\H by\c@pgf@counta
%\message{determining ticks for #1-axis: Wr := (width/max space between ticks) = \the\Wr, desirednumticks=max(\axisdefaulttryminticks, trunc(Wr)) = \the\desirednumticks, H#1=(axis range/(desirednumticks-1)) = \the\H}%
		%
		% SEARCH for the NEXT FEASABLE H.
		\edef\Hmacro{\pgf@sys@tonumber\H}%
		\ifpgfplots@cur@is@linear
			% CASE LINEAR AXIS
			\ifpgfplots@is@datascaled
				% This here works if the scaling trafo is linear.
				\expandafter\pgfplots@data@scale@inverse@trafo\expandafter{\Hmacro}%
				\let\Hmacro=\pgfmathresult
			\else
				\pgfmathfloatparsenumber{\Hmacro}%
				\let\Hmacro=\pgfmathresult
			\fi
			\expandafter\pgfmathfloat@decompose\pgfmathresult\relax\pgfmathfloat@a@S\H\pgfmathfloat@a@E
%\message{Got T^{-1}(H#1) = \Hmacro}%
			% modify the mantisse:
			\ifdim\H<2pt
				\ifdim\H<1.5pt
					\H=1.0pt
				\else
					\H=2.0pt
				\fi
			\else
				\ifdim\H<4.9999pt
					\ifdim\H<3.5pt
						\H=2.0pt\relax
					\else
						\H=5.0pt\relax
					\fi
				\else
					\ifdim\H<7.5pt
						\H=5.0pt\relax
					\else
						\H=1.0pt\relax
						\advance\pgfmathfloat@a@E by1
					\fi
				\fi
			\fi
%\message{Computed H#1=\the\H [unscaled]; transforming ... }%
			\pgfmathfloatcreate{\the\pgfmathfloat@a@S}{\pgf@sys@tonumber{\H}}{\the\pgfmathfloat@a@E}%
			\let\Hmacro=\pgfmathresult
			\ifpgfplots@is@datascaled
				\pgfplots@data@scale@trafo\Hmacro
			\else
				\pgfmathfloattofixed\Hmacro
			\fi
			\let\Hmacro=\pgfmathresult
%\message{got H#1=\Hmacro\ [transformed]}%
			\expandafter\H\Hmacro pt
			\expandafter\aftergroup\csname pgfplots@determinedefaultvalues@#1isuniformtrue\endcsname
			%
			% Now, we want to activate the Tick set {i*H, i in \Z}
			% compute I such that
			%   xmin = I * H + rest;  |rest| < H
			% -> I = round(xmin/H)
			% -> MIN = I * H
			%
			% but first: eliminate any 'affine' data scaling!
			\ifpgfplots@is@datascaled
				\advance\MIN by\csname pgfplots@data@scale@trafo@SHIFT@#1\endcsname pt
			\fi
			\pgfmathlog@invoke@expanded\pgfmathdivide@{%
				{\pgf@sys@tonumber\MIN}%
				{\pgf@sys@tonumber\H}%
			}%
			\pgfmathsetcount{\c@pgf@counta}{\pgfmathresult}%
			\ifdim\MIN<0pt
				% the truncation rounds TOWARDS 0 which is not what I want.
				\advance\c@pgf@counta by-1
			\fi
			\MIN=\H
			\multiply\MIN by\c@pgf@counta
			\ifpgfplots@is@datascaled
				\advance\MIN by-\csname pgfplots@data@scale@trafo@SHIFT@#1\endcsname pt
			\fi
		\else
			% CASE LOG AXIS
			%
			% search for the "best" H= j* log(10),  j an integer.
			%
			% And prefer j=1 if that is possible (otherwise minor
			% ticks are not useful).
			\pgfmathmultiply@{\Hmacro}{\reciproclogten}%
			\expandafter\H\pgfmathresult pt
%\message{ [ H / log(10) = \pgfmathresult}%
			\ifdim\H<2pt
				\H=1pt
			\else
				\ifnum\H<1pt
					\H=1pt
				\else
					\expandafter\pgfmathfloor\expandafter{\pgfmathresult}%
					\expandafter\H\pgfmathresult pt
				\fi
			\fi
			\ifdim\H=1pt
				\expandafter\aftergroup\csname pgfplots@determinedefaultvalues@#1isuniformtrue\endcsname
				\pgfplots@isuniformticktrue
			\else
				\expandafter\aftergroup\csname pgfplots@determinedefaultvalues@#1isuniformfalse\endcsname
				\pgfplots@isuniformtickfalse
			\fi
%\message{final H=\pgf@sys@tonumber{\H} * log(10)}%
			\H=\logten\H\relax
			% Now, we want to activate the Tick set 
			%   {lowest, lowest+H, ..., highest}
			%
			% Where 
			% 	lowest =  I * log(10) + rest, |rest| < log(10).
			% this is conceptionally different from the approach for
			% linear axes, because H = j*log(10).
			%
			% remember the original xmin in MINH:
			\MINH=\MIN
			%
			% and compute I and I*log(10) here:
			\expandafter\MIN\reciproclogten\MIN\relax
			\edef\pgfmathresult{\pgf@sys@tonumber{\MIN}}%
			\pgfmathsetcount{\c@pgf@counta}{\pgfmathresult}%
			\ifdim\MIN<0pt
				% the truncation rounds TOWARDS 0 which is not what I want.
				\advance\c@pgf@counta by-1
			\fi
			\expandafter\MIN\logten pt
			\multiply\MIN by\c@pgf@counta
			\ifpgfplots@isuniformtick
			\else
				% This here is a special case to move the first tick
				% near the lower axis limit.
				%
				% "Near" means either directly above or directly below ymin.
				% 
				% My application example is as follows:
				% Let H = 2*log(10).
				% Furthermore, ymin = 3e-6, ymax= 8e-2. That means we can choose either
				%    10^{-5}, 10^{-3}, 10^{-1}
				% or
				%    10^{-4}, 10^{-2}
				% as ticks. Well, I prefer the first one.
				%
				% HEURISTICS: start as near to ymin as possible!
				%
				% We check here if we can come nearer to ymin if we
				% shift the current tick by log(10):
				%  if( ymin - I * log(10) < 0.5*H ->  use I+1, that means add log(10).
				%
				% that's equivalent to 
				%  2*(ymin - I * log(10)) - H < 0.
				\advance\MINH by-\MIN
				\multiply\MINH by2
				\advance\MINH by-\H
				% 
				\ifdim\MINH<0pt
					\expandafter\advance\expandafter\MIN\logten pt
				\fi
			\fi
		\fi
		\MINH=\MIN
		\advance\MINH by\H
%\pgfplots@message{final H=\the\H}%
		\xdef\pgfplots@TMP{\pgf@sys@tonumber{\MIN},\pgf@sys@tonumber{\MINH},...,\pgf@sys@tonumber{\MAX}}%
		\xdef\pgfplots@TMPB{\pgf@sys@tonumber{\H}}%
		\aftergroup\pgfplots@determinedefaultvalues@needs@check@uniformtickfalse
	\fi
	\endgroup
	\expandafter\let\csname pgfplots@#1tick\endcsname=\pgfplots@TMP
	\expandafter\let\csname pgfplots@tick@distance@#1\endcsname=\pgfplots@TMPB
%\pgfplots@message{pgfplots.sty: #1tick set to \csname pgfplots@#1tick\endcsname [#1mode=\the\csname pgfplots@#1mode\endcsname,  #1min=\csname pgfplots@#1min\endcsname, #1max=\csname pgfplots@#1max\endcsname].}%
}

% Helper method for 
%  \pgfplots@apply@data@scale@trafo@to@options@for
% #1: the ticks
% #2: the trafo macro name 
% #3: the output macro name
\long\def\pgfplots@apply@data@scale@trafo@to@user@ticks#1#2\to#3{%
	\let#3=\pgfutil@empty
	\foreach \pgfplots@TMPB in {#1} {%
		\pgfmathfloatparsenumber{\pgfplots@TMPB}%
		\expandafter#2\expandafter{\pgfmathresult}%
		\ifx#3\pgfutil@empty
			\xdef#3{\pgfmathresult}%
		\else
			\xdef#3{#3,\pgfmathresult}%
		\fi
	}%
	%
}%

% Helper method for 
%  \pgfplots@apply@data@scale@trafo@to@options@for
% #1: the ticks ALREADY IN FLOAT FORMAT
% #2: the trafo macro name 
% #3: the output macro name
\long\def\pgfplots@apply@data@scale@trafo@to@user@ticks@isfloat#1#2\to#3{%
	\let#3=\pgfutil@empty
	\foreach \pgfplots@TMPB in {#1} {%
		#2{\pgfplots@TMPB}%
		\ifx#3\pgfutil@empty
			\xdef#3{\pgfmathresult}%
		\else
			\xdef#3{#3,\pgfmathresult}%
		\fi
	}%
	%
}%

% helper for \pgfplots@apply@data@scale@trafo@to@options@for.
\def\pgfplots@compute@number@order@for@trafo@isdimen#1\tocount#2{%
	\edef\pgfplots@TMP{\pgf@sys@tonumber{#1}}%
	\pgfmathfloatparsenumber{\pgfplots@TMP}%
	\expandafter\pgfmathfloat@decompose@E\pgfmathresult\relax#2
	\advance#2 by1
}

% helper for \pgfplots@apply@data@scale@trafo@to@options@for.
% 
\def\pgfplots@compute@number@order@for@trafo@isfloat#1\tocount#2{%
	\expandafter\pgfmathfloat@decompose@E#1\relax#2
	\advance#2 by1
}

% Initialises the data scale transformation and applies it to any
% user specified options.
%
% PRECONDITION:
%   - all axis limits are available in float representation
%   - \pgfplots@set@default@size@options has been called before
% POSTCONDITION:
%   - the scaling transformation is set up,
%   - all axis limits are transformed,
%   - any user input (like ticks and tick labels)
%     will reflect the changes.
\def\pgfplots@apply@data@scale@trafo@to@options@for#1{%
	% initialise data scale transformation 
	%   T(x) = 10^{q-m} * x
	%
	\begingroup
	\let\data@max@order=\c@pgf@counta
	\let\data@cur@order=\c@pgf@countb
	\let\data@dimen=\pgf@xa
	\let\data@tmp=\pgf@xb
	\let\data@dimen@order=\c@pgf@countc
	\let\data@EXPONENT=\c@pgf@countd
	\expandafter\let\expandafter\pgfplots@min@float\csname pgfplots@#1min\endcsname
	\expandafter\let\expandafter\pgfplots@max@float\csname pgfplots@#1max\endcsname
	\expandafter\let\expandafter\pgfplots@actual@trafo\csname pgfplots@datascaletrafo@#1\endcsname
	%
	% Step 1: compute 'm', the data order
	\pgfplots@compute@number@order@for@trafo@isfloat
		\pgfplots@min@float
		\tocount\data@cur@order
	%
	\data@max@order=\data@cur@order
	%
	\pgfplots@compute@number@order@for@trafo@isfloat
		\pgfplots@max@float
		\tocount\data@cur@order
	%
	\ifnum\data@cur@order>\data@max@order
		\data@max@order=\data@cur@order
	\fi
	%
	% Step 2: compute 'q', the #1-size of the axis.
	\expandafter\ifx\csname pgfplots@#1\endcsname\pgfutil@empty
		% We have 'width' or 'height'.
		%
		% Use the order of these parameters.
		\def\pgfplots@TMP{#1}%
		\def\pgfplots@TMPB{x}%
		\ifx\pgfplots@TMP\pgfplots@TMPB
			\expandafter\data@dimen\pgfplots@width\relax
		\else
			\def\pgfplots@TMPB{y}%
			\ifx\pgfplots@TMP\pgfplots@TMPB
				\expandafter\data@dimen\pgfplots@height\relax
			\fi
		\fi
		\pgfplots@compute@number@order@for@trafo@isdimen
			\data@dimen
			\tocount\data@dimen@order
		% This here is to avoid inaccuracies in the final
		% axis rectangle size, see \pgfplots@initsizes:
		%\advance\data@dimen@order by-1
	\else
		% FIXME:
		% we have either the 'x=1cm' or 'y=1cm' option!
		% How should I initialise the trafo!?
		\data@dimen@order=3
	\fi
	%
%\message{Direction #1: data max order=\the\data@max@order;  data dimen order=\the\data@dimen@order. }%
	\data@EXPONENT=\data@dimen@order
	\advance\data@EXPONENT by-\data@max@order
	% Now, I introduce a loop which shall avoid cancellation of
	% significant digits.
	%
	% Harmless Example: 
	%  if we have data shift = -3 and 
	%  max = 2e6, min = 1e6, then max-min = 1e6; T(max)-T(min) = 1e3 which is ok.
	%  In this case, the loop won't change anything.
	%
	% Critical Example:
	%  if we have data shift = -3 and
	%  max = 1980, min = 1930 then 
	%    T(max) = 1.98 and T(min) = 1.93
	%  and thus T(max)-T(min) = 0.05 . 
	%  Considering that this is the axis range
	%  in which tick labels and plot points need to be computed, we
	%  only have two or three digits left! That happens because the
	%  prefix '19' is common and is cancelled in the subtraction.
	%  Idea: while T(max)-T(min) < O(10^2) -> increase shift by +1
	%  (and make sure that T(max) < MAX_VALID_TEX_NUMBER).
	%
	\pgfplots@loop@CONTINUEtrue
	\expandafter\edef\csname pgfplots@data@scale@trafo@SHIFT@#1\endcsname{0}%
	\loop
	\expandafter\edef\csname pgfplots@data@scale@trafo@EXPONENT@#1\endcsname{\the\data@EXPONENT}%
	\pgfplots@actual@trafo{\pgfplots@min@float}%
	\let\pgfplots@min@fixed=\pgfmathresult
	\ifpgfplots@loop@CONTINUE
		\pgfplots@actual@trafo{\pgfplots@max@float}%
		\let\pgfplots@max@fixed=\pgfmathresult
		\expandafter\data@tmp\pgfplots@max@fixed pt
%\message{Current trafo EXPONENT for #1 direction: \the\data@EXPONENT; original #1 limits: [\pgfplots@min@float:\pgfplots@max@float]; current transformed #1 limits: [\pgfplots@min@fixed:\pgfplots@max@fixed]; cancellation check max-min running...}%
		\ifdim\data@tmp<0pt
			% I need absolute values here:
			\multiply\data@tmp by-1\relax
		\fi
		\pgfmathsubtract@{\pgfplots@max@fixed}{\pgfplots@min@fixed}%
		\expandafter\data@dimen\pgfmathresult pt
		\pgfplots@loop@CONTINUEfalse
		\ifdim\data@tmp<1500pt% a multiplication with '10' results in max = 15000 which is the upper limit.
			\ifdim\data@dimen<100pt% I guess if max-min = O(100), we have quite good accuracy
				\ifdim\data@dimen<0.0001pt
				\else
					\advance\data@EXPONENT by1
					\pgfplots@loop@CONTINUEtrue
				\fi
			\fi
		\fi
		%--------------------------------------------------
		% \ifdim\data@dimen>1200pt% FIXME : is this here ok!? CHECK IT!
		% 	\ifdim\data@dimen>7999pt
		% 		\advance\data@EXPONENT by-2
		% 	\else
		% 		\advance\data@EXPONENT by-1
		% 	\fi
		% 	\pgfplots@loop@CONTINUEfalse
		% \fi
		%-------------------------------------------------- 
	\repeat
	\xdef\pgfplots@TMP{\csname pgfplots@data@scale@trafo@EXPONENT@#1\endcsname}%
	\xdef\pgfplots@TMPB{\pgfplots@min@fixed}%
	\endgroup
%\message{Initialising the data scale transformation in direction #1 to 10^\pgfplots@TMP*#1 - \pgfplots@TMPB...}%
	% COMPLETE INITIALISATION:
	\ifpgfplots@EMERGENCY@FORCE@DATA@TRAFO@TO@IDENTITY
		\pgfplots@warning{The automatic scaling of input data has been DISABLED to allow unscalable commands. DISABLING THE DATA SCALING LIMITS THE DATA RANGE! I hope I can fix this issue as soon as possible. Please take a look at the manual for more information. If you get 'OVERFLOW' or 'UNDERFLOW' errors, you may need to disable the axispath.}%
		\def\pgfplots@TMP{0}%
		\def\pgfplots@TMPB{0}%
	\fi
	\expandafter\let\csname pgfplots@data@scale@trafo@EXPONENT@#1\endcsname\pgfplots@TMP
	\expandafter\let\csname pgfplots@data@scale@trafo@SHIFT@#1\endcsname\pgfplots@TMPB%
	%
	% ... and apply transformation to any user input
	%
	% Transform axis limits:
%\message{#1-limits BEFORE data transformation: [\csname pgfplots@#1min\endcsname:\csname pgfplots@#1max\endcsname]}%
	\expandafter\expandafter\csname pgfplots@datascaletrafo@#1\endcsname\expandafter{\csname pgfplots@#1min\endcsname}%
	\expandafter\global\expandafter\let\csname pgfplots@#1min\endcsname=\pgfmathresult
	%
	\expandafter\expandafter\csname pgfplots@datascaletrafo@#1\endcsname\expandafter{\csname pgfplots@#1max\endcsname}%
	\expandafter\global\expandafter\let\csname pgfplots@#1max\endcsname=\pgfmathresult
%\message{#1-limits after data transformation: [\csname pgfplots@#1min\endcsname:\csname pgfplots@#1max\endcsname]}%
	%
	% Convert any user-specified ticks:
	\edef\pgfplots@TMP{\csname pgfplots@#1tick\endcsname}%
	% this here should also work with 'xtick=\pgfutil@empty', the "No tick" command.
	\ifx\pgfplots@TMP\pgfutil@empty
	\else
		\pgfplots@toka=\expandafter{\csname pgfplots@datascaletrafo@#1\endcsname}%
		\def\pgfplots@TMPB{data}%
		\ifx\pgfplots@TMP\pgfplots@TMPB
			% we have #1tick = data
			%
			% Since we have entered this method, we know that these
			% coordinate ARE ALREADY in floating point representation.
			\expandafter\let\expandafter\pgfplots@TMP\csname pgfplots@firstplot@coords@#1\endcsname
			\pgfplotslist@TOK@a=\expandafter{\pgfplots@TMP}%
			\edef\pgfplots@TMP{{\the\pgfplotslist@TOK@a}\the\pgfplots@toka}%
			\expandafter\pgfplots@apply@data@scale@trafo@to@user@ticks@isfloat\pgfplots@TMP\to\pgfplots@TMPC
		\else
			\pgfplotslist@TOK@a=\expandafter{\pgfplots@TMP}%
%\message{Converting #1tick='\csname pgfplots@#1tick\endcsname'}%
			\edef\pgfplots@TMP{{\the\pgfplotslist@TOK@a}\the\pgfplots@toka}%
			\expandafter\pgfplots@apply@data@scale@trafo@to@user@ticks\pgfplots@TMP\to\pgfplots@TMPC
		\fi
		\expandafter\let\csname pgfplots@#1tick\endcsname=\pgfplots@TMPC
	\fi
	%
	% Convert any extra-ticks, see above.
	\edef\pgfplots@TMP{\csname pgfplots@extra@#1tick\endcsname}%
	\ifx\pgfplots@TMP\pgfutil@empty
	\else
		\pgfplots@toka=\expandafter{\csname pgfplots@datascaletrafo@#1\endcsname}%
		\edef\pgfplots@TMP{{\csname pgfplots@extra@#1tick\endcsname}\the\pgfplots@toka}%
		\expandafter\pgfplots@apply@data@scale@trafo@to@user@ticks\pgfplots@TMP\to\pgfplots@TMPC
		\expandafter\let\csname pgfplots@extra@#1tick\endcsname=\pgfplots@TMPC
	\fi
	%
	% Transform any explicit axis unit scalings:
	\expandafter\ifx\csname pgfplots@#1\endcsname\pgfutil@empty
	\else
%\message{Converting #1 unit scale='\csname pgfplots@#1\endcsname' ... }%
		\expandafter\pgfmathparse\expandafter{\csname pgfplots@#1\endcsname}%
		\expandafter\expandafter\csname pgfplots@inverse@datascaletrafo@tofixed@#1@noshift\endcsname\expandafter{\pgfmathresult}%
		\expandafter\edef\csname pgfplots@#1\endcsname{\pgfmathresult pt}%
%\message{to #1='\csname pgfplots@#1\endcsname'.}%
	\fi
}

\newif\ifpgfplots@determinedefaultvalues@xisuniform
\newif\ifpgfplots@determinedefaultvalues@yisuniform
\newif\ifpgfplots@determinedefaultvalues@needs@check@uniformtick

\def\pgfplots@determinedefaultvalues{%
	\ifpgfplots@limits@are@computed
	\else
		% EMPTY AXIS:
		\pgfplots@warning{You have a plot with empty range. This may produce errors!}%
	\fi
	%
	\pgfplots@set@default@size@options
	%
	\ifpgfplots@apply@datatrafo@x
		\let\pgfplots@xmin@unscaled@as@float=\pgfplots@xmin
		\let\pgfplots@xmax@unscaled@as@float=\pgfplots@xmax
		\pgfplots@apply@data@scale@trafo@to@options@for x%
	\else
		\def\pgfplots@TMPB{data}%
		\ifx\pgfplots@xtick\pgfplots@TMPB
			\let\pgfplots@xtick=\pgfplots@firstplot@coords@x
		\fi
		\let\pgfplots@xmin@unscaled@as@float=\pgfutil@empty
		\let\pgfplots@xmax@unscaled@as@float=\pgfutil@empty
	\fi
	\ifpgfplots@apply@datatrafo@y
		\let\pgfplots@ymin@unscaled@as@float=\pgfplots@ymin
		\let\pgfplots@ymax@unscaled@as@float=\pgfplots@ymax
		\pgfplots@apply@data@scale@trafo@to@options@for y%
	\else
		\def\pgfplots@TMPB{data}%
		\ifx\pgfplots@ytick\pgfplots@TMPB
			\let\pgfplots@ytick=\pgfplots@firstplot@coords@y
		\fi
		\let\pgfplots@ymin@unscaled@as@float=\pgfutil@empty
		\let\pgfplots@ymax@unscaled@as@float=\pgfutil@empty
	\fi
	\pgfplots@datascaletrafo@initialisedtrue
	%
	% From now on, we can always work with pgfmath.
	% We simply need to apply the data scaling trafo before doing so.
	\pgfplots@float@numerics@mode@xfalse
	\pgfplots@float@numerics@mode@yfalse
	%
	\pgfkeysgetvalue{/pgfplots/minor x tick num}\pgfplots@minor@xtick@num
	\pgfkeysgetvalue{/pgfplots/minor y tick num}\pgfplots@minor@ytick@num
	\ifpgfplots@xislinear
		\ifnum\pgfplots@minor@xtick@num=0\relax
			\pgfplots@xminorticksfalse
			\pgfplots@xminorgridsfalse
		\else
			\pgfplots@xminortickstrue
		\fi
	\fi
	\ifpgfplots@yislinear
		\ifnum\pgfplots@minor@ytick@num=0\relax
			\pgfplots@yminorticksfalse
			\pgfplots@yminorgridsfalse
		\else
			\pgfplots@yminortickstrue
		\fi
	\fi
	%
	\pgfplots@enlarge@limit@ifconfigured x
	\pgfplots@enlarge@limit@ifconfigured y
	%
	\pgfplots@avoid@empty@axis@range@for x%
	\pgfplots@avoid@empty@axis@range@for y%
	%
	\pgfplots@initsizes
	%
	\pgfplots@determinedefaultvalues@xisuniformtrue
	\pgfplots@determinedefaultvalues@yisuniformtrue
	\pgfplots@determinedefaultvalues@needs@check@uniformticktrue
	\ifx\pgfplots@xtick\pgfutil@empty
		\pgfplots@assign@default@tick@foraxis{x}%
	\fi
	\ifpgfplots@determinedefaultvalues@needs@check@uniformtick
		\ifpgfplots@xislinear
			\expandafter\pgfplots@checkisuniformLINEARtick\expandafter{\pgfplots@xtick}%
			\ifpgfplots@isuniformtick
				\pgfplots@determinedefaultvalues@xisuniformtrue
			\else
				\pgfplots@determinedefaultvalues@xisuniformfalse
			\fi
		\else
			\expandafter\pgfplots@checkisuniformLOGtick\expandafter{\pgfplots@xtick}%
			\ifpgfplots@isuniformtick
				\pgfplots@determinedefaultvalues@xisuniformtrue
			\else
				\pgfplots@determinedefaultvalues@xisuniformfalse
			\fi
		\fi
	\fi
	\ifpgfplots@determinedefaultvalues@xisuniform
	\else
		\pgfplots@xminorticksfalse
	\fi
	%
	%
	%
	\pgfplots@determinedefaultvalues@needs@check@uniformticktrue
	\ifx\pgfplots@ytick\pgfutil@empty
		\pgfplots@assign@default@tick@foraxis{y}%
	\fi
	\ifpgfplots@determinedefaultvalues@needs@check@uniformtick
		\ifpgfplots@yislinear
			\expandafter\pgfplots@checkisuniformLINEARtick\expandafter{\pgfplots@ytick}%
			\ifpgfplots@isuniformtick
				\pgfplots@determinedefaultvalues@yisuniformtrue
			\else
				\pgfplots@determinedefaultvalues@yisuniformfalse
			\fi
		\else
			\expandafter\pgfplots@checkisuniformLOGtick\expandafter{\pgfplots@ytick}%
			\ifpgfplots@isuniformtick
				\pgfplots@determinedefaultvalues@yisuniformtrue
			\else
				\pgfplots@determinedefaultvalues@yisuniformfalse
			\fi
		\fi
	\fi
	\ifpgfplots@determinedefaultvalues@yisuniform
	\else
		\pgfplots@yminorticksfalse
	\fi
	%
	\ifx\pgfplots@xticklabel\pgfutil@empty
		\ifpgfplots@xislinear
			\def\pgfplots@xticklabel{\axisdefaultticklabel}%
		\else
			\def\pgfplots@xticklabel{\axisdefaultticklabellog}%
		\fi
	\fi
	\ifx\pgfplots@extra@xticklabel\pgfutil@empty
		\let\pgfplots@extra@xticklabel=\pgfplots@xticklabel
	\fi
	\ifx\pgfplots@yticklabel\pgfutil@empty
		\ifpgfplots@yislinear
			\def\pgfplots@yticklabel{\axisdefaultticklabel}%
		\else
			\def\pgfplots@yticklabel{\axisdefaultticklabellog}%
		\fi
	\fi
	\ifx\pgfplots@extra@yticklabel\pgfutil@empty
		\let\pgfplots@extra@yticklabel=\pgfplots@yticklabel
	\fi
	%
	%
	\pgfplots@prepare@ZERO@coordinates
}

% This code is mainly interesting for bar plots.
%
% It precomputes x = 0 and y = 0 - which is not necessarily
% trivial in case of data scaling. Furthermore, it applies
% coordinate clipping to the resulting values and multiplies them
% with x- and y scale vectors.
\def\pgfplots@prepare@ZERO@coordinates{%
	\ifpgfplots@xislinear
		\ifpgfplots@apply@datatrafo@x
			\pgfplots@datascaletrafo@fromfixed@x{0}%
			\global\let\pgfplots@ZERO@x=\pgfmathresult
		\else
			\gdef\pgfplots@ZERO@x{0}%
		\fi
		\pgfmathmax@{\pgfplots@ZERO@x}{\pgfplots@xmin}%
		\global\let\pgfplots@ZERO@x=\pgfmathresult
	\else
		\global\let\pgfplots@ZERO@x=\pgfplots@xmin%
	\fi
	%
	\ifpgfplots@yislinear
		\ifpgfplots@apply@datatrafo@y
			\pgfplots@datascaletrafo@fromfixed@y{0}%
			\global\let\pgfplots@ZERO@y=\pgfmathresult
		\else
			\gdef\pgfplots@ZERO@y{0}%
		\fi
		\pgfmathmax@{\pgfplots@ZERO@y}{\pgfplots@ymin}%
		\global\let\pgfplots@ZERO@y=\pgfmathresult
	\else
		\global\let\pgfplots@ZERO@y=\pgfplots@ymin%
	\fi
	%
	\pgfinterruptboundingbox%
	\pgfpointxy{\pgfplots@ZERO@x}{\pgfplots@ZERO@y}%
	\xdef\pgfplots@ZERO@x{\the\pgf@x}%
	\xdef\pgfplots@ZERO@y{\the\pgf@y}%
	\endpgfinterruptboundingbox%
}%


% Helper method for initsizes.
%
% It computes a scaling such that \pgfplots@width = SCALE * ACTUAL WIDTH.
% 
% The actual width is 
% 	c + x*(xmax-xmin)
% based on
% - x*xmax = \pgfplots@xcoordmaxTEX
% - x*xmin = \pgfplots@xcoordminTEX
% - c = estimated, a constant for the axis label/tick labels
%
% Arguments: 
% #1: the output argument for the SCALE.
\def\pgfplots@initsizes@getXscale\into#1{%
	\begingroup
	\pgf@xa=\pgfplots@width\relax
	% EXPECTED WIDTH = X = \pgfplots@width
	% ACTUAL WIDTH = c + x * (xmax-xmin)
	% where c is a CONSTANT (for the axis labels/tick labels).
	% -> \pgfplots@tmpXscale = (X - c) / (x *(xmax-xmin))
	%
	% \pgf@xa := X-c:
	\ifpgfplots@scale@only@axis
	\else
		\advance\pgf@xa by-45pt% FIXME determine 'c' correctly!
	\fi
	\ifdim\pgf@xa<0pt
		\pgfplots@error{Error: Plot width `\pgfplots@width' is too small. This can't be realised while maintaining constant width for y-labels. Sorry, label width are only approximate. You will need to adjust your width.}%
		\pgf@xa=0pt
	\fi
	% \pgf@xb := x*(xmax-xmin):
	\pgf@xb=\pgfplots@xcoordmaxTEX
	\advance\pgf@xb by-\pgfplots@xcoordminTEX
	\pgfmathlog@invoke@expanded\pgfmathdivide@{%
		{\pgf@sys@tonumber\pgf@xa}%
		{\pgf@sys@tonumber\pgf@xb}%
	}%
%\pgfplots@message{pgfplots.sty: Computing 'x' such that 'width = c + x*(xmax-xmin)';
%	c=estimated, 
%	width-c =\the\pgf@xa,  
%	x*(xmax[=\the\pgfplots@xcoordmaxTEX] - xmin[=\the\pgfplots@xcoordminTEX)]) = \the\pgf@xb  
%	-> x-scale =#1 }%
	\pgfmath@smuggleone\pgfmathresult
	\endgroup
	\let#1=\pgfmathresult
}

% The same as \pgfplots@initsizes@getXscale, just for the height.
\def\pgfplots@initsizes@getYscale\into#1{%
	\begingroup
	\expandafter\pgf@xa\pgfplots@height\relax
	% EXPECTED WIDTH = X = \pgfplots@width
	% ACTUAL WIDTH = c + x * (xmax-xmin)
	% where c is a CONSTANT (for the axis labels/tick labels).
	% -> \pgfplots@tmpXscale = (X - c) / (x *(xmax-xmin))
	%
	% \pgf@xa := X-c:
	\ifpgfplots@scale@only@axis
	\else
		\advance\pgf@xa by-45pt\relax% FIXME determine 'c' correctly!
	\fi
	\ifdim\pgf@xa<0pt
		\pgfplots@error{Error: Plot height `\pgfplots@height' is too small. This can't be realised while maintaining constant height for x-labels. Sorry, label heights are only approximate. You will need to adjust your height.}%
		\pgf@xa=0pt
	\fi
	% \pgf@xb := x*(xmax-xmin):
	\pgf@xb=\pgfplots@ycoordmaxTEX
	\advance\pgf@xb by-\pgfplots@ycoordminTEX
	\pgfmathlog@invoke@expanded\pgfmathdivide@{%
		{\pgf@sys@tonumber\pgf@xa}%
		{\pgf@sys@tonumber\pgf@xb}%
	}%
%\pgfplots@message{pgfplots.sty: Computing 'y' such that 'height = c + y*(ymax-ymin)';
%	height=\pgfplots@height,
%	c=estimated,
%	height-c =\the\pgf@xa,  
%	y*(ymax[=\the\pgfplots@ycoordmaxTEX] - ymin[=\the\pgfplots@ycoordminTEX)]) = \the\pgf@xb  
%	-> y-scale =#1 }%
	\pgfmath@smuggleone\pgfmathresult
	\endgroup
	\let#1=\pgfmathresult
}


\newif\ifpgfplots@avoid@emptyrange@@range@is@approx@equal
% Checks whether axis limits in coordinate #1 are approximately equal.
%
% If that is the case, force a non-zero width of the range.
%
\def\pgfplots@avoid@empty@axis@range@for#1{%
	% Check if axis limits are empty:
	\begingroup
	\expandafter\let\expandafter\if@cur@is@scaled\csname ifpgfplots@apply@datatrafo@#1\endcsname
	\expandafter\let\expandafter\min\csname pgfplots@#1min\endcsname
	\expandafter\let\expandafter\max\csname pgfplots@#1max\endcsname
	\let\min@d=\pgf@xa
	\let\max@d=\pgf@xb
	\let\diff=\pgf@xc
	\expandafter\min@d\min pt
	\expandafter\max@d\max pt
	\diff=\max@d
	\advance\diff by-\min@d
	% FIXME : I need a RELATIVE check here!
	% but: real number point division is expensive
	\if@cur@is@scaled
		% this here should be sufficient because the axis
		% has absolute values of order O( 10^3 ) or so.
		\ifdim\diff<0.01pt
			\pgfplots@avoid@emptyrange@@range@is@approx@equaltrue
		\fi
	\else
		% there is no data scaling, so I should be much more defensive
		% with absolute thresholds...
		\ifdim\diff<0.01pt
			\pgfplots@avoid@emptyrange@@range@is@approx@equaltrue
		\fi
	\fi
	\ifpgfplots@avoid@emptyrange@@range@is@approx@equal
		\pgfplots@warning{Axis range for axis #1 (transformed: [\min:\max]) is approximately equal; enlargeing it.}%
		% the case \min ~= \max
		%
		% enlarge \max and shrink \min:
		\ifdim\max@d<0pt
			\ifdim\max@d<-1pt
				\max@d=0.8\max@d
				\min@d=1.2\min@d
			\else
				\advance\max@d by-1pt
				\advance\min@d by1pt
			\fi
		\else
			\ifdim\max@d>1pt
				\max@d=1.2\max@d
				\min@d=0.8\min@d
			\else
				\ifdim\max@d=0pt
					\if@cur@is@scaled
						% we can't simply add a constant in the
						% transformed range.
						%
						% So: set limits to [-1,1] = [-1.0e0,+1.0e0]
						\pgfmathfloatcreate{2}{1.0}{0}%
						\csname pgfplots@datascaletrafo@#1\endcsname{\pgfmathresult}%
						\let\min=\pgfmathresult
						\pgfmathfloatcreate{1}{1.0}{0}%
						\csname pgfplots@datascaletrafo@#1\endcsname{\pgfmathresult}%
						\let\max=\pgfmathresult
%\pgfplots@message{[trafo shift = \csname pgfplots@data@scale@trafo@EXPONENT@#1\endcsname; setting limits -1:1]}%
						\expandafter\min@d\min pt
						\expandafter\max@d\max pt
					\else
						\advance\max@d by1pt
						\advance\min@d by-1pt
					\fi
				\else
					\advance\max@d by1pt
					\advance\min@d by-1pt
				\fi
			\fi
		\fi
		\xdef\pgfplots@TMP{\pgf@sys@tonumber{\min@d}}%
		\xdef\pgfplots@TMPB{\pgf@sys@tonumber{\max@d}}%
%\pgfplots@message{ -> #1 = \pgfplots@TMP : \pgfplots@TMPB;}%
	\else
		\global\let\pgfplots@TMP=\min%
		\global\let\pgfplots@TMPB=\max%
	\fi
	\endgroup
	\expandafter\global\expandafter\let\csname pgfplots@#1min\endcsname=\pgfplots@TMP
	\expandafter\global\expandafter\let\csname pgfplots@#1max\endcsname=\pgfplots@TMPB
}

% PRECONDITION:
% 	none
% POSTCONDITION:
% 	\pgfplots@default@aspect@ratio is set.
\def\pgfplots@compute@default@aspect@ratio{%
	\expandafter\pgfmath@x\axisdefaultwidth
	\expandafter\pgfmath@y\axisdefaultheight
	\pgfmathlog@invoke@expanded\pgfmathdivide@{%
		{\pgf@sys@tonumber{\pgfmath@x}}%
		{\pgf@sys@tonumber{\pgfmath@y}}%
	}%
	\let\pgfplots@default@aspect@ratio=\pgfmathresult
}

\def\pgfplots@set@default@size@options{%
	% The axes 'x' and 'y' vectors will be scaled such that the total
	% size is (\axisdefaultwidth, \axisdefaultheight).
	%
	% If the user specifies ONE of width OR height, 
	% the plot will be resized; keeping the aspect ratio.
	%
	\let\pgfplots@default@aspect@ratio=\pgfutil@empty
	% CASES:
	% hasx := 'x' option non-empty
	% hasy := 'y' option non-empty
	% W := 'width' option non-empty
	% H := 'height' option non-empty
	%
	% hasx = 1 -> width is not interesting; we use 'x' option.
	% hasx = 0 -> determine final width:
	% 		W H
	% 		0 0 -> \axisdefaultwidth 
	% 		0 1 -> determine width out of H and the default aspect ratio
	% 		1 X -> ok, use the user parameter.
	%
	% hasy = 1 -> height is not interesting, we use 'y' option.
	% hasy = 0 -> determine final height:
	% 		W H
	% 		0 0 -> \axisdefaultheight
	% 		X 1 -> ok, use the user parameter
	% 		1 0 -> determine height out of W and the default aspect ratio
	%
	\ifx\pgfplots@x\pgfutil@empty
		\ifx\pgfplots@y\pgfutil@empty
			% hasx=0, hasy=0 
			%
			% -> KEEP ASPECT RATIO if just one W, or H is given!
			\ifx\pgfplots@width\pgfutil@empty
				\ifx\pgfplots@height\pgfutil@empty
					% The case hasx=0, hasy=0,  W=0 H=0:
					\let\pgfplots@width=\axisdefaultwidth
					\let\pgfplots@height=\axisdefaultheight
				\else
					% The case hasx=0, hasy=0,  W=0 H=1:
					\pgfplots@compute@default@aspect@ratio
					\expandafter\pgfmath@y\pgfplots@height
					\pgfmathlog@invoke@expanded\pgfmathmultiply@{%
						{\pgf@sys@tonumber{\pgfmath@y}}%
						{\pgfplots@default@aspect@ratio}%
					}%
					\edef\pgfplots@width{\pgfmathresult pt}%
				\fi
			\else
				\ifx\pgfplots@height\pgfutil@empty
					% The case hasx=0, hasy=0,  W=1 H=0:
					\pgfplots@compute@default@aspect@ratio
					\expandafter\pgfmath@x\pgfplots@width
					\pgfmathlog@invoke@expanded\pgfmathdivide@{%
						{\pgf@sys@tonumber{\pgfmath@x}}%
						{\pgfplots@default@aspect@ratio}%
					}%
					\edef\pgfplots@height{\pgfmathresult pt}%
				\else
					% The case hasx=0, hasy=0,  W=1 H=1:
				\fi
			\fi
		\else
			% hasx=0, hasy=1, W=0:
			\ifx\pgfplots@width\pgfutil@empty
				\let\pgfplots@width=\axisdefaultwidth
			\fi
		\fi
	\else
		\ifx\pgfplots@y\pgfutil@empty
			% hasx=1, hasy=0, H=0
			\ifx\pgfplots@height\pgfutil@empty
				\let\pgfplots@height=\axisdefaultheight
			\fi
		\fi
	\fi
}

% PRECONDITION: 
% 	- final axis limits are given in transformed range
% 	-  \pgfplots@set@default@size@options has been invoked before
% POSTCONDITION: 
% 	- the current x- and y unit vectors are changed;
% 	- \pgfplots@[xy]coord{min,max}TEX  are set
%
\def\pgfplots@initsizes{%
	% INIT.
	%
	%
	\pgfpointxy{\pgfplots@xmin}{\pgfplots@ymin}%
	\pgfplots@xcoordminTEX=\pgf@x
	\pgfplots@ycoordminTEX=\pgf@y
	\pgfpointxy{\pgfplots@xmax}{\pgfplots@ymax}%
	\pgfplots@xcoordmaxTEX=\pgf@x
	\pgfplots@ycoordmaxTEX=\pgf@y
	%
	%
	%-----------------------------------------
	% PROCESS THE 'width' and 'height' options
	%-----------------------------------------
	%
	% FIXME: make these variables LOCAL:
	%
	\let\pgfplots@rectangle@width=\pgfutil@empty
	\let\pgfplots@rectangle@height=\pgfutil@empty
	%
	\ifx\pgfplots@x\pgfutil@empty
		\ifx\pgfplots@width\pgfutil@empty
			\pgfplots@error{INTERNAL LOGIC ERROR! WIDTH NOT SET}%
		\fi
		\pgfplots@initsizes@getXscale\into\pgfplots@tmpXscale
		\ifpgfplots@scale@only@axis
			\let\pgfplots@rectangle@width=\pgfplots@width
		\fi
	\else
		\def\pgfplots@tmpXscale{1}%
	\fi
	%
	\ifx\pgfplots@y\pgfutil@empty
		\ifx\pgfplots@height\pgfutil@empty
			\pgfplots@error{INTERNAL LOGIC ERROR! HEIGHT NOT SET}%
		\fi
		\pgfplots@initsizes@getYscale\into\pgfplots@tmpYscale
		\ifpgfplots@scale@only@axis
			\let\pgfplots@rectangle@height=\pgfplots@height
		\fi
	\else
		\def\pgfplots@tmpYscale{1}%
	\fi
	%
	% 
	% assert( \pgfplots@tmpXscale != \pgfutil@empty && \pgfplots@tmpYscale != \pgfutil@empty )
	%
	% Apply scaling:
	\pgfpointxy{\pgfplots@tmpXscale}{\pgfplots@tmpYscale}%
	\edef\pgfplots@tmpscale@x{\the\pgf@x}%
	\edef\pgfplots@tmpscale@y{\the\pgf@y}%
	\ifx\pgfplots@x\pgfutil@empty
		\pgfsetxvec{\pgfqpoint{\pgfplots@tmpscale@x}{0pt}}%
	\else
		\pgfsetxvec{\pgfpoint{\pgfplots@x}{0pt}}%
	\fi
	\ifx\pgfplots@y\pgfutil@empty
		\pgfsetyvec{\pgfqpoint{0pt}{\pgfplots@tmpscale@y}}%
	\else
		\pgfsetyvec{\pgfpoint{0pt}{\pgfplots@y}}%
	\fi
	%
	% Determine final rectangle dimensions.
	% There are the following cases: 
	% 1. the user really wants a fixed dimension,
	%    i.e. he used 'scale only axis'.
	%    Then, we have to work to get the correct dimension!
	%
	%    Up to now, the scaling mechanism looses to many significant
	%    digits such that the final width/height differs by 1-2 pt.
	%
	%    If I am not mistaken, this does ONLY affect the final size,
	%    not the relative plot precision.
	%    
	%    FIXME : really compute the plot precision!
	% 
	% 2. The use specified width and/or height, but not 'scale only
	%    axis'. Accept inaccurate final widths/heights (see above).
	%
	% 3. The user supplied 'x' and or 'y'. Simply use them, its
	% accurate.
	%
	\pgfpointxy{\pgfplots@xmin}{\pgfplots@ymin}%
	\pgfplots@xcoordminTEX=\pgf@x
	\pgfplots@ycoordminTEX=\pgf@y
	\pgfpointxy{\pgfplots@xmax}{\pgfplots@ymax}%
	\ifx\pgfplots@rectangle@width\pgfutil@empty
		\pgfplots@xcoordmaxTEX=\pgf@x
	\else
		% this 'if' here should only make a difference of about
		% 1-2pt, not more.
		%
		% and I am quite sure that this inaccuracy (and this
		% work-around) only affects the
		% final size, not the relative plot accuracy.
		\pgfplots@xcoordmaxTEX=\pgfplots@xcoordminTEX
		\expandafter\advance\expandafter\pgfplots@xcoordmaxTEX\pgfplots@width
	\fi
	\ifx\pgfplots@rectangle@height\pgfutil@empty
		\pgfplots@ycoordmaxTEX=\pgf@y
	\else
		\pgfplots@ycoordmaxTEX=\pgfplots@ycoordminTEX
		\expandafter\advance\expandafter\pgfplots@ycoordmaxTEX\pgfplots@height
	\fi
%--------------------------------------------------
% \begingroup
% \pgf@x=\pgfplots@xcoordmaxTEX\advance\pgf@x by-\pgfplots@xcoordminTEX
% \pgf@y=\pgfplots@ycoordmaxTEX\advance\pgf@y by-\pgfplots@ycoordminTEX
% \message{Axis scaling x=\pgfplots@tmpXscale, y=\pgfplots@tmpYscale\ yields lower-left-corner (\pgfplots@xmin,\pgfplots@ymin) = (\the\pgfplots@xcoordminTEX,\the\pgfplots@ycoordminTEX) and upper right (\pgfplots@xmax,\pgfplots@ymax) = (\the\pgfplots@xcoordmaxTEX,\the\pgfplots@ycoordmaxTEX).   Axis dimensions are width=\the\pgf@x, height=\the\pgf@y.}%
% \endgroup
%-------------------------------------------------- 
}

\pgfdeclareshape{pgfplots@low@level@shape@INNER}{%
	% FIXME : convert this thing into a light-weight node which is
	% ONLY valid inside of axis descriptions.
	\savedanchor\lowerleftinnercorner{%
		\pgfqpoint{\pgfplots@xcoordminTEX}{\pgfplots@ycoordminTEX}%
	}%
	\savedanchor\upperrightinnercorner{%
		\pgfqpoint{\pgfplots@xcoordmaxTEX}{\pgfplots@ycoordmaxTEX}%
	}%
	\savedanchor\origin{%
		\pgfqpoint{\pgfplots@ZERO@x\relax}{\pgfplots@ZERO@y\relax}%
	}%
	%
	\nodeparts{image}%
	\anchor{image}{%
		\pgf@x=0pt
		\pgf@y=0pt
	}%
	%
	%
	\anchor{center}{%
		\lowerleftinnercorner
		\pgf@xa=\pgf@x
		\pgf@xb=\pgf@y
		\upperrightinnercorner
		\advance\pgf@xa by\pgf@x
		\pgf@x=.5\pgf@xa
		\advance\pgf@xb by\pgf@y
		\pgf@y=.5\pgf@xb
	}%
	\anchor{north}{%
		\lowerleftinnercorner
		\pgf@xa=\pgf@x
		\pgf@xb=\pgf@y
		\upperrightinnercorner
		\advance\pgf@xa by\pgf@x
		\pgf@x=.5\pgf@xa
	}%
	\anchor{north east}{\upperrightinnercorner}%
	\anchor{east}{%
		\lowerleftinnercorner
		\pgf@xa=\pgf@x
		\pgf@xb=\pgf@y
		\upperrightinnercorner
		\advance\pgf@xb by\pgf@y
		\pgf@y=.5\pgf@xb
	}%
	\anchor{south east}{%
		\lowerleftinnercorner
		\pgf@xa=\pgf@x
		\pgf@xb=\pgf@y
		\upperrightinnercorner
		\pgf@y=\pgf@xb
	}%
	\anchor{south}{%
		\lowerleftinnercorner
		\pgf@xa=\pgf@x
		\pgf@xb=\pgf@y
		\upperrightinnercorner
		\advance\pgf@xa by\pgf@x
		\pgf@x=.5\pgf@xa
		\pgf@y=\pgf@xb
	}%
	\anchor{south west}{\lowerleftinnercorner}%
	\anchor{west}{%
		\lowerleftinnercorner
		\pgf@xa=\pgf@x
		\pgf@xb=\pgf@y
		\upperrightinnercorner
		\pgf@x=\pgf@xa
		\advance\pgf@xb by\pgf@y
		\pgf@y=.5\pgf@xb
	}%
	\anchor{north west}{%
		\lowerleftinnercorner
		\pgf@xa=\pgf@x
		\pgf@xb=\pgf@y
		\upperrightinnercorner
		\pgf@x=\pgf@xa
	}%
	%%
	\anchor{origin}{%
		\origin
	}%
	\anchor{above origin}{%
		\origin
		\pgf@xa=\pgf@x
		\upperrightinnercorner
		\pgf@x=\pgf@xa
	}%
	\anchor{left of origin}{%
		\origin
		\pgf@xa=\pgf@y
		\lowerleftinnercorner
		\pgf@y=\pgf@xa
	}%
	\anchor{right of origin}{%
		\origin
		\pgf@xa=\pgf@y
		\upperrightinnercorner
		\pgf@y=\pgf@xa
	}%
	\anchor{below origin}{%
		\origin
		\pgf@xa=\pgf@x
		\lowerleftinnercorner
		\pgf@x=\pgf@xa
	}%
	%%
	%%
	%%
	%%
	\anchorborder{}%
	\backgroundpath{}%\pgfpathrectangle{\pgfpointorigin}{\upperrightinnercorner}}%
	\foregroundpath{}%
	\behindbackgroundpath{}%
	\beforebackgroundpath{}%
	\behindforegroundpath{}%
	\beforeforegroundpath{}%
}

% This is the main axis shape.
%
% It has one node part, which is the complete image. It provides a lot
% of anchors.
\pgfdeclareshape{pgfplots@low@level@shape}{%
	\savedanchor\upperrightcorner{
		\pgf@x=\wd\pgfnodepartimagebox
		\pgf@y=\ht\pgfnodepartimagebox
	}%
	\savedanchor\lowerleftinnercorner{%
		\pgfqpoint{\pgfplots@savedanchor@inner@lowerleft@x\relax}{\pgfplots@savedanchor@inner@lowerleft@y\relax}%
	}%
	\savedanchor\upperrightinnercorner{%
		\pgfqpoint{\pgfplots@savedanchor@inner@upperright@x\relax}{\pgfplots@savedanchor@inner@upperright@y\relax}%
	}%
	\savedanchor\origin{%
		\pgfqpoint{\pgfplots@ZERO@x\relax}{\pgfplots@ZERO@y\relax}%
	}%
	%
	\nodeparts{image}%
	\anchor{image}{%
		\pgf@x=0pt
		\pgf@y=0pt
	}%
	%
	%
	\anchor{center}{%
		\lowerleftinnercorner
		\pgf@xa=\pgf@x
		\pgf@xb=\pgf@y
		\upperrightinnercorner
		\advance\pgf@xa by\pgf@x
		\pgf@x=.5\pgf@xa
		\advance\pgf@xb by\pgf@y
		\pgf@y=.5\pgf@xb
	}%
	\anchor{north}{%
		\lowerleftinnercorner
		\pgf@xa=\pgf@x
		\pgf@xb=\pgf@y
		\upperrightinnercorner
		\advance\pgf@xa by\pgf@x
		\pgf@x=.5\pgf@xa
	}%
	\anchor{north east}{\upperrightinnercorner}%
	\anchor{east}{%
		\lowerleftinnercorner
		\pgf@xa=\pgf@x
		\pgf@xb=\pgf@y
		\upperrightinnercorner
		\advance\pgf@xb by\pgf@y
		\pgf@y=.5\pgf@xb
	}%
	\anchor{south east}{%
		\lowerleftinnercorner
		\pgf@xa=\pgf@x
		\pgf@xb=\pgf@y
		\upperrightinnercorner
		\pgf@y=\pgf@xb
	}%
	\anchor{south}{%
		\lowerleftinnercorner
		\pgf@xa=\pgf@x
		\pgf@xb=\pgf@y
		\upperrightinnercorner
		\advance\pgf@xa by\pgf@x
		\pgf@x=.5\pgf@xa
		\pgf@y=\pgf@xb
	}%
	\anchor{south west}{\lowerleftinnercorner}%
	\anchor{west}{%
		\lowerleftinnercorner
		\pgf@xa=\pgf@x
		\pgf@xb=\pgf@y
		\upperrightinnercorner
		\pgf@x=\pgf@xa
		\advance\pgf@xb by\pgf@y
		\pgf@y=.5\pgf@xb
	}%
	\anchor{north west}{%
		\lowerleftinnercorner
		\pgf@xa=\pgf@x
		\pgf@xb=\pgf@y
		\upperrightinnercorner
		\pgf@x=\pgf@xa
	}%
	%%
	\anchor{origin}{%
		\origin
	}%
	\anchor{above origin}{%
		\origin
		\pgf@xa=\pgf@x
		\upperrightinnercorner
		\pgf@x=\pgf@xa
	}%
	\anchor{left of origin}{%
		\origin
		\pgf@xa=\pgf@y
		\lowerleftinnercorner
		\pgf@y=\pgf@xa
	}%
	\anchor{right of origin}{%
		\origin
		\pgf@xa=\pgf@y
		\upperrightinnercorner
		\pgf@y=\pgf@xa
	}%
	\anchor{below origin}{%
		\origin
		\pgf@xa=\pgf@x
		\lowerleftinnercorner
		\pgf@x=\pgf@xa
	}%
	%
	%%
	%
	\anchor{outer north}{%
		\upperrightcorner
		\pgf@x=.5\pgf@x
	}%
	\anchor{outer north east}{\upperrightcorner}%
	\anchor{outer east}{%
		\upperrightcorner
		\pgf@y=.5\pgf@y
	}%
	\anchor{outer south east}{%
		\upperrightcorner
		\pgf@y=0pt
	}%
	\anchor{outer south}{%
		\upperrightcorner
		\pgf@x=.5\pgf@x
		\pgf@y=0pt
	}%
	\anchor{outer south west}{%
		\pgf@x=0pt
		\pgf@y=0pt
	}%
	\anchor{outer west}{%
		\upperrightcorner
		\pgf@x=0pt
		\pgf@y=.5\pgf@y
	}%
	\anchor{outer north west}{%
		\upperrightcorner
		\pgf@x=0pt
	}%
	\anchor{outer center}{%
		\upperrightcorner
		\pgf@x=.5\pgf@x
		\pgf@y=.5\pgf@y
	}%
	%
	%
	%%
	%%
	\anchor{above north}{%
		\upperrightcorner
		\pgf@xa=\pgf@y
		\lowerleftinnercorner
		\pgf@xb=\pgf@x
		\upperrightinnercorner
		\advance\pgf@xb by\pgf@x
		\pgf@x=.5\pgf@xb
		\pgf@y=\pgf@xa
	}%
	\anchor{above north east}{%
		\upperrightcorner
		\pgf@xa=\pgf@y
		\upperrightinnercorner
		\pgf@y=\pgf@xa
	}%
	\anchor{right of north east}{%
		\upperrightcorner
		\pgf@xa=\pgf@x
		\upperrightinnercorner
		\pgf@x=\pgf@xa
	}%
	\anchor{right of east}{%
		\upperrightcorner
		\pgf@xa=\pgf@x
		\upperrightinnercorner
		\pgf@xb=\pgf@y
		\lowerleftinnercorner
		\advance\pgf@xb by\pgf@y
		\pgf@x=\pgf@xa
		\pgf@y=.5\pgf@xb
	}%
	\anchor{right of south east}{%
		\upperrightcorner
		\pgf@xa=\pgf@x
		\lowerleftinnercorner
		\pgf@x=\pgf@xa
	}%
	\anchor{below south east}{%
		\upperrightinnercorner
		\pgf@y=0pt
	}%
	\anchor{below south}{%
		\lowerleftinnercorner
		\pgf@xa=\pgf@x
		\upperrightinnercorner
		\pgf@y=0pt
		\advance\pgf@xa by\pgf@x
		\pgf@x=.5\pgf@xa
	}%
	\anchor{below south west}{%
		\lowerleftinnercorner
		\pgf@y=0pt
	}%
	\anchor{left of south west}{%
		\lowerleftinnercorner
		\pgf@x=0pt
	}%
	\anchor{left of west}{%
		\lowerleftinnercorner
		\pgf@xa=\pgf@y
		\upperrightinnercorner
		\advance\pgf@xa by\pgf@y
		\pgf@y=.5\pgf@xa
		\pgf@x=0pt
	}%
	\anchor{left of north west}{%
		\upperrightinnercorner
		\pgf@x=0pt
	}%
	\anchor{above north west}{%
		\upperrightcorner
		\pgf@xa=\pgf@y
		\lowerleftinnercorner
		\pgf@y=\pgf@xa
	}%
	%%
	%%
	\anchorborder{%
		% Call a function that computes a border point. Since this
		% function will modify dimensions like \pgf@x, we must move them to
		% other dimensions.
		\@tempdima=\pgf@x
		\@tempdimb=\pgf@y
		\pgfpointborderrectangle%
			{\pgfpoint{\@tempdima}{\@tempdimb}}%
			{\pgfpointadd{\upperrightcorner}{\pgfpoint{\width}{\height}}}
	}%
	\backgroundpath{\pgfpathrectangle{\pgfpointorigin}{\upperrightcorner}}%
	\foregroundpath{}%
	\behindbackgroundpath{}%
	\beforebackgroundpath{}%
	\behindforegroundpath{}%
	\beforeforegroundpath{}%
}

\def\pgfplots@BEGIN@init@and@draw@axis{%
	\pgfplots@determinedefaultvalues
	\setbox\pgfnodepartimagebox=\hbox\bgroup\bgroup
		\pgfinterruptpicture
		\tikzpicture[/pgfplots/every axis]%
		%\pgfqkeys{/tikz}{every axis}%
		% set baseline for sub-picture to default value.
		% the baseline option will be applied to the OUTER picture.
		\pgfsetbaseline{\pgf@picminy}%
		%
		\scope
		\ifpgfplots@hide@x\else
			% compute tick position lists
			% 	\pgfplots@prepared@tick@positions@minor@x
			% and
			% 	\pgfplots@prepared@tick@positions@major@x
			\expandafter\pgfplots@prepare@ticks@for\expandafter{\pgfplots@xtick}{x}{0}%
			\pgfplots@drawgridlines@for{x}{0}%
		\fi
		\ifpgfplots@hide@y\else
			\expandafter\pgfplots@prepare@ticks@for\expandafter{\pgfplots@ytick}{y}{1}%
			\pgfplots@drawgridlines@for{y}{1}%
		\fi
		%
		\ifpgfplots@hide@x\else
			\pgfplots@drawticklines@for{x}{0}%
		\fi
		\ifpgfplots@hide@y\else
			\pgfplots@drawticklines@for{y}{1}%
		\fi
		%
		\pgfplots@drawaxis@lines
		%
		\ifpgfplots@hide@x\else
			\pgfplots@drawticklabels@for{x}{0}%
		\fi
		\ifpgfplots@hide@y\else
			\pgfplots@drawticklabels@for{y}{1}%
		\fi
			%
		\ifpgfplots@hide@x\else
			\ifx\pgfplots@extra@xtick\pgfutil@empty
			\else
				\expandafter\pgfplots@draw@extra@ticks@for\expandafter x\expandafter0\expandafter{\pgfplots@extra@xtick}%
			\fi
		\fi
		\ifpgfplots@hide@y\else
			\ifx\pgfplots@extra@ytick\pgfutil@empty
			\else
				\expandafter\pgfplots@draw@extra@ticks@for\expandafter y\expandafter1\expandafter{\pgfplots@extra@ytick}%
			\fi
		\fi
		\clip
						(\pgfplots@xcoordminTEX,	\pgfplots@ycoordminTEX) 
			rectangle	(\pgfplots@xcoordmaxTEX,	\pgfplots@ycoordmaxTEX);
}


\def\pgfplots@END@init@and@draw@axis{%
	\endscope%
}

% Writes output to \pgfplots@TMP
\def\pgfplots@filter@input@ticks@with@log#1{%
	\let\pgfplots@TMP=\pgfutil@empty
	\foreach \pgfplots@TMPB in {#1} {%
		\expandafter\pgfmathlog@\expandafter{\pgfplots@TMPB}%
		\ifx\pgfplots@TMP\pgfutil@empty
			\xdef\pgfplots@TMP{\pgfmathresult}%
		\else
			\xdef\pgfplots@TMP{\pgfplots@TMP,\pgfmathresult}%
		\fi
	}%
}

\tikzdeclarecoordinatesystem{axis}{\pgfplots@evalute@tikz@coord@system@interface#1\pgfplots@coord@end}


% Assigns \pgfmathresult := canvas coordinate (#2) for axis #1.
\long\def\pgfplots@evalute@tikz@coord@system@interface@for#1#2{%
	\expandafter\let\expandafter\if@datascaled@cur\csname ifpgfplots@apply@datatrafo@#1\endcsname
	\csname ifpgfplots@#1islinear\endcsname
		\if@datascaled@cur
			\csname pgfplots@datascaletrafo@fromfixed@#1\endcsname{#2}%
		\else
			\def\pgfmathresult{#2}%
		\fi
	\else
		\pgfmathlog@{#2}%
	\fi
}

\long\def\pgfplots@evalute@tikz@coord@system@interface#1,#2\pgfplots@coord@end{%
	\begingroup
	\pgfplots@evalute@tikz@coord@system@interface@for{x}{#1}%
	\let\pgfplots@evaluate@tikz@coord@x=\pgfmathresult
	\pgfplots@evalute@tikz@coord@system@interface@for{y}{#2}%
	\xdef\pgfplots@TMP{\pgfplots@evaluate@tikz@coord@x pt}%
	\xdef\pgfplots@TMPB{\pgfmathresult pt}%
	\endgroup
	\pgfpointxy{\pgfplots@TMP}{\pgfplots@TMPB}%
}

% In case of (semi-) logplots, this command will 
% - assign a filter which invokes \pgfmathlog@{} for each coordinate
% - replace any user-specified coordinate by its log.
%
% All subsequent commands will then work with logarithmic coordinates.
%
% @see pgfmathlog.sty for details about the implementation of log().
%
% PRECONDITION: 
% - The user input options have been set correctly, 
% - the option processing has not yet begun
%
% POSTCONDITION: 
% - any user input for log-axis has been replaced by its log
% - coordinate filters to compute logs are installed
%
% See also:
%     \pgfplots@apply@data@scale@trafo@to@options@for
\def\pgfplots@prepare@coord@filtering@for#1{%
	\expandafter\def\csname pgfplots@prepare@#1coord\endcsname##1{%
		\def\pgfmathresult{##1}%
	}%
	\pgfkeysgetvalue{/pgfplots/#1filter}\pgfplots@TMP
	\ifx\pgfplots@TMP\pgfutil@empty
	\else
		\expandafter\let\csname pgfplots@#1filter@backwcompat\endcsname=\pgfplots@TMP
		\pgfplots@toka={/pgfplots/#1filter is deprecated. Please use /pgfplots/#1 filter/.code={\def\pgfmathresult{\#1}}}%
		\pgfplots@warning{\the\pgfplots@toka}%
		\pgfkeys{/pgfplots/#1 filter/.code={\csname pgfplots@#1filter@backwcompat\endcsname{##1}\to\pgfmathresult}}%
	\fi
	\csname pgfplots@float@numerics@mode@#1false\endcsname
	\csname ifpgfplots@#1islinear\endcsname
		\ifpgfplots@disabledatascaling
			\csname pgfplots@apply@datatrafo@#1false\endcsname
			\pgfplots@apply@datatrafofalse
		\else
			\csname pgfplots@apply@datatrafo@#1true\endcsname
			\csname pgfplots@float@numerics@mode@#1true\endcsname
			\pgfplots@apply@datatrafotrue
		\fi
		\ifpgfplots@apply@datatrafo
			% Check for any existing axis limits:
			\expandafter\let\expandafter\pgfplots@TMP\csname pgfplots@#1min\endcsname
			\ifx\pgfplots@TMP\pgfutil@empty
			\else
				\expandafter\pgfmathfloatparsenumber\expandafter{\pgfplots@TMP}%
				\expandafter\global\expandafter\let\csname pgfplots@#1min\endcsname=\pgfmathresult
			\fi
			\expandafter\let\expandafter\pgfplots@TMP\csname pgfplots@#1max\endcsname
			\ifx\pgfplots@TMP\pgfutil@empty
			\else
				\expandafter\pgfmathfloatparsenumber\expandafter{\pgfplots@TMP}%
				\expandafter\global\expandafter\let\csname pgfplots@#1max\endcsname=\pgfmathresult
			\fi
		\fi
	\else
		\ifpgfplots@disablelogfilter
		\else
			% any user-specified axis limits:
			\expandafter\let\expandafter\pgfplots@TMP\csname pgfplots@#1min\endcsname
			\ifx\pgfplots@TMP\pgfutil@empty
			\else
				\expandafter\pgfmathlog@\expandafter{\pgfplots@TMP}%
				\expandafter\global\expandafter\let\csname pgfplots@#1min\endcsname=\pgfmathresult
			\fi
			\expandafter\let\expandafter\pgfplots@TMP\csname pgfplots@#1max\endcsname
			\ifx\pgfplots@TMP\pgfutil@empty
			\else
				\expandafter\pgfmathlog@\expandafter{\pgfplots@TMP}%
				\expandafter\global\expandafter\let\csname pgfplots@#1max\endcsname=\pgfmathresult
			\fi
			%
			% any user specified axis ticks:
			\expandafter\let\expandafter\pgfplots@TMP\csname pgfplots@#1tick\endcsname
			\ifx\pgfplots@TMP\pgfutil@empty
			\else
				\def\pgfplots@TMPB{data}%
				\ifx\pgfplots@TMP\pgfplots@TMPB
				\else
					\expandafter\pgfplots@filter@input@ticks@with@log\expandafter{\pgfplots@TMP}%
					\expandafter\edef\csname pgfplots@#1tick\endcsname{\pgfplots@TMP}%
				\fi
			\fi
			\expandafter\let\expandafter\pgfplots@TMP\csname pgfplots@extra@#1tick\endcsname
			\ifx\pgfplots@TMP\pgfutil@empty
			\else
				\expandafter\pgfplots@filter@input@ticks@with@log\expandafter{\pgfplots@TMP}%
				\expandafter\edef\csname pgfplots@extra@#1tick\endcsname{\pgfplots@TMP}%
			\fi
			%
			\expandafter\def\csname pgfplots@prepare@#1coord\endcsname##1{%
				\pgfmathlog@{##1}%
			}%
		\fi
	\fi
}

\def\pgfplots@create@axis@descriptions{%
	\ifpgfplots@hide@x
	\else
		\ifx\pgfplots@xlabel\pgfutil@empty
		\else
			\pgfplots@show@label{x}%
		\fi
	\fi
	\ifpgfplots@hide@y
	\else
		\ifx\pgfplots@ylabel\pgfutil@empty
		\else
			\pgfplots@show@label{y}%
		\fi
	\fi
	\ifx\pgfplots@title\pgfutil@empty
	\else
		\pgfplots@show@title
	\fi
	\pgfplots@createlegend
}

\def\pgfplots@datascaletrafo@undoshift@x#1{%
	\pgfmathsubtract@{#1}{\pgfplots@data@scale@trafo@SHIFT@x}%
}%
\def\pgfplots@datascaletrafo@redoshift@x#1{%
	\pgfmathadd@{#1}{\pgfplots@data@scale@trafo@SHIFT@x}%
}%
\def\pgfplots@datascaletrafo@undoshift@y#1{%
	\pgfmathsubtract@{#1}{\pgfplots@data@scale@trafo@SHIFT@y}%
}%
\def\pgfplots@datascaletrafo@redoshift@y#1{%
	\pgfmathadd@{#1}{\pgfplots@data@scale@trafo@SHIFT@y}%
}%

% DATA TRANSFORMATION T(x) = X - xmin
%
% Input: 
%    a number in the original data range, given in **floating** point representation
% Output: 
%    a fixed point number in transformed range
%    stored in \pgfmathresult
% @see
% \pgfplots@datascaletrafo@fromfixed@x
\def\pgfplots@datascaletrafo@x#1{%
	\pgfmathfloatshift@{#1}{\pgfplots@data@scale@trafo@EXPONENT@x}%
	\expandafter\pgfmathfloattofixed\expandafter{\pgfmathresult}%
	\expandafter\pgfmathsubtract@\expandafter{\pgfmathresult}{\pgfplots@data@scale@trafo@SHIFT@x}%
}
\def\pgfplots@datascaletrafo@x@noshift#1{%
	\pgfmathfloatshift@{#1}{\pgfplots@data@scale@trafo@EXPONENT@x}%
	\expandafter\pgfmathfloattofixed\expandafter{\pgfmathresult}%
}

% Overloaded function.
% Input:
%     a FIXED point number instead of a floating point one.
% @see \pgfplots@datascaletrafo@x
\def\pgfplots@datascaletrafo@fromfixed@x#1{%
	\pgfmathfloatparsenumber{#1}%
	\expandafter\pgfplots@datascaletrafo@x\expandafter{\pgfmathresult}%
}

\def\pgfplots@datascaletrafo@y#1{%
	\pgfmathfloatshift@{#1}{\pgfplots@data@scale@trafo@EXPONENT@y}%
	\expandafter\pgfmathfloattofixed\expandafter{\pgfmathresult}%
	\expandafter\pgfmathsubtract@\expandafter{\pgfmathresult}{\pgfplots@data@scale@trafo@SHIFT@y}%
}
\def\pgfplots@datascaletrafo@fromfixed@y#1{%
	\pgfmathfloatparsenumber{#1}%
	\expandafter\pgfplots@datascaletrafo@y\expandafter{\pgfmathresult}%
}
\def\pgfplots@datascaletrafo@y@noshift#1{%
	\pgfmathfloatshift@{#1}{\pgfplots@data@scale@trafo@EXPONENT@y}%
	\expandafter\pgfmathfloattofixed\expandafter{\pgfmathresult}%
}

% INVERSE transformation x = T^{-1}( X )
% Input: 
%    a fixed point number in transformed domain
% Output:
%    a number in the data domain, given in floating point
%    representation
%
% If the input number is approximately X=0, we will return x=0 as
% well.
%
% This allows to handle rounding inaccuracies and should not pose any
% problems.
\def\pgfplots@inverse@datascaletrafo@x#1{%
	\begingroup
	\pgfmathadd@{#1}{\pgfplots@data@scale@trafo@SHIFT@x}%
	\let\pgfplots@inverse@datascaletrafo@@shifted=\pgfmathresult
	\pgfmathapproxequalto@{\pgfplots@inverse@datascaletrafo@@shifted}{0.0}%
	\ifpgfmathcomparison
		\pgfmathfloatcreate{0}{0.0}{0}%
	\else
		\pgfmathfloatparsenumber{\pgfplots@inverse@datascaletrafo@@shifted}%
		\edef\pgfplots@data@scale@trafo@EXPONENT@x{-\pgfplots@data@scale@trafo@EXPONENT@x}%
		\expandafter\pgfmathfloatshift@\expandafter{\pgfmathresult}{\pgfplots@data@scale@trafo@EXPONENT@x}%
	\fi
	\pgfmath@smuggleone\pgfmathresult
	\endgroup
}
\def\pgfplots@inverse@datascaletrafo@x@noshift#1{%
	\begingroup
	\pgfmathapproxequalto@{#1}{0.0}%
	\ifpgfmathcomparison
		\pgfmathfloatcreate{0}{0.0}{0}%
	\else
		\pgfmathfloatparsenumber{#1}%
		\edef\pgfplots@data@scale@trafo@EXPONENT@x{-\pgfplots@data@scale@trafo@EXPONENT@x}%
		\expandafter\pgfmathfloatshift@\expandafter{\pgfmathresult}{\pgfplots@data@scale@trafo@EXPONENT@x}%
	\fi
	\pgfmath@smuggleone\pgfmathresult
	\endgroup
}

\def\pgfplots@inverse@datascaletrafo@y#1{%
	\begingroup
	\pgfmathadd@{#1}{\pgfplots@data@scale@trafo@SHIFT@y}%
	\let\pgfplots@inverse@datascaletrafo@@shifted=\pgfmathresult
	\pgfmathapproxequalto@{\pgfplots@inverse@datascaletrafo@@shifted}{0.0}%
	\ifpgfmathcomparison
		\pgfmathfloatcreate{0}{0.0}{0}%
	\else
		\pgfmathfloatparsenumber{\pgfplots@inverse@datascaletrafo@@shifted}%
		\edef\pgfplots@data@scale@trafo@EXPONENT@y{-\pgfplots@data@scale@trafo@EXPONENT@y}%
		\expandafter\pgfmathfloatshift@\expandafter{\pgfmathresult}{\pgfplots@data@scale@trafo@EXPONENT@y}%
	\fi
	\pgfmath@smuggleone\pgfmathresult
	\endgroup
}
% Overloaded function. This one does only apply the scaling, no shift.
\def\pgfplots@inverse@datascaletrafo@y@noshift#1{%
	\begingroup
	\pgfmathapproxequalto@{#1}{0.0}%
	\ifpgfmathcomparison
		\pgfmathfloatcreate{0}{0.0}{0}%
	\else
		\pgfmathfloatparsenumber{#1}%
		\edef\pgfplots@data@scale@trafo@EXPONENT@y{-\pgfplots@data@scale@trafo@EXPONENT@y}%
		\expandafter\pgfmathfloatshift@\expandafter{\pgfmathresult}{\pgfplots@data@scale@trafo@EXPONENT@y}%
	\fi
	\pgfmath@smuggleone\pgfmathresult
	\endgroup
}

% Overloaded function.
%
% In contrast to \pgfplots@inverse@datascaletrafo@x, this method
% returns a number in FIXED point representation.
% @see \pgfplots@inverse@datascaletrafo@x
\def\pgfplots@inverse@datascaletrafo@tofixed@x#1{%
	\pgfplots@inverse@datascaletrafo@x{#1}%
	\expandafter\pgfmathfloattofixed\expandafter{\pgfmathresult}%
}
\def\pgfplots@inverse@datascaletrafo@tofixed@y#1{%
	\pgfplots@inverse@datascaletrafo@y{#1}%
	\expandafter\pgfmathfloattofixed\expandafter{\pgfmathresult}%
}

% Overloaded function.
%
% This one does only apply the scale, no shift.
\def\pgfplots@inverse@datascaletrafo@tofixed@x@noshift#1{%
	\pgfplots@inverse@datascaletrafo@x@noshift{#1}%
	\expandafter\pgfmathfloattofixed\expandafter{\pgfmathresult}%
}
\def\pgfplots@inverse@datascaletrafo@tofixed@y@noshift#1{%
	\pgfplots@inverse@datascaletrafo@y@noshift{#1}%
	\expandafter\pgfmathfloattofixed\expandafter{\pgfmathresult}%
}

% Parses all options in #1 which are known in the currently active families.
%
% The result will be stored back into the TikZ-style named #1 without 
% further processing.
%
% Example:
% \tikzstyle{every axis}=[xmin=0,xmax=1,line width=1pt
% \pgfplots@set@keys@from@tikz@style\tmpmacro{every axis}{/pgfplots}
% 
% - sets axis options 'xmin' and 'xmax'
% - calls \tikzstyle{every axis}={line width=1pt}
% 
% I assume that this method is called within local TeX groups so
% nothing will be destroyed outside.
%
% #1:  A style name.
\def\pgfplots@set@keys@from@tikz@style#1{%
	\let\pgfplots@rmopts=\pgfutil@empty
	\pgfqkeysfiltered{/pgfplots}{/pgfplots/#1}%
	\pgfplots@set@keymacro@to@style\pgfplots@rmopts{#1}%
}

% The same as \pgfplots@set@keys@from@tikz@style  but this one appends
% unmatched options to style #2.
%
% #1:  A style name.
% #2:  A style name which will be filled with unprocessed options.
\def\pgfplots@set@keys@from@tikz@style@append@to#1#2{%
	\let\pgfplots@rmopts=\pgfutil@empty
	\pgfqkeysfiltered{/pgfplots}{/pgfplots/#1}%
	\pgfplots@append@keymacro@to@style\pgfplots@rmopts{#2}%
}

% #1:  A sequence of options.
% #2:  A style name which will be filled with unprocessed options.
\def\pgfplots@set@keys@and@append@remaining@to@style#1#2{%
	\let\pgfplots@rmopts=\pgfutil@empty
	\pgfqkeysfiltered{/pgfplots}{#1}%
	\pgfplots@append@keymacro@to@style\pgfplots@rmopts{#2}%
}%

% #1: input macro
\def\pgfplots@setkeys@from@macro#1{%
	\let\pgfplots@rmopts=\pgfutil@empty
	\def\pgfplots@TMP{\pgfqkeysfiltered{/pgfplots}}%
	\expandafter\pgfplots@TMP\expandafter{#1}%
}

% #1: macro
% #2: style name
\long\def\pgfplots@append@keymacro@to@style#1#2{%
	\pgfplotslist@TOK@a={#2/.append style=}%
	\pgfplotslist@TOK@b=\expandafter{#1}%
	\edef\pgfplots@setkeys@TMP{\the\pgfplotslist@TOK@a{\the\pgfplotslist@TOK@b}}%
	\expandafter\pgfplotsset\expandafter{\pgfplots@setkeys@TMP}%
%\pgfplots@message{tikzstyle{#2}+=[#1]}%
}

% #1: macro
% #2: style name
\long\def\pgfplots@set@keymacro@to@style#1#2{%
	\pgfplotslist@TOK@a={#2/.style=}%
	\pgfplotslist@TOK@b=\expandafter{#1}%
	\edef\pgfplots@setkeys@TMP{\the\pgfplotslist@TOK@a{\the\pgfplotslist@TOK@b}}%
	\expandafter\pgfplotsset\expandafter{\pgfplots@setkeys@TMP}%
%\pgfplots@message{tikzstyle{#2}=[#1]}%
}

\def\pgfplots@preset@keys{%
  \def\pgfplots@at{\pgfqpoint{\the\tikz@lastx}{\the\tikz@lasty}}%
}

% backwards compatibility:
\let\prettyprintnumber=\pgfmathprintnumber%

\def\pgfplots@set@options#1{%
	\pgfplots@preset@keys
	%
	% Temporarily assign families to 'name' and 'alias' options.
	% This allows to get the names - they should not be appended to
	% 'every axis'!
	\pgfkeys{%
		/tikz/domain/.belongs to family=/pgfplots,
		/pgfplots/domain/.code={\pgfkeysalso{/tikz/domain=##1}},
		/tikz/name/.belongs to family=/pgfplots/naming commands,
		/tikz/alias/.belongs to family=/pgfplots/naming commands,
		% and provide aliases in the '/pgfplots/' tree to avoid 
		% search path problems just for these two options:
		/pgfplots/name/.belongs to family=/pgfplots/naming commands,
		/pgfplots/alias/.belongs to family=/pgfplots/naming commands,
		/pgfplots/name/.code={\pgfkeysalso{/tikz/name=##1}},
		/pgfplots/alias/.code={\pgfkeysalso{/tikz/alias=##1}},
		%
		%
		/pgf/key filter handlers/append filtered to/.install key filter handler=\pgfplots@rmopts,
	}%
	\let\tikz@alias=\pgfutil@empty
	\let\tikz@fig@name=\pgfutil@empty
	%
	% Step 1: acquire ONLY 'xmode' and 'ymode' (necessary to decide
	% which axis style shall be loaded):
		\let\pgfplots@rmopts=\pgfutil@empty
		\pgfkeysinstallkeyfilter
			{/pgf/key filters/active families or no family}
			{{/pgf/key filters/false}{/pgf/key filters/false}}
		%
		\pgfqkeysactivatesinglefamilyandfilteroptions{/pgfplots/scale}%
			{/pgfplots}
			{#1}%
	\let\pgfplots@remaining@input=\pgfplots@rmopts
	%
	% Step 2: parse any pgfplots options out of styles.
	\pgfkeysactivatefamily{/pgfplots/style commands}%
	\pgfkeysinstallkeyfilter
		{/pgf/key filters/active families or no family}% DEBUG}
		{{/pgf/key filters/is descendant of=/pgfplots}% for keys without family
		 {/pgf/key filters/false}% for unknown keys
		}%
	%
	\pgfkeysactivatefamilies
		{/pgfplots,/pgfplots/naming commands,/pgfplots/tick,/pgfplots/legend,/pgfplots/descriptions,/pgfplots/scale}
		{\pgfplots@deactivefamiliescmd}%
		\pgfplots@set@keys@from@tikz@style{every axis}%
		\pgfkeysdeactivatefamily{/pgfplots/scale}%
		%
		\ifpgfplots@xislinear
			\ifpgfplots@yislinear
				\pgfplots@set@keys@from@tikz@style@append@to{every linear axis}{every axis}%
			\else
				\pgfplots@set@keys@from@tikz@style@append@to{every semilogy axis}{every axis}%
			\fi
		\else
			\ifpgfplots@yislinear
				\pgfplots@set@keys@from@tikz@style@append@to{every semilogx axis}{every axis}%
			\else
				\pgfplots@set@keys@from@tikz@style@append@to{every loglog axis}{every axis}%
			\fi
		\fi
	\pgfplots@deactivefamiliescmd
	%
	% Acquire style commands and nameing commands from direct input
	% options '#1' BEFORE the 'every' styles are processed:
	\pgfkeysactivatefamily{/pgfplots/naming commands}%
		\pgfplots@setkeys@from@macro\pgfplots@remaining@input%
		\let\pgfplots@remaining@input=\pgfplots@rmopts
	\pgfkeysdeactivatefamily{/pgfplots/naming commands}%
	%
	% Now, any 'name' and 'alias' options have been processed. 
	%
	% Remember their current meaning and reset the tikz options!
	\let\pgfplots@fig@name=\tikz@fig@name
	\let\pgfplots@fig@alias=\tikz@alias
	\let\tikz@alias=\pgfutil@empty
	\let\tikz@fig@name=\pgfutil@empty
	%
	% And protocol all named sub-nodes! Their positions need to be
	% updated later.
	\let\pgfplots@named@child@node@list=\pgfutil@empty
	\pgfkeysgetvalue{/tikz/name/.@cmd}\pgfplots@old@name@impl
	\pgfkeysgetvalue{/tikz/alias/.@cmd}\pgfplots@old@alias@impl
	\pgfkeysdef{/tikz/name}{%
		\xdef\pgfplots@named@child@node@list{\pgfplots@named@child@node@list,{##1}}%
		\pgfplots@old@name@impl##1\pgfeov
	}%
	\pgfkeysdef{/tikz/alias}{%
		\xdef\pgfplots@named@child@node@list{\pgfplots@named@child@node@list,{##1}}%
		\pgfplots@old@alias@impl##1\pgfeov
	}%
	%
	% Now, continue to process the 'every' styles. Please note that
	% the 'legend style={}' like options have already been processed;
	% their values are already inside of the associated 'every'
	% styles.
	%
	% What I am doing here is: set every pgfplots-option directly, and
	% discard it from the every-style. Any non-pgfplots-option will
	% be set in its context.
	%
	\pgfkeysactivatefamily{/pgfplots/legend}%
		\pgfplots@set@keys@from@tikz@style{every axis legend}%
	\pgfkeysdeactivatefamily{/pgfplots/legend}%
	%
	%
	\pgfkeysactivatefamily{/pgfplots/descriptions}%
		\pgfplots@set@keys@from@tikz@style{every axis label}%
		\pgfplots@set@keys@from@tikz@style{every axis x label}%
		\pgfplots@set@keys@from@tikz@style{every axis y label}%
		\pgfplots@set@keys@from@tikz@style{every axis title}%
	\pgfkeysdeactivatefamily{/pgfplots/descriptions}%
	%
	\pgfkeysactivatefamily{/pgfplots/tick}%
		\pgfplots@set@keys@from@tikz@style{every tick}%
		\pgfplots@set@keys@from@tikz@style{every minor tick}%
		\pgfplots@set@keys@from@tikz@style{every major tick}%
		\pgfplots@set@keys@from@tikz@style{every axis grid}%
		\pgfplots@set@keys@from@tikz@style{every minor grid}%
		\pgfplots@set@keys@from@tikz@style{every major grid}%
	\pgfkeysdeactivatefamily{/pgfplots/tick}%
	%
	\pgfkeysactivatefamily{/pgfplots}%
		\pgfplots@set@keys@from@tikz@style{every axis plot}%
	\pgfkeysdeactivatefamily{/pgfplots}%
	%
	\pgfkeysactivatefamily{/pgfplots/descriptions}%
	\pgfkeysactivatefamily{/pgfplots/tick}%
		\pgfplots@set@keys@from@tikz@style{every x tick label}%
		\pgfplots@set@keys@from@tikz@style{every y tick label}%
		\pgfplots@set@keys@from@tikz@style{every tick label}%
	\pgfkeysdeactivatefamily{/pgfplots/tick}%
	\pgfkeysdeactivatefamily{/pgfplots/descriptions}%
	%
	% Step 3: Set all remaining options of '#1'. They should have
	% highest precedence.
	\pgfkeysactivatefamilies
		{/pgfplots,/pgfplots/tick,/pgfplots/legend,/pgfplots/descriptions}%
		{\pgfplots@deactivefamiliescmd}%
		\expandafter
			\pgfplots@set@keys@and@append@remaining@to@style
		\expandafter
			{\pgfplots@remaining@input}%
			{every axis}%
	\pgfplots@deactivefamiliescmd
%\pgfkeysgetvalue{/tikz/every axis/.@cmd}\pgfplots@TMP
%\message{every axis is now '\meaning\pgfplots@TMP'}%
	%
	\pgfkeysdeactivatefamily{/pgfplots/style commands}%
	\global\pgfkeysgetvalue{/pgfplots/xmin}{\pgfplots@xmin}%
	\global\pgfkeysgetvalue{/pgfplots/xmax}{\pgfplots@xmax}%
	\global\pgfkeysgetvalue{/pgfplots/ymin}{\pgfplots@ymin}%
	\global\pgfkeysgetvalue{/pgfplots/ymax}{\pgfplots@ymax}%
	%
	\def\pgfplots@TMP{data}%
	\pgfplots@collect@firstplot@astickfalse
	\ifx\pgfplots@xtick\pgfplots@TMP
		\pgfplots@collect@firstplot@asticktrue
	\fi
	\ifx\pgfplots@ytick\pgfplots@TMP
		\pgfplots@collect@firstplot@asticktrue
	\fi
	\ifpgfplots@collect@firstplot@astick
		\let\pgfplots@firstplot@coords@x=\pgfutil@empty
		\let\pgfplots@firstplot@coords@y=\pgfutil@empty
	\fi
}

\def\pgfplots@install@abbrev@commands{
	\let\pgfplots@orig@path=\path
	\let\pgfplots@orig@plot=\plot
	%
	\let\axispath=\pgfplots@path
	%
	\let\addplot=\pgfplots@addplot
	\let\plot=\addplot
	%
	\def\logten{2.3025851}%
	\def\reciproclogten{0.434294}%
	%
	\def\logi##1{%
		\ifcase##1
		\or0
		\or0.693147
		\or1.098612
		\or1.386294
		\or1.60943791
		\or1.7917594
		\or1.94591014
		\or2.07944154
		\or2.197224
		\fi
	}%
	%
	\let\legend=\pgfplots@command@legend
	\let\addlegendentry=\pgfplots@addlegendentry
}

\def\pgfplots@environment{%
	\pgfutil@ifnextchar[{%
		\pgfplots@environment@opt
	}{%
		\pgfplots@environment@opt[]%
	}%
}%

% temporary (local) variables inside of axis
\newif\ifpgfplots@autocomputelimits
\newif\ifpgfplots@autocompute@xlim
\newif\ifpgfplots@autocompute@ylim
\newif\ifpgfplots@apply@datatrafo@x
\newif\ifpgfplots@apply@datatrafo@y
\newif\ifpgfplots@apply@datatrafo
\newif\ifpgfplots@float@numerics@mode@x
\newif\ifpgfplots@float@numerics@mode@y
\newif\ifpgfplots@datascaletrafo@initialised
\newif\ifpgfplots@draw@at@end
\newif\ifpgfplots@limits@are@computed
\newif\ifpgfplots@EMERGENCY@FORCE@DATA@TRAFO@TO@IDENTITY

% Extracts single components of an entry of
% \pgfplots@stored@plotlist
%
% They are defined as
% \pgfplots@stored@current@precmd
% \pgfplots@stored@current@cmd
% \pgfplots@stored@current@data
% \pgfplots@stored@current@postcmd
\def\pgfplots@stored@plotlist@EXTRACTENTRY#1#2#3#4{%
	\def\pgfplots@stored@current@precmd{#1}%
	\def\pgfplots@stored@current@cmd{#2}%
	\def\pgfplots@stored@current@data{#3}%
	\def\pgfplots@stored@current@postcmd{#4}%
}

\def\pgfplots@environment@opt[#1]{%
	\begingroup
	\pgfplots@install@abbrev@commands
	\pgfplots@stacked@initialise
	%
	%
	% The explicit specification of 'x' and 'y' as 1pt is to avoid
	% numeric overflow/underflow during scale computations:
	%
	% The scaling (i.e. proper values for 'x' and 'y') will be
	% determined later-on, dependend on the axis limits.  Since axis
	% limits are implicitly in units of 1pt, it is reasonable to use
	% '1pt' here as well.
	\pgfsetxvec{\pgfqpoint{1pt}{0pt}}%
	\pgfsetyvec{\pgfqpoint{0pt}{1pt}}%
	%
	\pgfplots@set@options{#1}%
	%
	% --------------------
	% Allocations:
	% --------------------
	\global\pgfplots@EMERGENCY@FORCE@DATA@TRAFO@TO@IDENTITYfalse
	\global\pgfplotslistnewempty\pgfplots@plotspeclist
	\global\pgfplotslistnewempty\pgfplots@legend
	\global\pgfplotslistnewempty\pgfplots@stored@plotlist
	\global\pgfplots@numplots=0
	\let\pgfplots@already@computed@legend@node=\pgfutil@empty
	%
	% --------------------
	% Option preprocessing
	% --------------------
	\pgfplots@prepare@coord@filtering@for x
	\pgfplots@prepare@coord@filtering@for y
	\ifpgfplots@apply@datatrafo
		\pgfplots@datascaletrafo@initialisedfalse
	\else
		\pgfplots@datascaletrafo@initialisedtrue% there is no trafo.
	\fi
	%
	\ifx\pgfplots@xmin\pgfutil@empty
		\pgfplots@autocompute@xlimtrue
		\gdef\pgfplots@xmin{16300}%
		\ifpgfplots@float@numerics@mode@x
			\pgfmathfloatcreate{1}{1.0}{324}%
			\global\let\pgfplots@xmin=\pgfmathresult
		\fi
	\fi
	\ifx\pgfplots@xmax\pgfutil@empty
		\pgfplots@autocompute@xlimtrue
		\gdef\pgfplots@xmax{-16300}%
		\ifpgfplots@float@numerics@mode@x
			\pgfmathfloatcreate{2}{1.0}{324}%
			\global\let\pgfplots@xmax=\pgfmathresult
		\fi
	\fi
	\ifx\pgfplots@ymin\pgfutil@empty
		\pgfplots@autocompute@ylimtrue
		\gdef\pgfplots@ymin{16300}%
		\ifpgfplots@float@numerics@mode@y
			\pgfmathfloatcreate{1}{1.0}{324}%
			\global\let\pgfplots@ymin=\pgfmathresult
		\fi
	\fi
	\ifx\pgfplots@ymax\pgfutil@empty
		\pgfplots@autocompute@ylimtrue
		\gdef\pgfplots@ymax{-16300}%
		\ifpgfplots@float@numerics@mode@y
			\pgfmathfloatcreate{2}{1.0}{324}%
			\global\let\pgfplots@ymax=\pgfmathresult
		\fi
	\fi
	%
	\global\pgfplots@limits@are@computedtrue
	\ifpgfplots@autocompute@xlim
		\pgfplots@draw@at@endtrue
		\pgfplots@autocomputelimitstrue
		\global\pgfplots@limits@are@computedfalse
	\fi
	\ifpgfplots@autocompute@ylim
		\pgfplots@draw@at@endtrue
		\pgfplots@autocomputelimitstrue
		\global\pgfplots@limits@are@computedfalse
	\fi
	%
	%
	% --------------------
	% Start axis:
	% either postponed to \end or directly.
	% --------------------
	%
	% Since I've introduced a public \numplots macro, I should 
	% make sure it's filled correctly:
	\pgfplots@draw@at@endtrue
	%
	\ifpgfplots@collect@firstplot@astick
		\pgfplots@draw@at@endtrue
	\fi
	\ifpgfplots@apply@datatrafo
		% ALWAYS TRUE. This allows to use inline plots because the data
		% scale transformation will be applied after I know that it should
		% be disabled.
		\pgfplots@draw@at@endtrue
	\fi
	\ifpgfplots@stackedmode
		% Stacked plots require special attention: they are drawn in
		% REVERSE order.
		\pgfplots@draw@at@endtrue
	\else
		% we have no stacked plots and thus no reversing.
		\pgfplots@stacked@reversefalse
	\fi
	%
	% any \path command is invalid inside of an axis.
	% Use \axispath instead:
	\def\numplots{\the\pgfplots@numplots}%
	\let\path=\pgfplots@replacement@for@tikz@path
	\let\closedcycle=\pgfplots@path@closed@cycle
	\ifpgfplots@draw@at@end
		\let\pgfplots@nextcommand=\relax
	\else
		\def\pgfplots@nextcommand{\pgfplots@BEGIN@init@and@draw@axis}%
	\fi
	\pgfplots@nextcommand
}
\def\endpgfplots@environment@opt{%
	\xdef\numplots{\the\pgfplots@numplots}%
	\pgfkeysvalueof{/pgfplots/before end axis/.@cmd}\pgfeov%
	%
	% restore old \path command:
	\let\path=\pgfplots@orig@path
	\let\plot=\pgfplots@orig@plot
	%
	%\end{axis}:
	% --------------------
	%  All plotting commands have been read.
	%  -> apply postponed drawing commands!
	% --------------------
	\ifpgfplots@draw@at@end
		\def\pgfplots@nextcommand{%
			\pgfplots@BEGIN@init@and@draw@axis
			\pgfplots@stacked@initialise
			\ifpgfplots@stacked@reverse
				% This here applies any scaling trafos and assembles a
				% NEW \pgfplots@stored@plotlist!
				\pgfplots@stacked@finalize@stored@plots
			\fi
			\pgfplotslistforeach\pgfplots@stored@plotlist\as\pgfplots@TMP{%
				\expandafter\pgfplots@stored@plotlist@EXTRACTENTRY\pgfplots@TMP
%\message{Processing stored plot with precommand '\meaning\pgfplots@stored@current@precmd';  pgfplots@plotcmd '\meaning\pgfplots@stored@current@cmd' postcommand '\meaning\pgfplots@stored@current@postcmd'}%
				\pgfplots@stored@current@precmd
				\ifx\pgfplots@stored@current@cmd\pgfutil@empty
					\pgfplots@stored@current@data%
				\else
					% this code here means we REALLY have a plotting
					% command!
					\ifpgfplots@stacked@reverse
						% has already been processed above, in \pgfplots@stacked@finalize@stored@plots.
					\else
						\ifpgfplots@apply@datatrafo
							% Apply the data scaling transformation
							\expandafter\pgfplots@coord@stream@finalize@storedcoords@START\pgfplots@stored@current@data\to\pgfplots@stored@current@data%
						\fi
					\fi
%\message{calling plotcmd and plotdata: pgfplots@plotcmd '\meaning\pgfplots@stored@current@cmd' data '\meaning\pgfplots@stored@current@data'}%
					\expandafter\pgfplots@stored@current@cmd\pgfplots@stored@current@data
				\fi
				\pgfplots@stored@current@postcmd
			}%
			\global\pgfplotslistnewempty\pgfplots@stored@plotlist% delete contents.
		}%
	\else
		\let\pgfplots@nextcommand=\relax
	\fi
	\pgfplots@nextcommand
	\pgfplots@END@init@and@draw@axis
	%
	\begingroup
	\pgfkeysvalueof{/pgfplots/after end axis/.@cmd}\pgfeov%
	\endgroup
	%
	%
	\begingroup
		% set coordinate system to (0,0) rectangle (1,1) for descriptions:
		\pgftransformxshift{\pgfplots@xcoordminTEX}%
		\pgftransformyshift{\pgfplots@ycoordminTEX}%
		%
		% create a 'current axis' node for anchor references. 
		\pgfmultipartnode{pgfplots@low@level@shape@INNER}{south west}{current axis}{\pgfusepath{discard}}%
		%
		\pgfplots@tmpa=\pgfplots@xcoordmaxTEX
		\advance\pgfplots@tmpa by-\pgfplots@xcoordminTEX
		\pgfsetxvec{\pgfqpoint{\pgfplots@tmpa}{0cm}}%
		%
		\pgfplots@tmpa=\pgfplots@ycoordmaxTEX
		\advance\pgfplots@tmpa by-\pgfplots@ycoordminTEX
		\pgfsetyvec{\pgfqpoint{0cm}{\pgfplots@tmpa}}%
		%
		\pgfplots@create@axis@descriptions
		\pgfkeysvalueof{/pgfplots/extra description/.@cmd}\pgfeov%
	\endgroup
	\endtikzpicture%
	\begingroup
		% Protocol sizes for the axis-shape.
		% I fear that needs to be done globally, do avoid all those
		% \endgroup's in and after \endpgfinterruptpicture ...
		\ifdim\pgf@picmaxx=-16000pt\relax%
			\pgf@picmaxx=0pt\relax%
			\pgf@picminx=0pt\relax%
			\pgf@picmaxy=0pt\relax%
			\pgf@picminy=0pt\relax%
		\fi%
		%
		\xdef\pgfplots@saveddimen@picminx{\the\pgf@picminx}%
		\xdef\pgfplots@saveddimen@picminy{\the\pgf@picminy}%
		%
		\pgf@xa=\pgfplots@xcoordmaxTEX
		\advance\pgf@xa by-\pgf@picminx
		\xdef\pgfplots@savedanchor@inner@upperright@x{\the\pgf@xa}%
		%
		\pgf@xa=\pgfplots@xcoordminTEX
		\advance\pgf@xa by-\pgf@picminx
		\xdef\pgfplots@savedanchor@inner@lowerleft@x{\the\pgf@xa}%
		%
		\pgf@xa=\pgfplots@ycoordmaxTEX
		\advance\pgf@xa by-\pgf@picminy
		\xdef\pgfplots@savedanchor@inner@upperright@y{\the\pgf@xa}%
		%
		\pgf@xa=\pgfplots@ycoordminTEX
		\advance\pgf@xa by-\pgf@picminy
		\xdef\pgfplots@savedanchor@inner@lowerleft@y{\the\pgf@xa}%
		%
		\pgf@xa=\pgfplots@ZERO@x
		\advance\pgf@xa by-\pgf@picminx
		\xdef\pgfplots@ZERO@x{\the\pgf@xa}%
		%
		\pgf@xa=\pgfplots@ZERO@y
		\advance\pgf@xa by-\pgf@picminy
		\xdef\pgfplots@ZERO@y{\the\pgf@xa}%
	\endgroup
	\endpgfinterruptpicture
	\egroup\egroup% end of pgfnodepartimagebox
	%
	\let\tikz@fig@name=\pgfplots@fig@name
	\tikz@fig@mustbenamed
    \pgftransformshift{\pgfplots@at}%
	\pgfmultipartnode{pgfplots@low@level@shape}{\pgfplots@anchorname}{\tikz@fig@name}{\pgfusepath{discard}}%
	\pgfplots@fig@alias
	\pgfnodealias{current axis}{\tikz@fig@name}%
	%
	\pgfplots@finally@correct@child@node@positions
	\pgfplots@stacked@finalize
	\endgroup
}

% Now, we need to process all named nodes inside of our
% axis-image.
%
% The situation at this point is as follows:
% 1. the complete axis image has been "typeset" into a box. That
% means its coordinate system is LOST up to those variables
% which have been saved explicitly.
%
% 2. the \pgfmultipartnode above knows about all axis anchors and
% saved dimensions.
%
% 3. All sub-nodes don't know about their position any more. Any
% saved anchors are wrong.
%
% The approach:
% 1. we shift each named node's saved anchors such that it's
% coordinate is valid inside of the TeX box.
%
% 2. we also shift each named node's saved anchors to reflect the
% axis' anchor.
%
% Afterwards, everything should be fine.
\def\pgfplots@finally@correct@child@node@positions{%
   \ifx\pgfplots@named@child@node@list\pgfutil@empty%
   \else%
      	\begingroup
		\pgftransformreset% FIXME: what's that for!? Copied from matrix code...
		%
		% Use the 'image' anchor here - the internal anchor
		% transformation matrix already has the shift for
		% \pgfplots@anchorname.
		\pgfpointanchor{\tikz@fig@name}{image}%
		\pgf@xa=\pgf@x
		\pgf@xb=\pgf@y
		\pgf@process{\pgfqpoint{\pgfplots@saveddimen@picminx}{\pgfplots@saveddimen@picminy}}%
		\advance\pgf@xa by-\pgf@x
		\advance\pgf@xb by-\pgf@y
		\pgf@x=\pgf@xa
		\pgf@y=\pgf@xb
		\edef\pgfplots@offset{\noexpand\pgfqpoint{\the\pgf@x}{\the\pgf@y}}%
		%
		\pgfutil@for\pgfplots@child@node@name:=\pgfplots@named@child@node@list\do{%
			\ifx\pgfplots@child@node@name\pgfutil@empty
			\else
				\expandafter\ifx\csname pgfplots@child@node@visited@\pgfplots@child@node@name\endcsname\relax%
					\pgfutil@ifundefined{pgf@sh@nt@\pgfplots@child@node@name}{%
						\pgfplots@warning{could not adjust coordinates of named node '\pgfplots@child@node@name' for reasons I do not understand! After finishing the image, it did no longer exist!? Sorry.}%
					}{%
						\pgf@shift@node{\pgfplots@child@node@name}{\pgfplots@offset}%
						\expandafter\let\csname pgfplots@child@node@visited@\pgfplots@child@node@name\endcsname=\pgfutil@empty%
					}%
				\fi
			\fi
		}%
		\endgroup
    \fi%
}%

\def\pgfplots@environment@axis{%
	\pgfutil@ifnextchar[{\pgfplots@@environment@axis}{\pgfplots@@environment@axis[]}%
}
\let\endpgfplots@environment@axis=\endpgfplots@environment@opt
\def\pgfplots@@environment@axis[#1]{%
	\pgfplots@environment@opt[/pgfplots/xmode=linear,/pgfplots/ymode=linear,#1]%
}

\def\pgfplots@environment@semilogxaxis{%
	\pgfutil@ifnextchar[{\pgfplots@@environment@semilogxaxis}{\pgfplots@@environment@semilogxaxis[]}%
}
\let\endpgfplots@environment@semilogxaxis=\endpgfplots@environment@opt
\def\pgfplots@@environment@semilogxaxis[#1]{%
	\pgfplots@environment@opt[/pgfplots/xmode=log,/pgfplots/ymode=linear,#1]%
}

\def\pgfplots@environment@semilogyaxis{%
	\pgfutil@ifnextchar[{\pgfplots@@environment@semilogyaxis}{\pgfplots@@environment@semilogyaxis[]}%
}
\let\endpgfplots@environment@semilogyaxis=\endpgfplots@environment@opt
\def\pgfplots@@environment@semilogyaxis[#1]{%
	\pgfplots@environment@opt[/pgfplots/xmode=linear,/pgfplots/ymode=log,#1]%
}

\def\pgfplots@environment@loglogaxis{%
	\pgfutil@ifnextchar[{\pgfplots@@environment@loglogaxis}{\pgfplots@@environment@loglogaxis[]}%
}
\let\endpgfplots@environment@loglogaxis=\endpgfplots@environment@opt
\def\pgfplots@@environment@loglogaxis[#1]{%
	\pgfplots@environment@opt[/pgfplots/xmode=log,/pgfplots/ymode=log,#1]%
}


\pgfutil@ifundefined{tikzaddtikzonlycommandshortcutlet}{%
	\def\tikzaddtikzonlycommandshortcutlet#1#2{%
		\expandafter\def\expandafter\tikz@installcommands\expandafter{\tikz@installcommands
			\let#1=#2
		}%
	}%
}{}

\tikzaddtikzonlycommandshortcutlet\axis\pgfplots@environment@axis
\tikzaddtikzonlycommandshortcutlet\endaxis\endpgfplots@environment@axis

\tikzaddtikzonlycommandshortcutlet\semilogxaxis\pgfplots@environment@semilogxaxis
\tikzaddtikzonlycommandshortcutlet\endsemilogxaxis\endpgfplots@environment@semilogxaxis

\tikzaddtikzonlycommandshortcutlet\semilogyaxis\pgfplots@environment@semilogyaxis
\tikzaddtikzonlycommandshortcutlet\endsemilogyaxis\endpgfplots@environment@semilogyaxis

\tikzaddtikzonlycommandshortcutlet\loglogaxis\pgfplots@environment@loglogaxis
\tikzaddtikzonlycommandshortcutlet\endloglogaxis\endpgfplots@environment@loglogaxis
