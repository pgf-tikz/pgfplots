%--------------------------------------------
%
% Package pgfplots
%
% Provides a user-friendly interface to create function plots (normal
% plots, semi-logplots and double-logplots).
% 
% It is based on Till Tantau's PGF package.
%
% Copyright 2007/2008 by Christian Feuersänger.
%
% This program is free software: you can redistribute it and/or modify
% it under the terms of the GNU General Public License as published by
% the Free Software Foundation, either version 3 of the License, or
% (at your option) any later version.
% 
% This program is distributed in the hope that it will be useful,
% but WITHOUT ANY WARRANTY; without even the implied warranty of
% MERCHANTABILITY or FITNESS FOR A PARTICULAR PURPOSE.  See the
% GNU General Public License for more details.
% 
% You should have received a copy of the GNU General Public License
% along with this program.  If not, see <http://www.gnu.org/licenses/>.
%
%--------------------------------------------

\input pgfplotsliststructure.code.tex
\input pgfnumtable.code.tex
\input pgfkeysfiltered.pgfplots.code.tex
\input pgfmathfloat.pgfplots.code.tex

% FIXME: reduce number of variables!
\newtoks\pgfplots@toka

\newcount\axisticknum
\newcount\pgfplots@tickposnum
\newcount\pgfplots@xmode
\newcount\pgfplots@ymode
\newcount\pgfplots@numplots
\newdimen\pgfplots@xcoordminTEX
\newdimen\pgfplots@xcoordmaxTEX
\newdimen\pgfplots@ycoordminTEX
\newdimen\pgfplots@ycoordmaxTEX
\newdimen\pgfplots@tmpa
\newdimen\pgfplots@tmpb
\newif\ifpgfplots@isuniformtick
\newif\ifpgfplots@addplot@table@from@macro
\newif\ifpgfplots@clip@limits
\newif\ifpgfplots@enlargelimits
\newif\ifpgfplots@enlargelimits@rel@thresh
\newif\ifpgfplots@enlargelimits@auto
\newif\ifpgfplots@xminorticks
\newif\ifpgfplots@xmajorticks
\newif\ifpgfplots@yminorticks
\newif\ifpgfplots@ymajorticks
\newif\ifpgfplots@xminorgrids
\newif\ifpgfplots@xmajorgrids
\newif\ifpgfplots@yminorgrids
\newif\ifpgfplots@ymajorgrids
\newif\ifpgfplots@scaled@ticks
\newif\ifpgfplots@identify@log@minor@tick@pos
\newif\ifpgfplots@disablelogfilter
\newif\ifpgfplots@disabledatascaling
\newif\ifpgfplots@hide
\newif\ifpgfplots@is@old@list@format
\newif\ifpgfplots@scale@only@axis
\let\pgfplots@TMP=\pgfutil@empty
\let\pgfplots@TMPB=\pgfutil@empty
\let\pgfplots@TMPC=\pgfutil@empty
\let\pgfnodepartimagebox=\pgfnodeparttextbox

\def\pgfplots@error#1{\PackageError{pgfplots}{#1}{}}%
\def\pgfplots@warning#1{\pgfplots@message{#1}}%
\def\pgfplots@message#1{%
	\edef\pgfplots@message@@{#1}%
	\expandafter\write\expandafter-\expandafter1\expandafter{\pgfplots@message@@}%
}%

\def\pgfplots@leftover{}
\def\pgfplots@rightover{}
\def\axisdefaultwidth{240pt}
\def\axisdefaultheight{207pt}


% Assigns list contents #2 to a list macro #1.
%
% The list contents may be provided in one of two formats:
% a) in the (deprecated) list format 'first\\second\\thirst\\'
% or
% b) in the PGF foreach list format 'first,second,third'.
\def\pgfplots@assign@list#1#2{%
	\def\pgfplots@TMP{#2}%
	\ifx\pgfplots@TMP\pgfutil@empty
		\pgfplotslistnewempty#1%
	\else
		\pgfplots@check@backwards@compatible@list@format #2\\\pgfplots@EOI
		\ifpgfplots@is@old@list@format
			\pgfplotslistnew#1{#2}%
		\else
			\pgfplots@foreach@to@list{#2}\to#1
		\fi
	\fi
}%

\def\pgfplots@EOI{\pgfplots@EOI}%

% Sets the boolean \ifpgfplots@is@old@list@format  to true if and only
% if the input is a list in the format '{first\\second\\}'.
%
% This format is deprecated, but is still provided for backwards
% compatibility.
%
% Usage:
% \pgfplots@check@backwards@compatible@list@format <argument>'\\'\pgfplots@EOI
% you NEED to append '\\\pgfplots@EOI' at the end.
\def\pgfplots@check@backwards@compatible@list@format#1\\#2\pgfplots@EOI{%
	\def\pgfplots@TMP{#2}%
	\ifx\pgfplots@TMP\pgfutil@empty
		\pgfplots@is@old@list@formatfalse
	\else
		\pgfplots@is@old@list@formattrue
	\fi
}%



% Creates a named plot cycle list.
%
% #1:  the name of the final list. Can be used in 'cycle list name'
% #2:  the list entries. You can use either a comma-separated list or
%      a '\\'-terminated list. The latter case also requires '\\'
%      AFTER the last entry.
\def\pgfcreateplotcyclelist#1#2{\pgfplots@assign@list#1{#2}}

\pgfcreateplotcyclelist{\blackwhiteplotspeclist}{%
	mark options={fill=gray},mark=*\\%
	mark options={fill=gray},mark=square*\\%
	mark options={fill=gray},mark=otimes*\\%
	mark=star\\%
	mark options={fill=gray},mark=diamond*\\%
	densely dashed,mark options={solid,fill=gray},mark=*\\%
	densely dashed,mark options={solid,fill=gray},mark=square*\\%
	densely dashed,mark options={solid,fill=gray},mark=otimes*\\%
	densely dashed,mark options={solid},mark=star\\%
	densely dashed,mark options={solid,fill=gray},mark=diamond*\\%
}
\pgfcreateplotcyclelist{\coloredplotspeclist}{%
	blue,mark options={fill=blue!80!black},mark=*\\%
	red,mark options={fill=red!80!black},mark=square*\\%
	brown!60!black,mark options={fill=brown!80!black},mark=otimes*\\%
	black,mark=star\\%
	blue,mark options={fill=blue!80!black},mark=diamond*\\%
	red,densely dashed,mark options={solid,fill=red!80!black},mark=*\\%
	brown!60!black,densely dashed,mark options={solid,fill=brown!80!black},mark=square*\\%
	black,densely dashed,mark options={solid,fill=gray},mark=otimes*\\%
	blue,densely dashed,mark=star,mark options=solid\\%
	red,densely dashed,mark options={solid,fill=red!80!black},mark=diamond*\\%
}

\pgfkeys{%
	/pgfplots/search path for tikz/.unknown/.code={%
		\let\searchname=\pgfkeyscurrentname%
		\pgfkeysalso{%
			/tikz/\searchname/.try=#1,
			/pgfplots/\searchname/.lastretry=#1
		}%
	},%
	/pgfplots/.is family,
	/pgfplots/scale/.is family,
	/pgfplots/legend/.is family,
	/pgfplots/tick/.is family,
	/pgfplots/descriptions/.is family,
	/pgfplots/style commands/.is family,
	/pgfplots/naming commands/.is family,
	/tikz/every axis/.style={},
	/tikz/every axis/.belongs to family=/pgfplots,
	/tikz/every semilogx axis/.style={},
	/tikz/every semilogx axis/.belongs to family=/pgfplots,
	/tikz/every semilogy axis/.style={},
	/tikz/every semilogy axis/.belongs to family=/pgfplots,
	/tikz/every loglog axis/.style={},
	/tikz/every loglog axis/.belongs to family=/pgfplots,
	/tikz/every linear axis/.style={},
	/tikz/every linear axis/.belongs to family=/pgfplots,
	/tikz/every axis plot/.style={},
	/tikz/every axis plot/.belongs to family=/pgfplots,
	/tikz/every axis label/.style={},
	/tikz/every axis label/.belongs to family=/pgfplots/descriptions,
	/tikz/every axis x label/.style={at={(0.5,0)},below,yshift=-15pt},
	/tikz/every axis x label/.belongs to family=/pgfplots/descriptions,
	/tikz/every axis y label/.style={at={(0,0.5)},xshift=-35pt,rotate=90},
	/tikz/every axis y label/.belongs to family=/pgfplots/descriptions,
	/tikz/every axis title/.style={at={(0.5,1)},above,yshift=6pt},
	/tikz/every axis title/.belongs to family=/pgfplots/descriptions,
	/tikz/every tick/.style={very thin,gray},
	/tikz/every tick/.belongs to family=/pgfplots/tick,
	/tikz/every minor tick/.style={},
	/tikz/every minor tick/.belongs to family=/pgfplots/tick,
	/tikz/every major tick/.style={},
	/tikz/every major tick/.belongs to family=/pgfplots/tick,
	/tikz/every axis grid/.style={style=help lines},
	/tikz/every axis grid/.belongs to family=/pgfplots/tick,
	/tikz/every minor grid/.style={},
	/tikz/every minor grid/.belongs to family=/pgfplots/tick,
	/tikz/every major grid/.style={},
	/tikz/every major grid/.belongs to family=/pgfplots/tick,
	/tikz/every tick label/.style={},
	/tikz/every tick label/.belongs to family=/pgfplots/tick,
	/tikz/every x tick label/.style={},
	/tikz/every x tick label/.belongs to family=/pgfplots/tick,
	/tikz/every extra x tick/.style={},
	/tikz/every extra x tick/.belongs to family=/pgfplots/tick,
	/pgfplots/extra x tick style/.belongs to family=/pgfplots/style commands,
	/pgfplots/extra x tick style/.code={%
		\pgfkeysalso{/tikz/every extra x tick/.append style={#1}},
	},
	/tikz/every x tick scale label/.style={at={(1,0)},yshift=-2em,left,inner sep=0pt},
	/tikz/every x tick scale label/.belongs to family=/pgfplots/tick,
	/tikz/every y tick label/.style={/tikz/auto y tick label/.try},
	/tikz/every y tick label/.belongs to family=/pgfplots/tick,
	/tikz/every extra y tick/.style={},
	/tikz/every extra y tick/.belongs to family=/pgfplots/tick,
	/pgfplots/extra y tick style/.belongs to family=/pgfplots/style commands,
	/pgfplots/extra y tick style/.code={%
		\pgfkeysalso{/tikz/every extra y tick/.append style={#1}},
	},
	/tikz/every y tick scale label/.style={at={(0,1)},above right,inner sep=0pt,yshift=0.3em},
	/tikz/every y tick scale label/.belongs to family=/pgfplots/tick,
	/tikz/every axis legend/.style={%
		cells={anchor=center},
		inner xsep=3pt,inner ysep=2pt,nodes={inner sep=2pt,text depth=0.15em},
		anchor=north east,%
		shape=rectangle,%
		fill=white,%
		draw=black,
		at={(0.98,0.98)},
	},
	/tikz/every axis legend/.belongs to family=/pgfplots/legend,
% tick options:
	/pgfplots/xticklabel/.store in=	\pgfplots@xticklabel,
	/pgfplots/xticklabel/.belongs to family=/pgfplots/tick,
	/pgfplots/xticklabel=,
	/pgfplots/xticklabels/.belongs to family=/pgfplots/tick,
	/pgfplots/xticklabels/.code={%
		\pgfplots@foreach@to@list{#1}\to\pgfplots@xticklabels
		\let\pgfplots@xticklabel=\pgfplots@user@ticklabel@list@x
	},
	/pgfplots/yticklabels/.belongs to family=/pgfplots/tick,
	/pgfplots/yticklabels/.code={%
		\pgfplots@foreach@to@list{#1}\to\pgfplots@yticklabels
		\let\pgfplots@yticklabel=\pgfplots@user@ticklabel@list@y
	},
	/pgfplots/extra x tick label/.store in=	\pgfplots@extra@xticklabel,
	/pgfplots/extra x tick label/.belongs to family=/pgfplots/tick,
	/pgfplots/extra x tick label=,
	/pgfplots/extra x tick labels/.belongs to family=/pgfplots/tick,
	/pgfplots/extra x tick labels/.code={%
		\pgfplots@foreach@to@list{#1}\to\pgfplots@extra@xticklabels
		\let\pgfplots@extra@xticklabel=\pgfplots@user@extra@ticklabel@list@x
	},
	/pgfplots/extra y tick labels/.code={%
		\pgfplots@foreach@to@list{#1}\to\pgfplots@extra@yticklabels
		\let\pgfplots@extra@yticklabel=\pgfplots@user@extra@ticklabel@list@y
	},
	/pgfplots/xtick/.store in=			\pgfplots@xtick,
	/pgfplots/xtick/.belongs to family=/pgfplots/tick,
	/pgfplots/xtick=,
	/pgfplots/extra x ticks/.store in=\pgfplots@extra@xtick,
	/pgfplots/extra x ticks/.belongs to family=/pgfplots/tick,
	/pgfplots/extra x ticks=,
	/pgfplots/xtickten/.store in=		\pgfplots@xtickten,
	/pgfplots/xtickten/.belongs to family=/pgfplots/tick,
	/pgfplots/xtickten=,
	/pgfplots/yticklabel/.store in=	\pgfplots@yticklabel,
	/pgfplots/yticklabel/.belongs to family=/pgfplots/tick,
	/pgfplots/yticklabel=,
	/pgfplots/extra y tick label/.store in=	\pgfplots@extra@yticklabel,
	/pgfplots/extra y tick label/.belongs to family=/pgfplots/tick,
	/pgfplots/extra y tick label=,
	/pgfplots/ytick/.store in=			\pgfplots@ytick,
	/pgfplots/ytick/.belongs to family=/pgfplots/tick,
	/pgfplots/ytick=,
	/pgfplots/extra y ticks/.store in=\pgfplots@extra@ytick,
	/pgfplots/extra y ticks/.belongs to family=/pgfplots/tick,
	/pgfplots/extra y ticks=,
	/pgfplots/ytickten/.store in=		\pgfplots@ytickten,
	/pgfplots/ytickten/.belongs to family=/pgfplots/tick,
	/pgfplots/ytickten=,
	/pgfplots/tick scale label code/.code={$\cdot 10^{#1}$},
	/pgfplots/tick scale label code/.belongs to family=/pgfplots/tick,
	/pgfplots/scaled ticks/.is if=pgfplots@scaled@ticks,
	/pgfplots/scaled ticks/.default=false,
	/pgfplots/scaled ticks/.belongs to family=/pgfplots/tick,
	/pgfplots/scaled ticks=true,
	/pgfplots/scale ticks above exponent/.store in=	\pgfplots@scale@ticks@above@exponent,
	/pgfplots/scale ticks above exponent/.belongs to family=/pgfplots/tick,
	/pgfplots/scale ticks above exponent=3,
	/pgfplots/scale ticks below exponent/.store in=	\pgfplots@scale@ticks@below@exponent,
	/pgfplots/scale ticks below exponent/.belongs to family=/pgfplots/tick,
	/pgfplots/scale ticks below exponent=-1,
	/pgfplots/subtickwidth/.store in=	\pgfplots@subtickwidth,
	/pgfplots/subtickwidth/.belongs to family=/pgfplots/tick,
	/pgfplots/subtickwidth=0.1cm,
	/pgfplots/tickwidth/.store in=		\pgfplots@tickwidth,
	/pgfplots/tickwidth/.belongs to family=/pgfplots/tick,
	/pgfplots/tickwidth=0.15cm,
	/pgfplots/minor tick length/.estore in=\pgfplots@subtickwidth,
	/pgfplots/minor tick length/.belongs to family=/pgfplots/tick,
	/pgfplots/major tick length/.estore in=\pgfplots@tickwidth,
	/pgfplots/major tick length/.belongs to family=/pgfplots/tick,
	/pgfplots/max space between ticks/.estore in=\axisdefaulttickwidth,
	/pgfplots/max space between ticks/.belongs to family=/pgfplots/tick,
	/pgfplots/max space between ticks=35,% the maximum space between adjacent ticks (in pt, but don't specify the unit 'pt')
	/pgfplots/try min ticks/.estore in=			\axisdefaulttryminticks,
	/pgfplots/try min ticks/.belongs to family=/pgfplots/tick,
	/pgfplots/try min ticks=4,
	/pgfplots/try min ticks log/.estore in=			\pgfplots@default@try@minticks@log,
	/pgfplots/try min ticks log/.belongs to family=/pgfplots/tick,
	/pgfplots/try min ticks log=3,
	/pgfplots/log plot exponent style/.style={/pgf/number format/fixed,/pgf/number format/precision=2},
	/pgfplots/log plot exponent style/.belongs to family=/pgfplots/tick,
	/pgfplots/log identify minor tick positions/.is if=pgfplots@identify@log@minor@tick@pos,
	/pgfplots/log identify minor tick positions/.belongs to family=/pgfplots/tick,
	/pgfplots/log identify minor tick positions=true,
	/pgfplots/log number format code/.code={{%
		\pgfmathlogtologten@{#1}%
		\ifpgfplots@identify@log@minor@tick@pos
			\expandafter\pgfplots@is@log@tick@a@minor@tick@pos\pgfmathresult\relax%
		\else
			\pgfmathcomparisonfalse
		\fi
		\ifpgfmathcomparison
			\pgfmathprintnumber[sci]{\pgfmathresult}%
		\else
			\pgfkeysalso{/pgfplots/log plot exponent style,/pgfplots/log base 10 number format code=\pgfmathresult}%
		\fi
	}},
	/pgfplots/log number format code/.belongs to family=/pgfplots/tick,
	/pgfplots/log base 10 number format code/.code={$10^{\pgfmathprintnumber{#1}}$},
	/pgfplots/log base 10 number format code/.belongs to family=/pgfplots/tick,
% sets \pgfplots@tickposnum to
% left=0
% right=1
% both=2
	/pgfplots/tickpos/.is choice,
	/pgfplots/tickpos/.belongs to family=/pgfplots/tick,
	/pgfplots/tickpos/left/.code	={\pgfplots@tickposnum=0},
	/pgfplots/tickpos/left/.belongs to family=/pgfplots/tick,
	/pgfplots/tickpos/right/.code	={\pgfplots@tickposnum=1},
	/pgfplots/tickpos/right/.belongs to family=/pgfplots/tick,
	/pgfplots/tickpos/both/.code	={\pgfplots@tickposnum=2},
	/pgfplots/tickpos/both/.belongs to family=/pgfplots/tick,
	/pgfplots/tickpos=both,
% 'axis' options:
	/pgfplots/anchor/.belongs to family=/pgfplots,
	/pgfplots/anchor/.store in=			\pgfplots@anchorname,
	/pgfplots/anchor=south west,
% tick options:
	/pgfplots/ticks/.is choice,
	/pgfplots/ticks/.belongs to family=/pgfplots/tick,
	/pgfplots/ticks/none/.belongs to family=/pgfplots/tick,
	/pgfplots/ticks/none/.code={%
		\pgfplots@xminorticksfalse
		\pgfplots@yminorticksfalse
		\pgfplots@xmajorticksfalse
		\pgfplots@ymajorticksfalse
	},
	/pgfplots/ticks/major/.belongs to family=/pgfplots/tick,
	/pgfplots/ticks/major/.code={%
		\pgfplots@xminorticksfalse
		\pgfplots@yminorticksfalse
		\pgfplots@xmajortickstrue
		\pgfplots@ymajortickstrue
	},
	/pgfplots/ticks/minor/.belongs to family=/pgfplots/tick,
	/pgfplots/ticks/minor/.code={%
		\pgfplots@xminortickstrue
		\pgfplots@yminortickstrue
		\pgfplots@xmajorticksfalse
		\pgfplots@ymajorticksfalse
	},
	/pgfplots/ticks/both/.belongs to family=/pgfplots/tick,
	/pgfplots/ticks/both/.code={%
		\pgfplots@xminortickstrue
		\pgfplots@yminortickstrue
		\pgfplots@xmajortickstrue
		\pgfplots@ymajortickstrue
	},
	/pgfplots/ticks=both,
	/pgfplots/grid/.is choice,
	/pgfplots/grid/.belongs to family=/pgfplots/tick,
	/pgfplots/grid/none/.belongs to family=/pgfplots/tick,
	/pgfplots/grid/none/.code={%
		\pgfplots@xminorgridsfalse
		\pgfplots@yminorgridsfalse
		\pgfplots@xmajorgridsfalse
		\pgfplots@ymajorgridsfalse
	},
	/pgfplots/grid/major/.belongs to family=/pgfplots/tick,
	/pgfplots/grid/major/.code={%
		\pgfplots@xminorgridsfalse
		\pgfplots@yminorgridsfalse
		\pgfplots@xmajorgridstrue
		\pgfplots@ymajorgridstrue
	},
	/pgfplots/grid/minor/.belongs to family=/pgfplots/tick,
	/pgfplots/grid/minor/.code={%
		\pgfplots@xminorgridstrue
		\pgfplots@yminorgridstrue
		\pgfplots@xmajorgridsfalse
		\pgfplots@ymajorgridsfalse
	},
	/pgfplots/grid/both/.belongs to family=/pgfplots/tick,
	/pgfplots/grid/both/.code={%
		\pgfplots@xminorgridstrue
		\pgfplots@yminorgridstrue
		\pgfplots@xmajorgridstrue
		\pgfplots@ymajorgridstrue
	},
	/pgfplots/grid=none,
	/pgfplots/grid/.default=major,
	/pgfplots/xminorticks/.is if=pgfplots@xminorticks,
	/pgfplots/xminorticks/.default=true,
	/pgfplots/xminorticks/.belongs to family=/pgfplots/tick,
	/pgfplots/xmajorticks/.is if=pgfplots@xmajorticks,
	/pgfplots/xmajorticks/.default=true,
	/pgfplots/xmajorticks/.belongs to family=/pgfplots/tick,
	/pgfplots/yminorticks/.is if=pgfplots@yminorticks,
	/pgfplots/yminorticks/.default=true,
	/pgfplots/yminorticks/.belongs to family=/pgfplots/tick,
	/pgfplots/ymajorticks/.is if=pgfplots@ymajorticks,
	/pgfplots/ymajorticks/.default=true,
	/pgfplots/ymajorticks/.belongs to family=/pgfplots/tick,
	/pgfplots/xminorgrids/.is if=pgfplots@xminorgrids,
	/pgfplots/xminorgrids/.default=true,
	/pgfplots/xminorgrids/.belongs to family=/pgfplots/tick,
	/pgfplots/xmajorgrids/.is if=pgfplots@xmajorgrids,
	/pgfplots/xmajorgrids/.default=true,
	/pgfplots/xmajorgrids/.belongs to family=/pgfplots/tick,
	/pgfplots/yminorgrids/.is if=pgfplots@yminorgrids,
	/pgfplots/yminorgrids/.default=true,
	/pgfplots/yminorgrids/.belongs to family=/pgfplots/tick,
	/pgfplots/ymajorgrids/.is if=pgfplots@ymajorgrids,
	/pgfplots/ymajorgrids/.default=true,
	/pgfplots/ymajorgrids/.belongs to family=/pgfplots/tick,
% legend options:
	/pgfplots/legend columns/.store in=\pgfplots@legend@columns,
	/pgfplots/legend columns/.belongs to family=/pgfplots/legend,
	/pgfplots/legend columns=1,
	/pgfplots/legend plot pos/.is choice,
	/pgfplots/legend plot pos/.belongs to family=/pgfplots/legend,
	/pgfplots/legend plot pos/left/.code=	{\def\pgfplots@legend@plot@pos{0}},
	/pgfplots/legend plot pos/left/.belongs to family=/pgfplots/legend,
	/pgfplots/legend plot pos/right/.code=	{\def\pgfplots@legend@plot@pos{1}},
	/pgfplots/legend plot pos/right/.belongs to family=/pgfplots/legend,
	/pgfplots/legend plot pos/none/.code=	{\def\pgfplots@legend@plot@pos{2}},
	/pgfplots/legend plot pos/none/.belongs to family=/pgfplots/legend,
	/pgfplots/legend plot pos=left,
% axis description options:
	/pgfplots/title/.store in=		\pgfplots@title,
	/pgfplots/title/.belongs to family=/pgfplots/descriptions,
	/pgfplots/title=,
	/pgfplots/xlabel/.store in=	\pgfplots@xlabel,
	/pgfplots/xlabel/.belongs to family=/pgfplots/descriptions,
	/pgfplots/xlabel=,
	/pgfplots/ylabel/.store in=	\pgfplots@ylabel,
	/pgfplots/ylabel/.belongs to family=/pgfplots/descriptions,
	/pgfplots/ylabel=,
% axis options:
	/pgfplots/at/.code={\tikz@scan@one@point\pgfplots@set@at#1},
	/pgfplots/at/.belongs to family=/pgfplots,
	/pgfplots/clip limits/.is if=pgfplots@clip@limits,
	/pgfplots/clip limits/.default=true,
	/pgfplots/clip limits=true,
	/pgfplots/clip limits/.belongs to family=/pgfplots,
	/pgfplots/xmin/.store in=		\pgfplots@xmin,
	/pgfplots/xmin/.belongs to family=/pgfplots,
	/pgfplots/xmin=,
	/pgfplots/xmax/.store in=		\pgfplots@xmax,
	/pgfplots/xmax/.belongs to family=/pgfplots,
	/pgfplots/xmax=,
	/pgfplots/ymin/.store in=		\pgfplots@ymin,
	/pgfplots/ymin/.belongs to family=/pgfplots,
	/pgfplots/ymin=,
	/pgfplots/ymax/.store in=		\pgfplots@ymax,
	/pgfplots/ymax/.belongs to family=/pgfplots,
	/pgfplots/ymax=,
	/pgfplots/xfilter/.store in=	\pgfplots@xfilter,
	/pgfplots/xfilter/.belongs to family=/pgfplots,
	/pgfplots/xfilter=,
	/pgfplots/yfilter/.store in=	\pgfplots@yfilter,
	/pgfplots/yfilter/.belongs to family=/pgfplots,
	/pgfplots/yfilter=,
	/pgfplots/width/.store in=		\pgfplots@width,
	/pgfplots/width/.belongs to family=/pgfplots,
	/pgfplots/width=,
	/pgfplots/height/.store in=	\pgfplots@height,
	/pgfplots/height/.belongs to family=/pgfplots,
	/pgfplots/height=,
	/pgfplots/execute at begin plot/.store in=\pgfplots@execute@at@begin@plot,
	/pgfplots/execute at begin plot/.belongs to family=/pgfplots,
	/pgfplots/execute at begin plot=,
	/pgfplots/execute at end plot/.store in=		\pgfplots@execute@at@end@plot,
	/pgfplots/execute at end plot/.belongs to family=/pgfplots,
	/pgfplots/execute at end plot=,
	/pgfplots/enlargelimits/.store in=		\pgfplots@enlargelimits,
	/pgfplots/enlargelimits/.belongs to family=/pgfplots,
	/pgfplots/enlargelimits=auto,
	/pgfplots/enlargelimits/.default=true,
	/pgfplots/x/.store in=		\pgfplots@x,
	/pgfplots/x/.belongs to family=/pgfplots,
	/pgfplots/x=,% is implicitly set by 'width' and/or '\axisdefaultwidth'
	/pgfplots/y/.store in=		\pgfplots@y,
	/pgfplots/y/.belongs to family=/pgfplots,
	/pgfplots/y=,% is implicitly set by 'width' and/or '\axisdefaultwidth'
	/pgfplots/cycle list/.code={\pgfplots@assign@list\autoplotspeclist{#1}},
	/pgfplots/cycle list/.belongs to family=/pgfplots,
	/pgfplots/cycle list name/.code={\let\autoplotspeclist=#1\relax},
	/pgfplots/cycle list name/.belongs to family=/pgfplots,
	/pgfplots/cycle list name=\coloredplotspeclist,
	/pgfplots/legend style/.belongs to family=/pgfplots/style commands,
	/pgfplots/legend style/.code={%
		\pgfkeysalso{/tikz/every axis legend/.append style={#1}},
	},
	/pgfplots/label style/.belongs to family=/pgfplots/style commands,
	/pgfplots/label style/.code={%
		\pgfkeysalso{/tikz/every axis label/.append style={#1}}%
	}%
	/pgfplots/x label style/.belongs to family=/pgfplots/style commands,
	/pgfplots/x label style/.code={%
		\pgfkeysalso{/tikz/every axis x label/.append style={#1}}%
	},
	/pgfplots/y label style/.belongs to family=/pgfplots/style commands,
	/pgfplots/y label style/.code={%
		\pgfkeysalso{/tikz/every axis y label/.append style={#1}}%
	},
	/pgfplots/title style/.belongs to family=/pgfplots/style commands,
	/pgfplots/title style/.code={%
		\pgfkeysalso{/tikz/every axis title/.append style={#1}}%
	},
	/pgfplots/tick style/.belongs to family=/pgfplots/style commands,
	/pgfplots/tick style/.code={%
		\pgfkeysalso{/tikz/every tick/.append style={#1}}%
	},
	/pgfplots/tick label style/.belongs to family=/pgfplots/style commands,
	/pgfplots/tick label style/.code={%
		\pgfkeysalso{/tikz/every tick label/.append style={#1}}%
	},
	/pgfplots/x tick label style/.belongs to family=/pgfplots/style commands,
	/pgfplots/x tick label style/.code={%
		\pgfkeysalso{/tikz/every x tick label/.append style={#1}}%
	},
	/pgfplots/y tick label style/.belongs to family=/pgfplots/style commands,
	/pgfplots/y tick label style/.code={%
		\pgfkeysalso{/tikz/every y tick label/.append style={#1}}%
	},
	/pgfplots/x tick scale label style/.belongs to family=/pgfplots/style commands,
	/pgfplots/x tick scale label style/.code={%
		\pgfkeysalso{/tikz/every x scale tick label/.append style={#1}}%
	},
	/pgfplots/y tick scale label style/.belongs to family=/pgfplots/style commands,
	/pgfplots/y tick scale label style/.code={%
		\pgfkeysalso{/tikz/every y scale tick label/.append style={#1}}%
	},
	/pgfplots/minor tick style/.belongs to family=/pgfplots/style commands,
	/pgfplots/minor tick style/.code={%
		\pgfkeysalso{/tikz/every minor tick/.append style={#1}}%
	},
	/pgfplots/major tick style/.belongs to family=/pgfplots/style commands,
	/pgfplots/major tick style/.code={%
		\pgfkeysalso{/tikz/every major tick/.append style={#1}}%
	},
	/pgfplots/grid style/.belongs to family=/pgfplots/style commands,
	/pgfplots/grid style/.code={%
		\pgfkeysalso{/tikz/every axis grid/.append style={#1}}%
	},
	/pgfplots/minor grid style/.belongs to family=/pgfplots/style commands,
	/pgfplots/minor grid style/.code={%
		\pgfkeysalso{/tikz/every minor grid/.append style={#1}}%
	},
	/pgfplots/major grid style/.belongs to family=/pgfplots/style commands,
	/pgfplots/major grid style/.code={%
		\pgfkeysalso{/tikz/every major grid/.append style={#1}}%
	},
	/pgfplots/disablelogfilter/.is if=pgfplots@disablelogfilter,
	/pgfplots/disablelogfilter/.default=true,
	/pgfplots/disablelogfilter/.belongs to family=/pgfplots,
	/pgfplots/disablelogfilter=false,
	/pgfplots/disabledatascaling/.is if=pgfplots@disabledatascaling,
	/pgfplots/disabledatascaling/.default=true,
	/pgfplots/disabledatascaling/.belongs to family=/pgfplots,
	/pgfplots/disabledatascaling=false,
	/pgfplots/hide axis/.is if=pgfplots@hide,
	/pgfplots/hide axis/.default=true,
	/pgfplots/hide axis/.belongs to family=/pgfplots,
	/pgfplots/hide axis=false,
	/pgfplots/scale only axis/.is if=pgfplots@scale@only@axis,
	/pgfplots/scale only axis/.default=true,
	/pgfplots/scale only axis/.belongs to family=/pgfplots,
	/pgfplots/scale only axis=false,
% sets \pgfplots@xmode to
% normal=0
% log=1
	/pgfplots/xmode/.is choice,
	/pgfplots/xmode/.belongs to family=/pgfplots/scale,
	/pgfplots/xmode/normal/.code={\pgfplots@xmode=0},
	/pgfplots/xmode/normal/.belongs to family=/pgfplots/scale,
	/pgfplots/xmode/linear/.code={\pgfplots@xmode=0},
	/pgfplots/xmode/linear/.belongs to family=/pgfplots/scale,
	/pgfplots/xmode/log/.code={\pgfplots@xmode=1},
	/pgfplots/xmode/log/.belongs to family=/pgfplots/scale,
	/pgfplots/xmode=linear,
	/pgfplots/ymode/.is choice,
	/pgfplots/ymode/.belongs to family=/pgfplots/scale,
	/pgfplots/ymode/normal/.code={\pgfplots@ymode=0},
	/pgfplots/ymode/normal/.belongs to family=/pgfplots/scale,
	/pgfplots/ymode/linear/.code={\pgfplots@ymode=0},
	/pgfplots/ymode/linear/.belongs to family=/pgfplots/scale,
	/pgfplots/ymode/log/.code={\pgfplots@ymode=1},
	/pgfplots/ymode/log/.belongs to family=/pgfplots/scale,
	/pgfplots/ymode=linear,
	/pgfplots/numeric table/x index/.store in=\pgfplots@plot@tbl@xindex,
	/pgfplots/numeric table/x index=0,
	/pgfplots/numeric table/x/.store in=\pgfplots@plot@tbl@x,
	/pgfplots/numeric table/x=,
	/pgfplots/numeric table/y index/.store in=\pgfplots@plot@tbl@yindex,
	/pgfplots/numeric table/y index=0,
	/pgfplots/numeric table/y/.store in=\pgfplots@plot@tbl@y,
	/pgfplots/numeric table/y=,
}

% Checks whether the tick position given as #1.#2=log10(T) belongs to
% T=i*10^j with an integer i>1.
%
% If T=i*10^j,  \ifpgfmathcomparison will be set to true and
% \pgfmathresult will contain T.
%
% Otherwise, \ifpgfmathcomparison will be set to false and
% pgfmathresult to #1.#2
%
% Arguments:
% #1.#2  the value log10(T)
%
% Implementation:  
% if T = i*10^j,  log10(T) = log10(i) + j.
% That means if log10(T) in \Z,  we have T = 10^j. If not, we need to
% check wether i is an integer. Please note that log10(i) < 1.
%
% Further note: log(T) < 0 <=>  j<0.
% In case j<0, we have 
%   #1.#2 = j + log(i) 
%         = - ( -j - log(i) ) 
%         = - ( -j - 1  + (1-log(i)) )
%         = #1 '.' #2 [ up to the '0.'
% that means #1 = j-1  and #2 = 1-log(i).
\def\pgfplots@is@log@tick@a@minor@tick@pos#1.#2\relax{%
	\pgfmathapproxequalto{#1.#2}{#1.0}%
	\ifpgfmathcomparison
		\pgfmathcomparisonfalse
		\def\pgfmathresult{#1.#2}%
	\else
		% I guess this won't happen too often. In fact, it's a very
		% special case.
		\begingroup
		\c@pgf@counta=#1\relax
		\ifnum\c@pgf@counta<0
			\advance\c@pgf@counta by-1
			\pgfmathsubtract@{1}{0.#2}%
			\expandafter\pgfplots@is@log@tick@a@minor@tick@pos@IDENTIFY@LOGi\pgfmathresult\relax
			\ifpgfmathcomparison
				\aftergroup\pgfmathcomparisontrue
				\edef\pgfmathresult{\pgfmathresult e\the\c@pgf@counta}%
			\else
			\aftergroup\pgfmathcomparisonfalse
				\def\pgfmathresult{#1.#2}%
			\fi
		\else
			\pgfplots@is@log@tick@a@minor@tick@pos@IDENTIFY@LOGi0.#2\relax
			\ifpgfmathcomparison
				\aftergroup\pgfmathcomparisontrue
				\edef\pgfmathresult{\pgfmathresult e\the\c@pgf@counta}%
			\else
			\aftergroup\pgfmathcomparisonfalse
				\def\pgfmathresult{#1.#2}%
			\fi
		\fi
		\pgfmath@smuggleone\pgfmathresult
		\endgroup
	\fi
}

% expects a positive number.
\def\pgfplots@is@log@tick@a@minor@tick@pos@IDENTIFY@LOGi0.#1\relax{%
	\pgfmathcomparisontrue
	\pgfmathapproxequalto{0.#1}{0.3010299956639}%
	\ifpgfmathcomparison
		\def\pgfmathresult{2}%
	\else
		\pgfmathapproxequalto{0.#1}{0.4771212547196}%
		\ifpgfmathcomparison
			\def\pgfmathresult{3}%
		\else
			\pgfmathapproxequalto{0.#1}{0.6020599913279}%
			\ifpgfmathcomparison
				\def\pgfmathresult{4}%
			\else
				\pgfmathapproxequalto{0.#1}{0.698970004}%
				\ifpgfmathcomparison
					\def\pgfmathresult{5}%
				\else
					\pgfmathapproxequalto{0.#1}{0.7781512503}%
					\ifpgfmathcomparison
						\def\pgfmathresult{6}%
					\else
						\pgfmathapproxequalto{0.#1}{0.8450980400}%
						\ifpgfmathcomparison
							\def\pgfmathresult{7}%
						\else
							\pgfmathapproxequalto{0.#1}{0.9030899869}%
							\ifpgfmathcomparison
								\def\pgfmathresult{8}%
							\else
								\pgfmathapproxequalto{0.#1}{0.954242509439}%
								\ifpgfmathcomparison
									\def\pgfmathresult{9}%
								\else
									\pgfmathcomparisonfalse
								\fi
							\fi
						\fi
					\fi
				\fi
			\fi
		\fi
	\fi
}


\def\pgfplots@set@at#1{\def\pgfplots@at{#1}}%



% converts a comma-separated list (PGF foreach)  to my internal list
% structure.
\long\def\pgfplots@foreach@to@list#1\to#2{%
	\pgfplotslistnewempty\pgfplots@TMP
	\begingroup
	\foreach \i in {#1} {%
		\globaldefs=1
		\expandafter\pgfplotslistpushback\i\to\pgfplots@TMP
		\globaldefs=0
	}%
	\endgroup
	\pgfplotslistcopy\pgfplots@TMP\to#2\relax
}

% Several tick options:
\long\def\axispreset#1{%
	\tikzstyle{every axis}+=[#1]%
}
\long\def\legendpreset#1{%
	\pgfplots@error{Sorry, legendpreset is now deprecated, along with the legend options text width and font. Legends are now TikZ-matrizes which provide better alignment and can be placed horizontally. See the manual for details.}%
}

% Checks whether we need to create a separate 'tick scale label',
% a node with ' * 10^3' on the side of the axis:
%
% PRECONDITION:
%    Axis limits for #1 are given. I need their values before any data
%    scale transformation has been applied.
%    If  
%    	\pgfplots@#1min@unscaled@as@float 
%    and
%    	\pgfplots@#1max@unscaled@as@float 
%    exist; I will use these macros.
%    Otherwise, I will use \pgfplots@#1min and \pgfplots@#1max;
%    assuming that no data scale transformation is active.
%    FIXME : does that need further attention?
\def\pgfplots@init@scaled@tick@for#1{%
	\begingroup
	% the \pgfplots@xmin@unscaled@as@float  is set just before the data
	% scale transformation is initialised.
	%
	% The variables are empty if there is no datascale transformation.
	\expandafter\let\expandafter\pgfplots@cur@min@unscaled\csname pgfplots@#1min@unscaled@as@float\endcsname
	\expandafter\let\expandafter\pgfplots@cur@max@unscaled\csname pgfplots@#1max@unscaled@as@float\endcsname
	%
	\ifx\pgfplots@cur@min@unscaled\pgfutil@empty
		\xdef\pgfplots@TMP{\csname pgfplots@#1min\endcsname}%
		\expandafter\pgfmathfloatparsenumber\expandafter{\pgfplots@TMP}%
		\let\pgfplots@cur@min@unscaled=\pgfmathresult
		\xdef\pgfplots@TMP{\csname pgfplots@#1max\endcsname}%
		\expandafter\pgfmathfloatparsenumber\expandafter{\pgfplots@TMP}%
		\let\pgfplots@cur@max@unscaled=\pgfmathresult
	\fi
	%
	\expandafter\pgfmathfloat@decompose@E\pgfplots@cur@min@unscaled\relax\pgfmathfloat@a@E
	\expandafter\pgfmathfloat@decompose@E\pgfplots@cur@max@unscaled\relax\pgfmathfloat@b@E
	\ifnum\pgfmathfloat@b@E<\pgfmathfloat@a@E
		\pgfmathfloat@b@E=\pgfmathfloat@a@E
	\fi
	\xdef\pgfplots@TMP{\pgfplots@scale@ticks@above@exponent}%
	\expandafter\ifnum\pgfplots@TMP<\pgfmathfloat@b@E
		% ok, scale it:
		\multiply\pgfmathfloat@b@E by-1
		\xdef\pgfplots@TMP{\the\pgfmathfloat@b@E}%
	\else
		\xdef\pgfplots@TMP{\pgfplots@scale@ticks@below@exponent}%
		\expandafter\ifnum\pgfplots@TMP>\pgfmathfloat@b@E
			% ok, scale it:
			\multiply\pgfmathfloat@b@E by-1
			\xdef\pgfplots@TMP{\the\pgfmathfloat@b@E}%
		\else
			% no scaling necessary:
			\xdef\pgfplots@TMP{0}%
		\fi
	\fi
	\endgroup
	\expandafter\let\csname pgfplots@tick@scale@#1\endcsname=\pgfplots@TMP%
}

\def\pgfplots@show@yticklabel#1#2{{%
	\let\ticknum=#1%
	\ifpgfplots@float@numerics@mode@y
		\pgfplots@inverse@datascaletrafo@y{#2}%
		% FIXME: do that for any linear axis, also if no datascale is
		% applied!
		\ifpgfplots@scaled@ticks
			\expandafter\pgfmathfloatshift@\expandafter{\pgfmathresult}{\pgfplots@tick@scale@y}%
		\fi
		% .. and this here provides \tick as fixed point repr:
		\expandafter\pgfmathfloattofixed\expandafter{\pgfmathresult}%
		\let\tick=\pgfmathresult
	\else
		\def\tick{#2}%
	\fi
	\pgfplots@yticklabel
}}

% x-Achsenbeschriftung zum #1.ten Tick:
% #1: die ticknum
% #2: der x wert dafuer
\def\pgfplots@show@xticklabel#1#2{{%
	\let\ticknum=#1%
	\ifpgfplots@float@numerics@mode@x
		\pgfplots@inverse@datascaletrafo@x{#2}%
		% FIXME: do that for any linear axis, also if no datascale is
		% applied!
		\ifpgfplots@scaled@ticks
			\expandafter\pgfmathfloatshift@\expandafter{\pgfmathresult}{\pgfplots@tick@scale@x}%
		\fi
		% .. and this here provides \tick as fixed point repr:
		\expandafter\pgfmathfloattofixed\expandafter{\pgfmathresult}%
		\let\tick=\pgfmathresult
	\else
		\def\tick{#2}%
	\fi
	\pgfplots@xticklabel
}}

\def\pgfplots@user@ticklabel@list@x{%
	\pgfplotslistselect\ticknum\of\pgfplots@xticklabels\to\tick
	\tick
}
\def\pgfplots@user@ticklabel@list@y{%
	\pgfplotslistselect\ticknum\of\pgfplots@yticklabels\to\tick
	\tick
}
\def\pgfplots@user@extra@ticklabel@list@x{%
	\pgfplotslistselect\ticknum\of\pgfplots@extra@xticklabels\to\tick
	\tick
}
\def\pgfplots@user@extra@ticklabel@list@y{%
	\pgfplotslistselect\ticknum\of\pgfplots@extra@yticklabels\to\tick
	\tick
}

\def\axisdefaultticklabel{%
	$\pgfmathprintnumber{\tick}$%
}

\def\axisdefaultticklabellog{%
	\pgfkeysgetvalue{/pgfplots/log number format code/.@cmd}\pgfplots@log@label@style
	\expandafter\pgfplots@log@label@style\tick\pgfeov
}
	

\def\pgfplots@show@label#1{%
	\node 
		[style=every axis,%
		style=every axis label,%
		style=every axis #1 label]
	{\csname pgfplots@#1label\endcsname};
}

\def\pgfplots@show@title{%
	\node%
		[style=every axis,%
		style=every axis title]
		{\pgfplots@title};
}

\def\pgfplots@drawyaxis@placecomputedtick#1[#2]{%
	\draw[#2]
	\ifcase\pgfplots@tickposnum
		(\pgfplots@xcoordminTEX\pgfplots@leftover,		\pgfplots@tmpa)
	--	(\pgfplots@xcoordminTEX+#1,	\pgfplots@tmpa)
	\or
		(\pgfplots@xcoordmaxTEX\pgfplots@rightover,		\pgfplots@tmpa)
	--	(\pgfplots@xcoordmaxTEX-#1,	\pgfplots@tmpa)
	\or
		(\pgfplots@xcoordminTEX\pgfplots@leftover,		\pgfplots@tmpa)
	--	(\pgfplots@xcoordminTEX+#1,	\pgfplots@tmpa)

		(\pgfplots@xcoordmaxTEX\pgfplots@rightover,		\pgfplots@tmpa)
	--	(\pgfplots@xcoordmaxTEX-#1,	\pgfplots@tmpa)
	\fi
	;
}%
\def\pgfplots@drawyaxis@placecomputedgridline[#1]{%
	\draw[#1]
		(\pgfplots@xcoordminTEX\pgfplots@leftover,	\pgfplots@tmpa)
	--	(\pgfplots@xcoordmaxTEX\pgfplots@rightover,	\pgfplots@tmpa)
	;
}%
% Y-Achsen und Achsenbeschriftung LOG:
% 1: Ticks
\def\pgfplots@drawyaxis#1{%
	\ifnum\pgfplots@ymode=0
		\ifpgfplots@scaled@ticks
			\pgfplots@init@scaled@tick@for y%
		\fi
	\fi
	\begingroup
	\newif\ifneedsminorloop
	\axisticknum=0
	\edef\pgfplots@TMP{#1}%
	\ifx\pgfplots@TMP\pgfutil@empty
	\else
		\ifpgfplots@yminorticks
			\needsminorlooptrue
		\fi
		\ifpgfplots@yminorgrids
			\needsminorlooptrue
		\fi
		\foreach \y in {#1} {	
			\pgfextracty{\pgfplots@tmpa}{\pgfpointxy{0}{\y}}%
			%
			\ifdim\pgfplots@tmpa<\pgfplots@ycoordminTEX
			\else
				\ifdim\pgfplots@tmpa>\pgfplots@ycoordmaxTEX
				\else
					% Achsenbeschriftung:
					\ifnum\pgfplots@tickposnum=1
						\node at 
							(\pgfplots@xcoordmaxTEX\pgfplots@leftover,\pgfplots@tmpa)
							[right,style=every tick label,style=every y tick label]
							{\pgfplots@show@yticklabel\axisticknum\y};
					\else
						\node at 
							(\pgfplots@xcoordminTEX\pgfplots@leftover,\pgfplots@tmpa)
							[left,style=every tick label,style=every y tick label]
							{\pgfplots@show@yticklabel\axisticknum\y};
					\fi
				\fi
			\fi
			\global\advance\axisticknum by1
			\scope
			\clip 			(\pgfplots@xcoordminTEX\pgfplots@leftover,	\pgfplots@ycoordminTEX\pgfplots@leftover) 
				rectangle	(\pgfplots@xcoordmaxTEX\pgfplots@rightover,	\pgfplots@ycoordmaxTEX\pgfplots@rightover);
			% Y-ticks left and/or right
			% in log:
			%  log( i*10^k ) = log\i + k\log10 -> draw ticks for i=1..9
			\ifneedsminorloop
				\foreach \i in {1,...,9} {
					% this code here is scoped inside of \foreach.
					\pgfextracty{\pgfplots@tmpb}{\pgfpointxy{0}{\logi\i}}%
					\advance\pgfplots@tmpa by\pgfplots@tmpb
					\ifpgfplots@yminorgrids
						\pgfplots@drawyaxis@placecomputedgridline[style=every axis grid,style=every minor grid]%
					\fi
					\ifpgfplots@yminorticks
						\pgfplots@drawyaxis@placecomputedtick\pgfplots@subtickwidth[style=every tick,style=every minor tick]%
					\fi
				}%
			\fi
			\ifpgfplots@ymajorgrids
				\pgfplots@drawyaxis@placecomputedgridline[style=every axis grid,style=every major grid]%
			\fi
			\ifpgfplots@ymajorticks
				\pgfplots@drawyaxis@placecomputedtick\pgfplots@tickwidth[style=every tick,style=every major tick]%
			\fi
			\endscope
		}%
	\fi
	\pgfplots@draw@tick@scale@label@for y%
	\endgroup
}

\def\pgfplots@draw@tick@scale@label@for#1{%
	\expandafter\ifnum\csname pgfplots@#1mode\endcsname=0
		\ifpgfplots@scaled@ticks
			\begingroup
				\xdef\pgfplots@TMP{\csname pgfplots@tick@scale@#1\endcsname}%
				\expandafter\c@pgf@counta\pgfplots@TMP\relax
				\multiply\c@pgf@counta by-1
				\ifnum\c@pgf@counta=0\relax
					\global\let\pgfplots@TMP=\pgfutil@empty
				\else
					\xdef\pgfplots@TMP{\the\c@pgf@counta}%
				\fi
			\endgroup
			\ifx\pgfplots@TMP\pgfutil@empty
			\else
				\edef\pgfplots@tick@scale@labels{\noexpand\pgfplots@invoke@pgfkeyscode{/pgfplots/tick scale label code/.@cmd}{\pgfplots@TMP}}%
				\node 
					[%
					xshift=\pgfplots@xcoordminTEX,%
					yshift=\pgfplots@ycoordminTEX,%
					x=\pgfplots@xcoordmaxTEX-\pgfplots@xcoordminTEX,%
					y=\pgfplots@ycoordmaxTEX-\pgfplots@ycoordminTEX,%
					style=every axis,%
					style=every tick label,%
					style=every #1 tick label,%
					style=every #1 tick scale label]
				{\pgfplots@tick@scale@labels};
			\fi
		\fi
	\fi
}

% Simply invokes the code of PGF key #1 with value #2, that means
% #1#2\pgfeov
\def\pgfplots@invoke@pgfkeyscode#1#2{%
	\pgfkeysgetvalue{#1}\pgfplots@invoke@pgfkeyscode@CODE
	\pgfplots@invoke@pgfkeyscode@CODE#2\pgfeov
}

\def\pgfplots@drawxaxis@placecomputedtick#1[#2]{%
	\draw[#2]
	\ifcase\pgfplots@tickposnum
		(\pgfplots@tmpa,	\pgfplots@ycoordminTEX\pgfplots@leftover)
	--	(\pgfplots@tmpa,	\pgfplots@ycoordminTEX+#1)
	\or
		(\pgfplots@tmpa,	\pgfplots@ycoordmaxTEX\pgfplots@rightover)
	--	(\pgfplots@tmpa,	\pgfplots@ycoordmaxTEX-#1)
	\or
		(\pgfplots@tmpa,	\pgfplots@ycoordminTEX\pgfplots@leftover)
	--	(\pgfplots@tmpa,	\pgfplots@ycoordminTEX+#1)

		(\pgfplots@tmpa,	\pgfplots@ycoordmaxTEX\pgfplots@rightover)
	--	(\pgfplots@tmpa,	\pgfplots@ycoordmaxTEX-#1)
	\fi
	;
}%
\def\pgfplots@drawxaxis@placecomputedgridline[#1]{%
	\draw[#1]
		(\pgfplots@tmpa,	\pgfplots@ycoordminTEX\pgfplots@leftover)
	--	(\pgfplots@tmpa,	\pgfplots@ycoordmaxTEX\pgfplots@rightover)
	;
}%

% #1: axis (x or y)
% #2: tick position list
\def\pgfplots@draw@extra@ticks@for#1#2{%
	\begingroup
	\edef\pgfplots@TMP{#2}%
	\ifx\pgfplots@TMP\pgfutil@empty
	\else
		\pgfplots@scaled@ticksfalse
		\csname pgfplots@#1minorticksfalse\endcsname
		\csname pgfplots@#1minorgridsfalse\endcsname
		\expandafter\let\expandafter\axis@TMP\csname pgfplots@extra@#1ticklabel\endcsname
		\expandafter\let\csname pgfplots@#1ticklabel\endcsname=\axis@TMP
		% FIXME: that search path - thing is far from perfect.
		% In fact, it doesn't even work for choice keys... with
		% unexpected results.
		\pgfqkeys{/pgfplots/search path for tikz}{/tikz/every extra #1 tick}%
		\csname pgfplots@draw#1axis\endcsname{#2}%
	\fi
	\endgroup
}

% X-Achsen und Achsenbeschriftung 
% 1: Ticks
\def\pgfplots@drawxaxis#1{%
	\ifnum\pgfplots@xmode=0
		\ifpgfplots@scaled@ticks
			\pgfplots@init@scaled@tick@for{x}%
		\fi
	\fi
	\begingroup
	\axisticknum=0
	\newif\ifneedsminorloop
	\edef\pgfplots@TMP{#1}%
	\ifx\pgfplots@TMP\pgfutil@empty
	\else
		\ifpgfplots@xminorticks
			\needsminorlooptrue
		\fi
		\ifpgfplots@xminorgrids
			\needsminorlooptrue
		\fi
		\foreach \x in {#1} {	
			\pgfextractx{\pgfplots@tmpa}{\pgfpointxy{\x}{0}}%
			%
			\ifdim\pgfplots@tmpa<\pgfplots@xcoordminTEX
			\else
				\ifdim\pgfplots@tmpa>\pgfplots@xcoordmaxTEX
				\else
					% Achsenbeschriftung:
					\ifnum\pgfplots@tickposnum=1
						\node at 
							(\pgfplots@tmpa,\pgfplots@ycoordmaxTEX\pgfplots@leftover)
							[above,style=every tick label,style=every x tick label]
							{\pgfplots@show@xticklabel\axisticknum\x};
					\else
						\node at 
							(\pgfplots@tmpa,\pgfplots@ycoordminTEX\pgfplots@leftover)
							[below,style=every tick label,style=every x tick label]
							{\pgfplots@show@xticklabel\axisticknum\x};
					\fi
				\fi
			\fi
			\global\advance\axisticknum by1
			%
			\scope
			\clip 			(\pgfplots@xcoordminTEX\pgfplots@leftover,	\pgfplots@ycoordminTEX\pgfplots@leftover) 
				rectangle	(\pgfplots@xcoordmaxTEX\pgfplots@rightover,	\pgfplots@ycoordmaxTEX\pgfplots@rightover);
			% X-Axis ticks bottom and top
			% in log:
			%  log( i*10^k ) = log\i + k\log10 -> draw ticks for i=1..9
			\ifneedsminorloop
				\foreach \i in {1,...,9} {
					\pgfextractx{\pgfplots@tmpb}{\pgfpointxy{\logi\i}{0}}%
					\advance\pgfplots@tmpa by\pgfplots@tmpb
					\ifpgfplots@xminorgrids
						\pgfplots@drawxaxis@placecomputedgridline[style=every axis grid,style=every minor grid]%
					\fi
					\ifpgfplots@xminorticks
						\pgfplots@drawxaxis@placecomputedtick\pgfplots@subtickwidth[style=every tick,style=every minor tick]%
					\fi
				}%
			\fi
			\ifpgfplots@xmajorgrids
				\pgfplots@drawxaxis@placecomputedgridline[every axis grid,style=every major grid]%
			\fi
			\ifpgfplots@xmajorticks
				\pgfplots@drawxaxis@placecomputedtick\pgfplots@tickwidth[every tick,style=every major tick]%
			\fi
			\endscope
		}%
	\fi
	\pgfplots@draw@tick@scale@label@for x%
	\endgroup
}

\def\pgfplots@rememberplotspec#1{{%
	\globaldefs=1
	\pgfplotslistpushback#1\to\pgfplots@plotspeclist
	\advance\pgfplots@numplots by1%
}}

\def\pgfplots@getautoplotspec into#1{%
	\pgfplotslistsize\autoplotspeclist\to\c@pgf@counta
	\ifnum\c@pgf@counta=0
		\let#1=\pgfutil@empty
	\else
		\c@pgf@countb=\pgfplots@numplots
		% offset modulo size:
		\loop
		\ifnum\c@pgf@countb<\c@pgf@counta
			\pgfplots@loop@CONTINUEfalse
		\else
			\pgfplots@loop@CONTINUEtrue
		\fi
		\ifpgfplots@loop@CONTINUE
			\advance\c@pgf@countb by-\c@pgf@counta
		\repeat
		\pgfplotslistselect\c@pgf@countb\of\autoplotspeclist\to#1
%\pgfplots@message{pgfplots.sty: using \string\autoplotspeclist\ specification no\#\the\c@pgf@countb (of \the\c@pgf@counta): #1}%
	\fi
}

% The main interface to draw a plot into an axis.
%
% Usage:
% \addplot 
% 	plot coordinates {
% 		(0,0)
% 		(1,1)
% 	};
% 
% or
%
% \addplot[color=blue,mark=*]
% 	plot coordinates {
% 		(0,0)
% 		(1,1)
% 	};
% 
% The first syntax will use the next plot specification in the list
% \autoplotspeclist
% and the first will use blue color and * markers. 
%
% The linespec. will be used in the legend.
\long\def\pgfplots@addplot{%
	\pgfutil@ifnextchar+{%
		\pgfplots@getautoplotspec into\nextplotspec
		\pgfplots@addplotimplAPPEND
	}{%
		\pgfutil@ifnextchar[{%
			\pgfplots@addplotimpl%
		}{%
			\pgfplots@getautoplotspec into\nextplotspec
			% the space after ']' is required here:
			% FIXME: 
			% - \addplot[]plot coordinates is NOT allowed!?
			\expandafter\pgfplots@addplotimpl\expandafter[\nextplotspec]
		}%
	}%
}

\long\def\pgfplots@addplotimplAPPEND+[{%
	\expandafter\pgfplots@addplotimpl\expandafter[\nextplotspec,
}

\long\def\pgfplots@addplotimpl[#1]{%
	\pgfplots@start@plot@with@options{#1}%
	\pgfutil@ifnextchar p{%
		\pgfplots@addplotimpl@plot{#1}%
	}{%
		\pgfplots@addplotimpl@plot{#1}plot
	}%
}

\def\pgfplots@addplotimpl@plot#1plot{%
	\pgfutil@ifnextchar[{%
		\pgfplots@addplotimpl@plot@withoptions{#1}%
	}{%
		\pgfplots@addplotimpl@plot@withoptions{#1}[]%
	}%
}

\def\pgfplots@addplotimpl@plot@withoptions#1[#2]{
	\pgfqkeys{/tikz}{#2}%
	\pgfutil@ifnextchar c{%
		\pgfplots@addplotimpl@coordinates{#1}plot 
	}{%
		\pgfutil@ifnextchar f{%
			\pgfplots@addplotimpl@f{#1}%
		}{%
			\pgfutil@ifnextchar t{%
				\pgfplots@addplotimpl@table{#1}%
			}{%
				\pgfutil@ifnextchar ({%
					\pgfplots@addplotimpl@expression{#1}%
				}{%
					\pgfplots@error{Sorry, the supplied plot command is unknown or unsupported by pgfplots!}%
				}%
			}%
		}%
	}%
}

% The plot expression thing does two things.
%
% 1. It disables any data scaling transformation,
% 2. It checks for 'domain' arguments inside of #1 (the options
% directly following the \addplot command).
\long\def\pgfplots@addplotimpl@expression#1#2;{%
	\pgfplots@EMERGENCY@FORCE@DATA@TRAFO@TO@IDENTITYtrue
	\pgfplots@addplotimpl@expression@limits{#1}%
	\axispath \draw[#1] plot #2;%
	\pgfplots@end@plot
}%

\def\pgfplots@addplotimpl@expression@limits#1{%
	\pgfqkeysfiltered
		{\pgfkeysequals{/tikz/domain}}%
		{\pgfkeysignore}%
		{/tikz}%
		{#1}%
	\expandafter\pgfplots@addplotimpl@expression@limits@domain\tikz@plot@domain\relax
}%

\def\pgfplots@addplotimpl@expression@limits@domain#1:#2\relax{%
	\ifpgfplots@autocompute@xlim
		\ifpgfplots@float@numerics@mode@x
			\pgfmathfloatparsenumber{#1}%
			\let\pgfplots@filter@tmp@x=\pgfmathresult
			\pgfmathfloatmin{\pgfplots@xmin}{\pgfplots@filter@tmp@x}%
			\let\pgfplots@xmin=\pgfmathresult
			\pgfmathfloatmax{\pgfplots@xmax}{\pgfplots@filter@tmp@x}%
			\let\pgfplots@xmax=\pgfmathresult
			%
			\pgfmathfloatparsenumber{#2}%
			\let\pgfplots@filter@tmp@x=\pgfmathresult
			\pgfmathfloatmin{\pgfplots@xmin}{\pgfplots@filter@tmp@x}%
			\let\pgfplots@xmin=\pgfmathresult
			\pgfmathfloatmax{\pgfplots@xmax}{\pgfplots@filter@tmp@x}%
			\let\pgfplots@xmax=\pgfmathresult
		\else
			\pgfmathmin{\pgfplots@xmin}{#1}%
			\let\pgfplots@xmin=\pgfmathresult
			\pgfmathmax{\pgfplots@xmax}{#1}%
			\let\pgfplots@xmax=\pgfmathresult
			\pgfmathmin{\pgfplots@xmin}{#2}%
			\let\pgfplots@xmin=\pgfmathresult
			\pgfmathmax{\pgfplots@xmax}{#2}%
			\let\pgfplots@xmax=\pgfmathresult
		\fi
	\fi
}

\def\pgfplots@addplotimpl@f#1f{%
	\pgfutil@ifnextchar i{\pgfplots@addplotimpl@file{#1}}{\pgfplots@addplotimpl@function{#1}}%
}%

\let\pgfplots@backupof@pgfplotxyfile=\pgfplotxyfile

\def\pgfplots@addplotimpl@function#1unction#2{%
  \def\tikz@plot@filename{\tikz@plot@prefix\tikz@plot@id}%  
  \iftikz@plot@raw@gnuplot%
    \def\tikz@plot@data{\pgfplotgnuplot[\tikz@plot@filename]{#2}}%
  \else%
    \iftikz@plot@parametric%   
      \def\tikz@plot@data{\pgfplotgnuplot[\tikz@plot@filename]{%
          set samples \tikz@plot@samples;
          set parametric;
          plot [t=\tikz@plot@domain] #2}}%
    \else%
      \def\tikz@plot@data{\pgfplotgnuplot[\tikz@plot@filename]{%
          set samples \tikz@plot@samples;
          plot [x=\tikz@plot@domain] #2}}%
    \fi%
  \fi%
  \def\pgfplotxyfile{\pgfplots@addplotimpl@gnuplotresult{#1}}%
  \tikz@plot@data
  \let\pgfplotxyfile=\pgfplots@backupof@pgfplotxyfile
}%

\def\pgfplots@addplotimpl@gnuplotresult#1#2{%
	\begingroup
	\openin1=#2
	\ifeof1
		\pgfplots@error{Sorry, the gnuplot-result file '#2' could not be found. Maybe you need to enable the shell-escape feature? For pdflatex, this is '>> pdflatex -shell-escape'. You can also invoke '>> gnuplot #2' manually.}%
		\aftergroup\pgfplots@loop@CONTINUEfalse
	\else
		\aftergroup\pgfplots@loop@CONTINUEtrue
	\fi
	\closein1
	\endgroup
	\ifpgfplots@loop@CONTINUE
		\pgfplots@addplotimpl@table{#1}table{#2};%
	\fi
}

\def\pgfplots@addplotimpl@file#1ile{%
	\pgfplots@addplotimpl@table{#1}table%
}%

\def\pgfplots@addplotimpl@table#1table{%
	\pgfutil@ifnextchar[{%
		\pgfplots@addplotimpl@table@getopts{#1}%
	}{%
		\pgfplots@addplotimpl@table@getopts{#1}[x index=0,y index=1]%
	}%
}%

\def\pgfplots@addplotimpl@table@f#1#2from{%
	\pgfplots@addplot@table@from@macrotrue
	\pgfplots@addplotimpl@table@START{#1}{#2}%
}

\def\pgfplots@addplotimpl@table@getopts#1[#2]{%
	\pgfutil@ifnextchar f{%
		\pgfplots@addplotimpl@table@f{#1}{#2}%
	}{%
		\pgfplots@addplot@table@from@macrofalse
		\pgfplots@addplotimpl@table@START{#1}{#2}%
	}%
}

\long\def\pgfplots@addplotimpl@table@START#1#2#3{%
	\begingroup
	\let\pgfplots@plot@tbl@x\pgfutil@empty
	\let\pgfplots@plot@tbl@y\pgfutil@empty
	\def\pgfplots@plot@tbl@xindex{0}%
	\def\pgfplots@plot@tbl@yindex{1}%
	\pgfqkeys{/pgfplots/numeric table}{#2}%
	\ifpgfplots@addplot@table@from@macro
		\pgfnumtablecopy#3\to\table
	\else
		\pgfnumtableread{#3} to \table
	\fi
	\ifx\pgfplots@plot@tbl@x\pgfutil@empty
		\pgfnumtablegetcolumnbyindex\pgfplots@plot@tbl@xindex\of\table\to\addplot@tbl@x
	\else
		\pgfnumtablegetcolumnbyname\pgfplots@plot@tbl@x\of\table\to\addplot@tbl@x
	\fi
	\ifx\pgfplots@plot@tbl@y\pgfutil@empty
		\pgfnumtablegetcolumnbyindex\pgfplots@plot@tbl@yindex\of\table\to\addplot@tbl@y
	\else
		\pgfnumtablegetcolumnbyname\pgfplots@plot@tbl@y\of\table\to\addplot@tbl@y
	\fi
	\pgfplots@toka={}%
	\loop
	\pgfplotslistcheckempty\addplot@tbl@x
	\ifpgfplotslistempty
		\pgfplots@loop@CONTINUEfalse
	\else
		\pgfplots@loop@CONTINUEtrue
	\fi
	\ifpgfplots@loop@CONTINUE
		\pgfplotslistpopfront\addplot@tbl@x\to\addplot@tbl@cur@x
		\pgfplotslistpopfront\addplot@tbl@y\to\addplot@tbl@cur@y
		\edef\pgfplots@TMP{\the\pgfplots@toka(\addplot@tbl@cur@x,\addplot@tbl@cur@y)}%
		\pgfplots@toka=\expandafter{\pgfplots@TMP}%
	\repeat
	\xdef\pgfplots@TMP{plot coordinates {\the\pgfplots@toka}}%
	\endgroup
	\def\pgfplots@TMPB{\pgfplots@addplotimpl@coordinates{#1}}%
	\expandafter\pgfplots@TMPB\pgfplots@TMP;
	\pgfplots@addplotimpl@gobblesemicolon
}

\def\pgfplots@addplotimpl@gobblesemicolon;{}

\def\pgfplots@start@plot@with@options#1{%
	\ifx\pgfplots@execute@at@begin@plot\pgfutil@empty
	\else
		\pgfplots@execute@at@begin@plot
	\fi
	\pgfplots@rememberplotspec{[#1]}%
}

\def\pgfplots@end@plot{%
	\ifx\pgfplots@execute@at@end@plot\pgfutil@empty
	\else
		\pgfplots@execute@at@end@plot
	\fi
}

\long\def\pgfplots@addplotimpl@coordinates#1#2;{%
%\tracingmacros=2\tracingcommands=2
	% #2 = plot coordinates {...}
	\ifpgfplots@filtercoords
		% FIXME : FREE THIS VARIABLE:
		\pgfplots@filtercoords@and@up@limits#2;\newplotcmdto\pgfplots@filteredcoordinateplotcmd
		\def\pgfplots@TMP{\pgfplots@path@impl{\draw[style=every axis plot,#1]}}%
		\expandafter\pgfplots@TMP\pgfplots@filteredcoordinateplotcmd%
	\else
		\ifpgfplots@draw@at@end
			\pgfplots@update@limits{#2;}%
		\fi
		\pgfplots@path@impl{\draw[style=every axis plot,#1]}#2;%
	\fi
	\pgfplots@end@plot
%\tracingmacros=0\tracingcommands=0
}%

\long\def\pgfplots@update@limits#1{%
	\pgfplots@foreach@plot@coord@do%
		\pgfplots@update@limits@for@one@point
	\for
	#1%
	\pgfplots@EOI
}

\newif\ifpgfplots@update@limits@for@one@point@ISCLIPPED
\def\pgf@math@ONE{1.0}%

% Updates the current x and y limits for point (#1,#2).
%
% The point coordinates may be given in floating point format, see
% below.
%
% Please note that if user specified limits are given, automatic
% limits are only applied to points which fall into the user specified
% clipping region.
%
% PRECONDITIONS:
%   - 'floating point numerics active' 
%   		=> coordinate filtering is active
%   		=> #1 and #2 are in floating point format
%   - no coordinate filters active 
%   		=> #1 and #2 MAY be macros, but they must be valid TeX
%   		length (without unit).
\long\def\pgfplots@update@limits@for@one@point#1#2{%
%\tracingmacros=2\tracingcommands=2
%\pgfplots@message{Updating limits for (#1,#2) ...}%
	\pgfplots@update@limits@for@one@point@ISCLIPPEDfalse
	\ifpgfplots@clip@limits
		\ifpgfplots@autocompute@xlim
		\else
			\ifpgfplots@float@numerics@mode@x
				\pgfmathfloatlessthan{#1}{\pgfplots@xmin}%
				\ifpgfmathfloatcomparison
					\pgfplots@update@limits@for@one@point@ISCLIPPEDtrue
				\fi
				\pgfmathfloatlessthan{\pgfplots@xmax}{#1}%
				\ifpgfmathfloatcomparison
					\pgfplots@update@limits@for@one@point@ISCLIPPEDtrue
				\fi
			\else
				\pgfmathlessthan{#1}{\pgfplots@xmin}%
				\ifx\pgfmathresult\pgf@math@ONE
					\pgfplots@update@limits@for@one@point@ISCLIPPEDtrue
				\fi
				\pgfmathlessthan{\pgfplots@xmax}{#1}%
				\ifx\pgfmathresult\pgf@math@ONE
					\pgfplots@update@limits@for@one@point@ISCLIPPEDtrue
				\fi
			\fi
		\fi
		\ifpgfplots@autocompute@ylim
		\else
			\ifpgfplots@float@numerics@mode@y
				\pgfmathfloatlessthan{#2}{\pgfplots@ymin}%
				\ifpgfmathfloatcomparison
					\pgfplots@update@limits@for@one@point@ISCLIPPEDtrue
				\fi
				\pgfmathfloatlessthan{\pgfplots@ymax}{#2}%
				\ifpgfmathfloatcomparison
					\pgfplots@update@limits@for@one@point@ISCLIPPEDtrue
				\fi
			\else
				\pgfmathlessthan{#2}{\pgfplots@ymin}%
				\ifx\pgfmathresult\pgf@math@ONE
					\pgfplots@update@limits@for@one@point@ISCLIPPEDtrue
				\fi
				\pgfmathlessthan{\pgfplots@ymax}{#2}%
				\ifx\pgfmathresult\pgf@math@ONE
					\pgfplots@update@limits@for@one@point@ISCLIPPEDtrue
				\fi
			\fi
		\fi
	\fi
	\ifpgfplots@update@limits@for@one@point@ISCLIPPED
	\else
		\ifpgfplots@autocompute@xlim
			\ifpgfplots@float@numerics@mode@x
				\pgfmathfloatmin{\pgfplots@xmin}{#1}%
				\let\pgfplots@xmin=\pgfmathresult
				\pgfmathfloatmax{\pgfplots@xmax}{#1}%
				\let\pgfplots@xmax=\pgfmathresult
			\else
				\pgfmathmin{\pgfplots@xmin}{#1}%
				\let\pgfplots@xmin=\pgfmathresult
				\pgfmathmax{\pgfplots@xmax}{#1}%
				\let\pgfplots@xmax=\pgfmathresult
			\fi
		\fi
		\ifpgfplots@autocompute@ylim
			\ifpgfplots@float@numerics@mode@y
				\pgfmathfloatmin{\pgfplots@ymin}{#2}%
				\let\pgfplots@ymin=\pgfmathresult
				\pgfmathfloatmax{\pgfplots@ymax}{#2}%
				\let\pgfplots@ymax=\pgfmathresult
			\else
				\pgfmathmin{\pgfplots@ymin}{#2}%
				\let\pgfplots@ymin=\pgfmathresult
				\pgfmathmax{\pgfplots@ymax}{#2}%
				\let\pgfplots@ymax=\pgfmathresult
			\fi
		\fi
		\pgfplots@limits@are@computedtrue
	\fi
%\pgfplots@message{Updated limits: (\pgfplots@xmin,\pgfplots@ymin) rectangle  (\pgfplots@xmax,\pgfplots@ymax).}%
%\tracingmacros=0\tracingcommands=0
}

% Filters coordinates and writes a new, modified 'plot coordinates'
% command to macro #2.
\long\def\pgfplots@filtercoords@and@up@limits#1\newplotcmdto#2{%
	\let#2=\pgfutil@empty
	\let\pgfplots@filter@output=#2
	\pgfplots@foreach@plot@coord@do%
		\pgfplots@filter@one@coord@pair
	\for
	#1%
	\pgfplots@EOI
	\edef#2{plot coordinates {\pgfplots@filter@output};}%
}%

\def\pgfplots@filter@one@coord@pair#1#2{%
	\ifx\pgfplots@xfilter\pgfutil@empty
		\def\pgfplots@filter@tmp@x{#1}%
	\else
		\let\pgfplots@filter@tmp@x=\pgfutil@empty
		\pgfplots@xfilter#1\to\pgfplots@filter@tmp@x
	\fi
	\ifx\pgfplots@yfilter\pgfutil@empty
		\def\pgfplots@filter@tmp@y{#2}%
	\else
		\let\pgfplots@filter@tmp@y=\pgfutil@empty
		\pgfplots@yfilter#2\to\pgfplots@filter@tmp@y
	\fi
	\ifx\pgfplots@filter@tmp@x\pgfutil@empty
		\pgfplots@message{NOTE: coordinate (#1,#2) has been dropped because of the x-coordinate filter.}%
	\else
		\ifx\pgfplots@filter@tmp@y\pgfutil@empty
			\pgfplots@message{NOTE: coordinate (#1,#2) has been dropped because of the y-coordinate filter.}%
		\else
			\ifpgfplots@draw@at@end
				% update also axis limits:
				\pgfplots@update@limits@for@one@point{\pgfplots@filter@tmp@x}{\pgfplots@filter@tmp@y}%
			\else
				% apply data transformation directly.
				\ifpgfplots@float@numerics@mode@x
					\pgfplots@datascaletrafo@x\pgfplots@filter@tmp@x
					\let\pgfplots@filter@tmp@x=\pgfmathresult
				\fi
				\ifpgfplots@float@numerics@mode@y
					\pgfplots@datascaletrafo@y\pgfplots@filter@tmp@y
					\let\pgfplots@filter@tmp@y=\pgfmathresult
				\fi
			\fi
			\pgfplots@toka=\expandafter{\pgfplots@filter@output}%
			\edef\pgfplots@filter@output{\the\pgfplots@toka (\pgfplots@filter@tmp@x,\pgfplots@filter@tmp@y)}%
%\pgfplots@message{FILTERING: appended \pgfplots@filter@tmp@x,\pgfplots@filter@tmp@y}%  -> results in \pgfplots@filter@output}%
		\fi
	\fi
}


% A looping command to loop through plot coordinates.
% For every point, #3{X}{Y} will be invoked.
%
% No scoping is used during this operation, so you can access outer
% variables.
\long\def\pgfplots@foreach@plot@coord@do#1\for plot coordinates #2;\pgfplots@EOI{%
	\pgfplots@foreach@plot@coord@ITERATE#1\n#2\pgfplots@EOI%
}

\def\pgfplots@foreach@plot@coord@ITERATE#1\n{%
	\pgfutil@ifnextchar;{%
		\pgfplots@foreach@plot@coord@FINISH@SEMICOLON%
	}{%
		\pgfutil@ifnextchar\pgfplots@EOI{%
			\pgfplots@foreach@plot@coord@FINISH%
		}{%
			\pgfplots@foreach@plot@coord@NEXT#1\n%
		}%
	}%
}

\def\pgfplots@foreach@plot@coord@NEXT#1\n(#2,#3){%
	#1{#2}{#3}%
	\pgfplots@foreach@plot@coord@ITERATE#1\n
}

\def\pgfplots@foreach@plot@coord@FINISH\pgfplots@EOI{}
\def\pgfplots@foreach@plot@coord@FINISH@SEMICOLON;\pgfplots@EOI{}

% INPUT: 
%   a sequence of points (SMeE,SMeE) (terminated with ';') 
%   where S=sign, M=mantisse, E = exponent.
%   Such input is generated in the case pgfplots@float@numerics@modetrue
%   by the coordinate filters.
% OUTPUT: 
%   proper sequence of fixed point coordinates which are the result of the
%   data scaling transformation (\pgfplots@datascaletrafo@x).
%
% The output will be written into the macro #2.
\long\def\pgfplots@apply@data@scaletrafo@to@plot@data#1\to#2{%
	\let#2=\pgfutil@empty
	\let\pgfplots@data@scaletrafo@result=\pgfutil@empty
	\pgfplots@foreach@plot@coord@do
		\pgfplots@apply@data@scaletrafo@to@one@point
	\for
	#1%
	\pgfplots@EOI
	\edef#2{plot coordinates {\pgfplots@data@scaletrafo@result};}%
}

% for use in \ifx:
\def\pgfplots@EOI{\pgfplots@EOI}%
	
\def\pgfplots@apply@data@scaletrafo@to@one@point#1#2{%
	\ifpgfplots@float@numerics@mode@x
		\pgfplots@datascaletrafo@x{#1}%
		\let\pgfplots@filter@tmp@x=\pgfmathresult
	\else
		\def\pgfplots@filter@tmp@x{#1}%
	\fi
	\ifpgfplots@float@numerics@mode@y
		\pgfplots@datascaletrafo@y{#2}%
		\let\pgfplots@filter@tmp@y=\pgfmathresult
	\else
		\def\pgfplots@filter@tmp@y{#2}%
	\fi
	\edef\pgfplots@data@scaletrafo@result{\pgfplots@data@scaletrafo@result (\pgfplots@filter@tmp@x,\pgfplots@filter@tmp@y)}%
}


\long\def\pgfplots@path#1;{%
	\pgfplots@path@impl{}#1;
}

% This thing here shall be used to replace any '\path' where \axispath
% shall be used.
\long\def\pgfplots@replacement@for@tikz@path#1;{%
	\axispath\path#1;%
}

% low-level \axispath implementation.
% #1:
%    the command which is responsable for drawing.
%    - If #1 is NOT '\pgfutil@empty', we expect #2 to contain only
%      EXPANDABLE DATA.
%      That's important for postponed floating point arithmetics in #2.
%    - If #1='\pgfutil@empty', we don't make any assumption about #2
%      and process it as-is.
%
% #2: see above.
\long\def\pgfplots@path@impl#1#2;{%
	\ifpgfplots@draw@at@end
		\pgfplotslistpushback{#1}\to\pgfplots@stored@plot@cmds
		\pgfplotslistpushback{#2;}\to\pgfplots@stored@plot@data
	\else
		#1#2;%
	\fi
}


% Assigns a legend.
% Syntax:
% \legend{entry 1\\entry2\\entry3}
\def\pgfplots@command@legend{
	\pgfutil@ifnextchar[{%
		\pgfplots@error{Sorry, legend options are now deprecated. Legends are now TikZ-matrizes which provide better alignment and can be placed horizontally. See the manual for details.}%
	}{%
		\pgfplots@command@legend@impl
	}%
}

\def\pgfplots@command@legend@impl#1{%
	\pgfplots@assign@list\pgfplots@TMPC{#1}%
	\global\let\pgfplots@legend=\pgfplots@TMPC
}


\def\pgfplots@addlegendentry#1{%
	{%
	\globaldefs=1% this will ONLY store the options and the legend into global variables
	\pgfplotslistpushback#1\to\pgfplots@legend
	}%
}

\def\pgfplots@pop@next@legend{{%
	\globaldefs=1
	\pgfplotslistcheckempty\pgfplots@plotspeclist
	\ifpgfplotslistempty
		\let\pgfplots@curplotlist=\pgfutil@empty
	\else
		\pgfplotslistpopfront\pgfplots@plotspeclist\to\pgfplots@curplotlist
	\fi
	%
	\pgfplotslistcheckempty\pgfplots@legend
	\ifpgfplotslistempty
		\let\pgfplots@curlegend=\pgfutil@empty
	\else
		\pgfplotslistpopfront\pgfplots@legend\to\pgfplots@curlegend
	\fi
	\advance\pgfplots@numplots by-1
}}

\long\def\pgfplots@show@small@legendplots#1{
	\draw#1
			[mark repeat=2,mark phase=2] 
			plot coordinates {
				(0cm,0cm) 
				(0.3cm,0cm)
				(0.6cm,0cm)%
			};%
}

% Typesets a legend node.
%
% It will either typeset a previously computed legend (which needs to be
% stored in the macro \pgfplots@already@computed@legend@node)
%
% or it creates a legend, stores the commands into the macro named
% above and typesets it.
\def\pgfplots@createlegend{%
	\ifx\pgfplots@already@computed@legend@node\pgfutil@empty
		\pgfplotslistcheckempty\pgfplots@legend
		\ifpgfplotslistempty
		\else
		%
		% 
		\begingroup
			% assemble a 
			% \matrix {
			% 	small plot  & legend1\\
			% 	small plot  & legend2\\
			% 	...
			% };
			% command [and using the 'legend columns' option]
			%
			% \pgfplotslist@TOK@a={
			% 	small plot  & legend1\\
			% 	small plot  & legend2\\
			% 	...
			% }
			% ( I have allocated the token registers in my
			% liststructure.sty)
			% 
			% \global\def\pgfplots@TMP{
			% 	\matrix {
			% 		\TOKL@TA
			% 	};
			% }
			% -> finally, \pgfplots@TMP will contain the complete command.
			\pgfplotslist@TOK@a={}%
			\let\curcolumnNum=\c@pgf@counta
			\let\maxcolumnCount=\c@pgf@countb
			\let\legendplotpos=\c@pgf@countc
			\legendplotpos\expandafter=\pgfplots@legend@plot@pos
			\curcolumnNum=0
			\maxcolumnCount=\pgfplots@legend@columns\relax
			%
			\loop
			\ifnum0<\pgfplots@numplots\relax
				\pgfplots@pop@next@legend
				\ifx\pgfplots@curlegend\pgfutil@empty
				\else
					\advance\curcolumnNum by1
					\ifcase\legendplotpos
						% legend plot pos=left 
						\pgfplotslist@TOK@b=\expandafter{%
								\expandafter\pgfplots@show@small@legendplots
								\expandafter{\pgfplots@curplotlist}%
								\pgfmatrixnextcell\node}%
						\pgfplotslist@TOK@b=\expandafter\expandafter\expandafter{\expandafter\the\expandafter\pgfplotslist@TOK@b\expandafter{\pgfplots@curlegend};}%
					\or
						% legend plot pos=right
						\pgfplotslist@TOK@b=\expandafter{%
							\expandafter\node
							\expandafter{\pgfplots@curlegend};%
							\pgfmatrixnextcell
							\pgfplots@show@small@legendplots}%
						\pgfplotslist@TOK@b=\expandafter\expandafter\expandafter{\expandafter\the\expandafter\pgfplotslist@TOK@b\expandafter{\pgfplots@curplotlist}}%
					\or
						% legend plot pos=none
						\pgfplotslist@TOK@b=\expandafter{\expandafter\node\expandafter{\pgfplots@curlegend};}%
					\fi
					\ifnum\curcolumnNum=\maxcolumnCount
						\pgfplotslist@TOK@b=\expandafter{\the\pgfplotslist@TOK@b\\}%
					\else
						\ifnum\pgfplots@numplots=0
							\pgfplotslist@TOK@b=\expandafter{\the\pgfplotslist@TOK@b\\}%
						\else
							\pgfplotslist@TOK@b=\expandafter{\the\pgfplotslist@TOK@b\pgfmatrixnextcell}%
						\fi
					\fi
					\edef\pgfplots@TMP{%
						\the\pgfplotslist@TOK@a
						\the\pgfplotslist@TOK@b
					}%
					\ifnum\curcolumnNum=\maxcolumnCount
						\curcolumnNum=0
					\fi
					\expandafter\pgfplotslist@TOK@a\expandafter{\pgfplots@TMP}%
				\fi
			\repeat
			\pgfplotslist@TOK@b={\matrix[every axis,every axis legend]}%
			\xdef\pgfplots@TMP{%
				\the\pgfplotslist@TOK@b {%
					\the\pgfplotslist@TOK@a
				};%
			}%
		\endgroup
		\let\pgfplots@already@computed@legend@node=\pgfplots@TMP
		\fi
	\fi
% \pgfkeysgetvalue{/tikz/every axis legend/.@cmd}{\pgfplots@TMP}%
% \pgfplots@message{Vor legende: every axis legend= \meaning\pgfplots@TMP}%
% \pgfkeysgetvalue{/tikz/every axis/.@cmd}{\pgfplots@TMP}%
% \pgfplots@message{Vor legende: every axis= \meaning\pgfplots@TMP}%
	\pgfplots@already@computed@legend@node
}

% Checks whether the argument to xtick or ytick is a UNIFORM tick
% sequence.
%
% A uniform tick sequence is 0,...,10 and 3,4,5 and -5,-4,-2 but 
% NOT 0,2,4 or 4,10.
%
% Furthermore, any NON-integer tick arguments are also assumed to be
% NOT uniform.
%
% INPUT:
% #1: a tick argument (i.e. something which can be put to 
%     \foreach \x in {#1})
%
% OUTPUT:
%    \pgfplots@isuniformticktrue
%    or
%    \pgfplots@isuniformtickfalse
% depending on the check.
% This variable will be set globally.
\def\pgfplots@checkisuniformLOGtick#1{%
	\begingroup
	\global\pgfplots@isuniformticktrue
	\newif\ifisfirst
	\newif\iftwo@are@equal
	\isfirsttrue
	\foreach \x in {#1}{%
		\pgfmathmultiply@\x\reciproclogten
		\let\cur=\pgfmathresult
		% check whether
		%  \cur - last == 1 (last = \pgfplots@TMPB)
		\ifisfirst
			\global\isfirstfalse
		\else
			\pgfmathsubtract@\cur\pgfplots@TMPB%
			\pgfmathapproxequaltomacro\pgfmathresult{1.0}{two@are@equal}%
			\iftwo@are@equal
			\else
				\global\pgfplots@isuniformtickfalse
				\breakforeach
			\fi
		\fi
		\global\let\pgfplots@TMPB=\cur
	}%
	\endgroup
}

\def\skipsuffixzero#1.#2|{
	{%
	\def\pgfplots@TMP{#2}%
	\def\pgfplots@TMPB{0}%
	\ifx\pgfplots@TMP\pgfplots@TMPB
		\global\def\pgfmathresult{#1}%
	\else
		\global\def\pgfmathresult{#1.#2}%
	\fi
	}%
}

\def\pgfmathlogtologten#1{%
	\pgfmathparse{#1}%
	\expandafter\pgfmathlogtologten@\expandafter{\pgfmathresult}%
}

% Simply divides #1 by log(10).
\def\pgfmathlogtologten@#1{%
	\pgfmathmultiply@{#1}\reciproclogten%
}%

% DEPRECATED.
\def\logtologtentomacro#1#2{%
	\pgfmathmultiply@{#1}\reciproclogten%
	\expandafter\skipsuffixzero\pgfmathresult|%
	\let#2=\pgfmathresult
}

% DEPRECATED.
\def\logtologten#1{%
	\pgfmathmultiply@{#1}\reciproclogten%
	\expandafter\skipsuffixzero\pgfmathresult|%
	\pgfmathresult
}

% helper method which computes log10*\x foreach \x in {#1}.
% The result will be \xdef'ed into #2.
\def\pgfplots@compute@tick@times@logten#1\to#2{%
	\global\let#2=\pgfutil@empty
	\foreach \pgfplots@TMPB in {#1} {%
		\pgfmathmultiply@\pgfplots@TMPB\logten%
		\ifx#2\pgfutil@empty
			\xdef#2{\pgfmathresult}%
		\else
			\xdef#2{#2,\pgfmathresult}%
		\fi
	}%
}

\def\pgfplots@init@enlarge@limit@booleans{%
	\pgfplots@enlargelimits@autofalse
	\pgfplots@enlargelimitsfalse
	\pgfplots@enlargelimits@rel@threshfalse
	\edef\pgfplots@TMP{\pgfplots@enlargelimits}%
	%
	\def\pgfplots@TMPB{true}%
	\ifx\pgfplots@TMP\pgfplots@TMPB
		\pgfplots@enlargelimitstrue
	\else
		\def\pgfplots@TMPB{false}%
		\ifx\pgfplots@TMP\pgfplots@TMPB
		\else
			\def\pgfplots@TMPB{auto}%
			\ifx\pgfplots@TMP\pgfplots@TMPB
				\pgfplots@enlargelimits@autotrue
			\else
				\begingroup
				% try to read it as number:
				\expandafter\pgf@xa\pgfplots@enlargelimits pt\relax
				\endgroup
				\pgfplots@enlargelimitstrue
				\pgfplots@enlargelimits@rel@threshtrue
			\fi
		\fi
	\fi
}

\def\pgfplots@enlarge@limit@for#1{%
	\begingroup
	\expandafter\let\expandafter\min\expandafter=\csname pgfplots@#1min\endcsname
	\expandafter\let\expandafter\max\expandafter=\csname pgfplots@#1max\endcsname
	\ifcase\csname pgfplots@#1mode\endcsname
		\ifpgfplots@enlargelimits@rel@thresh
			\edef\enlargepercent{\pgfplots@enlargelimits}%
		\else
			\def\enlargepercent{0.1}% FIXME : pack 10% as default into option 'enlargelimits' or so
		\fi
		\pgfmathsubtract@\max\min%
		\expandafter\pgf@xa\pgfmathresult pt
		\expandafter\pgf@xb\enlargepercent\pgf@xa
		\ifdim\pgf@xb>0.001pt
			% the case with 
			%   enlargeabsolute ~= 0
			% means that \min ~= \max.
			% It is handled in another method.
			\edef\enlargeabsolute{\pgf@sys@tonumber{\pgf@xb}}%
			%
			% compute xmin := xmin - enlargeabsolute
			\pgfmathsubtract@\min\enlargeabsolute%
			\let\min=\pgfmathresult
			\pgfmathadd@\max\enlargeabsolute%
			\let\max=\pgfmathresult
		\fi
	\or
		% compute
		%   \pgfmathsetmacro{\min}{\min - 0.5*\logten}%
		%   \pgfmathsetmacro{\max}{\max + 0.5*\logten}%
		% just a more efficiently (I hope so):
		\expandafter\pgfmath@x\logten pt
		\divide\pgfmath@x by2
		\pgfmathlog@invoke@expanded\pgfmathsubtract@{{\min}{\pgf@sys@tonumber{\pgfmath@x}}}%
		\let\min=\pgfmathresult
		\pgfmathlog@invoke@expanded\pgfmathadd@{{\max}{\pgf@sys@tonumber{\pgfmath@x}}}%
		\let\max=\pgfmathresult
	\fi
	\xdef\pgfplots@TMP{\min}%
	\xdef\pgfplots@TMPB{\max}%
	\endgroup
	\expandafter\let\csname pgfplots@#1min\endcsname=\pgfplots@TMP
	\expandafter\let\csname pgfplots@#1max\endcsname=\pgfplots@TMPB
}

% Computes tick positions using the current axis limits.
%
% Parameters:
% /pgfplots/max space between ticks
%    Determines the maximum space which is not filled by at least one
%    tick label (approximate, there is some rounding internally)
% /pgfplots/try min ticks
%    see manual
%
% Idea:
% We want ticks at each 
%    { i*H, i in \Z }.
% Of course, there shouldn't be TOO MUCH ticks. 
%
% Our heuristics is to set
%    desirednumticks = round(ACTUAL WIDTH / (max space between ticks) )
% and generate H = (axis range) / (desirednumticks).
%
% Since not all step sizes H look well, restrict H to a set of allowed 
% step sizes such as 
%   { 1, 1/2, 1/5, 1/10 },
% or, to be more precise:
%   { 1*10^e, 2*10^e, 5*10^e }
% -> round to the nearest matching number!
%
% For log-plots, 
% 	H in { j*log(10), j=1,2,3,... }
% where the usual case should be j = 1.
%
% Then, the resulting tick is
% TICK={MIN,MIN+H,...,MAX}
% where
%    MIN = I*H
% is chosen such that 
%    axis minimum limit = I*H + rest; |rest| < H.
%
% Again, log plots follow a slightly different approach: here,
%   MIN = I * log(10)
% is chosen such that
%    axis minimum limit = I*log(10) + rest; |rest| < log(10)
% while H = j*log(10), j>=1.
%
%
% PRECONDITION:
% - limits are correct
% - axis width/height is set correctly
%
% POSTCONDITION:
% - Tick for axis #1 is assigned
% - \ifpgfplots@needs@check@uniformtick is set
\def\pgfplots@assign@default@tick@foraxis#1{%
	\begingroup
	% Shortcut-names:
	\expandafter\let\expandafter\ifpgfplots@is@datascaled\csname ifpgfplots@float@numerics@mode@#1\endcsname
	\expandafter\let\expandafter\pgfplots@data@scale@trafo\csname pgfplots@datascaletrafo@#1\endcsname
	\expandafter\let\expandafter\pgfplots@data@scale@inverse@trafo\csname pgfplots@inverse@datascaletrafo@#1\endcsname
	%
	\let\desirednumticks=\c@pgf@countd
	\let\Wr=\pgf@xc
	\Wr=\csname pgfplots@#1coordmaxTEX\endcsname
	\advance\Wr by-\csname pgfplots@#1coordminTEX\endcsname
	% r = max place without ticks in pt -> choose desirednumticks >= W/r
	\expandafter\expandafter\divide\Wr\axisdefaulttickwidth
	\pgfmathsetcount{\desirednumticks}{\Wr}%
	\advance\desirednumticks by1
	\ifnum1=\csname pgfplots@#1mode\endcsname\relax
		\expandafter\ifnum\pgfplots@default@try@minticks@log>\desirednumticks
			\expandafter\desirednumticks\pgfplots@default@try@minticks@log\relax
		\fi
		\expandafter\ifx\csname pgfplots@#1tickten\endcsname\pgfutil@empty
		\else
			% log plot and tickten-option: provide special processing.
			\edef\pgfplots@TMP{\csname pgfplots@#1tickten\endcsname}%
			\expandafter\pgfplots@compute@tick@times@logten\pgfplots@TMP\to\pgfplots@TMP
			\expandafter\let\csname pgfplots@#1tick\endcsname=\pgfplots@TMP
			\aftergroup\pgfplots@needs@check@uniformticktrue
		\fi
	\else
		\expandafter\ifnum\axisdefaulttryminticks>\desirednumticks
			\expandafter\desirednumticks\axisdefaulttryminticks
		\fi
	\fi
	%
	\expandafter\ifx\csname pgfplots@#1tick\endcsname\pgfutil@empty
		% Ok, we have either log or linear axis and need default
		% ticks MIN,MIN+H,...,MAX.
		\let\MINH=\pgf@xa
		\let\H=\pgf@xb
		\let\MAX=\pgfplots@tmpa
		\let\MIN=\pgfplots@tmpb
		% compute step size 'H':
		\expandafter\MAX\csname pgfplots@#1max\endcsname pt
		\advance\MAX by0.001pt% avoid round errors
		%\expandafter\MIN\the\c@pgf@counta pt
		\expandafter\MIN\csname pgfplots@#1min\endcsname pt
		\H=\MAX
		\advance\H by-\MIN
		\c@pgf@counta=\desirednumticks
		\advance\c@pgf@counta by-1
		\divide\H by\c@pgf@counta
%\pgfplots@message{determining ticks for #1-axis: Wr := (width/max space between ticks) = \the\Wr, desirednumticks=max(\axisdefaulttryminticks, trunc(Wr)) = \the\desirednumticks, H=(axis range/(desirednumticks-1)) = \the\H}%
		%
		% SEARCH for the NEXT FEASABLE H.
		\edef\Hmacro{\pgf@sys@tonumber\H}%
		\ifcase\csname pgfplots@#1mode\endcsname
			% CASE LINEAR AXIS
			\ifpgfplots@is@datascaled
				\expandafter\pgfplots@data@scale@inverse@trafo\expandafter{\Hmacro}%
				\let\Hmacro=\pgfmathresult
			\else
				\pgfmathfloatparsenumber{\Hmacro}%
				\let\Hmacro=\pgfmathresult
			\fi
			\expandafter\pgfmathfloat@decompose\pgfmathresult\relax\pgfmathfloat@a@S\H\pgfmathfloat@a@E
%\pgfplots@message{Got T^{-1}(H) = [\H@S] \H@M * 10^\H@E}%
			% modify the mantisse:
			\ifdim\H<2pt
				\ifdim\H<1.5pt
					\H=1.0pt
				\else
					\H=2.0pt
				\fi
			\else
				\ifdim\H<4.9999pt
					\ifdim\H<3.5pt
						\H=2.0pt\relax
					\else
						\H=5.0pt\relax
					\fi
				\else
					\ifdim\H<7.5pt
						\H=5.0pt\relax
					\else
						\H=1.0pt\relax
						\advance\pgfmathfloat@a@E by1
					\fi
				\fi
			\fi
			\edef\Hmacro{\the\pgfmathfloat@a@S\pgf@sys@tonumber{\H}e\the\pgfmathfloat@a@E}%
			\ifpgfplots@is@datascaled
				\pgfplots@data@scale@trafo\Hmacro
			\else
				\pgfmathfloattofixed\Hmacro
			\fi
			\let\Hmacro=\pgfmathresult
%\pgfplots@message{I have computed H=\Hmacro}%
			\expandafter\H\Hmacro pt
			\expandafter\aftergroup\csname pgfplots@#1isuniformtrue\endcsname
			%
			% Now, we want to activate the Tick set {i*H, i in \Z}
			% compute I such that
			%   xmin = I * H + rest;  |rest| < H
			% -> I = round(xmin/H)
			% -> MIN = I * H
			\pgfmathlog@invoke@expanded\pgfmathdivide@{%
				{\pgf@sys@tonumber\MIN}%
				{\pgf@sys@tonumber\H}%
			}%
			\pgfmathsetcount{\c@pgf@counta}{\pgfmathresult}%
			\ifdim\MIN<0pt
				% the truncation rounds TOWARDS 0 which is not what I want.
				\advance\c@pgf@counta by-1
			\fi
			\MIN=\H
			\multiply\MIN by\c@pgf@counta
		\or
			% CASE LOG AXIS
			%
			% search for the "best" H= j* log(10),  j an integer.
			%
			% And prefer j=1 if that is possible (otherwise minor
			% ticks are not useful).
			\pgfmathmultiply@{\Hmacro}{\reciproclogten}%
			\expandafter\H\pgfmathresult pt
%\pgfplots@message{ [ H / log(10) = \pgfmathresult}%
			\ifdim\H<2pt
				\H=1pt
			\else
				\ifnum\H<1pt
					\H=1pt
				\else
					\expandafter\pgfmathfloor\expandafter{\pgfmathresult}%
					\expandafter\H\pgfmathresult pt
				\fi
			\fi
			\ifdim\H=1pt
				\expandafter\aftergroup\csname pgfplots@#1isuniformtrue\endcsname
				\pgfplots@isuniformticktrue
			\else
				\expandafter\aftergroup\csname pgfplots@#1isuniformfalse\endcsname
				\pgfplots@isuniformtickfalse
			\fi
%\pgfplots@message{final H=\pgf@sys@tonumber{\H} * log(10)}%
			\H=\logten\H\relax
			% Now, we want to activate the Tick set 
			%   {lowest, lowest+H, ..., highest}
			%
			% Where 
			% 	lowest =  I * log(10) + rest, |rest| < log(10).
			% this is conceptionally different from the approach for
			% linear axes, because H = j*log(10).
			%
			% remember the original xmin in MINH:
			\MINH=\MIN
			%
			% and compute I and I*log(10) here:
			\expandafter\MIN\reciproclogten\MIN\relax
			\edef\pgfmathresult{\pgf@sys@tonumber{\MIN}}%
			\pgfmathsetcount{\c@pgf@counta}{\pgfmathresult}%
			\ifdim\MIN<0pt
				% the truncation rounds TOWARDS 0 which is not what I want.
				\advance\c@pgf@counta by-1
			\fi
			\expandafter\MIN\logten pt
			\multiply\MIN by\c@pgf@counta
			\ifpgfplots@isuniformtick
			\else
				% This here is a special case to move the first tick
				% near the lower axis limit.
				%
				% "Near" means either directly above or directly below ymin.
				% 
				% My application example is as follows:
				% Let H = 2*log(10).
				% Furthermore, ymin = 3e-6, ymax= 8e-2. That means we can choose either
				%    10^{-5}, 10^{-3}, 10^{-1}
				% or
				%    10^{-4}, 10^{-2}
				% as ticks. Well, I prefer the first one.
				%
				% HEURISTICS: start as near to ymin as possible!
				%
				% We check here if we can come nearer to ymin if we
				% shift the current tick by log(10):
				%  if( ymin - I * log(10) < 0.5*H ->  use I+1, that means add log(10).
				%
				% that's equivalent to 
				%  2*(ymin - I * log(10)) - H < 0.
				\advance\MINH by-\MIN
				\multiply\MINH by2
				\advance\MINH by-\H
				% 
				\ifdim\MINH<0pt
					\expandafter\advance\expandafter\MIN\logten pt
				\fi
			\fi
		\fi
		\MINH=\MIN
		\advance\MINH by\H
%\pgfplots@message{final H=\the\H}%
		\xdef\pgfplots@TMP{\pgf@sys@tonumber{\MIN},\pgf@sys@tonumber{\MINH},...,\pgf@sys@tonumber{\MAX}}%
		\aftergroup\pgfplots@needs@check@uniformtickfalse
	\fi
	\endgroup
	\expandafter\let\csname pgfplots@#1tick\endcsname=\pgfplots@TMP
%\pgfplots@message{pgfplots.sty: #1tick set to \csname pgfplots@#1tick\endcsname [#1mode=\the\csname pgfplots@#1mode\endcsname,  #1min=\csname pgfplots@#1min\endcsname, #1max=\csname pgfplots@#1max\endcsname].}%
}

% Helper method for 
%  \pgfplots@apply@data@scale@trafo@to@options@for
% #1: the ticks
% #2: the trafo macro name 
% #3: the output macro name
\long\def\pgfplots@apply@data@scale@trafo@to@user@ticks#1#2\to#3{%
	\let#3=\pgfutil@empty
	\foreach \pgfplots@TMPB in {#1} {%
		\pgfmathfloatparsenumber{\pgfplots@TMPB}%
		\expandafter#2\expandafter{\pgfmathresult}%
		\ifx#3\pgfutil@empty
			\xdef#3{\pgfmathresult}%
		\else
			\xdef#3{#3,\pgfmathresult}%
		\fi
	}%
	%
}%

% helper for \pgfplots@apply@data@scale@trafo@to@options@for.
\def\pgfplots@compute@number@order@for@trafo@isdimen#1\tocount#2{%
	\edef\pgfplots@TMP{\pgf@sys@tonumber{#1}}%
	\pgfmathfloatparsenumber{\pgfplots@TMP}%
	\expandafter\pgfmathfloat@decompose@E\pgfmathresult\relax#2
	\advance#2 by1
}

% helper for \pgfplots@apply@data@scale@trafo@to@options@for.
% 
\def\pgfplots@compute@number@order@for@trafo@isfloat#1\tocount#2{%
	\expandafter\pgfmathfloat@decompose@E#1\relax#2
	\advance#2 by1
}

% Initialises the data scale transformation and applies it to any
% user specified options.
%
% PRECONDITION:
%   - all axis limits are available in float representation
%   - \pgfplots@set@default@size@options has been called before
% POSTCONDITION:
%   - the scaling transformation is set up,
%   - all axis limits are transformed,
%   - any user input (like ticks and tick labels)
%     will reflect the changes.
\def\pgfplots@apply@data@scale@trafo@to@options@for#1{%
	% initialise data scale transformation 
	%   T(x) = 10^{q-m} * x
	%
	\begingroup
	\let\data@max@order=\c@pgf@counta
	\let\data@cur@order=\c@pgf@countb
	\let\data@dimen=\pgf@xa
	\let\data@dimen@order=\c@pgf@countc
	%
	% Step 1: compute 'm', the data order
	\expandafter\pgfplots@compute@number@order@for@trafo@isfloat
		\csname pgfplots@#1min\endcsname
		\tocount\data@cur@order
	%
	\data@max@order=\data@cur@order
	%
	\expandafter\pgfplots@compute@number@order@for@trafo@isfloat
		\csname pgfplots@#1max\endcsname
		\tocount\data@cur@order
	%
	\ifnum\data@cur@order>\data@max@order
		\data@max@order=\data@cur@order
	\fi
	%
	% Step 2: compute 'q', the #1-size of the axis.
	\expandafter\ifx\csname pgfplots@#1\endcsname\pgfutil@empty
		% We have 'width' or 'height'.
		%
		% Use the order of these parameters.
		\def\pgfplots@TMP{#1}%
		\def\pgfplots@TMPB{x}%
		\ifx\pgfplots@TMP\pgfplots@TMPB
			\expandafter\data@dimen\pgfplots@width\relax
		\else
			\def\pgfplots@TMPB{y}%
			\ifx\pgfplots@TMP\pgfplots@TMPB
				\expandafter\data@dimen\pgfplots@height\relax
			\fi
		\fi
		\pgfplots@compute@number@order@for@trafo@isdimen
			\data@dimen
			\tocount\data@dimen@order
		% This here is to avoid inaccuracies in the final
		% axis rectangle size, see \pgfplots@initsizes:
		\advance\data@dimen@order by-1
	\else
		% FIXME:
		% we have either the 'x=1cm' or 'y=1cm' option!
		% How should I initialise the trafo!?
		\data@dimen@order=3
	\fi
	%
%\pgfplots@message{Direction #1: data max order=\the\data@max@order;  data dimen order=\the\data@dimen@order. }%
	\advance\data@dimen@order by-\data@max@order
	\xdef\pgfplots@TMP{\the\data@dimen@order}%
	\endgroup
%\pgfplots@message{Initialising the data scale transformation in direction #1 to \pgfplots@TMP...}%
	% COMPLETE INITIALISATION:
	\ifpgfplots@EMERGENCY@FORCE@DATA@TRAFO@TO@IDENTITY
		\pgfplots@warning{WARNING: the automatic scaling of input data has been DISABLED to allow unscalable commands. DISABLING THE DATA SCALING LIMITS THE DATA RANGE! I hope I can fix this issue as soon as possible. Please take a look at the manual for more information. If you get 'OVERFLOW' or 'UNDERFLOW' errors, you may need to disable the axispath.}%
		\def\pgfplots@TMP{0}%
		\expandafter\let\csname pgfplots@data@scale@trafo@SHIFT@#1\endcsname\pgfplots@TMP
	\else
		\expandafter\let\csname pgfplots@data@scale@trafo@SHIFT@#1\endcsname\pgfplots@TMP
	\fi
	%
	% ... and apply transformation to any user input
	%
	% Transform axis limits:
	\expandafter\expandafter\csname pgfplots@datascaletrafo@#1\endcsname\expandafter{\csname pgfplots@#1min\endcsname}%
	\expandafter\let\csname pgfplots@#1min\endcsname=\pgfmathresult
	%
	\expandafter\expandafter\csname pgfplots@datascaletrafo@#1\endcsname\expandafter{\csname pgfplots@#1max\endcsname}%
	\expandafter\let\csname pgfplots@#1max\endcsname=\pgfmathresult
	%
	% Convert any user-specified ticks:
	\edef\pgfplots@TMP{\csname pgfplots@#1tick\endcsname}%
	% this here should also work with 'xtick=\pgfutil@empty', the "No tick" command.
	\ifx\pgfplots@TMP\pgfutil@empty
	\else
%\pgfplots@message{Converting #1tick='\csname pgfplots@#1tick\endcsname'}%
		\pgfplots@toka=\expandafter{\csname pgfplots@datascaletrafo@#1\endcsname}%
		\edef\pgfplots@TMP{{\csname pgfplots@#1tick\endcsname}\the\pgfplots@toka}%
		\expandafter\pgfplots@apply@data@scale@trafo@to@user@ticks\pgfplots@TMP\to\pgfplots@TMPC
		\expandafter\let\csname pgfplots@#1tick\endcsname=\pgfplots@TMPC
	\fi
	%
	% Convert any extra-ticks, see above.
	\edef\pgfplots@TMP{\csname pgfplots@extra@#1tick\endcsname}%
	\ifx\pgfplots@TMP\pgfutil@empty
	\else
		\pgfplots@toka=\expandafter{\csname pgfplots@datascaletrafo@#1\endcsname}%
		\edef\pgfplots@TMP{{\csname pgfplots@extra@#1tick\endcsname}\the\pgfplots@toka}%
		\expandafter\pgfplots@apply@data@scale@trafo@to@user@ticks\pgfplots@TMP\to\pgfplots@TMPC
		\expandafter\let\csname pgfplots@extra@#1tick\endcsname=\pgfplots@TMPC
	\fi
	%
	% Transform any explicit axis unit scalings:
	\expandafter\ifx\csname pgfplots@#1\endcsname\pgfutil@empty
	\else
%\pgfplots@message{Converting #1 unit scale='\csname pgfplots@#1\endcsname' ... }%
		\expandafter\pgfmathparse\expandafter{\csname pgfplots@#1\endcsname}%
		\expandafter\expandafter\csname pgfplots@inverse@datascaletrafo@tofixed@#1\endcsname\expandafter{\pgfmathresult}%
		\expandafter\edef\csname pgfplots@#1\endcsname{\pgfmathresult pt}%
%\pgfplots@message{to #1='\csname pgfplots@#1\endcsname'.}%
	\fi
}

\def\pgfplots@determinedefaultvalues{%
	\ifpgfplots@limits@are@computed
	\else
		% EMPTY AXIS:
		\pgfplots@warning{WARNING: You have a plot with empty range. This may produce errors!}%
		%--------------------------------------------------
		% \ifpgfplots@autocompute@xlim
		% 	\ifpgfplots@float@numerics@mode@x
		% 		\def\pgfplots@xmin{00.0e0}%
		% 		\def\pgfplots@xmax{11.0e0}%
		% 	\else
		% 		\def\pgfplots@xmin{0}%
		% 		\def\pgfplots@xmax{1}%
		% 	\fi
		% \fi
		% \ifpgfplots@autocompute@ylim
		% 	\ifpgfplots@float@numerics@mode@y
		% 		\def\pgfplots@ymin{00.0e0}%
		% 		\def\pgfplots@ymax{11.0e0}%
		% 	\else
		% 		\def\pgfplots@ymin{0}%
		% 		\def\pgfplots@ymax{1}%
		% 	\fi
		% \fi
		%-------------------------------------------------- 
	\fi
	%
	\pgfplots@set@default@size@options
	%
	\ifpgfplots@float@numerics@mode@x
		\let\pgfplots@xmin@unscaled@as@float=\pgfplots@xmin
		\let\pgfplots@xmax@unscaled@as@float=\pgfplots@xmax
		\pgfplots@apply@data@scale@trafo@to@options@for x%
	\else
		\let\pgfplots@xmin@unscaled@as@float=\pgfutil@empty
		\let\pgfplots@xmax@unscaled@as@float=\pgfutil@empty
	\fi
	\ifpgfplots@float@numerics@mode@y
		\let\pgfplots@ymin@unscaled@as@float=\pgfplots@ymin
		\let\pgfplots@ymax@unscaled@as@float=\pgfplots@ymax
		\pgfplots@apply@data@scale@trafo@to@options@for y%
	\else
		\let\pgfplots@ymin@unscaled@as@float=\pgfutil@empty
		\let\pgfplots@ymax@unscaled@as@float=\pgfutil@empty
	\fi
	%
	\ifnum\pgfplots@xmode=0
		\pgfplots@xminorticksfalse
	\fi
	\ifnum\pgfplots@ymode=0
		\pgfplots@yminorticksfalse
	\fi
	%
	\pgfplots@init@enlarge@limit@booleans
	\ifpgfplots@enlargelimits
		% relax the sizes.
		%
		% Idea: if the user chose his xmin,xmax tight to his data,
		% this here will look better.
		\pgfplots@enlarge@limit@for x
		\pgfplots@enlarge@limit@for y
	\else
		\ifpgfplots@enlargelimits@auto
			\ifpgfplots@hide
				% there is no axis, so skip this enlargement (unless
				% the user explizitly requests it)
			\else
				% FIXME : this here should be user-configurable!
				\ifpgfplots@autocompute@xlim
					\pgfplots@enlarge@limit@for x
				\fi
				\ifpgfplots@autocompute@ylim
					\pgfplots@enlarge@limit@for y
				\fi
			\fi
		\fi
	\fi
	%
	\pgfplots@avoid@empty@axis@range@for x%
	\pgfplots@avoid@empty@axis@range@for y%
	%
	\pgfplots@initsizes
	%
	\newif\ifpgfplots@xisuniform
	\newif\ifpgfplots@yisuniform
	\pgfplots@xisuniformtrue
	\pgfplots@yisuniformtrue
	\newif\ifpgfplots@needs@check@uniformtick
	\pgfplots@needs@check@uniformticktrue
	\ifx\pgfplots@xtick\pgfutil@empty
		\pgfplots@assign@default@tick@foraxis{x}%
	\fi
	\ifpgfplots@needs@check@uniformtick
		\ifnum\pgfplots@xmode=1
			\expandafter\pgfplots@checkisuniformLOGtick\expandafter{\pgfplots@xtick}%
			\ifpgfplots@isuniformtick
				\pgfplots@xisuniformtrue
			\else
				\pgfplots@xisuniformfalse
			\fi
		\fi
	\fi
	\ifpgfplots@xisuniform
	\else
		\pgfplots@xminorticksfalse
	\fi
	%
	%
	%
	\pgfplots@needs@check@uniformticktrue
	\ifx\pgfplots@ytick\pgfutil@empty
		\pgfplots@assign@default@tick@foraxis{y}%
	\fi
	\ifpgfplots@needs@check@uniformtick
		\ifnum\pgfplots@ymode=1
			\expandafter\pgfplots@checkisuniformLOGtick\expandafter{\pgfplots@ytick}%
			\ifpgfplots@isuniformtick
				\pgfplots@yisuniformtrue
			\else
				\pgfplots@yisuniformfalse
			\fi
		\fi
	\fi
	\ifpgfplots@yisuniform
	\else
		\pgfplots@yminorticksfalse
	\fi
	%
	\ifx\pgfplots@xticklabel\pgfutil@empty
		\ifcase\pgfplots@xmode
			\def\pgfplots@xticklabel{\axisdefaultticklabel}%
		\or
			\def\pgfplots@xticklabel{\axisdefaultticklabellog}%
		\fi
	\fi
	\ifx\pgfplots@extra@xticklabel\pgfutil@empty
		\let\pgfplots@extra@xticklabel=\pgfplots@xticklabel
	\fi
	\ifx\pgfplots@yticklabel\pgfutil@empty
		\ifcase\pgfplots@ymode
			\def\pgfplots@yticklabel{\axisdefaultticklabel}%
		\or
			\def\pgfplots@yticklabel{\axisdefaultticklabellog}%
		\fi
	\fi
	\ifx\pgfplots@extra@yticklabel\pgfutil@empty
		\let\pgfplots@extra@yticklabel=\pgfplots@yticklabel
	\fi
}


% Helper method for initsizes.
%
% It computes a scaling such that \pgfplots@width = SCALE * ACTUAL WIDTH.
% 
% The actual width is 
% 	c + x*(xmax-xmin)
% based on
% - x*xmax = \pgfplots@xcoordmaxTEX
% - x*xmin = \pgfplots@xcoordminTEX
% - c = estimated, a constant for the axis label/tick labels
%
% Arguments: 
% #1: the output argument for the SCALE.
\def\pgfplots@initsizes@getXscale\into#1{%
	\expandafter\pgfplots@tmpa\pgfplots@width\relax
	% EXPECTED WIDTH = X = \pgfplots@width
	% ACTUAL WIDTH = c + x * (xmax-xmin)
	% where c is a CONSTANT (for the axis labels/tick labels).
	% -> \pgfplots@tmpXscale = (X - c) / (x *(xmax-xmin))
	%
	% \pgfplots@tmpa := X-c:
	\ifpgfplots@scale@only@axis
	\else
		\advance\pgfplots@tmpa by-45pt% FIXME determine 'c' correctly!
	\fi
	\ifdim\pgfplots@tmpa<0pt
		\pgfplots@error{Error: Plot width `\pgfplots@width' is too small. This can't be realised while maintaining constant width for y-labels. Sorry, label width are only approximate. You will need to adjust your width.}%
		\pgfplots@tmpa=0pt
	\fi
	% \pgfplots@tmpb := x*(xmax-xmin):
	\pgfplots@tmpb=\pgfplots@xcoordmaxTEX
	\advance\pgfplots@tmpb by-\pgfplots@xcoordminTEX
	\pgfmathlog@invoke@expanded\pgfmathdivide@{%
		{\pgf@sys@tonumber\pgfplots@tmpa}%
		{\pgf@sys@tonumber\pgfplots@tmpb}%
	}%
	\let#1=\pgfmathresult
%\pgfplots@message{pgfplots.sty: Computing 'x' such that 'width = c + x*(xmax-xmin)';
%	c=estimated, 
%	width-c =\the\pgfplots@tmpa,  
%	x*(xmax[=\the\pgfplots@xcoordmaxTEX] - xmin[=\the\pgfplots@xcoordminTEX)]) = \the\pgfplots@tmpb  
%	-> x-scale =#1 }%
}

% The same as \pgfplots@initsizes@getXscale, just for the height.
\def\pgfplots@initsizes@getYscale\into#1{%
	\expandafter\pgfplots@tmpa\pgfplots@height\relax
	% EXPECTED WIDTH = X = \pgfplots@width
	% ACTUAL WIDTH = c + x * (xmax-xmin)
	% where c is a CONSTANT (for the axis labels/tick labels).
	% -> \pgfplots@tmpXscale = (X - c) / (x *(xmax-xmin))
	%
	% \pgfplots@tmpa := X-c:
	\ifpgfplots@scale@only@axis
	\else
		\advance\pgfplots@tmpa by-45pt\relax% FIXME determine 'c' correctly!
	\fi
	\ifdim\pgfplots@tmpa<0pt
		\pgfplots@error{Error: Plot height `\pgfplots@height' is too small. This can't be realised while maintaining constant height for x-labels. Sorry, label heights are only approximate. You will need to adjust your height.}%
		\pgfplots@tmpa=0pt
	\fi
	% \pgfplots@tmpb := x*(xmax-xmin):
	\pgfplots@tmpb=\pgfplots@ycoordmaxTEX
	\advance\pgfplots@tmpb by-\pgfplots@ycoordminTEX
	\pgfmathlog@invoke@expanded\pgfmathdivide@{%
		{\pgf@sys@tonumber\pgfplots@tmpa}%
		{\pgf@sys@tonumber\pgfplots@tmpb}%
	}%
	\let#1=\pgfmathresult
%\pgfplots@message{pgfplots.sty: Computing 'y' such that 'height = c + y*(ymax-ymin)';
%	height=\pgfplots@height,
%	c=estimated,
%	height-c =\the\pgfplots@tmpa,  
%	y*(ymax[=\the\pgfplots@ycoordmaxTEX] - ymin[=\the\pgfplots@ycoordminTEX)]) = \the\pgfplots@tmpb  
%	-> y-scale =#1 }%
}


% Checks whether axis limits in coordinate #1 are approximately equal.
%
% If that is the case, force a non-zero width of the range.
%
\def\pgfplots@avoid@empty@axis@range@for#1{%
	% Check if axis limits are empty:
	\begingroup
	\expandafter\let\expandafter\if@cur@is@scaled\csname ifpgfplots@float@numerics@mode@#1\endcsname
	\expandafter\let\expandafter\min\csname pgfplots@#1min\endcsname
	\expandafter\let\expandafter\max\csname pgfplots@#1max\endcsname
	\let\min@d=\pgf@xa
	\let\max@d=\pgf@xb
	\let\diff=\pgf@xc
	\newif\ifrange@is@approx@equal
	\expandafter\min@d\min pt
	\expandafter\max@d\max pt
	\diff=\max@d
	\advance\diff by-\min@d
	% FIXME : I need a RELATIVE check here!
	% but: real number point division is expensive
	\if@cur@is@scaled
		% this here should be sufficient because the axis
		% has absolute values of order O( 10^3 ) or so.
		\ifdim\diff<1pt
			\range@is@approx@equaltrue
		\fi
	\else
		% there is no data scaling, so I should be much more defensive
		% with absolute thresholds...
		\ifdim\diff<0.05pt
			\range@is@approx@equaltrue
		\fi
	\fi
	\ifrange@is@approx@equal
%\pgfplots@message{avoid range is approx equal for coord #1: MODIFYING (\min,\max)...}%
		% the case \min ~= \max
		%
		% enlarge \max and shrink \min:
		\ifdim\max@d<0pt
			\ifdim\max@d<-1pt
				\max@d=0.8\max@d
				\min@d=1.2\min@d
			\else
				\advance\max@d by-1pt
				\advance\min@d by1pt
			\fi
		\else
			\ifdim\max@d>1pt
				\max@d=1.2\max@d
				\min@d=0.8\min@d
			\else
				\ifdim\max@d=0pt
					\if@cur@is@scaled
						% we can't simply add a constant in the
						% transformed range.
						%
						% So: set limits to [-1,1] = [-1.0e0,+1.0e0]
						\csname pgfplots@datascaletrafo@#1\endcsname{21.0e0}%
						\let\min=\pgfmathresult
						\csname pgfplots@datascaletrafo@#1\endcsname{11.0e0}%
						\let\max=\pgfmathresult
%\pgfplots@message{[trafo shift = \csname pgfplots@data@scale@trafo@SHIFT@#1\endcsname; setting limits -1:1]}%
						\expandafter\min@d\min pt
						\expandafter\max@d\max pt
					\else
						\advance\max@d by1pt
						\advance\min@d by-1pt
					\fi
				\else
					\advance\max@d by1pt
					\advance\min@d by-1pt
				\fi
			\fi
		\fi
		\xdef\pgfplots@TMP{\pgf@sys@tonumber{\min@d}}%
		\xdef\pgfplots@TMPB{\pgf@sys@tonumber{\max@d}}%
%\pgfplots@message{ -> #1 = \pgfplots@TMP : \pgfplots@TMPB;}%
	\else
		\global\let\pgfplots@TMP=\min%
		\global\let\pgfplots@TMPB=\max%
	\fi
	\endgroup
	\expandafter\let\csname pgfplots@#1min\endcsname=\pgfplots@TMP
	\expandafter\let\csname pgfplots@#1max\endcsname=\pgfplots@TMPB
}

% PRECONDITION:
% 	none
% POSTCONDITION:
% 	\pgfplots@default@aspect@ratio is set.
\def\pgfplots@compute@default@aspect@ratio{%
	\expandafter\pgfmath@x\axisdefaultwidth
	\expandafter\pgfmath@y\axisdefaultheight
	\pgfmathlog@invoke@expanded\pgfmathdivide@{%
		{\pgf@sys@tonumber{\pgfmath@x}}%
		{\pgf@sys@tonumber{\pgfmath@y}}%
	}%
	\let\pgfplots@default@aspect@ratio=\pgfmathresult
}

\def\pgfplots@set@default@size@options{%
	% The axes 'x' and 'y' vectors will be scaled such that the total
	% size is (\axisdefaultwidth, \axisdefaultheight).
	%
	% If the user specifies ONE of width OR height, 
	% the plot will be resized; keeping the aspect ratio.
	%
	\let\pgfplots@default@aspect@ratio=\pgfutil@empty
	% CASES:
	% hasx := 'x' option non-empty
	% hasy := 'y' option non-empty
	% W := 'width' option non-empty
	% H := 'height' option non-empty
	%
	% hasx = 1 -> width is not interesting; we use 'x' option.
	% hasx = 0 -> determine final width:
	% 		W H
	% 		0 0 -> \axisdefaultwidth 
	% 		0 1 -> determine width out of H and the default aspect ratio
	% 		1 X -> ok, use the user parameter.
	%
	% hasy = 1 -> height is not interesting, we use 'y' option.
	% hasy = 0 -> determine final height:
	% 		W H
	% 		0 0 -> \axisdefaultheight
	% 		X 1 -> ok, use the user parameter
	% 		1 0 -> determine height out of W and the default aspect ratio
	%
	\ifx\pgfplots@x\pgfutil@empty
		\ifx\pgfplots@y\pgfutil@empty
			% hasx=0, hasy=0 
			%
			% -> KEEP ASPECT RATIO if just one W, or H is given!
			\ifx\pgfplots@width\pgfutil@empty
				\ifx\pgfplots@height\pgfutil@empty
					% The case hasx=0, hasy=0,  W=0 H=0:
					\let\pgfplots@width=\axisdefaultwidth
					\let\pgfplots@height=\axisdefaultheight
				\else
					% The case hasx=0, hasy=0,  W=0 H=1:
					\pgfplots@compute@default@aspect@ratio
					\expandafter\pgfmath@y\pgfplots@height
					\pgfmathlog@invoke@expanded\pgfmathmultiply@{%
						{\pgf@sys@tonumber{\pgfmath@y}}%
						{\pgfplots@default@aspect@ratio}%
					}%
					\edef\pgfplots@width{\pgfmathresult pt}%
				\fi
			\else
				\ifx\pgfplots@height\pgfutil@empty
					% The case hasx=0, hasy=0,  W=1 H=0:
					\pgfplots@compute@default@aspect@ratio
					\expandafter\pgfmath@x\pgfplots@width
					\pgfmathlog@invoke@expanded\pgfmathdivide@{%
						{\pgf@sys@tonumber{\pgfmath@x}}%
						{\pgfplots@default@aspect@ratio}%
					}%
					\edef\pgfplots@height{\pgfmathresult pt}%
				\else
					% The case hasx=0, hasy=0,  W=1 H=1:
				\fi
			\fi
		\else
			% hasx=0, hasy=1, W=0:
			\ifx\pgfplots@width\pgfutil@empty
				\let\pgfplots@width=\axisdefaultwidth
			\fi
		\fi
	\else
		\ifx\pgfplots@y\pgfutil@empty
			% hasx=1, hasy=0, H=0
			\ifx\pgfplots@height\pgfutil@empty
				\let\pgfplots@height=\axisdefaultheight
			\fi
		\fi
	\fi
}

% PRECONDITION: 
% 	- final axis limits are given in transformed range
% 	-  \pgfplots@set@default@size@options has been invoked before
% POSTCONDITION: 
% 	- the current x- and y unit vectors are changed;
% 	- \pgfplots@[xy]coord{min,max}TEX  are set
%
\def\pgfplots@initsizes{%
	% INIT.
	%
	%
	\pgfpointxy{\pgfplots@xmin}{\pgfplots@ymin}%
	\pgfplots@xcoordminTEX=\pgf@x
	\pgfplots@ycoordminTEX=\pgf@y
	\pgfpointxy{\pgfplots@xmax}{\pgfplots@ymax}%
	\pgfplots@xcoordmaxTEX=\pgf@x
	\pgfplots@ycoordmaxTEX=\pgf@y
	%
	%
	%-----------------------------------------
	% PROCESS THE 'width' and 'height' options
	%-----------------------------------------
	%
	% FIXME: make these variables LOCAL:
	%
	\let\pgfplots@rectangle@width=\pgfutil@empty
	\let\pgfplots@rectangle@height=\pgfutil@empty
	%
	\ifx\pgfplots@x\pgfutil@empty
		\ifx\pgfplots@width\pgfutil@empty
			\pgfplots@error{INTERNAL LOGIC ERROR! WIDTH NOT SET}%
		\fi
		\pgfplots@initsizes@getXscale\into\pgfplots@tmpXscale
		\ifpgfplots@scale@only@axis
			\let\pgfplots@rectangle@width=\pgfplots@width
		\fi
	\else
		\def\pgfplots@tmpXscale{1}%
	\fi
	%
	\ifx\pgfplots@y\pgfutil@empty
		\ifx\pgfplots@height\pgfutil@empty
			\pgfplots@error{INTERNAL LOGIC ERROR! HEIGHT NOT SET}%
		\fi
		\pgfplots@initsizes@getYscale\into\pgfplots@tmpYscale
		\ifpgfplots@scale@only@axis
			\let\pgfplots@rectangle@height=\pgfplots@height
		\fi
	\else
		\def\pgfplots@tmpYscale{1}%
	\fi
	%
	% 
	% assert( \pgfplots@tmpXscale != \pgfutil@empty && \pgfplots@tmpYscale != \pgfutil@empty )
	%
	% Apply scaling:
	\pgfpointxy{\pgfplots@tmpXscale}{\pgfplots@tmpYscale}%
	\pgfplots@tmpa=\pgf@x
	\pgfplots@tmpb=\pgf@y
	\ifx\pgfplots@x\pgfutil@empty
		\pgfsetxvec{\pgfpoint{\pgfplots@tmpa}{0pt}}%
	\else
		\pgfsetxvec{\pgfpoint{\pgfplots@x}{0pt}}%
	\fi
	\ifx\pgfplots@y\pgfutil@empty
		\pgfsetyvec{\pgfpoint{0pt}{\pgfplots@tmpb}}%
	\else
		\pgfsetyvec{\pgfpoint{0pt}{\pgfplots@y}}%
	\fi
	%
	% Determine final rectangle dimensions.
	% There are the following cases: 
	% 1. the user really wants a fixed dimension,
	%    i.e. he used 'scale only axis'.
	%    Then, we have to work to get the correct dimension!
	%
	%    Up to now, the scaling mechanism looses to many significant
	%    digits such that the final width/height differs by 1-2 pt.
	%
	%    If I am not mistaken, this does ONLY affect the final size,
	%    not the relative plot precision.
	%    
	%    FIXME : really compute the plot precision!
	% 
	% 2. The use specified width and/or height, but not 'scale only
	%    axis'. Accept inaccurate final widths/heights (see above).
	%
	% 3. The user supplied 'x' and or 'y'. Simply use them, its
	% accurate.
	%
	\pgfpointxy{\pgfplots@xmin}{\pgfplots@ymin}%
	\pgfplots@xcoordminTEX=\pgf@x
	\pgfplots@ycoordminTEX=\pgf@y
	\pgfpointxy{\pgfplots@xmax}{\pgfplots@ymax}%
	\ifx\pgfplots@rectangle@width\pgfutil@empty
		\pgfplots@xcoordmaxTEX=\pgf@x
	\else
		% this 'if' here should only make a difference of about
		% 1-2pt, not more.
		%
		% and I am quite sure that this inaccuracy (and this
		% work-around) only affects the
		% final size, not the relative plot accuracy.
		\pgfplots@xcoordmaxTEX=\pgfplots@xcoordminTEX
		\expandafter\advance\expandafter\pgfplots@xcoordmaxTEX\pgfplots@width
	\fi
	\ifx\pgfplots@rectangle@height\pgfutil@empty
		\pgfplots@ycoordmaxTEX=\pgf@y
	\else
		\pgfplots@ycoordmaxTEX=\pgfplots@ycoordminTEX
		\expandafter\advance\expandafter\pgfplots@ycoordmaxTEX\pgfplots@height
	\fi
%--------------------------------------------------
% \begingroup
% \pgf@x=\pgfplots@xcoordmaxTEX\advance\pgf@x by-\pgfplots@xcoordminTEX
% \pgf@y=\pgfplots@ycoordmaxTEX\advance\pgf@y by-\pgfplots@ycoordminTEX
% \pgfplots@message{Axis scaling x=\pgfplots@tmpXscale, y=\pgfplots@tmpYscale\ yields lower-left-corner (\pgfplots@xmin,\pgfplots@ymin) = (\the\pgfplots@xcoordminTEX,\the\pgfplots@ycoordminTEX) and upper right (\pgfplots@xmax,\pgfplots@ymax) = (\the\pgfplots@xcoordmaxTEX,\the\pgfplots@ycoordmaxTEX).   Axis dimensions are width=\the\pgf@x, height=\the\pgf@y.}%
% \endgroup
%-------------------------------------------------- 
}


% This is the main axis shape.
%
% It has one node part, which is the complete image. It provides a lot
% of anchors.
\pgfdeclareshape{pgfplots@low@level@shape}{%
	\savedanchor\upperrightcorner{
		\pgf@x=\wd\pgfnodepartimagebox
		\pgf@y=\ht\pgfnodepartimagebox
	}%
	\savedanchor\lowerleftinnercorner{%
		\pgfpoint{\pgfplots@savedanchor@inner@lowerleft@x}{\pgfplots@savedanchor@inner@lowerleft@y}%
	}%
	\savedanchor\upperrightinnercorner{%
		\pgfpoint{\pgfplots@savedanchor@inner@upperright@x}{\pgfplots@savedanchor@inner@upperright@y}%
	}%
	%
	\nodeparts{image}%
	\anchor{image}{%
		\pgf@x=0pt
		\pgf@y=0pt
	}%
	%
	\anchor{outer north}{%
		\upperrightcorner
		\pgf@x=.5\pgf@x
	}%
	\anchor{outer north east}{\upperrightcorner}%
	\anchor{outer east}{%
		\upperrightcorner
		\pgf@y=.5\pgf@y
	}%
	\anchor{outer south east}{%
		\upperrightcorner
		\pgf@y=0pt
	}%
	\anchor{outer south}{%
		\upperrightcorner
		\pgf@x=.5\pgf@x
		\pgf@y=0pt
	}%
	\anchor{outer south west}{%
		\pgf@x=0pt
		\pgf@y=0pt
	}%
	\anchor{outer west}{%
		\upperrightcorner
		\pgf@x=0pt
		\pgf@y=.5\pgf@y
	}%
	\anchor{outer north west}{%
		\upperrightcorner
		\pgf@x=0pt
	}%
	\anchor{outer center}{%
		\upperrightcorner
		\pgf@x=.5\pgf@x
		\pgf@y=.5\pgf@y
	}%
	%
	%
	\anchor{center}{%
		\lowerleftinnercorner
		\pgf@xa=\pgf@x
		\pgf@xb=\pgf@y
		\upperrightinnercorner
		\advance\pgf@xa by\pgf@x
		\pgf@x=.5\pgf@xa
		\advance\pgf@xb by\pgf@y
		\pgf@y=.5\pgf@xb
	}%
	\anchor{north}{%
		\lowerleftinnercorner
		\pgf@xa=\pgf@x
		\pgf@xb=\pgf@y
		\upperrightinnercorner
		\advance\pgf@xa by\pgf@x
		\pgf@x=.5\pgf@xa
	}%
	\anchor{north east}{\upperrightinnercorner}%
	\anchor{east}{%
		\lowerleftinnercorner
		\pgf@xa=\pgf@x
		\pgf@xb=\pgf@y
		\upperrightinnercorner
		\advance\pgf@xb by\pgf@y
		\pgf@y=.5\pgf@xb
	}%
	\anchor{south east}{%
		\lowerleftinnercorner
		\pgf@xa=\pgf@x
		\pgf@xb=\pgf@y
		\upperrightinnercorner
		\pgf@y=\pgf@xb
	}%
	\anchor{south}{%
		\lowerleftinnercorner
		\pgf@xa=\pgf@x
		\pgf@xb=\pgf@y
		\upperrightinnercorner
		\advance\pgf@xa by\pgf@x
		\pgf@x=.5\pgf@xa
		\pgf@y=\pgf@xb
	}%
	\anchor{south west}{\lowerleftinnercorner}%
	\anchor{west}{%
		\lowerleftinnercorner
		\pgf@xa=\pgf@x
		\pgf@xb=\pgf@y
		\upperrightinnercorner
		\pgf@x=\pgf@xa
		\advance\pgf@xb by\pgf@y
		\pgf@y=.5\pgf@xb
	}%
	\anchor{north west}{%
		\lowerleftinnercorner
		\pgf@xa=\pgf@x
		\pgf@xb=\pgf@y
		\upperrightinnercorner
		\pgf@x=\pgf@xa
	}%
	%%
	%%
	\anchor{above north}{%
		\upperrightcorner
		\pgf@xa=\pgf@y
		\lowerleftinnercorner
		\pgf@xb=\pgf@x
		\upperrightinnercorner
		\advance\pgf@xb by\pgf@x
		\pgf@x=.5\pgf@xb
		\pgf@y=\pgf@xa
	}%
	\anchor{above north east}{%
		\upperrightcorner
		\pgf@xa=\pgf@y
		\upperrightinnercorner
		\pgf@y=\pgf@xa
	}%
	\anchor{right of north east}{%
		\upperrightcorner
		\pgf@xa=\pgf@x
		\upperrightinnercorner
		\pgf@x=\pgf@xa
	}%
	\anchor{right of east}{%
		\upperrightcorner
		\pgf@xa=\pgf@x
		\upperrightinnercorner
		\pgf@xb=\pgf@y
		\lowerleftinnercorner
		\advance\pgf@xb by\pgf@y
		\pgf@x=\pgf@xa
		\pgf@y=.5\pgf@xb
	}%
	\anchor{right of south east}{%
		\upperrightcorner
		\pgf@xa=\pgf@x
		\lowerleftinnercorner
		\pgf@x=\pgf@xa
	}%
	\anchor{below south east}{%
		\upperrightinnercorner
		\pgf@y=0pt
	}%
	\anchor{below south}{%
		\lowerleftinnercorner
		\pgf@xa=\pgf@x
		\upperrightinnercorner
		\pgf@y=0pt
		\advance\pgf@xa by\pgf@x
		\pgf@x=.5\pgf@xa
	}%
	\anchor{below south west}{%
		\lowerleftinnercorner
		\pgf@y=0pt
	}%
	\anchor{left of south west}{%
		\lowerleftinnercorner
		\pgf@x=0pt
	}%
	\anchor{left of west}{%
		\lowerleftinnercorner
		\pgf@xa=\pgf@y
		\upperrightinnercorner
		\advance\pgf@xa by\pgf@y
		\pgf@y=.5\pgf@xa
		\pgf@x=0pt
	}%
	\anchor{left of north west}{%
		\upperrightinnercorner
		\pgf@x=0pt
	}%
	\anchor{above north west}{%
		\upperrightcorner
		\pgf@xa=\pgf@y
		\lowerleftinnercorner
		\pgf@y=\pgf@xa
	}%
	%%
	%%
	%%
	\anchorborder{%
		% Call a function that computes a border point. Since this
		% function will modify dimensions like \pgf@x, we must move them to
		% other dimensions.
		\@tempdima=\pgf@x
		\@tempdimb=\pgf@y
		\pgfpointborderrectangle%
			{\pgfpoint{\@tempdima}{\@tempdimb}}%
			{\pgfpointadd{\upperrightcorner}{\pgfpoint{\width}{\height}}}
	}%
	\backgroundpath{\pgfpathrectangle{\pgfpointorigin}{\upperrightcorner}}%
	\foregroundpath{}%
	\behindbackgroundpath{}%
	\beforebackgroundpath{}%
	\behindforegroundpath{}%
	\beforeforegroundpath{}%
}

\def\pgfplots@BEGIN@init@and@draw@axis{%
	\pgfplots@determinedefaultvalues
	\setbox\pgfnodepartimagebox=\hbox\bgroup\bgroup
		\pgfinterruptpicture
		\tikzpicture%
		% set baseline for sub-picture to default value.
		% the baseline option will be applied to the OUTER picture.
		\pgfsetbaseline{\pgf@picminy}%
		%
		\scope[style=every axis]
		\ifpgfplots@hide
			\clip
							(\pgfplots@xcoordminTEX,	\pgfplots@ycoordminTEX) 
				rectangle	(\pgfplots@xcoordmaxTEX,	\pgfplots@ycoordmaxTEX);
		\else
			\expandafter\pgfplots@drawxaxis\expandafter{\pgfplots@xtick}%
			\expandafter\pgfplots@draw@extra@ticks@for\expandafter x\expandafter{\pgfplots@extra@xtick}%
			\expandafter\pgfplots@drawyaxis\expandafter{\pgfplots@ytick}%
			\expandafter\pgfplots@draw@extra@ticks@for\expandafter y\expandafter{\pgfplots@extra@ytick}%
			\draw[clip]
							(\pgfplots@xcoordminTEX,	\pgfplots@ycoordminTEX) 
				rectangle	(\pgfplots@xcoordmaxTEX,	\pgfplots@ycoordmaxTEX);
		\fi
}

\def\pgfplots@END@init@and@draw@axis{%
	\endscope%
}

% Writes output to \pgfplots@TMP
\def\pgfplots@filter@input@ticks@with@log#1{%
	\let\pgfplots@TMP=\pgfutil@empty
	\foreach \pgfplots@TMPB in {#1} {%
		\expandafter\pgfmathlog@\expandafter{\pgfplots@TMPB}%
		\ifx\pgfplots@TMP\pgfutil@empty
			\xdef\pgfplots@TMP{\pgfmathresult}%
		\else
			\xdef\pgfplots@TMP{\pgfplots@TMP,\pgfmathresult}%
		\fi
	}%
}

\tikzdeclarecoordinatesystem{axis}{\pgfplots@evalute@tikz@coord@system@interface#1\pgfplots@coord@end}


% Assigns \pgfmathresult := canvas coordinate (#2) for axis #1.
\long\def\pgfplots@evalute@tikz@coord@system@interface@for#1#2{%
	\expandafter\let\expandafter\if@datascaled@cur\csname ifpgfplots@float@numerics@mode@#1\endcsname
	\expandafter\ifcase\csname pgfplots@#1mode\endcsname
		\if@datascaled@cur
			\csname pgfplots@datascaletrafo@fromfixed@#1\endcsname{#2}%
		\else
			\def\pgfmathresult{#2}%
		\fi
	\or
		\pgfmathlog@{#2}%
	\fi
}

\long\def\pgfplots@evalute@tikz@coord@system@interface#1,#2\pgfplots@coord@end{%
	\begingroup
	\pgfplots@evalute@tikz@coord@system@interface@for{x}{#1}%
	\let\pgfplots@evaluate@tikz@coord@x=\pgfmathresult
	\pgfplots@evalute@tikz@coord@system@interface@for{y}{#2}%
	\xdef\pgfplots@TMP{\pgfplots@evaluate@tikz@coord@x pt}%
	\xdef\pgfplots@TMPB{\pgfmathresult pt}%
	\endgroup
	\pgfpointxy{\pgfplots@TMP}{\pgfplots@TMPB}%
}

% In case of (semi-) logplots, this command will 
% - assign a filter which invokes \pgfmathlog@{} for each coordinate
% - replace any user-specified coordinate by its log.
%
% All subsequent commands will then work with logarithmic coordinates.
%
% @see pgfmathlog.sty for details about the implementation of log().
%
% PRECONDITION: 
% - The user input options have been set correctly, 
% - the option processing has not yet begun
%
% POSTCONDITION: 
% - any user input for log-axis has been replaced by its log
% - coordinate filters to compute logs are installed
%
% See also:
%     \pgfplots@apply@data@scale@trafo@to@options@for
\def\pgfplots@prepare@coord@filtering@for#1{%
	\expandafter\ifcase\csname pgfplots@#1mode\endcsname
		\ifpgfplots@disabledatascaling
			\csname pgfplots@float@numerics@mode@#1false\endcsname
			\pgfplots@float@numerics@modefalse
		\else
			\csname pgfplots@float@numerics@mode@#1true\endcsname
			\pgfplots@float@numerics@modetrue
		\fi
		\ifpgfplots@float@numerics@mode
			% Install coordinate filters that
			% - read input as normalised floating point numbers,
			% - return [SIGN][MANTISSE]eEXPONENT
			% Motivation: 
			% 1. we can compute axis limits using such data
			% 2. we can remember the floating point representation for 
			%    the data scaling transformation.
			% ATTENTION:
			% this here is a preprocessing step! You will need to
			% apply the data scaling later!
			% see 
			% \pgfplots@apply@data@scaletrafo@to@plot@data
			\expandafter\let\expandafter\pgfplots@TMP\csname pgfplots@#1filter\endcsname
			\ifx\pgfplots@TMP\pgfutil@empty
				\expandafter\def\csname addplot@float@numerics@#1filter\endcsname##1\to##2{%
					\pgfmathfloatparsenumber{##1}%
					\let##2=\pgfmathresult
				}%
			\else
				\expandafter\let\csname pgfplots@#1filter@orig\endcsname=\pgfplots@TMP
				\expandafter\def\csname addplot@float@numerics@#1filter\endcsname##1\to##2{%
					\csname pgfplots@#1filter@orig\endcsname##1\to##2\relax
					\ifx##2\pgfutil@empty
					\else
						\expandafter\pgfmathfloatparsenumber\expandafter{##2}%
						\let##2=\pgfmathresult
					\fi
				}%
			\fi
			\expandafter\let\expandafter\pgfplots@TMP\csname addplot@float@numerics@#1filter\endcsname
			\expandafter\let\csname pgfplots@#1filter\endcsname\pgfplots@TMP
			%
			% Check for any existing axis limits:
			\expandafter\let\expandafter\pgfplots@TMP\csname pgfplots@#1min\endcsname
			\ifx\pgfplots@TMP\pgfutil@empty
			\else
				\expandafter\pgfmathfloatparsenumber\expandafter{\pgfplots@TMP}%
				\expandafter\let\csname pgfplots@#1min\endcsname=\pgfmathresult
			\fi
			\expandafter\let\expandafter\pgfplots@TMP\csname pgfplots@#1max\endcsname
			\ifx\pgfplots@TMP\pgfutil@empty
			\else
				\expandafter\pgfmathfloatparsenumber\expandafter{\pgfplots@TMP}%
				\expandafter\let\csname pgfplots@#1max\endcsname=\pgfmathresult
			\fi
		\fi
	\or
		\ifpgfplots@disablelogfilter
		\else
			% any user-specified axis limits:
			\expandafter\let\expandafter\pgfplots@TMP\csname pgfplots@#1min\endcsname
			\ifx\pgfplots@TMP\pgfutil@empty
			\else
				\expandafter\pgfmathlog@\expandafter{\pgfplots@TMP}%
				\expandafter\let\csname pgfplots@#1min\endcsname=\pgfmathresult
			\fi
			\expandafter\let\expandafter\pgfplots@TMP\csname pgfplots@#1max\endcsname
			\ifx\pgfplots@TMP\pgfutil@empty
			\else
				\expandafter\pgfmathlog@\expandafter{\pgfplots@TMP}%
				\expandafter\let\csname pgfplots@#1max\endcsname=\pgfmathresult
			\fi
			%
			% any user specified axis ticks:
			\expandafter\let\expandafter\pgfplots@TMP\csname pgfplots@#1tick\endcsname
			\ifx\pgfplots@TMP\pgfutil@empty
			\else
				\expandafter\pgfplots@filter@input@ticks@with@log\expandafter{\pgfplots@TMP}%
				\expandafter\edef\csname pgfplots@#1tick\endcsname{\pgfplots@TMP}%
			\fi
			\expandafter\let\expandafter\pgfplots@TMP\csname pgfplots@extra@#1tick\endcsname
			\ifx\pgfplots@TMP\pgfutil@empty
			\else
				\expandafter\pgfplots@filter@input@ticks@with@log\expandafter{\pgfplots@TMP}%
				\expandafter\edef\csname pgfplots@extra@#1tick\endcsname{\pgfplots@TMP}%
			\fi
			%
			\expandafter\let\expandafter\pgfplots@TMP\csname pgfplots@#1filter\endcsname
			\ifx\pgfplots@TMP\pgfutil@empty
				\expandafter\def\csname addplot@log@compute@#1filter\endcsname##1\to##2{%
					\pgfmathlog@{##1}%
					\let##2=\pgfmathresult
				}%
			\else
				\expandafter\let\csname pgfplots@#1filter@orig\endcsname=\pgfplots@TMP
				\expandafter\def\csname addplot@log@compute@#1filter\endcsname##1\to##2{%
					\pgfmathlog@{##1}%
					\ifx\pgfmathresult\pgfutil@empty
						\let##2=\pgfutil@empty
					\else
						\expandafter\expandafter\csname pgfplots@#1filter@orig\endcsname\pgfmathresult\to##2\relax
					\fi
				}%
			\fi
			\expandafter\let\expandafter\pgfplots@TMP\csname addplot@log@compute@#1filter\endcsname
			\expandafter\let\csname pgfplots@#1filter\endcsname\pgfplots@TMP
		\fi
	\fi
}

\def\pgfplots@create@axis@descriptions{%
	\ifpgfplots@hide
	\else
		\ifx\pgfplots@xlabel\pgfutil@empty
		\else
			\pgfplots@show@label{x}%
		\fi
		\ifx\pgfplots@ylabel\pgfutil@empty
		\else
			\pgfplots@show@label{y}%
		\fi
	\fi
	\ifx\pgfplots@title\pgfutil@empty
	\else
		\pgfplots@show@title
	\fi
	\pgfplots@createlegend
}

% DATA TRANSFORMATION T(x) = X
%
% Input: 
%    a number in the original data range, given in **floating** point representation
% Output: 
%    a fixed point number in transformed range
%    stored in \pgfmathresult
% @see
% \pgfplots@datascaletrafo@fromfixed@x
\def\pgfplots@datascaletrafo@x#1{%
	\pgfmathfloatshift{#1}{\pgfplots@data@scale@trafo@SHIFT@x}%
	\expandafter\pgfmathfloattofixed\expandafter{\pgfmathresult}%
}

% Overloaded function.
% Input:
%     a FIXED point number instead of a floating point one.
% @see \pgfplots@datascaletrafo@x
\def\pgfplots@datascaletrafo@fromfixed@x#1{%
	\pgfmathfloatparsenumber{#1}%
	\expandafter\pgfplots@datascaletrafo@x\expandafter{\pgfmathresult}%
}

\def\pgfplots@datascaletrafo@y#1{%
	\pgfmathfloatshift{#1}{\pgfplots@data@scale@trafo@SHIFT@y}%
	\expandafter\pgfmathfloattofixed\expandafter{\pgfmathresult}%
}
\def\pgfplots@datascaletrafo@fromfixed@y#1{%
	\pgfmathfloatparsenumber{#1}%
	\expandafter\pgfplots@datascaletrafo@y\expandafter{\pgfmathresult}%
}

% INVERSE transformation x = T^{-1}( X )
% Input: 
%    a fixed point number in transformed domain
% Output:
%    a number in the data domain, given in floating point
%    representation
%
% If the input number is approximately X=0, we will return x=0 as
% well.
%
% This allows to handle rounding inaccuracies and should not pose any
% problems.
\def\pgfplots@inverse@datascaletrafo@x#1{%
	\begingroup
	\pgfmathapproxequalto{#1}{0.0}%
	\ifpgfmathcomparison
		\def\pgfmathresult{00.0e0}%
	\else
		\pgfmathfloatparsenumber{#1}%
		\edef\pgfplots@data@scale@trafo@SHIFT@x{-\pgfplots@data@scale@trafo@SHIFT@x}%
		\expandafter\pgfmathfloatshift\expandafter{\pgfmathresult}{\pgfplots@data@scale@trafo@SHIFT@x}%
	\fi
	\pgfmath@smuggleone\pgfmathresult
	\endgroup
}

\def\pgfplots@inverse@datascaletrafo@y#1{%
	\begingroup
	\pgfmathapproxequalto{#1}{0.0}%
	\ifpgfmathcomparison
		\def\pgfmathresult{00.0e0}%
	\else
		\pgfmathfloatparsenumber{#1}%
		\edef\pgfplots@data@scale@trafo@SHIFT@y{-\pgfplots@data@scale@trafo@SHIFT@y}%
		\expandafter\pgfmathfloatshift\expandafter{\pgfmathresult}{\pgfplots@data@scale@trafo@SHIFT@y}%
	\fi
	\pgfmath@smuggleone\pgfmathresult
	\endgroup
}

% Overloaded function.
%
% In contrast to \pgfplots@inverse@datascaletrafo@x, this method
% returns a number in FIXED point representation.
% @see \pgfplots@inverse@datascaletrafo@x
\def\pgfplots@inverse@datascaletrafo@tofixed@x#1{%
	\pgfplots@inverse@datascaletrafo@x{#1}%
	\expandafter\pgfmathfloattofixed\expandafter{\pgfmathresult}%
}
\def\pgfplots@inverse@datascaletrafo@tofixed@y#1{%
	\pgfplots@inverse@datascaletrafo@y{#1}%
	\expandafter\pgfmathfloattofixed\expandafter{\pgfmathresult}%
}

% Parses all options in #1 which are known in an (pgfplots) families #2.
%
% The result will be stored back into the TikZ-style named #3 without 
% further processing.
%
% Example:
% \tikzstyle{every axis}=[xmin=0,xmax=1,line width=1pt
% \pgfplots@set@keys@from@tikz@style\tmpmacro{every axis}{/pgfplots}
% 
% - sets axis options 'xmin' and 'xmax'
% - calls \tikzstyle{every axis}={line width=1pt}
% 
% I assume that this method is called within local TeX groups so
% nothing will be destroyed outside.
%
% #1:  a temporary macro name which will be used to accumulate options
%      which will be appended to the style.
%      You just need to provide the name, it will be reset properly.
%      Make sure you don't use the same name in nested calls!
% #2:  A style name.
% #3:  A comma separated sequence of families.
\long\def\pgfplots@set@keys@from@tikz@style#1#2#3{%
%\pgfplots@message{set@keys@from@tikz@style{#2}{#3}...}%
	\let#1=\pgfutil@empty
	\pgfqkeysfamilyoptionsdetailed%
		{#3}%  family list
		{\pgfkeysisdescendantof{/pgfplots}}% for keys without family
		{\pgfkeyspredicateFALSE}% for unknown keys
		{\pgfkeysappendfilterednamestomacro#1}%
		{/pgfplots}%
		{/tikz/#2}%
	\pgfplots@set@keymacro@to@style\pgfplots@rmopts{#2}%
%\pgfplots@message{set@keys@from@tikz@style{#2}{#3}... END.}%
}

% The same as \pgfplots@set@keys@from@tikz@style  but this one appends
% unmatched options to style #4.
%
% #1:  a temporary macro name which will be used to accumulate options
%      which will be appended to the style.
%      You just need to provide the name, it will be reset properly.
%      Make sure you don't use the same name in nested calls!
% #2:  A style name.
% #3:  A comma separated sequence of families.
% #4:  A style name which will be filled with unprocessed options.
\long\def\pgfplots@set@keys@from@tikz@style@append@to#1#2#3#4{%
%\pgfplots@message{axis@set@keys@from@tikz@style@append@to{#2}{#3}{#4}...}%
	\let#1=\pgfutil@empty
	\pgfqkeysfamilyoptionsdetailed%
		{#3}%
		{\pgfkeysisdescendantof{/pgfplots}}% for keys without family
		{\pgfkeyspredicateFALSE}% for unknown keys
		{\pgfkeysappendfilterednamestomacro#1}%
		{/pgfplots}%
		{/tikz/#2}%
	\pgfplots@append@keymacro@to@style\pgfplots@rmopts{#4}%
%\pgfplots@message{axis@set@keys@from@tikz@style@append@to{#2}{#3}{#4}... END.}%
}

% #1:  a temporary macro name which will be used to accumulate options
%      which will be appended to the style.
%      You just need to provide the name, it will be reset properly.
%      Make sure you don't use the same name in nested calls!
% #2:  A sequence of options.
% #3:  A comma separated sequence of families.
% #4:  A style name which will be filled with unprocessed options.
\long\def\pgfplots@set@keys@and@append@remaining@to@style#1#2#3#4{%
%\pgfplots@message{axis@set@keys@and@append@remaining@to@style{...}{#3}{#4} ...}
	\let#1=\pgfutil@empty
	\pgfqkeysfamilyoptionsdetailed%
		{#3}%
		{\pgfkeysisdescendantof{/pgfplots}}% for keys without family
		{\pgfkeyspredicateFALSE}% for unknown keys
		{\pgfkeysappendfilterednamestomacro#1}%
		{/pgfplots}%
		{#2}%
	\pgfplots@append@keymacro@to@style\pgfplots@rmopts{#4}%
%\pgfplots@message{axis@set@keys@and@append@remaining@to@style{...}{#3}{#4} ... END.}%
}%

% #1: input macro
% #2: macro for remaining options
% #3: families
\long\def\pgfplots@setkeys@from@macro#1#2#3{%
	\let#2=\pgfutil@empty
	\def\pgfplots@TMP{%
		\pgfqkeysfamilyoptionsdetailed%
			{#3}%
			{\pgfkeysisdescendantof{/pgfplots}}% for keys without family
			{\pgfkeyspredicateFALSE}% for unknown keys
			{\pgfkeysappendfilterednamestomacro#2}%
			{/pgfplots}%
	}%
	\expandafter\pgfplots@TMP\expandafter{#1}%
}

% #1: macro
% #2: style name
\long\def\pgfplots@append@keymacro@to@style#1#2{%
	\def\pgfplots@setkeys@TMP{\tikzstyle{#2}+=[}%
	\expandafter\pgfplots@setkeys@TMP#1]%
%\pgfplots@message{tikzstyle{#2}+=[#1]}%
}

% #1: macro
% #2: style name
\long\def\pgfplots@set@keymacro@to@style#1#2{%
	\def\pgfplots@setkeys@TMP{\tikzstyle{#2}=[}%
	\expandafter\pgfplots@setkeys@TMP#1]%
%\pgfplots@message{tikzstyle{#2}=[#1]}%
}

\def\pgfplots@preset@keys{%
  \def\pgfplots@at{\pgfqpoint{\the\tikz@lastx}{\the\tikz@lasty}}%
}

% backwards compatibility:
\def\prettyprintnumber#1{\pgfmathprintnumber{#1}}%

\def\pgfplots@set@options#1{%
%\pgfplots@message{SET OPTS: VOR preset}%
	\pgfplots@preset@keys
	%
	% Temporarily assign families to 'name' and 'alias' options.
	% This allows to get the names - they should not be appended to
	% 'every axis'!
	\pgfkeys{%
		/tikz/domain/.belongs to family=/pgfplots,
		/pgfplots/domain/.code={\pgfkeysalso{/tikz/domain=##1}},
		/tikz/name/.belongs to family=/pgfplots/naming commands,
		/tikz/alias/.belongs to family=/pgfplots/naming commands,
		% and provide aliases in the '/pgfplots/' tree to avoid 
		% search path problems just for these two options:
		/pgfplots/name/.belongs to family=/pgfplots/naming commands,
		/pgfplots/alias/.belongs to family=/pgfplots/naming commands,
		/pgfplots/name/.code={\pgfkeysalso{/tikz/name=##1}},
		/pgfplots/alias/.code={\pgfkeysalso{/tikz/alias=##1}},
	}%
	\let\tikz@alias=\pgfutil@empty
	\let\tikz@fig@name=\pgfutil@empty
	%
%\pgfplots@message{SET OPTS: VOR mode}%
	% Step 1: acquire ONLY 'xmode' and 'ymode' (necessary to decide
	% which axis style shall be loaded):
	\let\pgfplots@remaining@input=\pgfutil@empty
	\pgfqkeysfamilyoptionsNone{/pgfplots/scale}%
		{\pgfkeysappendfilterednamestomacro\pgfplots@remaining@input}
		{/pgfplots}
		{#1}%
%\pgfplots@message{SET OPTS: VOR styles}%
	%
	% Step 2: parse any pgfplots options out of styles.
	\pgfplots@set@keys@from@tikz@style{\pgfplots@rmopts}{every axis}{/pgfplots,/pgfplots/style commands,/pgfplots/naming commands,/pgfplots/tick,/pgfplots/legend,/pgfplots/descriptions,/pgfplots/scale}%
	\ifcase\pgfplots@xmode
		\ifcase\pgfplots@ymode
			\pgfplots@set@keys@from@tikz@style@append@to{\pgfplots@rmopts}{every linear axis}{/pgfplots,/pgfplots/style commands,/pgfplots/naming commands,/pgfplots/tick,/pgfplots/legend,/pgfplots/descriptions}{every axis}%
		\or
			\pgfplots@set@keys@from@tikz@style@append@to{\pgfplots@rmopts}{every semilogy axis}{/pgfplots,/pgfplots/style commands,/pgfplots/naming commands,/pgfplots/tick,/pgfplots/legend,/pgfplots/descriptions}{every axis}%
		\fi
	\or
		\ifcase\pgfplots@ymode
			\pgfplots@set@keys@from@tikz@style@append@to{\pgfplots@rmopts}{every semilogx axis}{/pgfplots,/pgfplots/style commands,/pgfplots/naming commands,/pgfplots/tick,/pgfplots/legend,/pgfplots/descriptions}{every axis}%
		\or
			\pgfplots@set@keys@from@tikz@style@append@to{\pgfplots@rmopts}{every loglog axis}{/pgfplots,/pgfplots/style commands,/pgfplots/naming commands,/pgfplots/tick,/pgfplots/legend,/pgfplots/descriptions}{every axis}%
		\fi
	\fi
	% Acquire the style commands from direct input options BEFORE styles are processed:
\message{remaining after scale:}%
\show\pgfplots@remaining@input
	\pgfplots@setkeys@from@macro\pgfplots@remaining@input\pgfplots@rmopts{/pgfplots/style commands,/pgfplots/naming commands}%
\message{remaining:}%
\show\pgfplots@rmopts
	\let\pgfplots@remaining@input=\pgfplots@rmopts
	%
	% Now, any 'name' and 'alias' options have been processed. 
	%
	% Remember their current meaning and reset the tikz options!
	\let\pgfplots@fig@name=\tikz@fig@name
	\let\pgfplots@fig@alias=\tikz@alias
	\let\tikz@alias=\pgfutil@empty
	\let\tikz@fig@name=\pgfutil@empty
	%
	% And protocol all named sub-nodes! Their positions need to be
	% updated later.
	\let\pgfplots@named@child@node@list=\pgfutil@empty
	\pgfkeysgetvalue{/tikz/name/.@cmd}\pgfplots@old@name@impl
	\pgfkeysgetvalue{/tikz/alias/.@cmd}\pgfplots@old@alias@impl
	\pgfkeysdef{/tikz/name}{%
		\xdef\pgfplots@named@child@node@list{\pgfplots@named@child@node@list,{##1}}%
		\pgfplots@old@name@impl##1\pgfeov
	}%
	\pgfkeysdef{/tikz/alias}{%
		\xdef\pgfplots@named@child@node@list{\pgfplots@named@child@node@list,{##1}}%
		\pgfplots@old@alias@impl##1\pgfeov
	}%
	%
	\pgfplots@set@keys@from@tikz@style{\pgfplots@rmopts}{every axis legend}{/pgfplots/legend}%
\message{rmopts fuer every axis legend:}%
\show\pgfplots@rmopts
	\pgfplots@set@keys@from@tikz@style{\pgfplots@rmopts}{every axis plot}{/pgfplots}%
	\pgfplots@set@keys@from@tikz@style{\pgfplots@rmopts}{every axis label}{/pgfplots/descriptions}%
	\pgfplots@set@keys@from@tikz@style{\pgfplots@rmopts}{every axis x label}{/pgfplots/descriptions}%
	\pgfplots@set@keys@from@tikz@style{\pgfplots@rmopts}{every axis y label}{/pgfplots/descriptions}%
	\pgfplots@set@keys@from@tikz@style{\pgfplots@rmopts}{every axis title}{/pgfplots/descriptions}%
	\pgfplots@set@keys@from@tikz@style{\pgfplots@rmopts}{every tick}{/pgfplots/tick}%
	\pgfplots@set@keys@from@tikz@style{\pgfplots@rmopts}{every minor tick}{/pgfplots/tick}%
	\pgfplots@set@keys@from@tikz@style{\pgfplots@rmopts}{every major tick}{/pgfplots/tick}%
	\pgfplots@set@keys@from@tikz@style{\pgfplots@rmopts}{every axis grid}{/pgfplots/tick}%
	\pgfplots@set@keys@from@tikz@style{\pgfplots@rmopts}{every minor grid}{/pgfplots/tick}%
	\pgfplots@set@keys@from@tikz@style{\pgfplots@rmopts}{every major grid}{/pgfplots/tick}%
	\pgfplots@set@keys@from@tikz@style{\pgfplots@rmopts}{every x tick label}{/pgfplots/descriptions,/pgfplots/tick}%
	\pgfplots@set@keys@from@tikz@style{\pgfplots@rmopts}{every y tick label}{/pgfplots/descriptions,/pgfplots/tick}%
	\pgfplots@set@keys@from@tikz@style{\pgfplots@rmopts}{every tick label}{/pgfplots/descriptions,/pgfplots/tick}%
	%
%\pgfplots@message{SET OPTS: VOR eigentlichen opts}%
	\expandafter\pgfplots@set@keys@and@append@remaining@to@style\expandafter\pgfplots@rmopts\expandafter{\pgfplots@remaining@input}%
		{/pgfplots,/pgfplots/tick,/pgfplots/legend,/pgfplots/descriptions}%
		{every axis}%
%\pgfplots@message{setopts OK}%
}

\def\pgfplots@install@abbrev@commands{
	\let\pgfplots@orig@path=\path
	\let\pgfplots@orig@plot=\plot
	%
	\let\axispath=\pgfplots@path
	%
	\let\addplot=\pgfplots@addplot
	\let\plot=\addplot
	%
	\def\logten{2.3025851}%
	\def\reciproclogten{0.434294}%
	%
	\def\logi##1{%
		\ifcase##1
		\or0
		\or0.693147
		\or1.098612
		\or1.386294
		\or1.60943791
		\or1.7917594
		\or1.94591014
		\or2.07944154
		\or2.197224
		\fi
	}%
	%
	\let\legend=\pgfplots@command@legend
	\let\addlegendentry=\pgfplots@addlegendentry
}

\def\pgfplots@environment{%
	\pgfutil@ifnextchar[{%
		\pgfplots@environment@opt
	}{%
		\pgfplots@environment@opt[]%
	}%
}%
\def\pgfplots@environment@opt[#1]{%
	\begingroup
	\pgfplots@install@abbrev@commands
	%
	%
	% The explicit specification of 'x' and 'y' as 1pt is to avoid
	% numeric overflow/underflow during scale computations:
	%
	% The scaling (i.e. proper values for 'x' and 'y') will be
	% determined later-on, dependend on the axis limits.  Since axis
	% limits are implicitly in units of 1pt, it is reasonable to use
	% '1pt' here as well.
	\pgfsetxvec{\pgfpoint{1pt}{0pt}}%
	\pgfsetyvec{\pgfpoint{0pt}{1pt}}%
	%
	\pgfplots@set@options{#1}%
	%
	% --------------------
	% Allocations:
	% --------------------
	\pgfplotslistnewempty\pgfplots@plotspeclist
	\pgfplotslistnewempty\pgfplots@legend
	\pgfplotslistnewempty\pgfplots@stored@plot@cmds
	\pgfplotslistnewempty\pgfplots@stored@plot@data
	\newif\ifpgfplots@autocompute@xlim
	\newif\ifpgfplots@autocompute@ylim
	\newif\ifpgfplots@filtercoords
	\newif\ifpgfplots@float@numerics@mode@x
	\newif\ifpgfplots@float@numerics@mode@y
	\newif\ifpgfplots@float@numerics@mode
	\newif\ifpgfplots@draw@at@end
	\newif\ifpgfplots@limits@are@computed
	\newif\ifpgfplots@EMERGENCY@FORCE@DATA@TRAFO@TO@IDENTITY
	\pgfplots@numplots=0
	\let\pgfplots@already@computed@legend@node=\pgfutil@empty
	%
	% --------------------
	% Option preprocessing
	% --------------------
	\pgfplots@prepare@coord@filtering@for x
	\pgfplots@prepare@coord@filtering@for y
	%
	\ifx\pgfplots@xfilter\pgfutil@empty
	\else
		\pgfplots@filtercoordstrue
	\fi
	\ifx\pgfplots@yfilter\pgfutil@empty
	\else
		\pgfplots@filtercoordstrue
	\fi
	% 
	\ifx\pgfplots@xmin\pgfutil@empty
		\pgfplots@autocompute@xlimtrue
		\def\pgfplots@xmin{16300}%
		\ifpgfplots@float@numerics@mode@x
			\def\pgfplots@xmin{11.0e324}%
		\fi
	\fi
	\ifx\pgfplots@xmax\pgfutil@empty
		\pgfplots@autocompute@xlimtrue
		\def\pgfplots@xmax{-16300}%
		\ifpgfplots@float@numerics@mode@x
			\def\pgfplots@xmax{21.0e324}%
		\fi
	\fi
	\ifx\pgfplots@ymin\pgfutil@empty
		\pgfplots@autocompute@ylimtrue
		\def\pgfplots@ymin{16300}%
		\ifpgfplots@float@numerics@mode@y
			\def\pgfplots@ymin{11.0e324}%
		\fi
	\fi
	\ifx\pgfplots@ymax\pgfutil@empty
		\pgfplots@autocompute@ylimtrue
		\def\pgfplots@ymax{-16300}%
		\ifpgfplots@float@numerics@mode@y
			\def\pgfplots@ymax{21.0e324}%
		\fi
	\fi
	%
	\pgfplots@limits@are@computedtrue
	\ifpgfplots@autocompute@xlim
		\pgfplots@draw@at@endtrue
		\pgfplots@limits@are@computedfalse
	\fi
	\ifpgfplots@autocompute@ylim
		\pgfplots@draw@at@endtrue
		\pgfplots@limits@are@computedfalse
	\fi
	%
	%
	% --------------------
	% Start axis:
	% either postponed to \end or directly.
	% --------------------
	%
	% ALWAYS TRUE. This allows to use inline plots because the data
	% scale transformation will be applied after I know that it should
	% be disabled.
	\pgfplots@draw@at@endtrue
	%
	% any \path command is invalid inside of an axis.
	% Use \axispath instead:
	\let\path=\pgfplots@replacement@for@tikz@path
	%
	\ifpgfplots@draw@at@end
		\let\pgfplots@nextcommand=\pgfutil@empty
	\else
		\def\pgfplots@nextcommand{\pgfplots@BEGIN@init@and@draw@axis}%
	\fi
	\pgfplots@nextcommand
}
\def\endpgfplots@environment@opt{%
	% restore old \path command:
	\let\path=\pgfplots@orig@path
	\let\plot=\pgfplots@orig@plot
	%
	%\end{axis}:
	% --------------------
	%  All plotting commands have been read.
	%  -> apply postponed drawing commands!
	% --------------------
	\ifpgfplots@draw@at@end
		\def\pgfplots@nextcommand{%
			\pgfplots@BEGIN@init@and@draw@axis
			\pgfplotslistforeach\pgfplots@stored@plot@cmds\as\plotcmd{%
				\pgfplotslistpopfront\pgfplots@stored@plot@data\to\plotdata
				\ifpgfplots@float@numerics@mode
					\ifx\plotcmd\pgfutil@empty
						\plotdata
					\else
						% Apply the data scaling transformation
						\expandafter\pgfplots@apply@data@scaletrafo@to@plot@data\plotdata\to\plotdata
						\expandafter\plotcmd\plotdata
					\fi
				\else
					\expandafter\plotcmd\plotdata
				\fi
			}%
		}%
	\else
		\let\pgfplots@nextcommand=\pgfutil@empty
	\fi
	\pgfplots@nextcommand
	\pgfplots@END@init@and@draw@axis
	%
	\begingroup
		% set coordinate system to (0,0) rectangle (1,1) for descriptions:
		\pgftransformxshift{\pgfplots@xcoordminTEX}%
		\pgftransformyshift{\pgfplots@ycoordminTEX}%
		%
		\pgfplots@tmpa=\pgfplots@xcoordmaxTEX
		\advance\pgfplots@tmpa by-\pgfplots@xcoordminTEX
		\pgfsetxvec{\pgfpoint{\pgfplots@tmpa}{0cm}}%
		%
		\pgfplots@tmpa=\pgfplots@ycoordmaxTEX
		\advance\pgfplots@tmpa by-\pgfplots@ycoordminTEX
		\pgfsetyvec{\pgfpoint{0cm}{\pgfplots@tmpa}}%
		%
		\pgfplots@create@axis@descriptions
	\endgroup
%	\pgfsys@endpicture
	\endtikzpicture%
	\begingroup
		% Protocol sizes for the axis-shape.
		% I fear that needs to be done globally, do avoid all those
		% \endgroup's in and after \endpgfinterruptpicture ...
		\ifdim\pgf@picmaxx=-16000pt\relax%
			\pgf@picmaxx=0pt\relax%
			\pgf@picminx=0pt\relax%
			\pgf@picmaxy=0pt\relax%
			\pgf@picminy=0pt\relax%
		\fi%
		%
		\xdef\pgfplots@saveddimen@picminx{\the\pgf@picminx}%
		\xdef\pgfplots@saveddimen@picminy{\the\pgf@picminy}%
		%
		\pgf@xa=\pgfplots@xcoordmaxTEX
		\advance\pgf@xa by-\pgf@picminx
		\xdef\pgfplots@savedanchor@inner@upperright@x{\the\pgf@xa}%
		%
		\pgf@xa=\pgfplots@xcoordminTEX
		\advance\pgf@xa by-\pgf@picminx
		\xdef\pgfplots@savedanchor@inner@lowerleft@x{\the\pgf@xa}%
		%
		\pgf@xa=\pgfplots@ycoordmaxTEX
		\advance\pgf@xa by-\pgf@picminy
		\xdef\pgfplots@savedanchor@inner@upperright@y{\the\pgf@xa}%
		%
		\pgf@xa=\pgfplots@ycoordminTEX
		\advance\pgf@xa by-\pgf@picminy
		\xdef\pgfplots@savedanchor@inner@lowerleft@y{\the\pgf@xa}%
	\endgroup
	\endpgfinterruptpicture
	\egroup\egroup% end of pgfnodepartimagebox
	%
	\let\tikz@fig@name=\pgfplots@fig@name
	\tikz@fig@mustbenamed
    \pgftransformshift{\pgfplots@at}%
	\pgfmultipartnode{pgfplots@low@level@shape}{\pgfplots@anchorname}{\tikz@fig@name}{\pgfusepath{discard}}%
	\pgfplots@fig@alias
	%
	\pgfplots@finally@correct@child@node@positions
	\endgroup
}

% Now, we need to process all named nodes inside of our
% axis-image.
%
% The situation at this point is as follows:
% 1. the complete axis image has been "typeset" into a box. That
% means its coordinate system is LOST up to those variables
% which have been saved explicitly.
%
% 2. the \pgfmultipartnode above knows about all axis anchors and
% saved dimensions.
%
% 3. All sub-nodes don't know about their position any more. Any
% saved anchors are wrong.
%
% The approach:
% 1. we shift each named node's saved anchors such that it's
% coordinate is valid inside of the TeX box.
%
% 2. we also shift each named node's saved anchors to reflect the
% axis anchor.
%
% Afterwards, everything should be fine.
\def\pgfplots@finally@correct@child@node@positions{%
   \ifx\pgfplots@named@child@node@list\pgfutil@empty%
   \else%
      	\begingroup
		\pgftransformreset% FIXME: what's that for!? Copied from matrix code...
		%
		% Use the 'image' anchor here - the internal anchor
		% transformation matrix already has the shift for
		% \pgfplots@anchorname.
		\pgfpointanchor{\tikz@fig@name}{image}%
		\pgf@xa=\pgf@x
		\pgf@xb=\pgf@y
		\pgf@process{\pgfpoint{\pgfplots@saveddimen@picminx}{\pgfplots@saveddimen@picminy}}%
		\advance\pgf@xa by-\pgf@x
		\advance\pgf@xb by-\pgf@y
		\pgf@x=\pgf@xa
		\pgf@y=\pgf@xb
		\edef\pgfplots@offset{\noexpand\pgfqpoint{\the\pgf@x}{\the\pgf@y}}%
		%
		\pgfutil@for\pgfplots@child@node@name:=\pgfplots@named@child@node@list\do{%
			\ifx\pgfplots@child@node@name\pgfutil@empty
			\else
				\expandafter\ifx\csname pgfplots@child@node@visited@\pgfplots@child@node@name\endcsname\relax%
					\pgfutil@ifundefined{pgf@sh@nt@\pgfplots@child@node@name}{%
						\pgfplots@warning{Package pgfplots WARNING: could not adjust coordinates of named node '\pgfplots@child@node@name' for reasons I do not understand! After finishing the image, it did no longer exist!? Sorry.}%
					}{%
						\pgf@shift@node{\pgfplots@child@node@name}{\pgfplots@offset}%
						\expandafter\let\csname pgfplots@child@node@visited@\pgfplots@child@node@name\endcsname=\pgfutil@empty%
					}%
				\fi
			\fi
		}%
		\endgroup
    \fi%
}%

\def\pgfplots@environment@axis{%
	\pgfutil@ifnextchar[{\pgfplots@@environment@axis}{\pgfplots@@environment@axis[]}%
}
\let\endpgfplots@environment@axis=\endpgfplots@environment@opt
\def\pgfplots@@environment@axis[#1]{%
	\pgfplots@environment@opt[/pgfplots/xmode=linear,/pgfplots/ymode=linear,#1]%
}

\def\pgfplots@environment@semilogxaxis{%
	\pgfutil@ifnextchar[{\pgfplots@@environment@semilogxaxis}{\pgfplots@@environment@semilogxaxis[]}%
}
\let\endpgfplots@environment@semilogxaxis=\endpgfplots@environment@opt
\def\pgfplots@@environment@semilogxaxis[#1]{%
	\pgfplots@environment@opt[/pgfplots/xmode=log,/pgfplots/ymode=linear,#1]%
}

\def\pgfplots@environment@semilogyaxis{%
	\pgfutil@ifnextchar[{\pgfplots@@environment@semilogyaxis}{\pgfplots@@environment@semilogyaxis[]}%
}
\let\endpgfplots@environment@semilogyaxis=\endpgfplots@environment@opt
\def\pgfplots@@environment@semilogyaxis[#1]{%
	\pgfplots@environment@opt[/pgfplots/xmode=linear,/pgfplots/ymode=log,#1]%
}

\def\pgfplots@environment@loglogaxis{%
	\pgfutil@ifnextchar[{\pgfplots@@environment@loglogaxis}{\pgfplots@@environment@loglogaxis[]}%
}
\let\endpgfplots@environment@loglogaxis=\endpgfplots@environment@opt
\def\pgfplots@@environment@loglogaxis[#1]{%
	\pgfplots@environment@opt[/pgfplots/xmode=log,/pgfplots/ymode=log,#1]%
}


\let\pgfplots@ORIG@tikz@installcommands=\tikz@installcommands
\let\pgfplots@ORIG@tikz@uninstallcommands=\tikz@uninstallcommands

% this command here is installed at the beginning of every
% tikzpicture.
%
% It installs abbreviated commands - and I use it for the same
% purpose.
\def\tikz@installcommands{%
	\pgfplots@ORIG@tikz@installcommands
	%
	\let\axis=\pgfplots@environment@axis
	\let\endaxis=\endpgfplots@environment@axis
	%
	\let\semilogxaxis=\pgfplots@environment@semilogxaxis
	\let\endsemilogxaxis=\endpgfplots@environment@semilogxaxis
	%
	\let\semilogyaxis=\pgfplots@environment@semilogyaxis
	\let\endsemilogyaxis=\endpgfplots@environment@semilogyaxis
	%
	\let\loglogaxis=\pgfplots@environment@loglogaxis
	\let\endloglogaxis=\endpgfplots@environment@loglogaxis
}%
