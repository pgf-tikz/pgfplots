%--------------------------------------------
% $Header: /cvsroot/pgfplots/pgfplots/generic/pgfplots/libs/pgflibrarysurfshading.code.tex,v 1.3 2009/02/27 22:39:04 ludewich Exp $
%
% Package pgfplots
%
% Provides a user-friendly interface to create function plots (normal
% plots, semi-logplots and double-logplots).
% 
% It is based on Till Tantau's PGF package.
%
% Copyright 2007/2008 by Christian Feuersänger.
%
% This program is free software: you can redistribute it and/or modify
% it under the terms of the GNU General Public License as published by
% the Free Software Foundation, either version 3 of the License, or
% (at your option) any later version.
% 
% This program is distributed in the hope that it will be useful,
% but WITHOUT ANY WARRANTY; without even the implied warranty of
% MERCHANTABILITY or FITNESS FOR A PARTICULAR PURPOSE.  See the
% GNU General Public License for more details.
% 
% You should have received a copy of the GNU General Public License
% along with this program.  If not, see <http://www.gnu.org/licenses/>.
%
%--------------------------------------------

%--------------------------------------------------
% \def\pgfplotslibrarysurf@cols{2}
% \pgfplotslibrarysurfstreamstart
% \pgfplotslibrarysurfstreamcoord{\pgfqpoint{0pt}{10pt}}{0}
% \pgfplotslibrarysurfstreamcoord{\pgfqpoint{100pt}{10pt}}{100}
% \pgfplotslibrarysurfstreamcoord{\pgfqpoint{0pt}{300pt}}{300}
% \pgfplotslibrarysurfstreamcoord{\pgfqpoint{150pt}{300pt}}{1000}
% \pgfplotslibrarysurfstreamend
% \pgfplotslibrarysurfdraw
%-------------------------------------------------- 



\def\pgfplotslibrarysurf@width{15cm}%
\def\pgfplotslibrarysurf@height{15cm}%
\def\pgfplotslibrarysurf@cols{5}%
\def\pgfplotslibrarysurf@binarystream{1234}%
\def\pgfplotslibrarysurf@decode{-16348.9 16348.9 -16348.9 16348.9 0 10000}%

\def\pgfplotslibrarysurfstreamstart{%
	\pgfplotsapplistXnewempty\pgfplotslibrarysurf@binarystream@accum
}%
\def\pgfplotslibrarysurfstreamend{%
	\pgfplotsapplistXlet\pgfplotslibrarysurf@binarystream=\pgfplotslibrarysurf@binarystream@accum
}%

% #1: a pgf point.
% #2: a color coordinate in the range [0,1000]
\def\pgfplotslibrarysurfstreamcoord#1#2{%
	\pgfkeysvalueof{/pgfplots/bin/bytes/.@cmd}4\pgfeov
	\pgf@process{#1}%
	% The idea is to map 
	% the low-level point coordinates LINEARLY into [0,2^{8*<bytes>}].
	%
	% This is what the pdf standard expects for surface shadings.
	%
	% To do that, we simply map
	% [-16384,16384] linearly into [0,2^{32}]
	% and write the resulting integer in big endian binary format to
	% the pdf low level stream.
	%
	% The decode procedure tells the pdf viewer how to invert that
	% stuff.
	%
	%% DEBUG NOTE: This mapping appears to work correctly according to
	%% my tests.
	%% For bc -l test codes: 
	%% set ibase=16; 
	%% -4000 +  809658FA. / (2^20) * 8000
	%
	% FIXME : these coordinates ARE NOT USER SPACE CORRECTED.
	% SO: either use them as fill path for the surfaces rectangle
	% (which may work, I don't know) or make sure they are mapped
	% correctly into "user space".
	\pgf@sys@bp@correct\pgf@x%
\message{ENCODING x  = \the\pgf@x.}%
	\pgfplotsbinaryencodedimenmaplinearly\pgf@x
	\t@pgfplots@toka=\expandafter{\pgfplotsbinaryresult}%
	\pgf@sys@bp@correct\pgf@y%
\message{ENCODING y  = \the\pgf@y.}%
	\pgfplotsbinaryencodedimenmaplinearly\pgf@y
	\t@pgfplots@tokb=\expandafter{\pgfplotsbinaryresult}%
	{%
		\pgf@xa=#2pt
		\multiply\pgf@xa by10
		\afterassignment\pgfplotslibrarysurf@gobbletillrelax
		\expandafter\c@pgf@counta\the\pgf@xa\relax
		\xdef\pgfplots@glob@TMPa{\the\c@pgf@counta}%
	}%
\message{ENCODING C = \pgfplots@glob@TMPa.}%
	\pgfkeysvalueof{/pgfplots/bin/bytes/.@cmd}2\pgfeov
	\pgfplotsbinaryencodeunsigned\pgfplots@glob@TMPa
	\t@pgfplots@tokc=\expandafter{\pgfplotsbinaryresult}%
	\edef\pgfplots@loc@TMPa{\the\t@pgfplots@toka\the\t@pgfplots@tokb\the\t@pgfplots@tokc}%
	\expandafter\pgfplotsapplistXpushback\pgfplots@loc@TMPa\to\pgfplotslibrarysurf@binarystream@accum
}
\def\pgfplotslibrarysurf@gobbletillrelax#1\relax{}%

\def\pgfplotslibrarysurfdraw{%
%	\pgftext[at=\pgfqpoint{0pt}{0pt}]%
		{\pgfplotssys@do@surfshading}%
}%

\def\pgfplotssys@do@surfshading{%
	{%
		\pdfobj stream attr {%
			% stream length will be computed automatically
			%/Shading 
			/ShadingType 5
			/BitsPerCoordinate 32
			/BitsPerComponent 16
			/VerticesPerRow \pgfplotslibrarysurf@cols\space
			/ColorSpace /DeviceRGB
			/Decode [\pgfplotslibrarysurf@decode]
			/Function 
			<< 
				/FunctionType 3 
				/Domain [0.0 10000] 
				/Functions [ 
					<< 
						/FunctionType 2 
						/Domain [0.0 10000] 
						/C0 [0 0 1] /C1 [1 1 0] /N 1 
					>>  
					<< 
						/FunctionType 2 
						/Domain [0.0 10000] 
						/C0 [1 1 0] /C1 [1 0.5 0] /N 1 
					>>  
					<< 
						/FunctionType 2 
						/Domain [0.0 10000] 
						/C0 [1 0.5 0] /C1 [1 0 0] /N 1 
					>>
				]
				/Bounds [ 2000 6000 ] 
				/Encode [0 1  0 1 0 1] 
			>> 
		} {%
			\pgfplotslibrarysurf@binarystream
		}%
		%--------------------------------------------------
		% \pdfobj {%
		% 	/TemporaryShade % this can be deleted. I use it to check what works - substitute /Plotsurface with /TemporaryShade to activate it.
		% 	<<
		% 		/ShadingType 2 
		% 		/ColorSpace /DeviceRGB
		% 		/Domain [0.0 226.76788] 
		% 		/Coords [0.0 0 226.76788 0] 
		% 		/Function 
		% 		<< 
		% 			/FunctionType 3 
		% 			/Domain [0.0 226.76788] 
		% 			/Functions [ 
		% 				<< 
		% 					/FunctionType 2 
		% 					/Domain [0.0 226.76788] 
		% 					/C0 [0 0 1] /C1 [1 1 0] /N 1 
		% 				>>  
		% 				<< 
		% 					/FunctionType 2 
		% 					/Domain [0.0 226.76788] 
		% 					/C0 [1 1 0] /C1 [1 0.5 0] /N 1 
		% 				>>  
		% 				<< 
		% 					/FunctionType 2 
		% 					/Domain [0.0 226.76788] 
		% 					/C0 [1 0.5 0] /C1 [1 0 0] /N 1 
		% 				>>
		% 			]
		% 			/Bounds [ 75.5893 151.17859] 
		% 			/Encode [0 1  0 1 0 1] 
		% 		>> 
		% 		/Extend [false false] 
		% 	>> 
		% }%
		%-------------------------------------------------- 
		% I don't know how to insert the shading into a properly sized PDF entity - this here appears to work.
		\setbox\pgfutil@tempboxa=\hbox to\pgfplotslibrarysurf@width{%
			\vbox to\pgfplotslibrarysurf@height{%
				\vfil
				\pgfsys@invoke{%
					/Plotsurface sh% invoke 'sh'ading operator on /Plotsurface
				}%
			}%
			\hfil}%
		\pdfxform resources {%
			/Shading << /Plotsurface \the\pdflastobj\space 0 R >>
		}\pgfutil@tempboxa
		%--------------------------------------------------
		% \pgf@process{\pgfpoint{\pgf@max}{#2}}%
		% \pdfxform resources {%
		% 	/Shading << 
		% 		/Sh << 
		% /Length <BYTES>
		% 			/ShadingType 5
		% 			/BitsPerCoordinate 32
		% 			/BitsPerComponent 16
		% 			/VerticesPerRow \pgfplotslibrarysurf@cols
		% 			/ColorSpace /DeviceRGB
		% 			/Decode [xmin xmax ymin ymax c1min c1max c2min c2max c3min c3max ]
		% 		>> 
		% 	>>}\pgfutil@tempboxa% <<
		%-------------------------------------------------- 
		\xdef\pgfplots@glob@TMPa{
			\leavevmode
			\noexpand\pdfrefobj\the\pdflastobj
			\noexpand\pdfrefxform\the\pdflastxform}% 
	}%
	\pgfplots@glob@TMPa
}
