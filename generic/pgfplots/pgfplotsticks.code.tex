%--------------------------------------------
%
% Package pgfplots
%
% Provides a user-friendly interface to create function plots (normal
% plots, semi-logplots and double-logplots).
% 
% It is based on Till Tantau's PGF package.
%
% Copyright 2007/2008 by Christian Feuersänger.
%
% This program is free software: you can redistribute it and/or modify
% it under the terms of the GNU General Public License as published by
% the Free Software Foundation, either version 3 of the License, or
% (at your option) any later version.
% 
% This program is distributed in the hope that it will be useful,
% but WITHOUT ANY WARRANTY; without even the implied warranty of
% MERCHANTABILITY or FITNESS FOR A PARTICULAR PURPOSE.  See the
% GNU General Public License for more details.
% 
% You should have received a copy of the GNU General Public License
% along with this program.  If not, see <http://www.gnu.org/licenses/>.
%
%--------------------------------------------

\newif\ifpgfplots@log@tick@isminor@tick@pos
% Checks whether the tick position given as #1.#2=log10(T) belongs to
% T=i*10^j with an integer i>1.
%
% If T=i*10^j,  \ifpgfplots@log@tick@isminor@tick@pos will be set to true and
% \pgfmathresult will contain T.
%
% Otherwise, \ifpgfplots@log@tick@isminor@tick@pos will be set to false and
% pgfmathresult to #1.#2
%
% Arguments:
% #1.#2  the value log10(T)
%
% Implementation:  
% if T = i*10^j,  log10(T) = log10(i) + j.
% That means if log10(T) in \Z,  we have T = 10^j. If not, we need to
% check wether i is an integer. Please note that log10(i) < 1.
%
% Further note: log(T) < 0 <=>  j<0.
% In case j<0, we have 
%   #1.#2 = j + log(i) 
%         = - ( -j - log(i) ) 
%         = - ( -j - 1  + (1-log(i)) )
%         = #1 '.' #2 [ up to the '0.'
% that means #1 = j-1  and #2 = 1-log(i).
\def\pgfplots@is@log@tick@a@minor@tick@pos#1.#2\relax{%
	\pgfmathapproxequalto@{#1.#2}{#1.0}%
	\ifpgfmathcomparison
		% in MOST cases, this here will be true:
		\pgfplots@log@tick@isminor@tick@posfalse
		\def\pgfmathresult{#1.#2}%
	\else
		% I guess this won't happen too often. In fact, it's a very
		% special case.
		\begingroup
		\c@pgf@counta=#1\relax
		\ifnum\c@pgf@counta<0
			\advance\c@pgf@counta by-1
			\pgfmathsubtract@{1}{0.#2}%
			\expandafter\pgfplots@is@log@tick@a@minor@tick@pos@IDENTIFY@LOGi\pgfmathresult\relax
			\ifpgfplots@log@tick@isminor@tick@pos
				\aftergroup\pgfplots@log@tick@isminor@tick@postrue
				\edef\pgfmathresult{\pgfmathresult e\the\c@pgf@counta}%
			\else
			\aftergroup\pgfplots@log@tick@isminor@tick@posfalse
				\def\pgfmathresult{#1.#2}%
			\fi
		\else
			\pgfplots@is@log@tick@a@minor@tick@pos@IDENTIFY@LOGi0.#2\relax
			\ifpgfplots@log@tick@isminor@tick@pos
				\aftergroup\pgfplots@log@tick@isminor@tick@postrue
				\edef\pgfmathresult{\pgfmathresult e\the\c@pgf@counta}%
			\else
				\aftergroup\pgfplots@log@tick@isminor@tick@posfalse
				\def\pgfmathresult{#1.#2}%
			\fi
		\fi
		\pgfmath@smuggleone\pgfmathresult
		\endgroup
	\fi
}

% expects a positive number.
\def\pgfplots@is@log@tick@a@minor@tick@pos@IDENTIFY@LOGi0.#1\relax{%
	\pgfplots@log@tick@isminor@tick@postrue
	\pgfmathapproxequalto@{0.#1}{0.3010299956639}%
	\ifpgfmathcomparison
		\def\pgfmathresult{2}%
	\else
		\pgfmathapproxequalto@{0.#1}{0.4771212547196}%
		\ifpgfmathcomparison
			\def\pgfmathresult{3}%
		\else
			\pgfmathapproxequalto@{0.#1}{0.6020599913279}%
			\ifpgfmathcomparison
				\def\pgfmathresult{4}%
			\else
				\pgfmathapproxequalto@{0.#1}{0.698970004}%
				\ifpgfmathcomparison
					\def\pgfmathresult{5}%
				\else
					\pgfmathapproxequalto@{0.#1}{0.7781512503}%
					\ifpgfmathcomparison
						\def\pgfmathresult{6}%
					\else
						\pgfmathapproxequalto@{0.#1}{0.8450980400}%
						\ifpgfmathcomparison
							\def\pgfmathresult{7}%
						\else
							\pgfmathapproxequalto@{0.#1}{0.9030899869}%
							\ifpgfmathcomparison
								\def\pgfmathresult{8}%
							\else
								\pgfmathapproxequalto@{0.#1}{0.954242509439}%
								\ifpgfmathcomparison
									\def\pgfmathresult{9}%
								\else
									\pgfplots@log@tick@isminor@tick@posfalse
								\fi
							\fi
						\fi
					\fi
				\fi
			\fi
		\fi
	\fi
}

% Checks whether we need to create a separate 'tick scale label',
% a node with ' * 10^3' on the side of the axis:
%
% PRECONDITION:
%    Axis limits for #1 are given. I need their values before any data
%    scale transformation has been applied.
%    If  
%    	\pgfplots@#1min@unscaled@as@float 
%    and
%    	\pgfplots@#1max@unscaled@as@float 
%    exist; I will use these macros.
%    Otherwise, I will use \pgfplots@#1min and \pgfplots@#1max;
%    assuming that no data scale transformation is active.
%    FIXME : does that need further attention?
\def\pgfplots@init@scaled@tick@for#1{%
	\global\def\pgfplots@glob@TMPa{0}%
	\begingroup
	\ifcase\csname pgfplots@scaled@ticks@#1@choice\endcsname
	% CASE 0 : scaled #1 ticks=false: do nothing here.
	\or
		% CASE 1 : scaled #1 ticks=true:
		%--------------------------------
		% the \pgfplots@xmin@unscaled@as@float  is set just before the data
		% scale transformation is initialised.
		%
		% The variables are empty if there is no datascale transformation.
		\expandafter\let\expandafter\pgfplots@cur@min@unscaled\csname pgfplots@#1min@unscaled@as@float\endcsname
		\expandafter\let\expandafter\pgfplots@cur@max@unscaled\csname pgfplots@#1max@unscaled@as@float\endcsname
		%
		\ifx\pgfplots@cur@min@unscaled\pgfutil@empty
			\edef\pgfplots@loc@TMPa{\csname pgfplots@#1min\endcsname}%
			\expandafter\pgfmathfloatparsenumber\expandafter{\pgfplots@loc@TMPa}%
			\let\pgfplots@cur@min@unscaled=\pgfmathresult
			\edef\pgfplots@loc@TMPa{\csname pgfplots@#1max\endcsname}%
			\expandafter\pgfmathfloatparsenumber\expandafter{\pgfplots@loc@TMPa}%
			\let\pgfplots@cur@max@unscaled=\pgfmathresult
		\fi
		%
		\expandafter\pgfmathfloat@decompose@E\pgfplots@cur@min@unscaled\relax\pgfmathfloat@a@E
		\expandafter\pgfmathfloat@decompose@E\pgfplots@cur@max@unscaled\relax\pgfmathfloat@b@E
		\ifnum\pgfmathfloat@b@E<\pgfmathfloat@a@E
			\pgfmathfloat@b@E=\pgfmathfloat@a@E
		\fi
		\xdef\pgfplots@glob@TMPa{\pgfplots@scale@ticks@above@exponent}%
		\expandafter\ifnum\pgfplots@glob@TMPa<\pgfmathfloat@b@E
			% ok, scale it:
			\multiply\pgfmathfloat@b@E by-1
			\xdef\pgfplots@glob@TMPa{\the\pgfmathfloat@b@E}%
		\else
			\xdef\pgfplots@glob@TMPa{\pgfplots@scale@ticks@below@exponent}%
			\expandafter\ifnum\pgfplots@glob@TMPa>\pgfmathfloat@b@E
				% ok, scale it:
				\multiply\pgfmathfloat@b@E by-1
				\xdef\pgfplots@glob@TMPa{\the\pgfmathfloat@b@E}%
			\else
				% no scaling necessary:
				\xdef\pgfplots@glob@TMPa{0}%
			\fi
		\fi
	\or
		% CASE 2 : scaled #1 ticks=base 10:
		%--------------------------------
		\c@pgf@counta=\csname pgfplots@scaled@ticks@#1@arg\endcsname\relax
		%\multiply\c@pgf@counta by-1
		\xdef\pgfplots@glob@TMPa{\the\c@pgf@counta}%
	\or
		% CASE 3 : scaled #1 ticks=real:
		%--------------------------------
		\pgfmathfloatparsenumber{\csname pgfplots@scaled@ticks@#1@arg\endcsname}%
		\global\let\pgfplots@glob@TMPa=\pgfmathresult
	\fi
	\endgroup
	\expandafter\let\csname pgfplots@tick@scale@#1\endcsname=\pgfplots@glob@TMPa%
}

% x-axis tick labels for #1th tick
% #1: the axis (x or y)
% #2: the value
% #3,#4: coordinates for tikz
% #5: ticknumber
\def\pgfplots@show@ticklabel#1#2(#3,#4)#5{{%
	\csname ifpgfplots@#1ticklabel@interval\endcsname
		\pgfmathparse{#3}%
		\edef\pgfplots@show@ticklabel@coord@x@new{\pgfmathresult}%
		\pgfmathparse{#4}%
		\edef\pgfplots@show@ticklabel@coord@y@new{\pgfmathresult}%
		%
		\pgfplots@show@ticklabel@{#1}{#2}%
		\let\nexttick=\tick
		\ifx\pgfplots@show@ticklabel@LASTTICK\pgfutil@empty
			% its the first call. Simply remember arguments and wait
			% for interval boundary before proceeding.
		\else
			% acquire options of first interval boundary:
			\pgfplots@show@ticklabel@LASTTICK
			% compute new node position:
			\pgfmathparse{0.5*(\csname pgfplots@show@ticklabel@coord@#1\endcsname + \csname pgfplots@show@ticklabel@coord@#1@new\endcsname)}%
			\expandafter\edef\csname pgfplots@show@ticklabel@coord@#1\endcsname{\pgfmathresult}%
			\let\ticknum=\pgfplots@show@ticklabel@num\relax%
			\let\tick=\pgfplots@show@ticklabel@tick%
			\edef\pgfplots@loc@TMPa{at (\pgfplots@show@ticklabel@coord@x,\pgfplots@show@ticklabel@coord@y)}%
			\expandafter\node\pgfplots@loc@TMPa
				{\csname pgfplots@#1ticklabel\endcsname};%
		\fi
		\xdef\pgfplots@show@ticklabel@LASTTICK{%
			\noexpand\def\noexpand\pgfplots@show@ticklabel@tick{\nexttick}%
			\noexpand\def\noexpand\pgfplots@show@ticklabel@coord@x{\pgfplots@show@ticklabel@coord@x@new}%
			\noexpand\def\noexpand\pgfplots@show@ticklabel@coord@y{\pgfplots@show@ticklabel@coord@y@new}%
			\noexpand\edef\noexpand\pgfplots@show@ticklabel@num{#5}%
		}%
	\else
		\let\ticknum=#5\relax%
		\pgfplots@show@ticklabel@{#1}{#2}%
		\node at (#3,#4) {\csname pgfplots@#1ticklabel\endcsname};%
	\fi
}}

% Defines \tick by applying any necessary math to the (possibly
% transformed) tick value #2.
%
% #1: axis (x or y)
% #2: tick value.
\def\pgfplots@show@ticklabel@#1#2{%
	\csname ifpgfplots@apply@datatrafo@#1\endcsname
		\csname pgfplots@inverse@datascaletrafo@#1\endcsname{#2}%
		\ifcase\csname pgfplots@scaled@ticks@#1@choice\endcsname
		\or
			\expandafter\pgfmathfloatshift@\expandafter{\pgfmathresult}{\csname pgfplots@tick@scale@#1\endcsname}%
		\or
			\expandafter\pgfmathfloatshift@\expandafter{\pgfmathresult}{\csname pgfplots@tick@scale@#1\endcsname}%
		\or
			\expandafter\pgfmathfloatdivide@\expandafter{\pgfmathresult}{\csname pgfplots@tick@scale@#1\endcsname}%
		\fi
		% .. and this here provides \tick as fixed point repr:
		\expandafter\pgfmathfloattofixed\expandafter{\pgfmathresult}%
		\let\tick=\pgfmathresult
	\else
		\edef\tick{#2}%
		\pgfplots@if{pgfplots@#1islinear}{%
			\ifnum\csname pgfplots@scaled@ticks@#1@choice\endcsname=0
			\else
				\pgfmathfloatparsenumber{#2}%
				\ifnum\csname pgfplots@scaled@ticks@#1@choice\endcsname=3
					\expandafter\pgfmathfloatdivide@\expandafter{\pgfmathresult}{\csname pgfplots@tick@scale@#1\endcsname}%
				\else
					\expandafter\pgfmathfloatshift@\expandafter{\pgfmathresult}{\csname pgfplots@tick@scale@#1\endcsname}%
				\fi
				\expandafter\pgfmathfloattofixed\expandafter{\pgfmathresult}%
				\let\tick=\pgfmathresult
			\fi
		}{}%
	\fi
	\pgfkeysgetvalue{/pgfplots/#1 coord inv trafo/.@cmd}\pgfplots@loc@TMPa
	\ifx\pgfplots@loc@TMPa\pgfplots@empty@command@key
	\else
		\let\pgfmathresult=\tick%
		\expandafter\pgfplots@loc@TMPa\expandafter{\tick}\pgfeov
		\let\tick=\pgfmathresult
	\fi
}%

\def\pgfplots@user@ticklabel@list@x{%
	\pgfplotslistselectorempty\ticknum\of\pgfplots@xticklabels\to\tick
	\tick
}
\def\pgfplots@user@ticklabel@list@y{%
	\pgfplotslistselectorempty\ticknum\of\pgfplots@yticklabels\to\tick
	\tick
}
\def\pgfplots@user@extra@ticklabel@list@x{%
	\pgfplotslistselectorempty\ticknum\of\pgfplots@extra@xticklabels\to\tick
	\tick
}
\def\pgfplots@user@extra@ticklabel@list@y{%
	\pgfplotslistselectorempty\ticknum\of\pgfplots@extra@yticklabels\to\tick
	\tick
}

% Check if a label does not cross the x-axis
\def\pgfplots@ytick@check@tickshow{%
	\pgfplots@tickshowtrue
	\ifnum\pgfplots@yaxislinesnum=2 %
		\ifcase\pgfplots@xaxislinesnum\relax
			\ifdim\pgfplots@tmpa=\pgfplots@ymin@reg\relax
				\pgfplots@tickshowfalse
			\fi
			\ifdim\pgfplots@tmpa=\pgfplots@ymax@reg\relax
				\pgfplots@tickshowfalse
			\fi
		\or
			\ifdim\pgfplots@tmpa=\pgfplots@ymin@reg\relax
				\pgfplots@tickshowfalse
			\fi
		\or
			\ifdim\pgfplots@tmpa=\pgfplots@logical@ZERO@y pt
				\pgfplots@tickshowfalse
			\fi
		\or
			\ifdim\pgfplots@tmpa=\pgfplots@ymax@reg\relax
				\pgfplots@tickshowfalse
			\fi
		\fi
	\fi
}

% Fills the macros
%   \pgfplots@tick@beg@a \pgfplots@tick@end@a
%   \pgfplots@tick@beg@b \pgfplots@tick@end@b
% with coordinates such that 
%   (\pgfplots@tick@beg@a,\pgfplots@tmpa) -- (\pgfplots@tick@end@a,\pgfplots@tmpa)
% produces a correct tick line. 
%
% The '@b' variant is only used in case of \pgfplots@ytickposnum = 0
%
% #1 : the current axis (x or y).
% #2 : the axis which is NOT fixed. For x tick lines, this is the y
% axis and for y tick lines, this is the x axis (#1 = x or y)
% #3 : the current tick width
%
\def\pgfplots@prepare@tick@offsets@for@#1#2#3{%
	\ifcase\csname pgfplots@#1tickposnum\endcsname\relax
		%(\pgfplots@xcoordminTEX-\pgfplots@tick@offset, \pgfplots@tmpa) -- ++( #3, 0pt)
		\edef\pgfplots@tick@beg@a{\csname pgfplots@#2min\endcsname}%
		\pgfmathsubtract@{\pgfplots@tick@beg@a}{\pgfplots@tick@offset}%
		\let\pgfplots@tick@beg@a=\pgfmathresult
		\pgfmathadd@{\pgfplots@tick@beg@a}{#3}%
		\let\pgfplots@tick@end@a=\pgfmathresult
		%
		%(\pgfplots@xcoordmaxTEX+\pgfplots@tick@offset, \pgfplots@tmpa)	-- ++(-#3, 0pt)
		\edef\pgfplots@tick@beg@b{\csname pgfplots@#2max\endcsname}%
		\pgfmathadd@{\pgfplots@tick@beg@b}{\pgfplots@tick@offset}%
		\let\pgfplots@tick@beg@b=\pgfmathresult
		\pgfmathsubtract@{\pgfplots@tick@beg@b}{#3}%
		\let\pgfplots@tick@end@b=\pgfmathresult%
	\or
		% (\pgfplots@xcoordminTEX-\pgfplots@tick@offset, \pgfplots@tmpa) -- ++( #3, 0pt);
		\edef\pgfplots@tick@beg@a{\csname pgfplots@#2min\endcsname}%
		\pgfmathsubtract@{\pgfplots@tick@beg@a}{\pgfplots@tick@offset}%
		\let\pgfplots@tick@beg@a=\pgfmathresult
		\pgfmathadd@{\pgfplots@tick@beg@a}{#3}%
		\let\pgfplots@tick@end@a=\pgfmathresult
	\or
		% (\pgfplots@ZERO@x      -\pgfplots@tick@offset, \pgfplots@tmpa) -- ++( #3, 0pt);
		\pgfmathsubtract@{\csname pgfplots@logical@ZERO@#2\endcsname}{\pgfplots@tick@offset}%
		\let\pgfplots@tick@beg@a=\pgfmathresult
		\pgfmathadd@{\pgfplots@tick@beg@a}{#3}%
		\let\pgfplots@tick@end@a=\pgfmathresult%
	\or
		% (\pgfplots@xcoordmaxTEX+\pgfplots@tick@offset, \pgfplots@tmpa)	-- ++(-#3, 0pt);
		\edef\pgfplots@tick@beg@a{\csname pgfplots@#2max\endcsname}%
		\pgfmathadd@{\pgfplots@tick@beg@a}{\pgfplots@tick@offset}%
		\let\pgfplots@tick@beg@a=\pgfmathresult
		\pgfmathsubtract@{\pgfplots@tick@beg@a}{#3}%
		\let\pgfplots@tick@end@a=\pgfmathresult%
	\fi
}%

\def\pgfplots@drawyaxis@placecomputedtick{%
	\pgfpathmoveto{\pgfqpointxy{\pgfplots@tick@beg@a}{\pgfplots@curtickpos}}%
	\pgfpathlineto{\pgfqpointxy{\pgfplots@tick@end@a}{\pgfplots@curtickpos}}%
	\ifnum\pgfplots@ytickposnum=0\relax
		\pgfpathmoveto{\pgfqpointxy{\pgfplots@tick@beg@b}{\pgfplots@curtickpos}}%
		\pgfpathlineto{\pgfqpointxy{\pgfplots@tick@end@b}{\pgfplots@curtickpos}}%
	\fi
}%
\def\pgfplots@drawyaxis@placecomputedgridline{%
	\pgfpathmoveto{\pgfqpointxy{\pgfplots@xmin}{\pgfplots@curgridpos}}%
	\pgfpathlineto{\pgfqpointxy{\pgfplots@xmax}{\pgfplots@curgridpos}}%
}%

\newif\ifpgfplots@needsminorloop

\def\pgfplots@draw@tick@scale@label@for#1{%
	\csname ifpgfplots@#1islinear\endcsname
		\ifnum\csname pgfplots@scaled@ticks@#1@choice\endcsname=0
			\global\let\pgfplots@glob@TMPa=\pgfutil@empty
		\else
			\begingroup
				\xdef\pgfplots@glob@TMPa{\csname pgfplots@tick@scale@#1\endcsname}%
				\ifnum\csname pgfplots@scaled@ticks@#1@choice\endcsname=3
				\else
					\expandafter\c@pgf@counta\pgfplots@glob@TMPa\relax
					\multiply\c@pgf@counta by-1
					\ifnum\c@pgf@counta=0\relax
						\global\let\pgfplots@glob@TMPa=\pgfutil@empty
					\else
						\xdef\pgfplots@glob@TMPa{\the\c@pgf@counta}%
					\fi
				\fi
			\endgroup
		\fi
		\ifx\pgfplots@glob@TMPa\pgfutil@empty
		\else
			\edef\pgfplots@tick@scale@labels{\noexpand\pgfplots@invoke@pgfkeyscode{/pgfplots/#1tick scale label code/.@cmd}{\pgfplots@glob@TMPa}}%
			\node 
				[%
				xshift=\pgfplots@xcoordminTEX,%
				yshift=\pgfplots@ycoordminTEX,%
				x=\pgfplots@xcoordmaxTEX-\pgfplots@xcoordminTEX,% FIXME WRONG!
				y=\pgfplots@ycoordmaxTEX-\pgfplots@ycoordminTEX,%
				/pgfplots/every tick label,%
				/pgfplots/every #1 tick label,%
				/pgfplots/every #1 tick scale label]
			{\pgfplots@tick@scale@labels};
		\fi
	\fi
}

% Check if the current tick position, stored in \pgfplots@tmpa, 
% does not cross the y-axis.
%
% This is just a special case for centered axis lines.
\def\pgfplots@xtick@check@tickshow{%
	\pgfplots@tickshowtrue
	\ifnum\pgfplots@xaxislinesnum=2 %
		\ifcase\pgfplots@yaxislinesnum\relax
			\ifdim\pgfplots@tmpa=\pgfplots@xmin@reg\relax
				\pgfplots@tickshowfalse
			\fi
			\ifdim\pgfplots@tmpa=\pgfplots@xmax@reg\relax
				\pgfplots@tickshowfalse
			\fi
		\or
			\ifdim\pgfplots@tmpa=\pgfplots@xmin@reg\relax
				\pgfplots@tickshowfalse
			\fi
		\or
			\ifdim\pgfplots@tmpa=\pgfplots@logical@ZERO@x pt %
				\pgfplots@tickshowfalse
			\fi
		\or
			\ifdim\pgfplots@tmpa=\pgfplots@xmax@reg\relax
				\pgfplots@tickshowfalse
			\fi
		\fi
	\fi
}

\def\pgfplots@drawxaxis@placecomputedtick{%
	\pgfpathmoveto{\pgfqpointxy{\pgfplots@curtickpos}{\pgfplots@tick@beg@a}}%
	\pgfpathlineto{\pgfqpointxy{\pgfplots@curtickpos}{\pgfplots@tick@end@a}}%
	\ifnum\pgfplots@xtickposnum=0\relax
		\pgfpathmoveto{\pgfqpointxy{\pgfplots@curtickpos}{\pgfplots@tick@beg@b}}%
		\pgfpathlineto{\pgfqpointxy{\pgfplots@curtickpos}{\pgfplots@tick@end@b}}%
	\fi
}%
\def\pgfplots@drawxaxis@placecomputedgridline{%
	\pgfpathmoveto{\pgfqpointxy{\pgfplots@curgridpos}{\pgfplots@ymin}}%
	\pgfpathlineto{\pgfqpointxy{\pgfplots@curgridpos}{\pgfplots@ymax}}%
}%

% #1: axis (x or y)
% #2: axis index (0 or 1)
% #3: tick position list
\def\pgfplots@draw@extra@ticks@for#1#2#3{%
	\begingroup
	\def\pgfplots@scaled@ticks@x@choice{0}%
	\def\pgfplots@scaled@ticks@y@choice{0}%
	\csname pgfplots@#1minorticksfalse\endcsname
	\csname pgfplots@#1minorgridsfalse\endcsname
	\expandafter\let\expandafter\axis@TMP\csname pgfplots@extra@#1ticklabel\endcsname
	\expandafter\let\csname pgfplots@#1ticklabel\endcsname=\axis@TMP
	\pgfplotsset{/pgfplots/every extra #1 tick}% FIXME : used 'search path for tikz' before...
	\pgfplots@prepare@ticks@for{#3}{#1}{#2}%
	\pgfplots@drawgridlines@for{#1}{#2}%
	\pgfplots@drawticklines@for{#1}{#2}%
	\pgfplots@drawticklabels@for{#1}{#2}%
	\endgroup
}

% Computes final major and minor tick positions into global lists
% \pgfplots@prepared@tick@positions@major@x
% and
% \pgfplots@prepared@tick@positions@minor@x.
%
% The major tick list contains tuples (canvas position,logical position)
% while the minor tick list contains just the canvas position.
%
% #1: the tick list.
% #2: the axis name as character, i.e. 'x' or 'y'
% #3: the axis name as integer, i.e. 0 or 1
%
% PRECONDITION:  
% 	- \pgfplots@determinedefaultvalues has been executed.
% 		That means particularly that \pgfplots@[xy][min,max] are available in TeX point
% 		range (after datascaling and logs).
\def\pgfplots@prepare@ticks@for#1#2#3{%
	\begingroup
	\expandafter\let\expandafter\ifpgfplots@islinear\csname ifpgfplots@#2islinear\endcsname
	\expandafter\let\expandafter\ifpgfplots@minorticks\csname ifpgfplots@#2minorticks\endcsname
	\expandafter\let\expandafter\ifpgfplots@minorgrids\csname ifpgfplots@#2minorgrids\endcsname
	% these lists need to be global such that I can fill them inside
	% of \foreach statements. And, yes: I have also added a TeX group
	% on my own (but that's not the problem).
	\global\pgfplotslistnewempty\pgfplots@prepared@tick@positions@major
	\global\pgfplotslistnewempty\pgfplots@prepared@tick@positions@minor
	\edef\pgfplots@loc@TMPa{#1}%
	\ifx\pgfplots@loc@TMPa\pgfutil@empty
	\else
		\ifpgfplots@minorticks
			\pgfplots@needsminorlooptrue
		\else
			\ifpgfplots@minorgrids
				\pgfplots@needsminorlooptrue
			\else
				\pgfplots@needsminorloopfalse
			\fi
		\fi
		\ifpgfplots@needsminorloop
			\ifpgfplots@islinear
				\expandafter\let\expandafter\pgfplots@minor@tick@num\csname pgfplots@minor@#2tick@num\endcsname
				\begingroup
				\c@pgf@counta=\pgfplots@minor@tick@num\relax
				\advance\c@pgf@counta by1\relax
				\pgfplots@tmpa=\csname pgfplots@tick@distance@#2\endcsname pt %
				\divide\pgfplots@tmpa by\c@pgf@counta
				\edef\pgfmathresult{\pgf@sys@tonumber{\pgfplots@tmpa}}%
				\pgfmath@smuggleone\pgfmathresult
				\endgroup
				\let\pgfplots@minor@tick@dist=\pgfmathresult
			\else
				\def\pgfplots@minor@tick@num{9}%
			\fi
		\fi
		\foreach \x in {#1} {%
			\pgfplots@tmpa=\x pt %
			%--------------------------------------------------
			% \ifcase#3\relax
			% 	\pgfextractx{\pgfplots@tmpa}{\pgfpointxy{\x}{0}}%
			% \or
			% 	\pgfextracty{\pgfplots@tmpa}{\pgfpointxy{0}{\x}}%
			% \fi
			%-------------------------------------------------- 
			\csname pgfplots@#2tick@check@tickshow\endcsname
			%
			\ifpgfplots@tickshow
				\ifdim\pgfplots@tmpa<\csname pgfplots@#2min@reg\endcsname
				\else
					\ifdim\pgfplots@tmpa>\csname pgfplots@#2max@reg\endcsname
					\else
						\expandafter\pgfplotslistpushbackglobal\x\to\pgfplots@prepared@tick@positions@major
					\fi
				\fi
			\fi
			% X-Axis ticks bottom and top
			% in log:
			%  log( i*10^k ) = log\pgfplots@i + k\log10 -> draw ticks for i=1..9
			\ifpgfplots@needsminorloop
				\foreach \pgfplots@i in {1,...,\pgfplots@minor@tick@num} {%
					\begingroup
					\ifpgfplots@islinear
						\pgfplots@tmpa=\pgfplots@minor@tick@dist pt %
						\pgfplots@tmpa=\pgfplots@i\pgfplots@tmpa
						%\pgfextractx{\pgfplots@tmpa}{\pgfpointxy{\pgfplots@i*\pgfplots@minor@tick@dist}{0}}%
					\else
						\pgfplots@tmpa=\logi\pgfplots@i pt %
						%\pgfextractx{\pgfplots@tmpa}{\pgfpointxy{\logi\pgfplots@i}{0}}%
					\fi
					\edef\pgfmathresult{\the\pgfplots@tmpa}%
					\pgfmath@smuggleone\pgfmathresult
					\endgroup
					\advance\pgfplots@tmpa by\pgfmathresult\relax
					\ifdim\pgfplots@tmpa<\csname pgfplots@#2min@reg\endcsname
					\else
						\ifdim\pgfplots@tmpa>\csname pgfplots@#2max@reg\endcsname
						\else
							\edef\pgfmathresult{\pgf@sys@tonumber{\pgfplots@tmpa}}%
							\expandafter\pgfplotslistpushbackglobal\pgfmathresult\to\pgfplots@prepared@tick@positions@minor
						\fi
					\fi
				}%
			\fi
		}%
	\fi
	\endgroup
	\expandafter\let\csname pgfplots@prepared@tick@positions@minor@#2\endcsname=\pgfplots@prepared@tick@positions@minor
	\expandafter\let\csname pgfplots@prepared@tick@positions@major@#2\endcsname=\pgfplots@prepared@tick@positions@major
	\global\let\pgfplots@prepared@tick@positions@major=\relax
	\global\let\pgfplots@prepared@tick@positions@minor=\relax
}%

% #1 : the verbatim axis name (either 'x' or 'y')
% #2 : the index of the axis  (either 0 or 1)
\def\pgfplots@drawgridlines@for#1#2{%
	\begingroup
	\expandafter\let\expandafter\ifpgfplots@major\csname ifpgfplots@#1majorgrids\endcsname
	\expandafter\let\expandafter\ifpgfplots@minor\csname ifpgfplots@#1minorgrids\endcsname
	\expandafter\let\expandafter\pgfplots@prepared@tick@positions@major@\csname pgfplots@prepared@tick@positions@major@#1\endcsname
	\expandafter\let\expandafter\pgfplots@prepared@tick@positions@minor@\csname pgfplots@prepared@tick@positions@minor@#1\endcsname
	\pgfplots@loop@CONTINUEfalse
	\ifpgfplots@major
		\pgfplots@loop@CONTINUEtrue
	\fi
	\ifpgfplots@minor
		\pgfplots@loop@CONTINUEtrue
	\fi
	\ifpgfplots@loop@CONTINUE
		\scope
		\pgfinterruptboundingbox% FIXME : won't work correctly for 3D!?
		%--------------------------------------------------
		% \ifcase#2\relax
		% 	\clip 			(\pgfplots@xcoordminTEX,	\pgfplots@ycoordminTEX-5cm) 
		% 		rectangle	(\pgfplots@xcoordmaxTEX,	\pgfplots@ycoordmaxTEX+5cm);
		% \or
		% 	\clip 			(\pgfplots@xcoordminTEX-5cm,	\pgfplots@ycoordminTEX) 
		% 		rectangle	(\pgfplots@xcoordmaxTEX+5cm,	\pgfplots@ycoordmaxTEX);
		% \fi
		%-------------------------------------------------- 
		\endpgfinterruptboundingbox%
		\ifpgfplots@major
			\draw[%
				/pgfplots/every axis grid,
				/pgfplots/every major grid,
				/pgfplots/every axis #1 grid,
				/pgfplots/every major #1 grid]%
			\pgfextra
			\pgfplotslistforeach\pgfplots@prepared@tick@positions@major@\as\pgfplots@curgridpos{%
				\csname pgfplots@draw#1axis@placecomputedgridline\endcsname
			}%
			\endpgfextra;
		\fi
		%
		\ifpgfplots@minor
			\draw[%
				/pgfplots/every axis grid,
				/pgfplots/every minor grid,
				/pgfplots/every axis #1 grid,
				/pgfplots/every minor #1 grid]%
			\pgfextra
			\pgfplotslistforeach\pgfplots@prepared@tick@positions@minor@\as\pgfplots@curgridpos{%
				\csname pgfplots@draw#1axis@placecomputedgridline\endcsname
			}%
			\endpgfextra;
		\fi
		\endscope
	\fi
	\endgroup
}

% #1 : the verbatim axis name (either 'x' or 'y')
% #2 : the index of the axis  (either 0 or 1)
\def\pgfplots@drawticklines@for#1#2{%
	\scope
	\expandafter\let\expandafter\ifpgfplots@major\csname ifpgfplots@#1majorticks\endcsname
	\expandafter\let\expandafter\ifpgfplots@minor\csname ifpgfplots@#1minorticks\endcsname
	\expandafter\let\expandafter\pgfplots@prepared@tick@positions@major@\csname pgfplots@prepared@tick@positions@major@#1\endcsname
	\expandafter\let\expandafter\pgfplots@prepared@tick@positions@minor@\csname pgfplots@prepared@tick@positions@minor@#1\endcsname
	%--------------------------------------------------
	% \pgfinterruptboundingbox%
	% \ifcase#2\relax % FIXME : WON'T WORK FOR 3D!
	% 	\clip 			(\pgfplots@xcoordminTEX,	\pgfplots@ycoordminTEX-5cm) 
	% 		rectangle	(\pgfplots@xcoordmaxTEX,	\pgfplots@ycoordmaxTEX+5cm);
	% \or
	% 	\clip 			(\pgfplots@xcoordminTEX-5cm,	\pgfplots@ycoordminTEX) 
	% 		rectangle	(\pgfplots@xcoordmaxTEX+5cm,	\pgfplots@ycoordmaxTEX);
	% \fi
	% \endpgfinterruptboundingbox%
	%-------------------------------------------------- 
	\ifpgfplots@major
		\draw[%
			/pgfplots/every tick,
			/pgfplots/every major tick,
			/pgfplots/every #1 tick,
			/pgfplots/every major #1 tick]%
		\pgfextra
		\pgfmathparse{\pgfplots@tickwidth}%
		\let\pgfplots@tickwidth@=\pgfmathresult
		\ifcase#2\relax
			\pgfmathmultiply@{\pgfplots@tickwidth@}{\pgfplots@y@inverseveclength}%
		\or
			\pgfmathmultiply@{\pgfplots@tickwidth@}{\pgfplots@x@inverseveclength}%
		\fi
		\let\pgfplots@tickwidth@=\pgfmathresult
		\let\pgfplots@tickwidth=\pgfmathresult
		\ifcase\csname pgfplots@#1tickalignnum\endcsname\relax
			\def\pgfplots@tick@offset{0}%
		\or
			\let\pgfplots@tick@offset=\pgfplots@tickwidth@%
		\or
			\pgfmathmultiply@{0.5}{\pgfplots@tickwidth@}%
			\let\pgfplots@tick@offset=\pgfmathresult%
		\fi
		\ifcase#2\relax
			\pgfplots@prepare@tick@offsets@for@{#1}{y}{\pgfplots@tickwidth@}%
		\else
			\pgfplots@prepare@tick@offsets@for@{#1}{x}{\pgfplots@tickwidth@}%
		\fi
		\pgfplotslistforeach\pgfplots@prepared@tick@positions@major@\as\pgfplots@curtickpos{%
			\csname pgfplots@draw#1axis@placecomputedtick\endcsname
		}%
		\endpgfextra;
	\fi
	%
	\ifpgfplots@minor
		\draw[%
			/pgfplots/every tick,
			/pgfplots/every minor tick,
			/pgfplots/every #1 tick,
			/pgfplots/every minor #1 tick]%
		\pgfextra
		\pgfmathparse{\pgfplots@subtickwidth}%
		\let\pgfplots@subtickwidth@=\pgfmathresult
		\ifcase#2\relax
			\pgfmathmultiply@{\pgfplots@subtickwidth@}{\pgfplots@y@inverseveclength}%
		\or
			\pgfmathmultiply@{\pgfplots@subtickwidth@}{\pgfplots@x@inverseveclength}%
		\fi
		\let\pgfplots@subtickwidth@=\pgfmathresult
		\let\pgfplots@subtickwidth=\pgfmathresult
		\ifcase\csname pgfplots@#1tickalignnum\endcsname\relax
			\def\pgfplots@tick@offset{0}%
		\or
			\let\pgfplots@tick@offset=\pgfplots@subtickwidth@%
		\or
			\pgfmathmultiply@{0.5}{\pgfplots@subtickwidth@}%
			\let\pgfplots@tick@offset=\pgfmathresult%
		\fi
		\ifcase#2\relax
			\pgfplots@prepare@tick@offsets@for@{#1}{y}{\pgfplots@subtickwidth@}%
		\else
			\pgfplots@prepare@tick@offsets@for@{#1}{x}{\pgfplots@subtickwidth@}%
		\fi
		\pgfplotslistforeach\pgfplots@prepared@tick@positions@minor@\as\pgfplots@curtickpos{%
			\csname pgfplots@draw#1axis@placecomputedtick\endcsname
		}%
		\endpgfextra;
	\fi
	\endscope
}

% #1 : the verbatim axis name (either 'x' or 'y')
% #2 : the index of the axis  (either 0 or 1)
\def\pgfplots@drawticklabels@for#1#2{%
	\begingroup
	\expandafter\let\expandafter\ifpgfplots@major\csname ifpgfplots@#1majorticks\endcsname
	\expandafter\let\expandafter\ifpgfplots@islinear\csname ifpgfplots@#1islinear\endcsname
	\expandafter\let\expandafter\pgfplots@prepared@tick@positions@major@\csname pgfplots@prepared@tick@positions@major@#1\endcsname
	\ifpgfplots@major
		\ifpgfplots@islinear
			\pgfplots@init@scaled@tick@for{#1}%
		\fi
		\begingroup
		\pgfkeys{/tikz/every node/.append style={/pgfplots/every tick label,/pgfplots/every #1 tick label}}%
		\ifcase\csname pgfplots@#1tickalignnum\endcsname\relax
			\def\pgfmathresult{0}%
		\or
			\pgfmathparse{\pgfplots@tickwidth}%
		\or
			\pgfmathmultiply{0.5}{\pgfplots@tickwidth}%
		\fi
		\expandafter\pgfmathmultiply@\expandafter{\pgfmathresult}{\csname pgfplots@#1@inverseveclength\endcsname}%
		\let\pgfplots@tick@offset=\pgfmathresult
		\ifnum\csname pgfplots@#1ticklabelposnum\endcsname=0
			% the choice '[xy]ticklabel pos=default':
			\expandafter\let\expandafter\pgfplots@tmpposnum\csname pgfplots@#1tickposnum\endcsname
		\else
			\expandafter\let\expandafter\pgfplots@tmpposnum\csname pgfplots@#1ticklabelposnum\endcsname
		\fi
		\ifcase#2\relax
			\ifcase\pgfplots@tmpposnum\relax
				\let\pgfplots@tick@origin=\pgfplots@ymin%
			\or
				\let\pgfplots@tick@origin=\pgfplots@ymin%
			\or
				\let\pgfplots@tick@origin=\pgfplots@logical@ZERO@y
			\or
				\let\pgfplots@tick@origin=\pgfplots@ymax%
			\fi
			\def\pgfplots@tickposchoicea{\tikzset{above}}%
			\def\pgfplots@tickposchoiceb{\tikzset{below}}%
		\or
			\ifcase\pgfplots@tmpposnum\relax
				\let\pgfplots@tick@origin=\pgfplots@xmin%
			\or
				\let\pgfplots@tick@origin=\pgfplots@xmin@reg%
			\or
				\let\pgfplots@tick@origin=\pgfplots@logical@ZERO@x
			\or
				\let\pgfplots@tick@origin=\pgfplots@xmax@reg%
			\fi
			\def\pgfplots@tickposchoicea{\tikzset{right}}%
			\def\pgfplots@tickposchoiceb{\tikzset{left}}%
		\fi
		\ifnum\pgfplots@tmpposnum=3
			\pgfplots@tickposchoicea
			\pgfmathadd@{\pgfplots@tick@origin}{\pgfplots@tick@offset}%
		\else
			\pgfplots@tickposchoiceb
			\pgfmathsubtract@{\pgfplots@tick@origin}{\pgfplots@tick@offset}%
		\fi
		\let\pgfplots@tick@origin=\pgfmathresult%
		\def\pgfplots@ticknum{0}%
		\xdef\pgfplots@show@ticklabel@LASTTICK{}%
		\pgfplotslistforeachungrouped\pgfplots@prepared@tick@positions@major@\as\pgfplots@curtickpos{%
			\ifcase#2\relax
				\pgfplots@show@ticklabel
					{#1}{\pgfplots@curtickpos}(\pgfplots@curtickpos,\pgfplots@tick@origin)%
					{\pgfplots@ticknum}%
			\or
				\pgfplots@show@ticklabel
					{#1}{\pgfplots@curtickpos}(\pgfplots@tick@origin,\pgfplots@curtickpos)%
					{\pgfplots@ticknum}%
			\fi
			\begingroup
			\c@pgf@counta=\pgfplots@ticknum\relax
			\advance\c@pgf@counta by1
			\edef\pgfplots@ticknum{\the\c@pgf@counta}%
			\pgfmath@smuggleone\pgfplots@ticknum
			\endgroup
		}%
		\endgroup
		\pgfplots@draw@tick@scale@label@for #1%
	\fi
	\endgroup
}

\newif\ifpgfplots@checkuniform@isfirst
% Checks whether the argument to xtick or ytick is a UNIFORM tick
% sequence.
%
% A uniform tick sequence is 0,...,10 and 3,4,5 and -5,-4,-2 but 
% NOT 0,2,4 or 4,10.
%
% Furthermore, any NON-integer tick arguments are also assumed to be
% NOT uniform.
%
% INPUT:
% #1: a tick argument (i.e. something which can be put to 
%     \foreach \x in {#1})
%
% OUTPUT:
%    \pgfplots@isuniformticktrue
%    or
%    \pgfplots@isuniformtickfalse
% depending on the check.
% This variable will be set globally.
\def\pgfplots@checkisuniformLOGtick#1{%
	\begingroup
	\global\pgfplots@isuniformticktrue
	\pgfplots@checkuniform@isfirsttrue
	\foreach \x in {#1}{%
		\pgfmathmultiply@\x\reciproclogten
		\let\cur=\pgfmathresult
		% check whether
		%  \cur - last == 1 (last = \pgfplots@glob@TMPb)
		\ifpgfplots@checkuniform@isfirst
			\global\pgfplots@checkuniform@isfirstfalse
		\else
			\pgfmathsubtract@\cur\pgfplots@glob@TMPb%
			\pgfmathapproxequalto@\pgfmathresult{1.0}%
			\ifpgfmathcomparison
			\else
				\global\pgfplots@isuniformtickfalse
				\breakforeach
			\fi
		\fi
		\global\let\pgfplots@glob@TMPb=\cur
	}%
	\endgroup
}

% Checks whether the linear tick sequence #1 is a uniform tick.
%
% It also assigns pgfplots@tick@distance@#1  as the distance.
%
% see \pgfplots@checkisuniformLOGtick for details.
%
% #1: a tick sequence (expanded)
% #2: a macro which will be filled with the tick distance. This is
% only valid if \pgfplots@isuniformticktrue.
\def\pgfplots@checkisuniformLINEARtick#1#2{%
	\begingroup
	\global\pgfplots@isuniformticktrue
	\pgfplots@checkuniform@isfirsttrue
	\global\let\pgfplots@glob@TMPb=\pgfutil@empty
	\global\def\pgfplots@glob@TMPa{1}%
	\foreach \x in {#1}{%
		\ifx\pgfplots@glob@TMPb\pgfutil@empty
		\else
			\pgfmathsubtract@\x\pgfplots@glob@TMPb
			\ifpgfplots@checkuniform@isfirst
				% remember first distance h = x_1 - x_0
				\global\let\pgfplots@glob@TMPa=\pgfmathresult
				\global\pgfplots@checkuniform@isfirstfalse
			\else
				% check whether x_i - x_{i-1} = h
				\pgfmathapproxequalto@\pgfmathresult\pgfplots@glob@TMPa%
				\ifpgfmathcomparison
				\else
					\global\pgfplots@isuniformtickfalse
					\breakforeach
				\fi
			\fi
		\fi
		\global\let\pgfplots@glob@TMPb=\x%
	}%
	\endgroup
	\let#2=\pgfplots@glob@TMPa
}

% helper method which computes log10*\x foreach \x in {#1}.
% The result will be \xdef'ed into #2.
\def\pgfplots@compute@tick@times@logten#1\to#2{%
	\global\let#2=\pgfutil@empty
	\foreach \pgfplots@loc@TMPb in {#1} {%
		\pgfmathmultiply@\pgfplots@loc@TMPb\logten%
		\ifx#2\pgfutil@empty
			\xdef#2{\pgfmathresult}%
		\else
			\xdef#2{#2,\pgfmathresult}%
		\fi
	}%
}

% Computes tick positions using the current axis limits.
%
% Parameters:
% /pgfplots/max space between ticks
%    Determines the maximum space which is not filled by at least one
%    tick label (approximate, there is some rounding internally)
% /pgfplots/try min ticks
%    see manual
%
% Idea:
% We want ticks at each 
%    { i*H, i in \Z }.
% Of course, there shouldn't be TOO MUCH ticks. 
%
% Our heuristics is to set
%    desirednumticks = round(ACTUAL WIDTH / (max space between ticks) )
% and generate H = (axis range) / (desirednumticks).
%
% Since not all step sizes H look well, restrict H to a set of allowed 
% step sizes such as 
%   { 1, 1/2, 1/5, 1/10 },
% or, to be more precise:
%   { 1*10^e, 2*10^e, 5*10^e }
% -> round to the nearest matching number!
%
% For log-plots, 
% 	H in { j*log(10), j=1,2,3,... }
% where the usual case should be j = 1.
%
% Then, the resulting tick is
% TICK={MIN,MIN+H,...,MAX}
% where
%    MIN = I*H
% is chosen such that 
%    axis minimum limit = I*H + rest; |rest| < H.
%
% Again, log plots follow a slightly different approach: here,
%   MIN = I * log(10)
% is chosen such that
%    axis minimum limit = I*log(10) + rest; |rest| < log(10)
% while H = j*log(10), j>=1.
%
%
% PRECONDITION:
% - limits are correct
% - axis width/height is set correctly
%
% POSTCONDITION:
% - Tick for axis #1 is assigned
% - \ifpgfplots@determinedefaultvalues@needs@check@uniformtick is set
%
% REMARKS:
% - this algorithms works also if the data range has been transformed
%   with a LINEAR transformation.
%   ATTENTION: as of 2008-05-15, the scaling trafo is AFFINE LINEAR.
%   That means we have to eliminate the 'affine' shifting before the
%   algorithms works correctly.
\def\pgfplots@assign@default@tick@foraxis#1{%
	\begingroup
	% Shortcut-names:
	\expandafter\let\expandafter\ifpgfplots@is@datascaled\csname ifpgfplots@apply@datatrafo@#1\endcsname
	% Attention here: use UNSHIFTET scalings, see remark above
	\expandafter\let\expandafter\pgfplots@data@scale@trafo\csname pgfplots@datascaletrafo@#1@noshift\endcsname
	\expandafter\let\expandafter\pgfplots@data@scale@inverse@trafo\csname pgfplots@inverse@datascaletrafo@#1@noshift\endcsname
	\expandafter\let\expandafter\ifpgfplots@cur@is@linear\csname ifpgfplots@#1islinear\endcsname
	%
	\let\desirednumticks=\c@pgf@countd
	\let\Wr=\pgf@xc
	\Wr=\csname pgfplots@#1coordmaxTEX\endcsname
	\advance\Wr by-\csname pgfplots@#1coordminTEX\endcsname
	% r = max place without ticks in pt -> choose desirednumticks >= W/r
	\expandafter\expandafter\divide\Wr\axisdefaulttickwidth
	\pgfmathsetcount{\desirednumticks}{\Wr}%
	\advance\desirednumticks by1
	\csname ifpgfplots@#1islinear\endcsname
		\expandafter\ifnum\axisdefaulttryminticks>\desirednumticks
			\desirednumticks=\axisdefaulttryminticks
		\fi
	\else
		\expandafter\ifnum\pgfplots@default@try@minticks@log>\desirednumticks
			\desirednumticks=\pgfplots@default@try@minticks@log\relax
		\fi
		\expandafter\ifx\csname pgfplots@#1tickten\endcsname\pgfutil@empty
		\else
			% log plot and tickten-option: provide special processing.
			\edef\pgfplots@loc@TMPa{\csname pgfplots@#1tickten\endcsname}%
			\expandafter\pgfplots@compute@tick@times@logten\pgfplots@loc@TMPa\to\pgfplots@glob@TMPa
			\expandafter\let\csname pgfplots@#1tick\endcsname=\pgfplots@glob@TMPa
		\fi
	\fi
	%
	\expandafter\ifx\csname pgfplots@#1tick\endcsname\pgfutil@empty
		% Ok, we have either log or linear axis and need default
		% ticks MIN,MIN+H,...,MAX.
		\let\MINH=\pgf@xa
		\let\H=\pgf@xb
		\let\MAX=\pgf@ya
		\let\MIN=\pgf@yb
		% compute step size 'H':
		\expandafter\MAX\csname pgfplots@#1max\endcsname pt
		\advance\MAX by0.001pt % avoid round errors
		%\expandafter\MIN\the\c@pgf@counta pt
		\expandafter\MIN\csname pgfplots@#1min\endcsname pt
		\H=\MAX
		\advance\H by-\MIN
%\message{Axis limit #1: [\the\MIN:\the\MAX], diff = \the\H.}%
		\c@pgf@counta=\desirednumticks
		\advance\c@pgf@counta by-1
		\divide\H by\c@pgf@counta
%\message{determining ticks for #1-axis: Wr := (width/max space between ticks) = \the\Wr, desirednumticks=max(\axisdefaulttryminticks, trunc(Wr)) = \the\desirednumticks, H#1=(axis range/(desirednumticks-1)) = \the\H}%
		%
		% SEARCH for the NEXT FEASABLE H.
		\edef\Hmacro{\pgf@sys@tonumber\H}%
		\ifpgfplots@cur@is@linear
			% CASE LINEAR AXIS
			\ifpgfplots@is@datascaled
				% This here works if the scaling trafo is linear.
				\expandafter\pgfplots@data@scale@inverse@trafo\expandafter{\Hmacro}%
				\let\Hmacro=\pgfmathresult
			\else
				\pgfmathfloatparsenumber{\Hmacro}%
				\let\Hmacro=\pgfmathresult
			\fi
			\expandafter\pgfmathfloat@decompose\pgfmathresult\relax\pgfmathfloat@a@S\H\pgfmathfloat@a@E
%\message{Got T^{-1}(H#1) = \Hmacro}%
			% modify the mantisse:
			\ifdim\H<2pt
				\ifdim\H<1.5pt
					\H=1.0pt
				\else
					\H=2.0pt
				\fi
			\else
				\ifdim\H<4.9999pt
					\ifdim\H<3.5pt
						\H=2.0pt\relax
					\else
						\H=5.0pt\relax
					\fi
				\else
					\ifdim\H<7.5pt
						\H=5.0pt\relax
					\else
						\H=1.0pt\relax
						\advance\pgfmathfloat@a@E by1
					\fi
				\fi
			\fi
%\message{Computed H#1=\the\H [unscaled]; transforming ... }%
			\pgfmathfloatcreate{\the\pgfmathfloat@a@S}{\pgf@sys@tonumber{\H}}{\the\pgfmathfloat@a@E}%
			\let\Hmacro=\pgfmathresult
			\ifpgfplots@is@datascaled
				\pgfplots@data@scale@trafo\Hmacro
			\else
				\pgfmathfloattofixed\Hmacro
			\fi
			\let\Hmacro=\pgfmathresult
%\message{got H#1=\Hmacro\ [transformed]}%
			\expandafter\H\Hmacro pt
			\expandafter\aftergroup\csname pgfplots@determinedefaultvalues@#1isuniformtrue\endcsname
			%
			% Now, we want to activate the Tick set {i*H, i in \Z}
			% compute I such that
			%   xmin = I * H + rest;  |rest| < H
			% -> I = round(xmin/H)
			% -> MIN = I * H
			%
			% but first: eliminate any 'affine' data scaling!
			\ifpgfplots@is@datascaled
				\advance\MIN by\csname pgfplots@data@scale@trafo@SHIFT@#1\endcsname pt
			\fi
			\pgfmathlog@invoke@expanded\pgfmathdivide@{%
				{\pgf@sys@tonumber\MIN}%
				{\pgf@sys@tonumber\H}%
			}%
			\pgfmathsetcount{\c@pgf@counta}{\pgfmathresult}%
			\ifdim\MIN<0pt
				% the truncation rounds TOWARDS 0 which is not what I want.
				\advance\c@pgf@counta by-1
			\fi
			\MIN=\H
			\multiply\MIN by\c@pgf@counta
			\ifpgfplots@is@datascaled
				\advance\MIN by-\csname pgfplots@data@scale@trafo@SHIFT@#1\endcsname pt
			\fi
		\else
			% CASE LOG AXIS
			%
			% search for the "best" H= j* log(10),  j an integer.
			%
			% And prefer j=1 if that is possible (otherwise minor
			% ticks are not useful).
			\pgfmathmultiply@{\Hmacro}{\reciproclogten}%
			\let\Hmacrobaseten=\pgfmathresult
			\expandafter\H\pgfmathresult pt
%\message{ [ H / log(10) = \pgfmathresult ]}%
			\ifdim\H<2pt
				\H=1pt
			\else
				\ifnum\H<1pt
					\H=1pt
				\else
					\expandafter\pgfmathfloor\expandafter{\pgfmathresult}%
					\expandafter\H\pgfmathresult pt
				\fi
			\fi
			\ifdim\H=1pt
				\expandafter\aftergroup\csname pgfplots@determinedefaultvalues@#1isuniformtrue\endcsname
				\pgfplots@isuniformticktrue
			\else
				\expandafter\aftergroup\csname pgfplots@determinedefaultvalues@#1isuniformfalse\endcsname
				\pgfplots@isuniformtickfalse
			\fi
%\message{final H=\pgf@sys@tonumber{\H} * log(10)}%
			\H=\logten\H\relax
			% Now, we want to activate the Tick set 
			%   {lowest, lowest+H, ..., highest}
			%
			% Where 
			% 	lowest =  I * log(10) + rest, |rest| < log(10).
			% this is conceptionally different from the approach for
			% linear axes, because H = j*log(10).
			%
			% remember the original xmin in MINH:
			\MINH=\MIN
			%
			% and compute I and I*log(10) here:
			\expandafter\MIN\reciproclogten\MIN\relax
			\edef\pgfmathresult{\pgf@sys@tonumber{\MIN}}%
			\pgfmathsetcount{\c@pgf@counta}{\pgfmathresult}%
			\ifdim\MIN<0pt
				% the truncation rounds TOWARDS 0 which is not what I want.
				\advance\c@pgf@counta by-1
			\fi
			\expandafter\MIN\logten pt
			\multiply\MIN by\c@pgf@counta
			\ifpgfplots@isuniformtick
			\else
				% This here is a special case to move the first tick
				% near the lower axis limit.
				%
				% "Near" means either directly above or directly below ymin.
				% 
				% My application example is as follows:
				% Let H = 2*log(10).
				% Furthermore, ymin = 3e-6, ymax= 8e-2. That means we can choose either
				%    10^{-5}, 10^{-3}, 10^{-1}
				% or
				%    10^{-4}, 10^{-2}
				% as ticks. Well, I prefer the first one.
				%
				% HEURISTICS: start as near to ymin as possible!
				%
				% We check here if we can come nearer to ymin if we
				% shift the current tick by log(10):
				%  if( ymin - I * log(10) < 0.5*H ->  use I+1, that means add log(10).
				%
				% that's equivalent to 
				%  2*(ymin - I * log(10)) - H < 0.
				\advance\MINH by-\MIN
				\multiply\MINH by2
				\advance\MINH by-\H
				% 
				\ifdim\MINH<0pt
					\expandafter\advance\expandafter\MIN\logten pt
				\fi
			\fi
		\fi
		\MINH=\MIN
		\advance\MINH by\H
		% Ok, now it can happen that only ONE tick label is placed in
		% this range.
		% That's useless, so check for it.
		%
		% That's the case if
		%    MIN < ORIGMIN && MAX < MIN+2 H 
		% MIN < ORIGMIN by construction (ok, MIN <= ORIGMIN by
		% construction, but I don't care about this case). 
		% So: check only the second condition.
		\pgfplots@tmpa=\MINH
		\advance\pgfplots@tmpa by\H
		\ifdim\MAX<\pgfplots@tmpa
			\expandafter\aftergroup\csname pgfplots@determinedefaultvalues@#1isuniformfalse\endcsname
			% ok, do something special.
			%
			% The idea is now to place ticks at
			% 10^{i*h} with properly choosen 'h'.
			%
			% So: apply basically the SAME code as above for linear
			% axis, just everything log 10!  And keep in mind that all
			% coordinates are actually given as natural logarithms.
			\expandafter\MIN\csname pgfplots@#1min\endcsname pt
			\H=\MAX
			\advance\H by-\MIN
			\H=\reciproclogten\H
%\message{Axis limit #1: [\the\MIN:\the\MAX], diff/log(10) = \the\H.}%
			\c@pgf@counta=\desirednumticks\relax
			\advance\c@pgf@counta by-1
			\ifnum\c@pgf@counta>2
				% subtract one more. This algorithm here produces more
				% ticks than the normal one which is designed for 10^i
				\advance\c@pgf@counta by-1
			\fi
			\divide\H by\c@pgf@counta\relax
%\message{determining ticks for #1-axis: Wr := (width/max space between ticks) = \the\Wr, desirednumticks=max(\axisdefaulttryminticks, trunc(Wr)) = \the\desirednumticks, H#1=(axis range/(desirednumticks-1)) = \the\H}%
			%
			% SEARCH for the NEXT FEASABLE H.
			\edef\Hmacro{\pgf@sys@tonumber\H}%
			\pgfmathfloatparsenumber{\Hmacro}%
			\expandafter\pgfmathfloat@decompose\pgfmathresult\relax\pgfmathfloat@a@S\H\pgfmathfloat@a@E
			% Determine properly snapped mantisse...
			\ifdim\H<2pt
				\ifdim\H<1.5pt
					\H=1.0pt
				\else
					\H=2.0pt
				\fi
			\else
				\ifdim\H<4.9999pt
					\ifdim\H<3.5pt
						\H=2.0pt\relax
					\else
						\H=5.0pt\relax
					\fi
				\else
					\ifdim\H<7.5pt
						\H=5.0pt\relax
					\else
						\H=1.0pt\relax
						\advance\pgfmathfloat@a@E by1
					\fi
				\fi
			\fi
			\pgfmathfloatcreate{\the\pgfmathfloat@a@S}{\pgf@sys@tonumber{\H}}{\the\pgfmathfloat@a@E}%
			\expandafter\pgfmathfloattofixed\expandafter{\pgfmathresult}%
			\let\Hmacro=\pgfmathresult
			\H=\Hmacro pt %
			% Ok, our step size h for 10^{i*h} is ready!
%\message{determined step size 10^{\Hmacro}}%
			% Now, we want to activate the Tick set {10^{i*H}, i in \Z}
			% compute I such that
			%   10^{min} = 10^{I * H + rest};  |rest| < H
			% -> I = round(xmin/H)
			% -> MIN = I * H
			% BUT EVERYTHING to log(10) basis!
			\MIN=\reciproclogten\MIN\relax
			\pgfmathlog@invoke@expanded\pgfmathdivide@{%
				{\pgf@sys@tonumber\MIN}%
				{\Hmacro}%
			}%
			\pgfmathsetcount{\c@pgf@counta}{\pgfmathresult}%
			\ifdim\MIN<0pt
				% the truncation rounds TOWARDS 0 which is not what I want.
				\advance\c@pgf@counta by-1
			\fi
			\MIN=\H\relax
			\multiply\MIN by\c@pgf@counta\relax
			%
			% convert back to basis 'e':
			\MIN=\logten\MIN\relax
			\H=\logten\H\relax
			\MINH=\MIN\relax
			\advance\MINH by\H\relax
		\fi
%\pgfplots@message{final H=\the\H.}%
		\xdef\pgfplots@glob@TMPa{\pgf@sys@tonumber{\MIN},\pgf@sys@tonumber{\MINH},...,\pgf@sys@tonumber{\MAX}}%
		\xdef\pgfplots@glob@TMPb{\pgf@sys@tonumber{\H}}%
		\aftergroup\pgfplots@determinedefaultvalues@needs@check@uniformtickfalse
	\else
		\expandafter\global\expandafter\let\expandafter\pgfplots@glob@TMPa\csname pgfplots@#1tick\endcsname
		\gdef\pgfplots@glob@TMPb{}% will be computed later, in 'check uniform tick'
		\aftergroup\pgfplots@determinedefaultvalues@needs@check@uniformticktrue
	\fi
	\endgroup
	\expandafter\let\csname pgfplots@#1tick\endcsname=\pgfplots@glob@TMPa
	\expandafter\let\csname pgfplots@tick@distance@#1\endcsname=\pgfplots@glob@TMPb
%\pgfplots@message{pgfplots.sty: #1tick set to \csname pgfplots@#1tick\endcsname [#1min=\csname pgfplots@#1min\endcsname, #1max=\csname pgfplots@#1max\endcsname].}%
}

% Helper method for 
%  \pgfplots@apply@data@scale@trafo@to@options@for
% #1: the ticks
% #2: the trafo macro name 
% #3: the output macro name
\long\def\pgfplots@apply@data@scale@trafo@to@user@ticks#1#2\to#3{%
	\let#3=\pgfutil@empty
	\foreach \pgfplots@loc@TMPb in {#1} {%
		\pgfmathfloatparsenumber{\pgfplots@loc@TMPb}%
		\expandafter#2\expandafter{\pgfmathresult}%
		\ifx#3\pgfutil@empty
			\xdef#3{\pgfmathresult}%
		\else
			\xdef#3{#3,\pgfmathresult}%
		\fi
	}%
	%
}%

% Helper method for 
%  \pgfplots@apply@data@scale@trafo@to@options@for
% #1: the ticks ALREADY IN FLOAT FORMAT
% #2: the trafo macro name 
% #3: the output macro name
\long\def\pgfplots@apply@data@scale@trafo@to@user@ticks@isfloat#1#2\to#3{%
	\let#3=\pgfutil@empty
	\foreach \pgfplots@loc@TMPb in {#1} {%
		#2{\pgfplots@loc@TMPb}%
		\ifx#3\pgfutil@empty
			\xdef#3{\pgfmathresult}%
		\else
			\xdef#3{#3,\pgfmathresult}%
		\fi
	}%
	%
}%
