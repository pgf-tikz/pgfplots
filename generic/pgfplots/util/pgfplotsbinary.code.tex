%--------------------------------------------
% $Header: /cvsroot/pgfplots/pgfplots/generic/pgfplots/util/pgfplotsbinary.code.tex,v 1.11 2009/04/01 18:10:40 ludewich Exp $
%
% Package pgfplots
%
% Provides a user-friendly interface to create function plots (normal
% plots, semi-logplots and double-logplots).
% 
% It is based on Till Tantau's PGF package.
%
% Copyright 2007/2008 by Christian Feuersänger.
%
% This program is free software: you can redistribute it and/or modify
% it under the terms of the GNU General Public License as published by
% the Free Software Foundation, either version 3 of the License, or
% (at your option) any later version.
% 
% This program is distributed in the hope that it will be useful,
% but WITHOUT ANY WARRANTY; without even the implied warranty of
% MERCHANTABILITY or FITNESS FOR A PARTICULAR PURPOSE.  See the
% GNU General Public License for more details.
% 
% You should have received a copy of the GNU General Public License
% along with this program.  If not, see <http://www.gnu.org/licenses/>.
%
%--------------------------------------------

% This file provides a self-contained package which does only need
% pgfkeys.
%
% It provides a method to convert TeX numbers (integers and
% dimensions) into binary format (macros with catcode 12).
%

\edef\pgfplotsbinaryatcode{\the\catcode`\@ }
\catcode`\@=11

\begingroup
% prepare ONE-BYTE conversion [0-255] -> binary character:
\count0=0
\catcode`\^^00=12	\expandafter\xdef\csname pgfp@bin@\the\count0 \endcsname{^^00}\advance\count0 by1
\catcode`\^^01=12	\expandafter\xdef\csname pgfp@bin@\the\count0 \endcsname{^^01}\advance\count0 by1
\catcode`\^^02=12	\expandafter\xdef\csname pgfp@bin@\the\count0 \endcsname{^^02}\advance\count0 by1
\catcode`\^^03=12	\expandafter\xdef\csname pgfp@bin@\the\count0 \endcsname{^^03}\advance\count0 by1
\catcode`\^^04=12	\expandafter\xdef\csname pgfp@bin@\the\count0 \endcsname{^^04}\advance\count0 by1
\catcode`\^^05=12	\expandafter\xdef\csname pgfp@bin@\the\count0 \endcsname{^^05}\advance\count0 by1
\catcode`\^^06=12	\expandafter\xdef\csname pgfp@bin@\the\count0 \endcsname{^^06}\advance\count0 by1
\catcode`\^^07=12	\expandafter\xdef\csname pgfp@bin@\the\count0 \endcsname{^^07}\advance\count0 by1
\catcode`\^^08=12	\expandafter\xdef\csname pgfp@bin@\the\count0 \endcsname{^^08}\advance\count0 by1
\begingroup
\catcode`\^^09=12	\expandafter\xdef\csname pgfp@bin@\the\count0 \endcsname{^^09}\endgroup\advance\count0 by1
\begingroup
\catcode`\^^0a=12\expandafter\xdef\csname pgfp@bin@\the\count0 \endcsname{^^0a}\endgroup\advance\count0 by1
\begingroup
\catcode`\^^0b=12	\expandafter\xdef\csname pgfp@bin@\the\count0 \endcsname{^^0b}\endgroup\advance\count0 by1
\catcode`\^^0c=12	\expandafter\xdef\csname pgfp@bin@\the\count0 \endcsname{^^0c}\advance\count0 by1
\begingroup
\catcode`\^^0d=12	\expandafter\xdef\csname pgfp@bin@\the\count0 \endcsname{^^0d}\endgroup\advance\count0 by1
\catcode`\^^0e=12	\expandafter\xdef\csname pgfp@bin@\the\count0 \endcsname{^^0e}\advance\count0 by1
\catcode`\^^0f=12	\expandafter\xdef\csname pgfp@bin@\the\count0 \endcsname{^^0f}\advance\count0 by1
\catcode`\^^10=12	\expandafter\xdef\csname pgfp@bin@\the\count0 \endcsname{^^10}\advance\count0 by1
\catcode`\^^11=12	\expandafter\xdef\csname pgfp@bin@\the\count0 \endcsname{^^11}\advance\count0 by1
\catcode`\^^12=12	\expandafter\xdef\csname pgfp@bin@\the\count0 \endcsname{^^12}\advance\count0 by1
\catcode`\^^13=12	\expandafter\xdef\csname pgfp@bin@\the\count0 \endcsname{^^13}\advance\count0 by1
\catcode`\^^14=12	\expandafter\xdef\csname pgfp@bin@\the\count0 \endcsname{^^14}\advance\count0 by1
\catcode`\^^15=12	\expandafter\xdef\csname pgfp@bin@\the\count0 \endcsname{^^15}\advance\count0 by1
\catcode`\^^16=12	\expandafter\xdef\csname pgfp@bin@\the\count0 \endcsname{^^16}\advance\count0 by1
\catcode`\^^17=12	\expandafter\xdef\csname pgfp@bin@\the\count0 \endcsname{^^17}\advance\count0 by1
\catcode`\^^18=12	\expandafter\xdef\csname pgfp@bin@\the\count0 \endcsname{^^18}\advance\count0 by1
\catcode`\^^19=12	\expandafter\xdef\csname pgfp@bin@\the\count0 \endcsname{^^19}\advance\count0 by1
\catcode`\^^1a=12	\expandafter\xdef\csname pgfp@bin@\the\count0 \endcsname{^^1a}\advance\count0 by1
\catcode`\^^1b=12	\expandafter\xdef\csname pgfp@bin@\the\count0 \endcsname{^^1b}\advance\count0 by1
\catcode`\^^1c=12	\expandafter\xdef\csname pgfp@bin@\the\count0 \endcsname{^^1c}\advance\count0 by1
\catcode`\^^1d=12	\expandafter\xdef\csname pgfp@bin@\the\count0 \endcsname{^^1d}\advance\count0 by1
\catcode`\^^1e=12	\expandafter\xdef\csname pgfp@bin@\the\count0 \endcsname{^^1e}\advance\count0 by1
\catcode`\^^1f=12	\expandafter\xdef\csname pgfp@bin@\the\count0 \endcsname{^^1f}\advance\count0 by1
\begingroup
\catcode`\^^20=12^^I\expandafter\xdef\csname^^Ipgfp@bin@\the\count0^^I\endcsname{^^20}\endgroup\advance\count0 by1
\catcode`\^^21=12	\expandafter\xdef\csname pgfp@bin@\the\count0 \endcsname{^^21}\advance\count0 by1
\catcode`\^^22=12	\expandafter\xdef\csname pgfp@bin@\the\count0 \endcsname{^^22}\advance\count0 by1
\catcode`\^^23=12	\expandafter\xdef\csname pgfp@bin@\the\count0 \endcsname{^^23}\advance\count0 by1
\catcode`\^^24=12	\expandafter\xdef\csname pgfp@bin@\the\count0 \endcsname{^^24}\advance\count0 by1
\catcode`\^^25=12	\expandafter\xdef\csname pgfp@bin@\the\count0 \endcsname{^^25}\advance\count0 by1
\catcode`\^^26=12	\expandafter\xdef\csname pgfp@bin@\the\count0 \endcsname{^^26}\advance\count0 by1
\catcode`\^^27=12	\expandafter\xdef\csname pgfp@bin@\the\count0 \endcsname{^^27}\advance\count0 by1
\catcode`\^^28=12	\expandafter\xdef\csname pgfp@bin@\the\count0 \endcsname{^^28}\advance\count0 by1
\catcode`\^^29=12	\expandafter\xdef\csname pgfp@bin@\the\count0 \endcsname{^^29}\advance\count0 by1
\catcode`\^^2a=12	\expandafter\xdef\csname pgfp@bin@\the\count0 \endcsname{^^2a}\advance\count0 by1
\catcode`\^^2b=12	\expandafter\xdef\csname pgfp@bin@\the\count0 \endcsname{^^2b}\advance\count0 by1
\catcode`\^^2c=12	\expandafter\xdef\csname pgfp@bin@\the\count0 \endcsname{^^2c}\advance\count0 by1
\catcode`\^^2d=12	\expandafter\xdef\csname pgfp@bin@\the\count0 \endcsname{^^2d}\advance\count0 by1
\catcode`\^^2e=12	\expandafter\xdef\csname pgfp@bin@\the\count0 \endcsname{^^2e}\advance\count0 by1
\catcode`\^^2f=12	\expandafter\xdef\csname pgfp@bin@\the\count0 \endcsname{^^2f}\advance\count0 by1
\catcode`\^^30=12	\expandafter\xdef\csname pgfp@bin@\the\count0 \endcsname{^^30}\advance\count0 by1
\catcode`\^^31=12	\expandafter\xdef\csname pgfp@bin@\the\count0 \endcsname{^^31}\advance\count0 by1
\catcode`\^^32=12	\expandafter\xdef\csname pgfp@bin@\the\count0 \endcsname{^^32}\advance\count0 by1
\catcode`\^^33=12	\expandafter\xdef\csname pgfp@bin@\the\count0 \endcsname{^^33}\advance\count0 by1
\catcode`\^^34=12	\expandafter\xdef\csname pgfp@bin@\the\count0 \endcsname{^^34}\advance\count0 by1
\catcode`\^^35=12	\expandafter\xdef\csname pgfp@bin@\the\count0 \endcsname{^^35}\advance\count0 by1
\catcode`\^^36=12	\expandafter\xdef\csname pgfp@bin@\the\count0 \endcsname{^^36}\advance\count0 by1
\catcode`\^^37=12	\expandafter\xdef\csname pgfp@bin@\the\count0 \endcsname{^^37}\advance\count0 by1
\catcode`\^^38=12	\expandafter\xdef\csname pgfp@bin@\the\count0 \endcsname{^^38}\advance\count0 by1
\catcode`\^^39=12	\expandafter\xdef\csname pgfp@bin@\the\count0 \endcsname{^^39}\advance\count0 by1
\catcode`\^^3a=12	\expandafter\xdef\csname pgfp@bin@\the\count0 \endcsname{^^3a}\advance\count0 by1
\catcode`\^^3b=12	\expandafter\xdef\csname pgfp@bin@\the\count0 \endcsname{^^3b}\advance\count0 by1
\catcode`\^^3c=12	\expandafter\xdef\csname pgfp@bin@\the\count0 \endcsname{^^3c}\advance\count0 by1
\catcode`\^^3d=12	\expandafter\xdef\csname pgfp@bin@\the\count0 \endcsname{^^3d}\advance\count0 by1
\catcode`\^^3e=12	\expandafter\xdef\csname pgfp@bin@\the\count0 \endcsname{^^3e}\advance\count0 by1
\catcode`\^^3f=12	\expandafter\xdef\csname pgfp@bin@\the\count0 \endcsname{^^3f}\advance\count0 by1
\catcode`\^^40=12	\expandafter\xdef\csname pgfp@bin@\the\count0 \endcsname{^^40}\advance\count0 by1
\catcode`\^^41=12	\expandafter\xdef\csname pgfp@bin@\the\count0 \endcsname{^^41}\advance\count0 by1
\catcode`\^^42=12	\expandafter\xdef\csname pgfp@bin@\the\count0 \endcsname{^^42}\advance\count0 by1
\catcode`\^^43=12	\expandafter\xdef\csname pgfp@bin@\the\count0 \endcsname{^^43}\advance\count0 by1
\catcode`\^^44=12	\expandafter\xdef\csname pgfp@bin@\the\count0 \endcsname{^^44}\advance\count0 by1
\catcode`\^^45=12	\expandafter\xdef\csname pgfp@bin@\the\count0 \endcsname{^^45}\advance\count0 by1
\catcode`\^^46=12	\expandafter\xdef\csname pgfp@bin@\the\count0 \endcsname{^^46}\advance\count0 by1
\catcode`\^^47=12	\expandafter\xdef\csname pgfp@bin@\the\count0 \endcsname{^^47}\advance\count0 by1
\catcode`\^^48=12	\expandafter\xdef\csname pgfp@bin@\the\count0 \endcsname{^^48}\advance\count0 by1
\catcode`\^^49=12	\expandafter\xdef\csname pgfp@bin@\the\count0 \endcsname{^^49}\advance\count0 by1
\catcode`\^^4a=12	\expandafter\xdef\csname pgfp@bin@\the\count0 \endcsname{^^4a}\advance\count0 by1
\catcode`\^^4b=12	\expandafter\xdef\csname pgfp@bin@\the\count0 \endcsname{^^4b}\advance\count0 by1
\catcode`\^^4c=12	\expandafter\xdef\csname pgfp@bin@\the\count0 \endcsname{^^4c}\advance\count0 by1
\catcode`\^^4d=12	\expandafter\xdef\csname pgfp@bin@\the\count0 \endcsname{^^4d}\advance\count0 by1
\catcode`\^^4e=12	\expandafter\xdef\csname pgfp@bin@\the\count0 \endcsname{^^4e}\advance\count0 by1
\catcode`\^^4f=12	\expandafter\xdef\csname pgfp@bin@\the\count0 \endcsname{^^4f}\advance\count0 by1
\catcode`\^^50=12	\expandafter\xdef\csname pgfp@bin@\the\count0 \endcsname{^^50}\advance\count0 by1
\catcode`\^^51=12	\expandafter\xdef\csname pgfp@bin@\the\count0 \endcsname{^^51}\advance\count0 by1
\catcode`\^^52=12	\expandafter\xdef\csname pgfp@bin@\the\count0 \endcsname{^^52}\advance\count0 by1
\catcode`\^^53=12	\expandafter\xdef\csname pgfp@bin@\the\count0 \endcsname{^^53}\advance\count0 by1
\catcode`\^^54=12	\expandafter\xdef\csname pgfp@bin@\the\count0 \endcsname{^^54}\advance\count0 by1
\catcode`\^^55=12	\expandafter\xdef\csname pgfp@bin@\the\count0 \endcsname{^^55}\advance\count0 by1
\catcode`\^^56=12	\expandafter\xdef\csname pgfp@bin@\the\count0 \endcsname{^^56}\advance\count0 by1
\catcode`\^^57=12	\expandafter\xdef\csname pgfp@bin@\the\count0 \endcsname{^^57}\advance\count0 by1
\catcode`\^^58=12	\expandafter\xdef\csname pgfp@bin@\the\count0 \endcsname{^^58}\advance\count0 by1
\catcode`\^^59=12	\expandafter\xdef\csname pgfp@bin@\the\count0 \endcsname{^^59}\advance\count0 by1
\catcode`\^^5a=12	\expandafter\xdef\csname pgfp@bin@\the\count0 \endcsname{^^5a}\advance\count0 by1
\catcode`\^^5b=12	\expandafter\xdef\csname pgfp@bin@\the\count0 \endcsname{^^5b}\advance\count0 by1
\begingroup
\catcode`\|=0
\catcode`\\=12	|expandafter|xdef|csname pgfp@bin@|the|count0 |endcsname{^^5c}|endgroup\advance\count0 by1
\catcode`\^^5d=12	\expandafter\xdef\csname pgfp@bin@\the\count0 \endcsname{^^5d}\advance\count0 by1
\begingroup
\catcode`\^^5e=12	\expandafter\xdef\csname pgfp@bin@\the\count0 \endcsname{^}\endgroup\advance\count0 by1
\catcode`\^^5f=12	\expandafter\xdef\csname pgfp@bin@\the\count0 \endcsname{^^5f}\advance\count0 by1
\catcode`\^^60=12	\expandafter\xdef\csname pgfp@bin@\the\count0 \endcsname{^^60}\advance\count0 by1
\catcode`\^^61=11	\expandafter\xdef\csname pgfp@bin@\the\count0 \endcsname{^^61}\advance\count0 by1
\catcode`\^^62=11	\expandafter\xdef\csname pgfp@bin@\the\count0 \endcsname{^^62}\advance\count0 by1
\catcode`\^^63=11	\expandafter\xdef\csname pgfp@bin@\the\count0 \endcsname{^^63}\advance\count0 by1
\catcode`\^^64=11	\expandafter\xdef\csname pgfp@bin@\the\count0 \endcsname{^^64}\advance\count0 by1
\catcode`\^^65=11	\expandafter\xdef\csname pgfp@bin@\the\count0 \endcsname{^^65}\advance\count0 by1
\catcode`\^^66=11	\expandafter\xdef\csname pgfp@bin@\the\count0 \endcsname{^^66}\advance\count0 by1
\catcode`\^^67=11	\expandafter\xdef\csname pgfp@bin@\the\count0 \endcsname{^^67}\advance\count0 by1
\catcode`\^^68=11	\expandafter\xdef\csname pgfp@bin@\the\count0 \endcsname{^^68}\advance\count0 by1
\catcode`\^^69=11	\expandafter\xdef\csname pgfp@bin@\the\count0 \endcsname{^^69}\advance\count0 by1
\catcode`\^^6a=11	\expandafter\xdef\csname pgfp@bin@\the\count0 \endcsname{^^6a}\advance\count0 by1
\catcode`\^^6b=11	\expandafter\xdef\csname pgfp@bin@\the\count0 \endcsname{^^6b}\advance\count0 by1
\catcode`\^^6c=11	\expandafter\xdef\csname pgfp@bin@\the\count0 \endcsname{^^6c}\advance\count0 by1
\catcode`\^^6d=11	\expandafter\xdef\csname pgfp@bin@\the\count0 \endcsname{^^6d}\advance\count0 by1
\catcode`\^^6e=11	\expandafter\xdef\csname pgfp@bin@\the\count0 \endcsname{^^6e}\advance\count0 by1
\catcode`\^^6f=11	\expandafter\xdef\csname pgfp@bin@\the\count0 \endcsname{^^6f}\advance\count0 by1
\catcode`\^^70=11	\expandafter\xdef\csname pgfp@bin@\the\count0 \endcsname{^^70}\advance\count0 by1
\catcode`\^^71=11	\expandafter\xdef\csname pgfp@bin@\the\count0 \endcsname{^^71}\advance\count0 by1
\catcode`\^^72=11	\expandafter\xdef\csname pgfp@bin@\the\count0 \endcsname{^^72}\advance\count0 by1
\catcode`\^^73=11	\expandafter\xdef\csname pgfp@bin@\the\count0 \endcsname{^^73}\advance\count0 by1
\catcode`\^^74=11	\expandafter\xdef\csname pgfp@bin@\the\count0 \endcsname{^^74}\advance\count0 by1
\catcode`\^^75=11	\expandafter\xdef\csname pgfp@bin@\the\count0 \endcsname{^^75}\advance\count0 by1
\catcode`\^^76=11	\expandafter\xdef\csname pgfp@bin@\the\count0 \endcsname{^^76}\advance\count0 by1
\catcode`\^^77=11	\expandafter\xdef\csname pgfp@bin@\the\count0 \endcsname{^^77}\advance\count0 by1
\catcode`\^^78=11	\expandafter\xdef\csname pgfp@bin@\the\count0 \endcsname{^^78}\advance\count0 by1
\catcode`\^^79=11	\expandafter\xdef\csname pgfp@bin@\the\count0 \endcsname{^^79}\advance\count0 by1
\catcode`\^^7a=11	\expandafter\xdef\csname pgfp@bin@\the\count0 \endcsname{^^7a}\advance\count0 by1
\begingroup
\catcode`\q=1
\catcode`\^^7b=12	\expandafter\xdef\csname pgfp@bin@\the\count0 \endcsname q^^7b}\endgroup\advance\count0 by1
\catcode`\^^7c=12	\expandafter\xdef\csname pgfp@bin@\the\count0 \endcsname{^^7c}\advance\count0 by1
\begingroup
\catcode`\q=2
\catcode`\^^7d=12	\expandafter\xdef\csname pgfp@bin@\the\count0 \endcsname{^^7dq\endgroup\advance\count0 by1
\catcode`\^^7e=12	\expandafter\xdef\csname pgfp@bin@\the\count0 \endcsname{^^7e}\advance\count0 by1
\catcode`\^^7f=12	\expandafter\xdef\csname pgfp@bin@\the\count0 \endcsname{^^7f}\advance\count0 by1
\catcode`\^^80=12	\expandafter\xdef\csname pgfp@bin@\the\count0 \endcsname{^^80}\advance\count0 by1
\catcode`\^^81=12	\expandafter\xdef\csname pgfp@bin@\the\count0 \endcsname{^^81}\advance\count0 by1
\catcode`\^^82=12	\expandafter\xdef\csname pgfp@bin@\the\count0 \endcsname{^^82}\advance\count0 by1
\catcode`\^^83=12	\expandafter\xdef\csname pgfp@bin@\the\count0 \endcsname{^^83}\advance\count0 by1
\catcode`\^^84=12	\expandafter\xdef\csname pgfp@bin@\the\count0 \endcsname{^^84}\advance\count0 by1
\catcode`\^^85=12	\expandafter\xdef\csname pgfp@bin@\the\count0 \endcsname{^^85}\advance\count0 by1
\catcode`\^^86=12	\expandafter\xdef\csname pgfp@bin@\the\count0 \endcsname{^^86}\advance\count0 by1
\catcode`\^^87=12	\expandafter\xdef\csname pgfp@bin@\the\count0 \endcsname{^^87}\advance\count0 by1
\catcode`\^^88=12	\expandafter\xdef\csname pgfp@bin@\the\count0 \endcsname{^^88}\advance\count0 by1
\catcode`\^^89=12	\expandafter\xdef\csname pgfp@bin@\the\count0 \endcsname{^^89}\advance\count0 by1
\catcode`\^^8a=12	\expandafter\xdef\csname pgfp@bin@\the\count0 \endcsname{^^8a}\advance\count0 by1
\catcode`\^^8b=12	\expandafter\xdef\csname pgfp@bin@\the\count0 \endcsname{^^8b}\advance\count0 by1
\catcode`\^^8c=12	\expandafter\xdef\csname pgfp@bin@\the\count0 \endcsname{^^8c}\advance\count0 by1
\catcode`\^^8d=12	\expandafter\xdef\csname pgfp@bin@\the\count0 \endcsname{^^8d}\advance\count0 by1
\catcode`\^^8e=12	\expandafter\xdef\csname pgfp@bin@\the\count0 \endcsname{^^8e}\advance\count0 by1
\catcode`\^^8f=12	\expandafter\xdef\csname pgfp@bin@\the\count0 \endcsname{^^8f}\advance\count0 by1
\catcode`\^^90=12	\expandafter\xdef\csname pgfp@bin@\the\count0 \endcsname{^^90}\advance\count0 by1
\catcode`\^^91=12	\expandafter\xdef\csname pgfp@bin@\the\count0 \endcsname{^^91}\advance\count0 by1
\catcode`\^^92=12	\expandafter\xdef\csname pgfp@bin@\the\count0 \endcsname{^^92}\advance\count0 by1
\catcode`\^^93=12	\expandafter\xdef\csname pgfp@bin@\the\count0 \endcsname{^^93}\advance\count0 by1
\catcode`\^^94=12	\expandafter\xdef\csname pgfp@bin@\the\count0 \endcsname{^^94}\advance\count0 by1
\catcode`\^^95=12	\expandafter\xdef\csname pgfp@bin@\the\count0 \endcsname{^^95}\advance\count0 by1
\catcode`\^^96=12	\expandafter\xdef\csname pgfp@bin@\the\count0 \endcsname{^^96}\advance\count0 by1
\catcode`\^^97=12	\expandafter\xdef\csname pgfp@bin@\the\count0 \endcsname{^^97}\advance\count0 by1
\catcode`\^^98=12	\expandafter\xdef\csname pgfp@bin@\the\count0 \endcsname{^^98}\advance\count0 by1
\catcode`\^^99=12	\expandafter\xdef\csname pgfp@bin@\the\count0 \endcsname{^^99}\advance\count0 by1
\catcode`\^^9a=12	\expandafter\xdef\csname pgfp@bin@\the\count0 \endcsname{^^9a}\advance\count0 by1
\catcode`\^^9b=12	\expandafter\xdef\csname pgfp@bin@\the\count0 \endcsname{^^9b}\advance\count0 by1
\catcode`\^^9c=12	\expandafter\xdef\csname pgfp@bin@\the\count0 \endcsname{^^9c}\advance\count0 by1
\catcode`\^^9d=12	\expandafter\xdef\csname pgfp@bin@\the\count0 \endcsname{^^9d}\advance\count0 by1
\catcode`\^^9e=12	\expandafter\xdef\csname pgfp@bin@\the\count0 \endcsname{^^9e}\advance\count0 by1
\catcode`\^^9f=12	\expandafter\xdef\csname pgfp@bin@\the\count0 \endcsname{^^9f}\advance\count0 by1
\catcode`\^^a0=12	\expandafter\xdef\csname pgfp@bin@\the\count0 \endcsname{^^a0}\advance\count0 by1
\catcode`\^^a1=12	\expandafter\xdef\csname pgfp@bin@\the\count0 \endcsname{^^a1}\advance\count0 by1
\catcode`\^^a2=12	\expandafter\xdef\csname pgfp@bin@\the\count0 \endcsname{^^a2}\advance\count0 by1
\catcode`\^^a3=12	\expandafter\xdef\csname pgfp@bin@\the\count0 \endcsname{^^a3}\advance\count0 by1
\catcode`\^^a4=12	\expandafter\xdef\csname pgfp@bin@\the\count0 \endcsname{^^a4}\advance\count0 by1
\catcode`\^^a5=12	\expandafter\xdef\csname pgfp@bin@\the\count0 \endcsname{^^a5}\advance\count0 by1
\catcode`\^^a6=12	\expandafter\xdef\csname pgfp@bin@\the\count0 \endcsname{^^a6}\advance\count0 by1
\catcode`\^^a7=12	\expandafter\xdef\csname pgfp@bin@\the\count0 \endcsname{^^a7}\advance\count0 by1
\catcode`\^^a8=12	\expandafter\xdef\csname pgfp@bin@\the\count0 \endcsname{^^a8}\advance\count0 by1
\catcode`\^^a9=12	\expandafter\xdef\csname pgfp@bin@\the\count0 \endcsname{^^a9}\advance\count0 by1
\catcode`\^^aa=12	\expandafter\xdef\csname pgfp@bin@\the\count0 \endcsname{^^aa}\advance\count0 by1
\catcode`\^^ab=12	\expandafter\xdef\csname pgfp@bin@\the\count0 \endcsname{^^ab}\advance\count0 by1
\catcode`\^^ac=12	\expandafter\xdef\csname pgfp@bin@\the\count0 \endcsname{^^ac}\advance\count0 by1
\catcode`\^^ad=12	\expandafter\xdef\csname pgfp@bin@\the\count0 \endcsname{^^ad}\advance\count0 by1
\catcode`\^^ae=12	\expandafter\xdef\csname pgfp@bin@\the\count0 \endcsname{^^ae}\advance\count0 by1
\catcode`\^^af=12	\expandafter\xdef\csname pgfp@bin@\the\count0 \endcsname{^^af}\advance\count0 by1
\catcode`\^^b0=12	\expandafter\xdef\csname pgfp@bin@\the\count0 \endcsname{^^b0}\advance\count0 by1
\catcode`\^^b1=12	\expandafter\xdef\csname pgfp@bin@\the\count0 \endcsname{^^b1}\advance\count0 by1
\catcode`\^^b2=12	\expandafter\xdef\csname pgfp@bin@\the\count0 \endcsname{^^b2}\advance\count0 by1
\catcode`\^^b3=12	\expandafter\xdef\csname pgfp@bin@\the\count0 \endcsname{^^b3}\advance\count0 by1
\catcode`\^^b4=12	\expandafter\xdef\csname pgfp@bin@\the\count0 \endcsname{^^b4}\advance\count0 by1
\catcode`\^^b5=12	\expandafter\xdef\csname pgfp@bin@\the\count0 \endcsname{^^b5}\advance\count0 by1
\catcode`\^^b6=12	\expandafter\xdef\csname pgfp@bin@\the\count0 \endcsname{^^b6}\advance\count0 by1
\catcode`\^^b7=12	\expandafter\xdef\csname pgfp@bin@\the\count0 \endcsname{^^b7}\advance\count0 by1
\catcode`\^^b8=12	\expandafter\xdef\csname pgfp@bin@\the\count0 \endcsname{^^b8}\advance\count0 by1
\catcode`\^^b9=12	\expandafter\xdef\csname pgfp@bin@\the\count0 \endcsname{^^b9}\advance\count0 by1
\catcode`\^^ba=12	\expandafter\xdef\csname pgfp@bin@\the\count0 \endcsname{^^ba}\advance\count0 by1
\catcode`\^^bb=12	\expandafter\xdef\csname pgfp@bin@\the\count0 \endcsname{^^bb}\advance\count0 by1
\catcode`\^^bc=12	\expandafter\xdef\csname pgfp@bin@\the\count0 \endcsname{^^bc}\advance\count0 by1
\catcode`\^^bd=12	\expandafter\xdef\csname pgfp@bin@\the\count0 \endcsname{^^bd}\advance\count0 by1
\catcode`\^^be=12	\expandafter\xdef\csname pgfp@bin@\the\count0 \endcsname{^^be}\advance\count0 by1
\catcode`\^^bf=12	\expandafter\xdef\csname pgfp@bin@\the\count0 \endcsname{^^bf}\advance\count0 by1
\catcode`\^^c0=12	\expandafter\xdef\csname pgfp@bin@\the\count0 \endcsname{^^c0}\advance\count0 by1
\catcode`\^^c1=12	\expandafter\xdef\csname pgfp@bin@\the\count0 \endcsname{^^c1}\advance\count0 by1
\catcode`\^^c2=12	\expandafter\xdef\csname pgfp@bin@\the\count0 \endcsname{^^c2}\advance\count0 by1
\catcode`\^^c3=12	\expandafter\xdef\csname pgfp@bin@\the\count0 \endcsname{^^c3}\advance\count0 by1
\catcode`\^^c4=12	\expandafter\xdef\csname pgfp@bin@\the\count0 \endcsname{^^c4}\advance\count0 by1
\catcode`\^^c5=12	\expandafter\xdef\csname pgfp@bin@\the\count0 \endcsname{^^c5}\advance\count0 by1
\catcode`\^^c6=12	\expandafter\xdef\csname pgfp@bin@\the\count0 \endcsname{^^c6}\advance\count0 by1
\catcode`\^^c7=12	\expandafter\xdef\csname pgfp@bin@\the\count0 \endcsname{^^c7}\advance\count0 by1
\catcode`\^^c8=12	\expandafter\xdef\csname pgfp@bin@\the\count0 \endcsname{^^c8}\advance\count0 by1
\catcode`\^^c9=12	\expandafter\xdef\csname pgfp@bin@\the\count0 \endcsname{^^c9}\advance\count0 by1
\catcode`\^^ca=12	\expandafter\xdef\csname pgfp@bin@\the\count0 \endcsname{^^ca}\advance\count0 by1
\catcode`\^^cb=12	\expandafter\xdef\csname pgfp@bin@\the\count0 \endcsname{^^cb}\advance\count0 by1
\catcode`\^^cc=12	\expandafter\xdef\csname pgfp@bin@\the\count0 \endcsname{^^cc}\advance\count0 by1
\catcode`\^^cd=12	\expandafter\xdef\csname pgfp@bin@\the\count0 \endcsname{^^cd}\advance\count0 by1
\catcode`\^^ce=12	\expandafter\xdef\csname pgfp@bin@\the\count0 \endcsname{^^ce}\advance\count0 by1
\catcode`\^^cf=12	\expandafter\xdef\csname pgfp@bin@\the\count0 \endcsname{^^cf}\advance\count0 by1
\catcode`\^^d0=12	\expandafter\xdef\csname pgfp@bin@\the\count0 \endcsname{^^d0}\advance\count0 by1
\catcode`\^^d1=12	\expandafter\xdef\csname pgfp@bin@\the\count0 \endcsname{^^d1}\advance\count0 by1
\catcode`\^^d2=12	\expandafter\xdef\csname pgfp@bin@\the\count0 \endcsname{^^d2}\advance\count0 by1
\catcode`\^^d3=12	\expandafter\xdef\csname pgfp@bin@\the\count0 \endcsname{^^d3}\advance\count0 by1
\catcode`\^^d4=12	\expandafter\xdef\csname pgfp@bin@\the\count0 \endcsname{^^d4}\advance\count0 by1
\catcode`\^^d5=12	\expandafter\xdef\csname pgfp@bin@\the\count0 \endcsname{^^d5}\advance\count0 by1
\catcode`\^^d6=12	\expandafter\xdef\csname pgfp@bin@\the\count0 \endcsname{^^d6}\advance\count0 by1
\catcode`\^^d7=12	\expandafter\xdef\csname pgfp@bin@\the\count0 \endcsname{^^d7}\advance\count0 by1
\catcode`\^^d8=12	\expandafter\xdef\csname pgfp@bin@\the\count0 \endcsname{^^d8}\advance\count0 by1
\catcode`\^^d9=12	\expandafter\xdef\csname pgfp@bin@\the\count0 \endcsname{^^d9}\advance\count0 by1
\catcode`\^^da=12	\expandafter\xdef\csname pgfp@bin@\the\count0 \endcsname{^^da}\advance\count0 by1
\catcode`\^^db=12	\expandafter\xdef\csname pgfp@bin@\the\count0 \endcsname{^^db}\advance\count0 by1
\catcode`\^^dc=12	\expandafter\xdef\csname pgfp@bin@\the\count0 \endcsname{^^dc}\advance\count0 by1
\catcode`\^^dd=12	\expandafter\xdef\csname pgfp@bin@\the\count0 \endcsname{^^dd}\advance\count0 by1
\catcode`\^^de=12	\expandafter\xdef\csname pgfp@bin@\the\count0 \endcsname{^^de}\advance\count0 by1
\catcode`\^^df=12	\expandafter\xdef\csname pgfp@bin@\the\count0 \endcsname{^^df}\advance\count0 by1
\catcode`\^^e0=12	\expandafter\xdef\csname pgfp@bin@\the\count0 \endcsname{^^e0}\advance\count0 by1
\catcode`\^^e1=12	\expandafter\xdef\csname pgfp@bin@\the\count0 \endcsname{^^e1}\advance\count0 by1
\catcode`\^^e2=12	\expandafter\xdef\csname pgfp@bin@\the\count0 \endcsname{^^e2}\advance\count0 by1
\catcode`\^^e3=12	\expandafter\xdef\csname pgfp@bin@\the\count0 \endcsname{^^e3}\advance\count0 by1
\catcode`\^^e4=12	\expandafter\xdef\csname pgfp@bin@\the\count0 \endcsname{^^e4}\advance\count0 by1
\catcode`\^^e5=12	\expandafter\xdef\csname pgfp@bin@\the\count0 \endcsname{^^e5}\advance\count0 by1
\catcode`\^^e6=12	\expandafter\xdef\csname pgfp@bin@\the\count0 \endcsname{^^e6}\advance\count0 by1
\catcode`\^^e7=12	\expandafter\xdef\csname pgfp@bin@\the\count0 \endcsname{^^e7}\advance\count0 by1
\catcode`\^^e8=12	\expandafter\xdef\csname pgfp@bin@\the\count0 \endcsname{^^e8}\advance\count0 by1
\catcode`\^^e9=12	\expandafter\xdef\csname pgfp@bin@\the\count0 \endcsname{^^e9}\advance\count0 by1
\catcode`\^^ea=12	\expandafter\xdef\csname pgfp@bin@\the\count0 \endcsname{^^ea}\advance\count0 by1
\catcode`\^^eb=12	\expandafter\xdef\csname pgfp@bin@\the\count0 \endcsname{^^eb}\advance\count0 by1
\catcode`\^^ec=12	\expandafter\xdef\csname pgfp@bin@\the\count0 \endcsname{^^ec}\advance\count0 by1
\catcode`\^^ed=12	\expandafter\xdef\csname pgfp@bin@\the\count0 \endcsname{^^ed}\advance\count0 by1
\catcode`\^^ee=12	\expandafter\xdef\csname pgfp@bin@\the\count0 \endcsname{^^ee}\advance\count0 by1
\catcode`\^^ef=12	\expandafter\xdef\csname pgfp@bin@\the\count0 \endcsname{^^ef}\advance\count0 by1
\catcode`\^^f0=12	\expandafter\xdef\csname pgfp@bin@\the\count0 \endcsname{^^f0}\advance\count0 by1
\catcode`\^^f1=12	\expandafter\xdef\csname pgfp@bin@\the\count0 \endcsname{^^f1}\advance\count0 by1
\catcode`\^^f2=12	\expandafter\xdef\csname pgfp@bin@\the\count0 \endcsname{^^f2}\advance\count0 by1
\catcode`\^^f3=12	\expandafter\xdef\csname pgfp@bin@\the\count0 \endcsname{^^f3}\advance\count0 by1
\catcode`\^^f4=12	\expandafter\xdef\csname pgfp@bin@\the\count0 \endcsname{^^f4}\advance\count0 by1
\catcode`\^^f5=12	\expandafter\xdef\csname pgfp@bin@\the\count0 \endcsname{^^f5}\advance\count0 by1
\catcode`\^^f6=12	\expandafter\xdef\csname pgfp@bin@\the\count0 \endcsname{^^f6}\advance\count0 by1
\catcode`\^^f7=12	\expandafter\xdef\csname pgfp@bin@\the\count0 \endcsname{^^f7}\advance\count0 by1
\catcode`\^^f8=12	\expandafter\xdef\csname pgfp@bin@\the\count0 \endcsname{^^f8}\advance\count0 by1
\catcode`\^^f9=12	\expandafter\xdef\csname pgfp@bin@\the\count0 \endcsname{^^f9}\advance\count0 by1
\catcode`\^^fa=12	\expandafter\xdef\csname pgfp@bin@\the\count0 \endcsname{^^fa}\advance\count0 by1
\catcode`\^^fb=12	\expandafter\xdef\csname pgfp@bin@\the\count0 \endcsname{^^fb}\advance\count0 by1
\catcode`\^^fc=12	\expandafter\xdef\csname pgfp@bin@\the\count0 \endcsname{^^fc}\advance\count0 by1
\catcode`\^^fd=12	\expandafter\xdef\csname pgfp@bin@\the\count0 \endcsname{^^fd}\advance\count0 by1
\catcode`\^^fe=12	\expandafter\xdef\csname pgfp@bin@\the\count0 \endcsname{^^fe}\advance\count0 by1
\catcode`\^^ff=12	\expandafter\xdef\csname pgfp@bin@\the\count0 \endcsname{^^ff}\advance\count0 by1
\ifnum\count0=256
\else
	\pgfplots@error{Internal logic Error: initialisation failed. I have only \the\count0\space bytes initialised.}
\fi
\endgroup

\def\pgfplotsbinary@apphighorderbytes@BIGENDIAN#1{\xdef\pgfplotsbinaryresult{#1\pgfplotsbinaryresult}}%
\def\pgfplotsbinary@apphighorderbytes@LITTLEENDIAN#1{\xdef\pgfplotsbinaryresult{\pgfplotsbinaryresult#1}}%

\def\pgfplotsbinaryencode@badic@unsigned@PAD@LITTLEENDIAN{%
	% pad with zeros:
	\ifcase\c@pgfplotsbin@byteno
		% ok.
	\or
		% one byte missing.
		\pgfplotsbinary@apphighorderbytes{\pgfplotsbinary@ZERO@HIGHEST}%
	\or
		% two bytes missing.
		\pgfplotsbinary@apphighorderbytes{\pgfplotsbinary@ZERO\pgfplotsbinary@ZERO@HIGHEST}%
	\or
		\pgfplotsbinary@apphighorderbytes{\pgfplotsbinary@ZERO\pgfplotsbinary@ZERO\pgfplotsbinary@ZERO@HIGHEST}%
	\or
		\pgfplotsbinary@apphighorderbytes{\pgfplotsbinary@ZERO\pgfplotsbinary@ZERO\pgfplotsbinary@ZERO\pgfplotsbinary@ZERO@HIGHEST}%
	\or
		\pgfplotsbinary@apphighorderbytes{\pgfplotsbinary@ZERO\pgfplotsbinary@ZERO\pgfplotsbinary@ZERO\pgfplotsbinary@ZERO\pgfplotsbinary@ZERO@HIGHEST}%
	\or
		\pgfplotsbinary@apphighorderbytes{\pgfplotsbinary@ZERO\pgfplotsbinary@ZERO\pgfplotsbinary@ZERO\pgfplotsbinary@ZERO\pgfplotsbinary@ZERO\pgfplotsbinary@ZERO@HIGHEST}%
	\or
		\pgfplotsbinary@apphighorderbytes{\pgfplotsbinary@ZERO\pgfplotsbinary@ZERO\pgfplotsbinary@ZERO\pgfplotsbinary@ZERO\pgfplotsbinary@ZERO\pgfplotsbinary@ZERO\pgfplotsbinary@ZERO@HIGHEST}%
	\or
		\pgfplotsbinary@apphighorderbytes{\pgfplotsbinary@ZERO\pgfplotsbinary@ZERO\pgfplotsbinary@ZERO\pgfplotsbinary@ZERO\pgfplotsbinary@ZERO\pgfplotsbinary@ZERO\pgfplotsbinary@ZERO\pgfplotsbinary@ZERO@HIGHEST}%
	\else
		\pgfplots@error{Sorry, I can't process byte no \the\c@pgfplotsbin@byteno... you may need to change bytes=\pgfplotsbinary@bytes.}%
	\fi
}%
\def\pgfplotsbinaryencode@badic@unsigned@PAD@BIGENDIAN{%
	% pad with zeros:
	\ifcase\c@pgfplotsbin@byteno
		% ok.
	\or
		% one byte missing.
		\pgfplotsbinary@apphighorderbytes{\pgfplotsbinary@ZERO@HIGHEST}%
	\or
		% two bytes missing.
		\pgfplotsbinary@apphighorderbytes{\pgfplotsbinary@ZERO@HIGHEST\pgfplotsbinary@ZERO}%
	\or
		\pgfplotsbinary@apphighorderbytes{\pgfplotsbinary@ZERO@HIGHEST\pgfplotsbinary@ZERO\pgfplotsbinary@ZERO}%
	\or
		\pgfplotsbinary@apphighorderbytes{\pgfplotsbinary@ZERO@HIGHEST\pgfplotsbinary@ZERO\pgfplotsbinary@ZERO\pgfplotsbinary@ZERO}%
	\or
		\pgfplotsbinary@apphighorderbytes{\pgfplotsbinary@ZERO@HIGHEST\pgfplotsbinary@ZERO\pgfplotsbinary@ZERO\pgfplotsbinary@ZERO\pgfplotsbinary@ZERO}%
	\or
		\pgfplotsbinary@apphighorderbytes{\pgfplotsbinary@ZERO@HIGHEST\pgfplotsbinary@ZERO\pgfplotsbinary@ZERO\pgfplotsbinary@ZERO\pgfplotsbinary@ZERO\pgfplotsbinary@ZERO}%
	\or
		\pgfplotsbinary@apphighorderbytes{\pgfplotsbinary@ZERO@HIGHEST\pgfplotsbinary@ZERO\pgfplotsbinary@ZERO\pgfplotsbinary@ZERO\pgfplotsbinary@ZERO\pgfplotsbinary@ZERO\pgfplotsbinary@ZERO}%
	\or
		\pgfplotsbinary@apphighorderbytes{\pgfplotsbinary@ZERO@HIGHEST\pgfplotsbinary@ZERO\pgfplotsbinary@ZERO\pgfplotsbinary@ZERO\pgfplotsbinary@ZERO\pgfplotsbinary@ZERO\pgfplotsbinary@ZERO\pgfplotsbinary@ZERO}%
	\else
		\pgfplots@error{Sorry, I can't process byte no \the\c@pgfplotsbin@byteno... you may need to change bytes=\pgfplotsbinary@bytes.}%
	\fi
}%

\expandafter\def\csname pgfplotsbinarysetbytes@1\endcsname{%
	\def\pgfplotsbinary@add@signed@largest@absolute{\advance\c@pgfplotsbin@input by 127 }%
	\def\pgfplotsbinaryencodesignedmaplinearly@prepare{%
		% warning: \pgfplotsbinary@bytes is NOT necessarily 1 (ASCII
		% encoding features)
		\c@pgfplotsbin@byteno=\pgfplotsbinary@bytes\relax
		\c@pgfplotsbin@basis=\pgfplotsbinary@basis\relax
		\divide\c@pgfplotsbin@input by16909320 % ~= (2^31-1) /  (2^(8*1-1) -1)
	}%
}%
\expandafter\def\csname pgfplotsbinarysetbytes@2\endcsname{%
	\def\pgfplotsbinary@add@signed@largest@absolute{\advance\c@pgfplotsbin@input by 32767 }%
	\def\pgfplotsbinaryencodesignedmaplinearly@prepare{%
		\c@pgfplotsbin@byteno=\pgfplotsbinary@bytes\relax
		\c@pgfplotsbin@basis=\pgfplotsbinary@basis\relax
		\divide\c@pgfplotsbin@input by65538 % ~= (2^31-1) /  (2^(8*2-1) -1)
	}%
}%
\expandafter\def\csname pgfplotsbinarysetbytes@3\endcsname{%
	\def\pgfplotsbinary@add@signed@largest@absolute{\advance\c@pgfplotsbin@input by 8388607 }%
	\def\pgfplotsbinaryencodesignedmaplinearly@prepare{%
		\c@pgfplotsbin@byteno=\pgfplotsbinary@bytes\relax
		\c@pgfplotsbin@basis=\pgfplotsbinary@basis\relax
		\divide\c@pgfplotsbin@input by256 % ~= (2^31-1) /  (2^(8*3-1) -1)
	}%
}%
\expandafter\def\csname pgfplotsbinarysetbytes@4\endcsname{%
	\def\pgfplotsbinary@add@signed@largest@absolute{%
		\advance\c@pgfplotsbin@input by 2147483647 % this is the *absolute* largest int that TeX can handle.
	}%
	\def\pgfplotsbinaryencodesignedmaplinearly@prepare{%
		\c@pgfplotsbin@byteno=\pgfplotsbinary@bytes\relax
		\c@pgfplotsbin@basis=\pgfplotsbinary@basis\relax
	}%
}%
\def\pgfplotsbinarysetbytes@@{%
	\def\pgfplotsbinaryencodesignedmaplinearly@prepare{%
		\c@pgfplotsbin@byteno=\pgfplotsbinary@bytes\relax
		\c@pgfplotsbin@basis=\pgfplotsbinary@basis\relax
		\pgfplots@error{Sorry, but I can't perform \string\pgfplotsbinaryencodesignedmaplinearly\space for bytes=\pgfplotsbinary@bytes\space yet... bytes=4 is the maximum.}%
	}%
	\def\pgfplotsbinary@add@signed@largest@absolute{
		\advance\c@pgfplotsbin@input by 2147483647 
	}%
}%
\expandafter\let\csname pgfplotsbinary@bytes@5\endcsname=\pgfplotsbinarysetbytes@@
\expandafter\let\csname pgfplotsbinary@bytes@6\endcsname=\pgfplotsbinarysetbytes@@
\expandafter\let\csname pgfplotsbinary@bytes@7\endcsname=\pgfplotsbinarysetbytes@@
\expandafter\let\csname pgfplotsbinary@bytes@8\endcsname=\pgfplotsbinarysetbytes@@

\pgfqkeys{/pgfplots/bin}{%
	% ordering not yet implemented; uses always BIG ENDIAN.
	ordering/.is choice,%
	ordering/big endian/.code={%
		\def\pgfplotsbinary@byteorder{0}%
		\let\pgfplotsbinary@apphighorderbytes=\pgfplotsbinary@apphighorderbytes@BIGENDIAN
		\let\pgfplotsbinaryencode@badic@unsigned@PAD=\pgfplotsbinaryencode@badic@unsigned@PAD@BIGENDIAN
	},%
	ordering/net/.style={/pgfplots/bin/ordering/big endian},%
	ordering/little endian/.code={%
		\def\pgfplotsbinary@byteorder{1}%
		\let\pgfplotsbinary@apphighorderbytes=\pgfplotsbinary@apphighorderbytes@LITTLEENDIAN
		\let\pgfplotsbinaryencode@badic@unsigned@PAD=\pgfplotsbinaryencode@badic@unsigned@PAD@LITTLEENDIAN
	},
	ordering/big endian,%
	%
	% The standard method - it results in binary encoded numbers.
	binary encoding/.code={%
		\expandafter\let\expandafter\pgfplotsbinary@ZERO\csname pgfp@bin@0\endcsname
		\expandafter\let\expandafter\pgfplotsbinary@ZERO@LINEARMAP\csname pgfp@bin@128\endcsname
		\let\pgfplotsbinary@ZERO@HIGHEST=\pgfplotsbinary@ZERO
		\def\pgfplotsbinary@basis{256}%
		\let\pgfplotsbinary@hook=\relax
		\def\pgfplotsbinary@hook@signed@linearmap{%
			\ifnum\c@pgfplotsbin@byteno=0
				\advance\c@pgfplotsbin@input by128
				\ifnum\c@pgfplotsbin@input>255
					\pgfplots@error{Sorry, there are not enough bytes to store the current number. I tried to write \the\c@pgfplotsbin@input...}%
				\fi
			\fi
		}%
		\def\pgfplotsbinarysetbytes##1{%
			\pgfutil@ifundefined{pgfplotsbinarysetbytes@##1}{%
				\pgfplots@error{Sorry, I can't write binary output with '##1' bytes yet...}%
			}{%
				\edef\pgfplotsbinary@bytes{##1}%
				\csname pgfplotsbinarysetbytes@##1\endcsname
			}%
		}%
	},%
	%
	% This applies 'binary encoding' and encodes the resulting bytes
	% in Hex. It corresponds to the ASCIIHexEncode in postscript or
	% pdf.
	% Please note that 'bytes' sets the number of binary bytes - the
	% actual encoding length is exactly twice as large.
	ASCIIHexEncode/.code={%
		\expandafter\let\expandafter\pgfplotsbinary@ZERO\csname pgfp@bin@48\endcsname
		\expandafter\let\expandafter\pgfplotsbinary@ZERO@LINEARMAP\csname pgfp@bin@56\endcsname
		\let\pgfplotsbinary@ZERO@HIGHEST=\pgfplotsbinary@ZERO
		\pgfkeysalso{/pgfplots/bin/ordering/big endian}%
		\def\pgfplotsbinary@basis{16}%
		\def\pgfplotsbinary@hook@hex{%
			\ifnum\c@pgfplotsbin@input<10
				\advance\c@pgfplotsbin@input by48
			\else
				\advance\c@pgfplotsbin@input by55
			\fi
		}%
		\let\pgfplotsbinary@hook=\pgfplotsbinary@hook@hex
		\def\pgfplotsbinary@hook@signed@linearmap{%
			\ifnum\c@pgfplotsbin@byteno=0
				\advance\c@pgfplotsbin@input by8
				\ifnum\c@pgfplotsbin@input>16
					\pgfplots@error{Sorry, there are not enough bytes to store the current number. I tried to write \the\c@pgfplotsbin@input...}%
				\fi
			\fi
			\pgfplotsbinary@hook@hex
		}%
		\def\pgfplotsbinarysetbytes##1{%
			\pgfutil@ifundefined{pgfplotsbinarysetbytes@##1}{%
				\pgfplots@error{Sorry, I can't write binary output with '##1' bytes yet...}%
			}{%
				\csname pgfplotsbinarysetbytes@##1\endcsname
				\begingroup
					\count0=##1\relax
					\multiply\count0 by2
					\xdef\pgfplotsbinary@glob@TMP{\the\count0 }%
				\endgroup
				\let\pgfplotsbinary@bytes=\pgfplotsbinary@glob@TMP
			}%
		}%
	},%
	%
	% 
	% This applies 'binary encoding' and encodes the resulting bytes
	% using a base 85 encoding. It corresponds to the ASCII85Encode in postscript or
	% pdf.
	% Handle this method with care - it works just for ONE number, not for a stream of
	% numbers as in pdf. Therefore, it might not be useful at all.
	% Please note that 'bytes' will be ignored; ASCII85Encode assumes
	% 4 binary bytes and uses 5 bytes to encode them.
	%
	% @ATTENTION bytes is ALWAYS 4, regardless of the setting of
	% 'bytes'!
	ASCII85Encode/.code={%
		\expandafter\let\expandafter\pgfplotsbinary@ZERO\csname pgfp@bin@33\endcsname
		\expandafter\let\expandafter\pgfplotsbinary@ZERO@LINEARMAP\csname pgfp@bin@42\endcsname
		\let\pgfplotsbinary@ZERO@HIGHEST=\pgfplotsbinary@ZERO
		\pgfkeysalso{/pgfplots/bin/ordering/big endian}%
		\edef\pgfplotsbinary@ASCII@specialzero{\pgfplotsbinary@ZERO\pgfplotsbinary@ZERO\pgfplotsbinary@ZERO\pgfplotsbinary@ZERO\pgfplotsbinary@ZERO}%
		\expandafter\def\expandafter\pgfplotsbinaryencode@badic@unsigned@PAD\expandafter{%
			\pgfplotsbinaryencode@badic@unsigned@PAD
			\ifx\pgfplotsbinaryresult\pgfplotsbinary@ASCII@specialzero
				% PDF standard: 0 is encoded as 'z':
				\gdef\pgfplotsbinaryresult{z}%
			\fi
		}%
		\def\pgfplotsbinary@basis{85}%
		\def\pgfplotsbinary@hook{%
			\advance\c@pgfplotsbin@input by33
		}%
		\def\pgfplotsbinary@hook@signed@linearmap{%
			\advance\c@pgfplotsbin@input by33
			\ifnum\c@pgfplotsbin@byteno=0
				\advance\c@pgfplotsbin@input by42
				\ifnum\c@pgfplotsbin@input>85
					\pgfplots@error{Sorry, there are not enough bytes to store the current number. I tried to write \the\c@pgfplotsbin@input...}%
				\fi
			\fi
		}%
		% I know, that does only work efficiently if bytes=4 for every
		% encoded number.
		\def\pgfplotsbinarysetbytes##1{%
			\def\pgfplotsbinary@bytes{5}%
			\csname pgfplotsbinarysetbytes@4\endcsname
		}%
		\pgfplotsbinarysetbytes4%
	},%
	binary encoding,%
	%
	bytes/.code={\pgfplotsbinarysetbytes{#1}},%
	bytes=4,
	%
	% Irreversibly change to VERBATIM output for debugging:
	debug mode/.code={%
		\let\pgfplotsbinary@apphighorderbytes@ORIG=\pgfplotsbinary@apphighorderbytes
		\def\pgfplotsbinary@ZERO{[Pad-0]}%
		\let\pgfplotsbinary@ZERO@HIGHEST=\pgfplotsbinary@ZERO
		\def\pgfplotsbinary@ZERO@LINEARMAP{[Pad-128]}%
		\def\pgfplotsbinary@apphighorderbytes##1{%
			\pgfutil@ifnextchar\csname{%
				\pgfplotsbinary@apphighorderbytes@DEBUG@csname
			}{%
				\pgfplotsbinary@apphighorderbytes@DEBUG@normal
			}%
			##1\relax
		}%
	},%
	% Write pdf objects in binary form. This does only work with
	% pdftex, and its output is only useful in conjunction with
	% \pdfcompresslevel=0
	% and a text editor.
	% Usage:
	% \pgfkeys{/pgfplots/bin/debug to pdf={\pgfplotsbinaryencodeunsigned}{1,2,3,...,16}}
	%
	% works only with pdftex
	debug to pdf/.code 2 args={%
		\foreach \num in {#2} {%
			#1{\num}%
			\immediate \pdfobj stream attr {
				/Decimal \num\space
				/Routine (\string#1)
			} {%
				\pgfplotsbinaryresult
			}%
		}%
	},%
}



\def\pgfplotsbinary@apphighorderbytes@DEBUG@csname \csname pgfp@bin@#1\endcsname\relax{%
	\pgfplotsbinary@apphighorderbytes@ORIG{[#1]}%
}%
\def\pgfplotsbinary@apphighorderbytes@DEBUG@normal#1\relax{%
	\pgfplotsbinary@apphighorderbytes@ORIG{#1}%
}%

%\pgfkeys{/pgfplots/bin/debug mode}


\countdef\c@pgfplotsbin@input=0
\countdef\c@pgfplotsbin@tmpa=1
\countdef\c@pgfplotsbin@tmpb=2
\countdef\c@pgfplotsbin@byteno=3
\countdef\c@pgfplotsbin@basis=4
\def\pgfplotsbinaryempty{}

%--------------------------------------------------
% input:    unsigned int x, b, n;
% output:   unsigned int y[n];
% for (i=0; i<n; i++) y[i] = 0;
% i=0;
% while (x > 0)
% { y[i] = x % b; /* entspricht x mod b   */
%  x = x / b;    /* ganzzahlige Division */
%  i++;
%  }
%-------------------------------------------------- 
% with x = #1
% b = basis
% will store stuff into \pgfplotsbinaryresult in binary format
%
% PRECONDITION:
% - \pgfplotsbinaryresult= empty!
% - \c@pgfplotsbin@byteno=\pgfplotsbinary@bytes
\def\pgfplotsbinaryencode@badic@unsigned@{%
	\ifnum\c@pgfplotsbin@input>0
		\c@pgfplotsbin@tmpa=\c@pgfplotsbin@input
		\divide\c@pgfplotsbin@tmpa by\c@pgfplotsbin@basis\relax
		\c@pgfplotsbin@tmpb=\c@pgfplotsbin@tmpa\relax
		\multiply\c@pgfplotsbin@tmpa by\c@pgfplotsbin@basis\relax
		\advance\c@pgfplotsbin@input by -\c@pgfplotsbin@tmpa\relax
		\advance\c@pgfplotsbin@byteno by-1
		\pgfplotsbinary@hook% hooks for modifications.
		\pgfplotsbinary@apphighorderbytes{\csname pgfp@bin@\the\c@pgfplotsbin@input\endcsname}%
		\c@pgfplotsbin@input=\c@pgfplotsbin@tmpb
%\message{RESULT SO FAR byte no \the\c@pgfplotsbin@byteno: \pgfplotsbinaryresult}%
		\expandafter\pgfplotsbinaryencode@badic@unsigned@
	\else
		\pgfplotsbinaryencode@badic@unsigned@PAD
%\message{RESULT SO FAR byte no \the\c@pgfplotsbin@byteno: \pgfplotsbinaryresult}%
	\fi
}%


% Defines \pgfplotsbinaryresult to be the binary representation of an
% unsigned integer.
%
% The representation will use unsigned dual number representation.
%
% The assignment to \pgfplotsbinaryresult will be globally.
% #1: an unsigned integer. It won't be transformed in any way, so make
% sure it fits into the configured number of bytes. It is an error if
% the number is too large or too small. Please note that only unsigned
% numbers are supported with this method.
%
% FIXME : fix  > 2^30
\def\pgfplotsbinaryencodeunsigned#1{%
	\begingroup
	\global\let\pgfplotsbinaryresult=\pgfplotsbinaryempty
	\c@pgfplotsbin@input=#1 %
	\c@pgfplotsbin@byteno=\pgfplotsbinary@bytes\relax
	\c@pgfplotsbin@basis=\pgfplotsbinary@basis\relax
	\pgfplotsbinaryencode@badic@unsigned@%
	\endgroup
}%

% An implementation for signed integers which maps the signed integer linearly into 
% the unsigned data range before it proceeds.
%
% The idea is thus, to first introduce a linear mapping
%
% phi : [- smallest_possible, +largest_possible ] -> [0, 256^bytes-1 ]
%
% A signed integer in TeX is in [ - (2^31-1), 2^31-1 ].
% Thus, we should map
%
% phi : [  -(2^31-1), 2^31-1 ] -> [ 0, 2^32-1 ].
%
% A simpler case is to employ the symmetry in TeX's registers and
% leave one out, i.e. to map to 2^32-2:
%
% phi : [  -(2^31-1), 2^31-1 ] -> [ 0, 2^32-2 ].
%
% Then,
%
% phi(x) = ( x + 2^31 -1 ) / (2^31-1 + 2^31-1) * (2^32-2) = x+ 2^31-1.
%
% The same map phi(x) = x + 2^31 -1 with target space  [0, 2^32-1 ]
% could be realized with the input space [- (2^31-1), 2^31 ].
%
% I am using this encoding procedure, phi(x) = x + 2^31 -1.
%
% As a consequence, the binary pattern FF FF FF FF does never occur as
% result of the mapping.
%
% To invert the mapping (i.e. to decode the result), set up the unique
% linear map
%
% psi : [ 0, 2^32-1 ] -> [ -(2^31-1), 2^31 ].
%
% Then, psi( phi(x) ) = x and the decoding procedure is correct.
%
% This doesn't need TeX register arithmetics on the whole range.
%
% REMARK: the whole operation does also work if bytes<4 (i.e. we have
% less than 32 bits in the target range). In this case, the mapping is
% phi : [  -(2^31-1), 2^31-1 ] -> [ 0, 2^{8*bytes}-1 ]
% and a further, *lossy* quantization still will be applied. The
% quantization step is an integer division performed in signed number
% arithmetics (i.e. it is symmetric around 0). 
\def\pgfplotsbinaryencodesignedmaplinearly#1{%
	\begingroup
	\global\let\pgfplotsbinaryresult=\pgfplotsbinaryempty
	\c@pgfplotsbin@input=#1 %
	\pgfplotsbinaryencodesignedmaplinearly@prepare
	\ifnum\c@pgfplotsbin@input<0
		% compute + 2^31 - 1
		\pgfplotsbinary@add@signed@largest@absolute
	\else
		% change zero padding such that positive numbers
		% get the EFFECT of + 2^31.
		\let\pgfplotsbinary@ZERO@HIGHEST=\pgfplotsbinary@ZERO@LINEARMAP
		\let\pgfplotsbinary@hook=\pgfplotsbinary@hook@signed@linearmap
		% and compute the -1 explicitly here:
		\advance\c@pgfplotsbin@input by-1
	\fi
	\pgfplotsbinaryencode@badic@unsigned@%
	\endgroup
}%

% Encodes a dimen (like 1pt or \dimen0) in binary form.
%
% The encoding works by mapping #1 linearly into the allowed integer
% range using a quantization technique to respect the (possibly)
% restricted number of bytes.
%
% The implementation is fast and uses only integer arithmetics.
% It relies on \pgfplotsbinaryencodesignedmaplinearly and a scale.
%
% So, what we do is to setup a linear map into binary range with k
% bytes. The range of a TeX dimen is precisely (in units of pt)
%  [ -(2^30-1) / 2^16, (2^30 -1) / 2^16 ] = [-16383.99998, 16383.99998]
%
% Thus, for an input dimen x, we set up the mapping
% 	phi(x) = 2^16 * x * 2
% which maps 
%  phi: [ -(2^30-1) / 2^16, (2^30-1) / 2^16 ] -> [-(2^31 -2), 2^31-2].
%
% I simply use the \pgfplotsbinaryencodesignedmaplinearly to process
% this further. To simplify the computation, I simply compute
%   phi_signed( phi(x) ), 
% where phi_signed denotes an application of
% \pgfplotsbinaryencodesignedmaplinearly:
%   phi_signed( y ) = y +2^31 -1,
%   phi_signed( phi(x) ) = 2^16 * 2 * x + 2^31 - 1.
% This is NOT a linear map to [0,2^32-1] as promised. 
% But, we can setup an inverse transformation PHI (which is linear) anyway
% such that
%   PHI( phi_signed(phi(x)) ) = x
% and that's all I want. Do do that, we use the unique linear decoder map 
%   PHI : [ 0,2^32-1 ] -> [ -16383.999992, 16384 ].
%
% This is not exacty the input range of before, but using it results
% in a proper decoder. The difference is due to the non-unique zero
% representation in TeX's arithmetics.
% 
%
% REMARK: the whole operation does also work if bytes<4 (i.e. we have
% less than 32 bits in the target range). In this case, a further
% *lossy* quantization step is applied in phi_signed. The inverse
% transformations are the same, however. See
% \pgfplotsbinaryencodesignedmaplinearly for details about the
% quantization step (or try it out).
%
	%
	%% DEBUG NOTE: This mapping appears to work correctly according to
	%% my tests.
	%% For bc -l test codes: 
	%% ibase=16; 
	%% -4000 +  809658FA. / (2^20) * 8000
\def\pgfplotsbinaryencodedimenmaplinearly#1{%
	\begingroup
	\dimen0=#1\relax
	\c@pgfplotsbin@input=\dimen0
	\multiply\c@pgfplotsbin@input by2
%\message{LOWLEVEL ENCODING '\the\c@pgfplotsbin@input' with linear map}%
	\pgfplotsbinaryencodesignedmaplinearly\c@pgfplotsbin@input
	\endgroup
}%

\catcode`\@=\pgfplotsbinaryatcode
\endinput
