\subsection{Label units}
\label{sec:units}

 \PGFPlots\ has the capability of supporting units. This provides quick customization of the plot as well as the addition of units in labels. In order to
 begin using the unit support one has to use the key |use units|. Currently it only supports predefined SI units but a per-user customization is also
 implemented such that it can be used in any way you like.
\begin{pgfplotsxykey}{\x\ unit=\marg{unit} (initially |empty|)}
  These keys sets the unit in their respective axis. In SI units you could for instance set the |x unit| in Newton as |x unit=N|.
\end{pgfplotsxykey}
\begin{pgfplotsxykey}{\x\ unit prefix=\marg{prefix} (initially |empty|)}
  These keys sets the prefix of the unit. If a value on the |y axis| is in kilo you would set the |y unit prefix=k|. This command will not intervene with
  the basis of the axis system. I.e. a prefix as just mentioned will not divide every |y axis| number by 1000. In order to do this see key \meta{axis}| SI prefix|. See Section \ref{sec:SI:prefix}.
\end{pgfplotsxykey}
Often one just have to utilize the above mentioned keys. It is the basis of the unit typesetting system provided by \PGFPlots. 
\begin{codeexample}[]
  \begin{tikzpicture}
    \begin{axis}[use units,
      x unit=m,x unit prefix=k,
      y unit=N,y unit prefix=m,
      xlabel=Distance,ylabel=Force]
      \addplot coordinates {
          (1,2.3)
          (2,2.7)
          (3,2.1)
          (4,1.8)
          (5,1.5)
          (6,1.1)
      };
    \end{axis}
  \end{tikzpicture}
\end{codeexample}

Below is an example of what would be yielded according to the styles
\begin{codeexample}[code only]
  \pgfplotsset{use units,x unit=T,xlabel=Temperature,ylabel=Nothing} 
  % x label becomes ``Temperature [T]'', y label as ``Nothing''
  \pgfplotsset{use units,x unit prefix=m,xlabel=Temperature,ylabel=Nothing} 
  % x label becomes ``Temperature'', y label as ``Nothing''
\end{codeexample}
Notice the second example. Only setting the prefix will not activate the unit typesetting. Therefore one should ensure to use the |x unit| key first.

For typesetting the units one can also change the appearance. For instance one might not like the square brackets which surround the unit. These can
luckily be changed using the below keys.
\begin{pgfplotskeylist}{unit marking pre=\marg{pre} (initially |left[|),unit marking post=\marg{post} (initially |right]|),unit markings=\mchoice{parenthesis,square braces,slash space} (initially |square braces|)}
  These keys sets the surroundings of the unit. The initial yields $\left[\frac{1}{2}\right]$ such that you can typeset fractions in units. These can
  easily be set using the option key |unit markings| where the options typesets as the following
  \begin{codeexample}[code only]
    \pgfplotsset{x unit=T,unit markings=parenthesis} % x unit becomes ``\left(T\right)''
    \pgfplotsset{x unit=T,unit markings=square braces} % x unit becomes ``\left[T\right]''
    \pgfplotsset{x unit=T,unit markings=slash space} % x unit becomes ``/ T''
  \end{codeexample}
  Of course you can just manually set each of them with the |unit marking pre| and |unit marking post| keys. Just remember that they are typeset within a \$\$.
\end{pgfplotskeylist}

One will typically typeset the unit with a specific font so to supply this one has the option of changing the command.
\begin{pgfplotskey}{unit code=\marg{typeset code} (initially |mathrm|)}
  This can be utilized to great extend. As a default the typesetting of the units is as |\mathrm{|\meta{unit prefix}\meta{unit}|}|. But if one for instance
  wishes to utilize the package |siunitx| one can just set the key as
  \begin{codeexample}[code only]
    \pgfplotsset{unit code=\si}
  \end{codeexample}
  which would yield the unit as |\si{|\meta{unit prefix}\meta{unit}|}|. The most important thing is that the command needs exactly one argument. So if you
  would like a command as |\mathrm{\mathbf{#1}}| you would need to redefine a new command and insert it as
  \begin{codeexample}[code only]
    \newcommand\mathrmbf{1}{\mathrm{\mathbf{#1}}}
    \pgfplotsset{unit code=\mathrmbf}
  \end{codeexample}
\end{pgfplotskey}


\subsubsection{Preset SI prefixes}
\label{sec:SI:prefix}
To support the SI system a number of preset keys are defined. This should yield a more intuitive way of supplying the prefix as well as add some more
functionality. 
\begin{pgfplotsxykeylist}{\x\ SI prefix=\mchoice{yocto,\dots,milli,centi,deci,deca,hecto,kilo,\dots,yotta} (initially |none|),change \x\
      base=\mchoice{true,false} (initially |false|)}
  These keys sets the prefix of the unit. The allowed prefixes are:

  \begin{center}
    \begin{tabular}{>{\ttfamily}cc}
      \toprule
      \rm Prefix & Power\\
      \midrule
      yocto & -24\\
      zepto & -21\\
      atto & -18\\
      femto & -15\\
      pico & -12\\
      nano& -9\\
      micro & -6\\
      milli & -3\\
      centi& -2\\
      deci& -1\\
\bottomrule
    \end{tabular}\qquad
    \begin{tabular}{>{\ttfamily}cc}
      \toprule
      \rm Prefix & Power\\
      \midrule
      deca & 1\\
      hecto & 2\\
      kilo & 3\\
      mega & 6\\
      giga & 9\\
      tera& 12\\
      peta & 15\\
      exa & 18\\
      zetta& 21\\
      yotta& 24\\
\bottomrule
    \end{tabular}

  \end{center}

  As well as resetting the base of the axis if the key |change |\meta{axis}| base=true|. Just \textbf{remember} to
  set the |change |\meta{axis}| base| before using the \meta{axis}| SI prefix| key. 

  See the utilization as in the example below.
  \begin{codeexample}[]
    \begin{tikzpicture}
      \begin{axis}[use units,change x base,
        x SI prefix=kilo,x unit=m,
        y SI prefix=milli,y unit=N,
        xlabel=Distance,ylabel=Force]
    \addplot coordinates {
        (1000,1)
        (2000,1.1)
        (3000,1.2)
        (4000,1.3)
    };
      \end{axis}
    \end{tikzpicture}
  \end{codeexample}
  Notice that the |x axis| has changed base without displaying the $\cdot 10^{3}$. This is done by using the key |change x base|. Even though you have used
  the key |y SI prefix=milli| the base isn't changed on the |y axis|. Try adding |change y base| and see the result!
\end{pgfplotsxykeylist}


The above keys are the easy implementation of the base change. Below is a further customization of the base change. 

\begin{pgfplotskey}{axis base prefix={axis \marg{axis} base \marg{base} prefix \marg{prefix}} (initially empty)}
  One can utilize this key to customize further of the base and setting the prefix.
  \begin{codeexample}[code only]
    \pgfplotsset{change x base,axis base prefix={axis x base -3 prefix k}
    \pgfplotsset{change x base,x SI prefix=kilo}
  \end{codeexample}
  The above two commands are thus equivalent. Remember that the base operates in opposite of the meaning of the prefix!
\end{pgfplotskey}
