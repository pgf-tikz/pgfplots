% main=manual.tex


\section{More examples}
This section contains a catalogue of different \PGFPlots\ features by example.
\label{sec:examples}%
\begingroup
\subsection{Legend position and appearance}
\subsubsection{Legend in the lower left corner}
\begin{lstlisting}
\begin{tikzpicture}
	\begin{loglogaxis}[
		xlabel={\textsc{Dof}},
		ylabel={$L_2$ Error},
		legend style={at={(0.03,0.03)},anchor=south west}
	]
	\addplot coordinates {
		(5,		8.312e-02)
		(17,	2.547e-02)
		(49,	7.407e-03)
		(129,	2.102e-03)
		(321,	5.874e-04)
		(769,	1.623e-04)
		(1793,	4.442e-05)
		(4097,	1.207e-05)
		(9217,	3.261e-06)
	};

	....

	\legend{$d=2$,$d=3$,$d=4$,$d=5$,$d=6$}
	\end{loglogaxis}
\end{tikzpicture}
\end{lstlisting}
\def\plots{%
	\addplot coordinates {
		(5,		8.312e-02)
		(17,	2.547e-02)
		(49,	7.407e-03)
		(129,	2.102e-03)
		(321,	5.874e-04)
		(769,	1.623e-04)
		(1793,	4.442e-05)
		(4097,	1.207e-05)
		(9217,	3.261e-06)
	};

	\addplot coordinates {
		(7,		8.472e-02)
		(31,	3.044e-02)
		(111,	1.022e-02)
		(351,	3.303e-03)
		(1023,	1.039e-03)
		(2815,	3.196e-04)
		(7423,	9.658e-05)
		(18943,	2.873e-05)
		(47103,	8.437e-06)
	};

	\addplot coordinates {
		(9,	7.881e-02)
		(49,	3.243e-02)
		(209,	1.232e-02)
		(769,	4.454e-03)
		(2561,	1.551e-03)
		(7937,	5.236e-04)
		(23297,	1.723e-04)
		(65537,	5.545e-05)
		(178177,	1.751e-05)
	};

	\addplot coordinates {
		(11,	6.887e-02)
		(71,	3.177e-02)
		(351,	1.341e-02)
		(1471,	5.334e-03)
		(5503,	2.027e-03)
		(18943,	7.415e-04)
		(61183,	2.628e-04)
		(187903,	9.063e-05)
		(553983,	3.053e-05)
	};

	\addplot coordinates {
		(13,	5.755e-02)
		(97,	2.925e-02)
		(545,	1.351e-02)
		(2561,	5.842e-03)
		(10625,	2.397e-03)
		(40193,	9.414e-04)
		(141569,	3.564e-04)
		(471041,	1.308e-04)
		(1496065,	4.670e-05)
	};
	\legend{$d=2$,$d=3$,$d=4$,$d=5$,$d=6$}
}%
{%
\begin{center}
\begin{tikzpicture}
	\begin{loglogaxis}[
		xlabel={\textsc{Dof}},
		ylabel={$L_2$ Error},
		legend style={at={(0.03,0.03)},anchor=south west}
	]
	\plots
	\end{loglogaxis}
\end{tikzpicture}
\end{center}
}

\subsubsection{Horizontal Legends}
\begin{lstlisting}
\begin{tikzpicture}
	\begin{loglogaxis}[
		xlabel={\textsc{Dof}},
		ylabel={$L_2$ Error},
		legend style={
			at={(0.5,0.98)},anchor=north,
			legend columns=2}
	]
	\addplot coordinates {
		(5,		8.312e-02)
		(17,	2.547e-02)
		(49,	7.407e-03)
		(129,	2.102e-03)
		(321,	5.874e-04)
		(769,	1.623e-04)
		(1793,	4.442e-05)
		(4097,	1.207e-05)
		(9217,	3.261e-06)
	};

	....

	\legend{$d=2$,$d=3$,$d=4$,$d=5$,$d=6$}
	\end{loglogaxis}
\end{tikzpicture}
\end{lstlisting}
{%
\begin{center}
\begin{tikzpicture}
	\begin{loglogaxis}[
		xlabel={\textsc{Dof}},
		ylabel={$L_2$ Error},
		legend style={
			at={(0.5,0.98)},anchor=north,
			legend columns=2}
	]
	\plots
	\end{loglogaxis}
\end{tikzpicture}
\end{center}
}

\subsubsection{Horizontal Legends (2)}
\begin{lstlisting}
\begin{tikzpicture}
	\begin{loglogaxis}[
		xlabel={\textsc{Dof}},
		ylabel={$L_2$ Error},
		legend style={
			at={(0.5,-0.2)},anchor=north,
			legend columns=-1}
	]
	\addplot coordinates {
		(5,		8.312e-02)
		(17,	2.547e-02)
		(49,	7.407e-03)
		(129,	2.102e-03)
		(321,	5.874e-04)
		(769,	1.623e-04)
		(1793,	4.442e-05)
		(4097,	1.207e-05)
		(9217,	3.261e-06)
	};

	....

	\legend{$d=2$,$d=3$,$d=4$,$d=5$,$d=6$}
	\end{loglogaxis}
\end{tikzpicture}
\end{lstlisting}
{%
\begin{center}
\begin{tikzpicture}
	\begin{loglogaxis}[
		xlabel={\textsc{Dof}},
		ylabel={$L_2$ Error},
		legend style={
			at={(0.5,-0.2)},anchor=north,
			legend columns=-1}
	]
	\plots
	\end{loglogaxis}
\end{tikzpicture}
\end{center}
}

\subsubsection{Modifying legend spacing}
\label{sec:legendexamples:cols}%
\begin{lstlisting}
% 'every axis legend' and 'legend style' are mostly the same:
\tikzstyle{every axis legend}+=[
	at={(0.5,1.02)},anchor=south,
	inner sep=0pt,
	legend columns=-1
]%
\begin{tikzpicture}
	\begin{loglogaxis}[
		xlabel={\textsc{Dof}},
		ylabel={$L_2$ Error}
	]
	\addplot coordinates {
		(5,		8.312e-02)
		(17,	2.547e-02)
		(49,	7.407e-03)
		(129,	2.102e-03)
		(321,	5.874e-04)
		(769,	1.623e-04)
		(1793,	4.442e-05)
		(4097,	1.207e-05)
		(9217,	3.261e-06)
	};

	....

	\legend{$d=2$,$d=3$,$d=4$,$d=5$,$d=6$}
	\end{loglogaxis}
\end{tikzpicture}
\end{lstlisting}
{%
\tikzstyle{every axis legend}+=[
	at={(0.5,1.02)},anchor=south,
	inner sep=0pt,
	legend columns=-1
]%
\begin{center}
\begin{tikzpicture}
	\begin{loglogaxis}[
		xlabel={\textsc{Dof}},
		ylabel={$L_2$ Error}
	]
	\plots
	\end{loglogaxis}
\end{tikzpicture}
\end{center}
}

\subsubsection{Modifying legend's small plots}
\label{sec:legendexamples:plotpos}%
\begin{lstlisting}
\begin{tikzpicture}
	\begin{loglogaxis}[
		legend plot pos=right,
		xlabel={\textsc{Dof}},
		ylabel={$L_2$ Error}
	]
	\addplot coordinates {
		(5,		8.312e-02)
		(17,	2.547e-02)
		(49,	7.407e-03)
		(129,	2.102e-03)
		(321,	5.874e-04)
		(769,	1.623e-04)
		(1793,	4.442e-05)
		(4097,	1.207e-05)
		(9217,	3.261e-06)
	};

	....

	\legend{$d=2$,$d=3$,$d=4$,$d=5$,$d=6$}
	\end{loglogaxis}
\end{tikzpicture}
\end{lstlisting}
{%
\begin{center}
\begin{tikzpicture}
	\begin{loglogaxis}[
		legend plot pos=right,
		xlabel={\textsc{Dof}},
		ylabel={$L_2$ Error}
	]
	\plots
	\end{loglogaxis}
\end{tikzpicture}
\end{center}
}

\subsection{Font size and line width}
\begin{lstlisting}
\tikzstyle{every axis legend}+=%
	[at={(0.03,0.03)},anchor=south west]%
\tikzstyle{every tick}+=[line width=0.6pt]%
\begin{tikzpicture}[font=\large,line width=1pt]
	\begin{loglogaxis}[
		xlabel={\textsc{Dof}},
		ylabel={$L_2$ Error}
	]
	\addplot ....
	...
	\end{loglogaxis}
\end{tikzpicture}
\end{lstlisting}

{%
\tikzstyle{every axis legend}+=
	[at={(0.03,0.03)},anchor=south west]%
\tikzstyle{every tick}+=[line width=0.6pt]
\begin{center}
\begin{tikzpicture}[font=\large,line width=1pt]
	\begin{loglogaxis}[
		xlabel={\textsc{Dof}},
		ylabel={$L_2$ Error}
	]
	\plots
	\end{loglogaxis}
\end{tikzpicture}
\end{center}
}%

\subsection{Changing line specifications}
\subsubsection{Using another, predefined list}
{%
\begin{lstlisting}
\begin{tikzpicture}
	\begin{loglogaxis}[
		cycle list name=\blackwhiteplotspeclist,
		xlabel={\textsc{Dof}},
		ylabel={$L_2$ Error}
	]
	...
	\end{loglogaxis}
\end{tikzpicture}
\end{lstlisting}

\begin{center}
\begin{tikzpicture}
	\begin{loglogaxis}[
		cycle list name=\blackwhiteplotspeclist,
		xlabel={\textsc{Dof}},
		ylabel={$L_2$ Error}
	]
	\plots
	\end{loglogaxis}
\end{tikzpicture}
\end{center}

\subsubsection{Defining new lists}
\begin{lstlisting}
\tikzstyle{every axis legend}+=[
	at={(1.03,1)},anchor=north west
]%
\begin{tikzpicture}
\begin{loglogaxis}[
	% will cycle through these three elements:
	cycle list={%
		red,dotted,mark=-,mark options={solid}\\%
		black,dashed,mark=pentagon,mark options={solid}\\%
		mark options={fill=blue!40},mark=diamond*,blue\\%
	},
	xlabel={\textsc{Dof}},
	ylabel={$L_2$ Error}
]
...
\end{loglogaxis}
\end{tikzpicture}
\end{lstlisting}

\tikzstyle{every axis legend}+=[
	at={(1.03,1)},anchor=north west
]%
\begin{center}
\begin{tikzpicture}
	\begin{loglogaxis}[
		% will cycle through these three elements:
		cycle list={%
			red,dotted,mark=-,mark options={solid}\\%
			black,dashed,mark=pentagon,mark options={solid}\\%
			mark options={fill=blue!40},mark=diamond*,blue\\%
		},
		xlabel={\textsc{Dof}},
		ylabel={$L_2$ Error}
	]
	\plots
	\end{loglogaxis}
\end{tikzpicture}
\end{center}
}%

\subsection{Changing the ticks}
\subsubsection{Placing ticks at $10^i$}
\begin{lstlisting}
\begin{tikzpicture}
	\begin{loglogaxis}[
		xtickten={0,2,3,4,6,...,10},% place tickmarks at 10^0, 10^2,...
		ytickten={-6,-4,...,2},
		xlabel={\textsc{Dof}},
		ylabel={$L_2$ Error}
	]
	....
	\end{loglogaxis}
\end{tikzpicture}
\end{lstlisting}

\begin{center}
\begin{tikzpicture}
	\begin{loglogaxis}[
		xtickten={0,2,3,4,6,...,10},
		ytickten={-6,-4,...,2},
		xlabel={\textsc{Dof}},
		ylabel={$L_2$ Error}
	]
	\plots
	\end{loglogaxis}
\end{tikzpicture}
\end{center}

\subsubsection{Placing ticks anywhere}
\begin{lstlisting}
\begin{tikzpicture}
	\begin{loglogaxis}[
		xtick={12,9897,1468864}
		xlabel={\textsc{Dof}},
		ylabel={$L_2$ Error}
	]
	....
	\end{loglogaxis}
\end{tikzpicture}
\end{lstlisting}

\begin{center}
\begin{tikzpicture}
	\begin{loglogaxis}[
		xtick={12,9897,1468864},%2.5,9.2,14.2},
		xlabel={\textsc{Dof}},
		ylabel={$L_2$ Error}
	]
	\plots
	\end{loglogaxis}
\end{tikzpicture}
\end{center}

\subsection{Annotating plots}
\label{sec:annot:plot}%
\subsubsection{Example: Placing Data Cursors}
\begin{lstlisting}
\tikzstyle{every pin}=[fill=white,draw=black,font=\footnotesize]
\begin{tikzpicture}
	\begin{loglogaxis}[
		xlabel={\textsc{Dof}},
		ylabel={$L_2$ Error}
	]

	....

	\addplot coordinates {
		(11,	6.887e-02)
		(71,	3.177e-02)
		(351,	1.341e-02)
		(1471,	5.334e-03)
		(5503,	2.027e-03)
		(18943,	7.415e-04)
		(61183,	2.628e-04)
		(187903,	9.063e-05)
		(553983,	3.053e-05)
	};
	...

	\node[coordinate,pin=above:Bad!] 
		at (axis cs:5503,2.027e-03) {};
	\node[coordinate,pin=above right:Good!] 
		at (axis cs:187903,9.063e-05)	{};
	\end{loglogaxis}
\end{tikzpicture}
\end{lstlisting}
\begin{center}
{
\tikzstyle{every pin}=[fill=white,draw=black,font=\footnotesize]
\begin{tikzpicture}
	\begin{loglogaxis}[
		xlabel={\textsc{Dof}},
		ylabel={$L_2$ Error}
	]
	\plots

	\node[coordinate,pin=above:Bad!] 
		at (axis cs:5503,2.027e-03) {};
	\node[coordinate,pin=above right:Good!] 
		at (axis cs:187903,9.063e-05)	{};
	\end{loglogaxis}
\end{tikzpicture}
}
\end{center}

\subsubsection{Example: Slopes of line segments}
\begin{lstlisting}
\begin{tikzpicture}
	\begin{loglogaxis}[
		xlabel=\textsc{Dof},
		ylabel=$L_2$ Error
	]
	\draw 
			(axis cs:1793,4.442e-05)
		|-	(axis cs:4097,1.207e-05)
		node[near start,left] {$\frac{dy}{dx} = -1.58$};

	\addplot coordinates {
		(5,		8.312e-02)
		(17,	2.547e-02)
		(49,	7.407e-03)
		(129,	2.102e-03)
		(321,	5.874e-04)
		(769,	1.623e-04)
		(1793,	4.442e-05)
		(4097,	1.207e-05)
		(9217,	3.261e-06)
	};
	...
	\end{loglogaxis}
\end{tikzpicture}
\end{lstlisting}
\begin{center}
\begin{tikzpicture}
	\begin{loglogaxis}[
		xlabel=\textsc{Dof},
		ylabel=$L_2$ Error
	]
	\draw 
			(axis cs:1793,4.442e-05)
		|-	(axis cs:4097,1.207e-05)
		node[near start,left] {$\frac{dy}{dx} = -1.58$};
	\plots
	
	\end{loglogaxis}
\end{tikzpicture}
\end{center}

\subsection{Vertical alignment with the \texttt{baseline} option}
\label{sec:align}%
The default axis anchor is \texttt{south west}, which means that the picture coordinate $(0,0)$ is the lower left corner of the axis. As a consequence, the \Tikz\ option ``\texttt{baseline}'' allows vertical alignment of adjacent plots, see figure~\ref{fig:baseline:example} for an example.
\begin{figure}
	\centering
		\noindent
		\lstinline!\begin{tikzpicture}...\end{tikzpicture}!:\par
		\noindent
		\begin{tikzpicture}%
		\begin{axis}[width=0.4\linewidth,xlabel=A normal sized $x$ label]
		\addplot[smooth,blue,mark=*] coordinates {
			(-1,	1)
			(-0.75,	0.5625)
			(-0.5,	0.25)
			(-0.25,	0.0625)
			(0,		0)
			(0.25,	0.0625)
			(0.5,	0.25)
			(0.75,	0.5625)
			(1,		1)
		};
		\end{axis}
		\end{tikzpicture}%
		\hspace{0.15cm}
		\begin{tikzpicture}%
		\begin{axis}[width=0.4\linewidth,xlabel={xlabel: $\displaystyle \sum_{i=0}^N n_i $ }]
		\addplot[smooth,blue,mark=*] coordinates {
			(-1,	1)
			(-0.75,	0.5625)
			(-0.5,	0.25)
			(-0.25,	0.0625)
			(0,		0)
			(0.25,	0.0625)
			(0.5,	0.25)
			(0.75,	0.5625)
			(1,		1)
		};
		\end{axis}
		\end{tikzpicture}
		\hfill
		\vskip 0.3cm%
		\noindent
		\lstinline!\begin{tikzpicture}[baseline]...\end{tikzpicture}!:\par
		\noindent
		\begin{tikzpicture}[baseline]%
		\begin{axis}[width=0.4\linewidth,xlabel=A normal sized $x$ label]
		\addplot[smooth,blue,mark=*] coordinates {
			(-1,	1)
			(-0.75,	0.5625)
			(-0.5,	0.25)
			(-0.25,	0.0625)
			(0,		0)
			(0.25,	0.0625)
			(0.5,	0.25)
			(0.75,	0.5625)
			(1,		1)
		};
		\end{axis}
		\end{tikzpicture}%
		\hspace{0.15cm}
		\begin{tikzpicture}[baseline]%
		\begin{axis}[width=0.4\linewidth,xlabel={xlabel: $\displaystyle \sum_{i=0}^N n_i $ }]
		\addplot[smooth,blue,mark=*] coordinates {
			(-1,	1)
			(-0.75,	0.5625)
			(-0.5,	0.25)
			(-0.25,	0.0625)
			(0,		0)
			(0.25,	0.0625)
			(0.5,	0.25)
			(0.75,	0.5625)
			(1,		1)
		};
		\end{axis}
		\end{tikzpicture}
		\hfill

	\caption{An example of the \Tikz-``\texttt{baseline}'' option for vertical alignment. Top row: without ``\texttt{baseline}'', bottom row: with ``\texttt{baseline}''.}
	\label{fig:baseline:example}
\end{figure}

\subsection{Horizontal Alignment}
{%
\label{sec:halign}%
The last subsection used the \texttt{baseline}-option to allow \emph{vertical} alignment of plots. You can also apply \emph{horizontal} alignment, if you place multiple \texttt{axes} into a single \texttt{tikzpicture} and use the `\texttt{anchor}'-option:
\def\plots{
\addplot[red,only marks,mark options={fill=red,scale=0.8},mark=*] coordinates {
	(0.190600,	0.860900)
	(0.194400,	0.908000)
	(0.179900,	0.140700)
	(0.212400,	0.294200)
	(0.218900,	0.779100)
	(0.201700,	0.388400)
	(0.209100,	0.679900)
	(0.196200,	0.470800)
	(0.199900,	0.812700)
	(0.199300,	0.850500)
	(0.204300,	0.980300)
	(0.209500,	0.953000)
	(0.176200,	0.145300)
	(0.207400,	0.351900)
	(0.199000,	0.304500)
	(0.202200,	0.256300)
	(0.197000,	0.744300)
	(0.181100,	0.998900)
	(0.193400,	0.554900)
	(0.218300,	0.506500)
	(0.193800,	0.201000)
	(0.208100,	0.531900)
	(0.204100,	0.635500)
	(0.217000,	0.069800)
	(0.196100,	0.710300)
	(0.198900,	0.440900)
	(0.213000,	0.591900)
	(0.206000,	0.030700)
	(0.186000,	0.395900)
	(0.206500,	0.082500)
	(0.195500,	0.672000)
	(0.221400,	-0.013500)
	(0.202300,	0.222300)
	(0.193300,	0.959700)
};
\addplot[black,only marks,mark options={fill=black,scale=0.8},mark=square*] coordinates {
	(0.005300,	0.369500)
	(-0.016100,	0.610000)
	(-0.007400,	0.252100)
	(0.002900,	0.699600)
	(0.010900,	0.232000)
	(-0.014100,	0.152400)
	(-0.008000,	-0.003100)
	(-0.009100,	0.639900)
	(-0.008700,	0.125800)
	(-0.003700,	0.897400)
	(0.009600,	0.432200)
	(-0.009100,	0.770800)
	(-0.030300,	0.974200)
	(-0.004000,	0.675900)
	(-0.002100,	0.089500)
	(0.003300,	0.319700)
	(0.006100,	0.863300)
	(0.000600,	0.263700)
	(-0.008900,	0.471500)
	(0.016000,	0.505700)
	(0.000000,	0.602600)
	(-0.012300,	0.404400)
	(-0.007000,	0.911900)
	(0.006700,	0.747200)
	(0.009000,	0.825300)
	(-0.001500,	0.534400)
	(-0.020600,	0.336300)
	(0.012600,	0.039800)
	(0.011600,	0.944100)
	(-0.003600,	0.068000)
	(-0.022400,	0.575400)
	(-0.002900,	0.184500)
};
\addplot[blue] coordinates {
	(0.093947,	-0.011481)
	(0.101957,	0.494273)
	(0.109967,	1.000027)
};
}%
\begin{lstlisting}
\begin{tikzpicture}
	\begin{axis}[width=4cm,scale only axis]
	\addplot[red,only marks,mark options={fill=red,scale=0.8},mark=*] coordinates {
		(0.190600,	0.860900)
		(0.194400,	0.908000)
		(0.179900,	0.140700)
		(0.212400,	0.294200)
		....
		(0.202300,	0.222300)
		(0.193300,	0.959700)
	};
	\addplot[black,only marks,mark options={fill=black,scale=0.8},mark=square*] coordinates {
		(0.005300,	0.369500)
		(-0.016100,	0.610000)
		(-0.007400,	0.252100)
		(0.002900,	0.699600)
		(0.010900,	0.232000)
		....
		(-0.003600,	0.068000)
		(-0.022400,	0.575400)
		(-0.002900,	0.184500)
	};
	\addplot[blue] coordinates {
		(0.093947,	-0.011481)
		(0.101957,	0.494273)
		(0.109967,	1.000027)
	};
	\end{axis}

	\draw[thick,gray,->] 
		(0,0) -- (0cm,-0.6cm) 
		node[black,midway,left] {0.6cm$\to$};

	\begin{axis}[%
		at={(0cm,-0.6cm)},
		anchor=north west,
		width=4cm,scale only axis,
		height=0.8cm,
		ytick=\empty
	]

	\addplot[red,only marks,mark options={fill=red,scale=0.8},mark=*] coordinates {
		(0.968555,	0)
		(0.030884,	0)
		...
		(0.468750,	0)
		(0,	0)
	};
	\addplot[black,only marks,mark options={fill=black,scale=0.8},mark=square*] coordinates {
		(0.367188,	0)
		(0.312500,	0)
		...
		(0.125000,	0)
		(0.156250,	0)
	};
	\end{axis}
\end{tikzpicture}
\end{lstlisting}
The result of this alignment is shown in figure~\ref{fig:halign}.

\begin{figure}
{\centering
\begin{minipage}[t]{4.4cm}%
\vspace{0pt}%
\centering
\begin{tikzpicture}
	\begin{axis}[width=4cm,scale only axis]
	\plots
	\end{axis}
\end{tikzpicture}

\begin{tikzpicture}
	\begin{axis}[%
		width=4cm,scale only axis,
		height=0.8cm,
		ytick=\empty
	]

	\addplot[red,only marks,mark options={fill=red,scale=0.8},mark=*] coordinates {
		(0.968555,	0)
		(0.030884,	0)
		(0.750000,	0)
		(0.468750,	0)
		(1.000000,	0)
		(0.030176,	0)
		(0.750000,	0)
		(0.468750,	0)
		(0,	0)
	};
	\addplot[black,only marks,mark options={fill=black,scale=0.8},mark=square*] coordinates {
		(0.367188,	0)
		(0.312500,	0)
		(0.136719,	0)
		(0.828125,	0)
		(0.656250,	0)
		(0.343750,	0)
		(0.125000,	0)
		(0.156250,	0)
	};
	\end{axis}
\end{tikzpicture}
\end{minipage}
\hspace{1cm}
\begin{minipage}[t]{4.4cm}%
\vspace{0pt}%
\centering
\begin{tikzpicture}
	\begin{axis}[width=4cm,scale only axis]
	\plots
	\end{axis}

	\draw[thick,gray,->] 
		(0,0) -- (0cm,-0.6cm) 
		node[black,midway,left] {0.6cm$\to$};

	\begin{axis}[%
		at={(0cm,-0.6cm)},
		anchor=north west,
		width=4cm,scale only axis,
		height=0.8cm,
		ytick=\empty
	]

	\addplot[red,only marks,mark options={fill=red,scale=0.8},mark=*] coordinates {
		(0.968555,	0)
		(0.030884,	0)
		(0.750000,	0)
		(0.468750,	0)
		(1.000000,	0)
		(0.030176,	0)
		(0.750000,	0)
		(0.468750,	0)
		(0,	0)
	};
	\addplot[black,only marks,mark options={fill=black,scale=0.8},mark=square*] coordinates {
		(0.367188,	0)
		(0.312500,	0)
		(0.136719,	0)
		(0.828125,	0)
		(0.656250,	0)
		(0.343750,	0)
		(0.125000,	0)
		(0.156250,	0)
	};
	\end{axis}
\end{tikzpicture}
\end{minipage}

}
	\caption{Demonstration of horizontal alignment with the `\texttt{anchor}' option. Left: two \texttt{tikzpicture}s placed above each other; right: one \texttt{tikzpicture} with horizontal alignment using the `\texttt{anchor}'-option. The source code is in the text.}
	\label{fig:halign}
\end{figure}
}

\subsection{Bounding box restrictions}
\label{sec:bounding:box:example}%
{%
The following figure is centered and encapsulated with an \lstinline!\fbox! to show its bounding box.
\setlength{\fboxsep}{0pt}%
\begin{center}
\fbox{%
\begin{tikzpicture}%
	\begin{pgfinterruptboundingbox}
	\begin{loglogaxis}[
		name=my plot,
		title=A title,
		xlabel={\textsc{Dof}},
		ylabel={$L_2$ Error},
		legend style={at={(1.03,1)},anchor=north west},
	]
		\plots
	\end{loglogaxis}
	\end{pgfinterruptboundingbox}

	\useasboundingbox (my plot.below south west) rectangle (my plot.above north east);
\end{tikzpicture}%
}%
\end{center}
}%
\begin{lstlisting}
\setlength{\fboxsep}{0pt}%
\begin{center}
\fbox{%
\begin{tikzpicture}%
  \begin{pgfinterruptboundingbox}
  \begin{loglogaxis}[
	name=my plot,
	title=A title,
	xlabel={\textsc{Dof}},
	ylabel={$L_2$ Error},
	legend style={at={(1.03,1)},anchor=north west},
  ]
	...
  \end{loglogaxis}
  \end{pgfinterruptboundingbox}

  \useasboundingbox 
  			  (my plot.below south west) 
	rectangle (my plot.above north east);
\end{tikzpicture}%
}%
\end{center}
\end{lstlisting}
You can combine any anchors to get different bounding boxes. See page~\pageref{option:anchor} for details about anchors.

\subsection{Gridlines}
\label{sec:gridlines}%
\begin{lstlisting}
\begin{tikzpicture}
	\begin{loglogaxis}[
		xlabel={\textsc{Dof}},
		ylabel={$L_2$ Error},
		grid=major
	]
	...
	\end{loglogaxis}
\end{tikzpicture}
	
\begin{tikzpicture}
	\begin{loglogaxis}[
		xlabel={\textsc{Dof}},
		ylabel={$L_2$ Error},
		grid=both
	]
	...
	\end{loglogaxis}
\end{tikzpicture}
\end{lstlisting}
\begin{center}
\begin{tikzpicture}
	\begin{loglogaxis}[
		xlabel={\textsc{Dof}},
		ylabel={$L_2$ Error},
		grid=major
	]
	\plots
	\end{loglogaxis}
\end{tikzpicture}
\hfill
\begin{tikzpicture}
	\begin{loglogaxis}[
		xlabel={\textsc{Dof}},
		ylabel={$L_2$ Error},
		grid=both
	]
	\plots
	\end{loglogaxis}
\end{tikzpicture}
\end{center}

\endgroup
