\section{Basic Level Commands}
\label{sec:pgfplots:lowlevel}
This section documents commands which provide access to more basic elements of \PGFPlots. Most of them are closely related to the basic level of \pgfname, especially various point commands which are specific to an axis.

However, most elements in this section are only interesting for advanced users -- and perhaps only for special cases.

\begin{command}{\pgfplotsextra\marg{low-level commands}}
	A command to execute \marg{low-level commands} in \PGFPlots. Since any drawing commands inside of an axis need to be postponed until the axis is complete and the scaling has been initialised, it is not possible to simply draw any paths.
	Instead, it is necessary to draw them as soon as the axis is finished. This is done automatically for every \Tikz\ path -- and it is also done manually if you write |\pgfplotsextra|\marg{commands}.
\begin{codeexample}[]
\begin{tikzpicture}
	\begin{axis}[xmin=0,xmax=3,ymin=0,ymax=5]
	\pgfplotsextra{%
		\pgfpathmoveto{\pgfplotspointaxisxy{1}{2}}%
		\pgfpathlineto{\pgfplotspointaxisxy{2}{4}}%
		\pgfusepath{stroke}%
	}
	\end{axis}
\end{tikzpicture}
\end{codeexample}
	The example above initialises an axis and executes the basic level path commands as soon as the axis is ready. The execution of multiple |\path|, |\addplot| and |\pgfplotsextra| commands is in the same sequence as they occur in the environment\footnote{Except for stacked plots where the sequence may be reverse, see the key \texttt{reverse stack plots}.}.%
\end{command}

\begin{commandlist}{%
	\pgfplotspointaxisxy\marg{x coordinate}\marg{y coordinate},%
	\pgfplotspointaxisxyz\marg{x coordinate}\marg{y coordinate}\marg{z coordinate}}
	Point commands like |\pgfpointxy| which take logical, absolute coordinates and return a low--level point. Every transformation from user transformations to logarithms are applied.

	Since the transformations are initialised after the axis is complete, this command needs to be postponed (see |\pgfplotsextra|).
\end{commandlist}

\begin{commandlist}{%
	\pgfplotspointrelaxisxy\marg{rel x coordinate}\marg{rel y coordinate},%
	\pgfplotspointrelaxisxyz\marg{rel x coordinate}\marg{rel y coordinate}\marg{rel z coordinate}}
	Point commands which take \emph{relative} coordinates such that $x=0$ is the \emph{lower} $x$ axis limit and $x=1$ the \emph{upper} $x$ axis limit.

	These commands are used for |rel axis cs|.

	Please note that the transformations are only initialised if the axis is complete! This means you need to provide |\pgfplotsextra|.
\end{commandlist}

\begin{commandlist}{%
	\pgfplotspointdescriptionxy\marg{$x$ fraction}\marg{$y$ fraction},%
	\pgfplotsqpointdescriptionxy\marg{$x$ fraction}\marg{$y$ fraction}}%
	Point commands such that |{0}{0}| is the lower left corner of the axis' bounding box and |{1}{1}| the upper right one; everything else is in-between. The `|q|' variant is quicker as it doesn't invoke the math parser on its arguments.

	They are used for |axis description cs|, see section~\ref{pgfplots:sec:axis:description:cs}.
\end{commandlist}

\begin{commandlist}{%
	\pgfplotspointunitx,%
	\pgfplotspointunity,%
	\pgfplotspointunitz}%
	Low--level point commands which return the $x$, $y$ or $z$ unit vectors.

	The point |\pgfplotspointxyz{1}{0}{0}| is the same as |\pgfplotspointunitx|, the |{0}{1}{0}| coordinate the unit $y$ vector and the |{0}{0}{1}| coordinate the unit $z$ vector.

	The unit $z$ vector is only defined for three dimensional axes.
\end{commandlist}

\begin{commandlist}{%
	\pgfplotsunitxlength,%
	\pgfplotsunitylength,%
	\pgfplotsunitzlength,%
	\pgfplotsunitxinvlength,%
	\pgfplotsunityinvlength,%
	\pgfplotsunitzinvlength}%
	Macros which expand to the vector length $\lVert x_i \rVert$ of the respective unit vector $x_i$ or the inverse vector length, $1/\lVert x_i \rVert$. These macros can be used inside of |\pgfmathparse|, for example.

	The $x_i$ are the |\pgfplotspointunitx| variants.
\end{commandlist}

\begin{commandlist}{\pgfplotspointaxisorigin}
	A point coordinate at the origin, $(0,0,0)$. If the origin is not part of the axis limits, the nearest point on the boundary is returned instead.

	This is the same coordinate as returned by the |origin| anchor.
\end{commandlist}

\begin{command}{\pgfplotsqpointoutsideofaxis\marg{three-char-string}\marg{coordinate}\marg{normal distance}}
	Provides a point coordinate on one of the available four axes in case of a two dimensional figure or on one of the available twelve axes in case of a three dimensional figure.
	
	The desired axis is uniquely identified by a three character string, provided as first argument to the command. The first of the three characters is `|0|' if the $x$ coordinate of the specified axis passes through the lower axis limit. It is `|1|', if the $x$ coordinate of the specified axis passes through the upper axis limit. Furthermore, it is `|2|' if it passes through the origin. The second character is also either |0|, |1| or |2| and it characterizes the position on the $y$ axis. The third character is for the third dimension, the $z$ axis. It should be left at `|0|' for two dimensional plots. However, \emph{one} of the three characters should be `|v|', meaning the axis \underline varies. For example, |v01| denotes $\{ (x,y_{\text{min}},z_{\text{max}}) \vert x \in \R \}$.
	
	The second argument, \meta{coordinate} is the logical coordinate on that axis. Since two coordinate of the axis are fixed, \meta{coordinate} refers to the \underline varying component of the axis. It must be a number without unit; no math expressions are supported here.

	The third argument \meta{normal distance} is a dimension like |10pt|. It shifts the coordinate away from the designated axis in direction of the outer normal vector. The outer normal vector always points away from the axis. It is computed using
	|\pgfplotspointouternormalvectorofaxis|.

	There are several variants of this command which are documented in the source code. One of them is particularly useful:
\end{command}

\begin{command}{\pgfplotsqpointoutsideofaxisrel\marg{three-char-string}\marg{axis fraction}\marg{normal distance}}
	This point coordinate is a variant of |\pgfplotsqpointoutsideofaxis| which allows to provide an \meta{axis fraction} instead of an absolute coordinate. The fraction is a number between $0$ (lower axis limit) and $1$ (upper axis limit), i.e.\ it is given in percent of the total axis. It is possible to provide negative values or values larger than one.

	The |\pgfplotsqpointoutsideofaxisrel| command is similar in spirit to |rel axis cs|.

	There is one speciality in conjunction with reversed axes: if the axis has been reversed by |x dir=reverse| and, in addition, |allow reversal of rel axis cs| is true, the value $0$ denotes the \emph{upper} limit while $1$ denotes the \emph{lower} limit. The effect is that coordinates won't change just because of axis reversal.
\index{allow reversal of rel axis cs}%
\end{command}

\begin{command}{\pgfplotspointouternormalvectorofaxis\marg{three-char-string}}
	A point command which yields the outer normal vector of the respective axis. The normal vector has length $1$ (computed with |\pgfpointnormalised|). It is the same normal vector used inside of |\pgfplotsqpointoutsideofaxis| and its variants.

	The output of this command will be cached and re-used during the lifetime of an axis. 
\end{command}

\begin{command}{\pgfplotsticklabelaxisspec\marg{x, y or z}}
	Expands to the three-character-identification for the axis containing tick labels for the chosen axis, either \meta{x}, \meta{y} or \meta{z}.
\end{command}

\begin{command}{\pgfplotsvalueoflargesttickdimen\marg{x, y or z}}
	Expands to the largest distance of a tick position to its tick label bounding box in direction of the outer unit normal vector. It does also include the value of the |ticklabel shift| key.

	This value is used for |ticklabel cs|.
\end{command}

\begin{commandlist}{%
	\pgfplotstransformcoordinatex\marg{x coordinate of an axis},%
	\pgfplotstransformcoordinatey\marg{y coordinate of an axis},%
	\pgfplotstransformcoordinatey\marg{z coordinate of an axis}}
	Defines |\pgfmathresult| to be the low-level \PGF\ coordinate corresponding to the input argument.

	The command applies any |[xyz] coord trafo| keys, data scalings and/or logarithms or whatever \PGFPlots\ does to map input coordinates to internal coordinates.

	The result can be used inside of a |\pgfpointxy| statement (i.e.\ it still needs to be scaled with the respective \PGF\ unit vector).
\begin{codeexample}[]
\begin{tikzpicture}
	\begin{axis}[xmin=0,xmax=2,ymin=0,ymax=5]
	\pgfplotsextra{%
		\pgfplotstransformcoordinatex{1}%
		\let\xcoord=\pgfmathresult
		\pgfplotstransformcoordinatey{1}%
		\let\ycoord=\pgfmathresult
		\pgfpathcircle
			{\pgfqpointxy{\xcoord}{\ycoord}}
			{5pt}%
		\pgfusepath{fill}%
	}%
	\end{axis}
\end{tikzpicture}
\end{codeexample}
	Please note that the transformations are only initialised if the axis is complete! This means you need to provide |\pgfplotsextra| as is shown in the example above.
\end{commandlist}

\begin{command}{\pgfplotsconvertunittocoordinate\marg{x, y or z}\marg{dimension}}
	Converts a dimension (with unit!) to a corresponding $x$, $y$ or $z$ coordinate. The result will be written to |\pgfmathresult| (without units).

	It is possible to use the result as arguments for the |\pgfpointxyz| commands.

	The effect is to multiply \marg{dimension} with the inverse length of the unit vector for the specified axis. These lengths are precomputed in \PGFPlots\ so the operation is fast.
\begin{codeexample}[code only]
\pgfplotsconvertunittocoordinate{x}{5pt}
% now, the command uses exactly 5pt in x direction:
\pgfqpointxyz{\pgfmathresult}{4}{3}
\end{codeexample}
\end{command}

\begin{command}{\ifpgfplotsthreedim\meta{true code}\textbackslash else\meta{else code}\textbackslash fi}%
	A \TeX\ |\if| which evaluate the \meta{true code} if the axis is three dimensional and the \meta{else code} if not.
\end{command}
