
\section{Use Cases involving Scatter Plots}

Assuming that we are more or less familiar with the basics of the preceding
tutorials, we would like to draw a scatter plot. A scatter plot is one in which
markers indicate the important information.

There are many different kinds of scatter plots and this section covers a
couple of them.

\subsection{Scatter Plot Use Case A}
\label{sec:tut3:usecaseA}

In this subsection, we address the following scatter plot use case: assume that
we are given a couple of special $(x,y)$ coordinates along with color data at
every vertex. We would like to draw markers at the positions and choose
individual colors depending on the color data.


\subsubsection{Importing the Data File}

We assume that our input data is given as a table containing much more columns
than we need. The first couple of rows are as follows:

    \lstinputlisting[
        columns=fixed,
        breaklines=false,
        tabsize=15,
        lastline=8,
            ]{plotdata/concat_VV_together_grid.dat}

What we need is the first and second column to get the $x$- and $y$-coordinate
values, respectively, and the third column |f(x)| to choose color values. The
color values are very small and have a high range: there are values of order
$10^{-6}$ and there are values of order $1$. Such ranges are best shown on a
logarithmic scale, which is why we will resort to some logarithmic scale on the
absolute values of this column. Thus, a requirement will be to accept a math
expression (involving logs) on the color data column.

Note that the data file (and all others referenced in this manual) are shipped
with \PGFPlots{}; you can find them in the subfolder
\texttt{doc/latex/pgfplots/plotdata}.

We learned already how to read table data from a file, so our first step is
relatively straightforward.

\begin{codeexample}[]
\begin{tikzpicture}
\begin{axis}
    \addplot+ [only marks] table
        {concat_VV_together_grid.dat};
\end{axis}
\end{tikzpicture}
\end{codeexample}

Here, the only non-trivial variation is the option |only marks| which is given
after the plus sign. Keep in mind that |\addplot+[|\meta{options}|]| means that
\PGFPlots{} shall combine the set of options of its |cycle list| with
\meta{options}. In our case, |only marks| does what it says. The |only marks|
plot handler is the most simple scatter plot: it uses the same color for every
marker.

Note that |\addplot table| takes the first column as |x| and the second as |y|
(which matches our input file perfectly).


\subsubsection{Fine Tuning}

We agree that our initial import has unsuitable displayed limits: there is too
much white space around the interesting plot area. In addition, the markers
overlap because they are too large. We can modify the appearance as follows:

\begin{codeexample}[]
\begin{tikzpicture}
\begin{axis}[
    enlargelimits=false,
]
    \addplot+ [only marks,mark size=0.6pt]
        table {concat_VV_together_grid.dat};
\end{axis}
\end{tikzpicture}
\end{codeexample}

As before, we assume that we add more options after |\begin{axis}|.
Consequently, we introduced suitable indentation and a trailing comma after the
option. Note that |enlargelimits| is typically active; it means that
\PGFPlots{} increases the displayed range by $10\%$ by default. Deactivating it
produces tight limits according to the input data.

Our second option is |mark size| -- using an absolute size (about the radius or
half size of the marker).


\subsubsection{Color Coding According To Input Data}

We are quite close to our goal, except for the colors. As discussed, our input
file contains three columns and the third one should be used to provide color
information. In our case, the data file has a column named |f(x)|.
%
\begin{codeexample}[]
\begin{tikzpicture}
    \begin{axis}[
        enlargelimits=false,
        colorbar,
    ]
        \addplot+ [
            only marks,
            scatter,
            point meta={
                ln(1e-6+abs(\thisrow{f(x)}))/ln(10)
            },
            mark size=0.6pt,
        ] table {concat_VV_together_grid.dat};
    \end{axis}
\end{tikzpicture}
\end{codeexample}
%
We added a couple of options to our example: the options |scatter|, and
|point meta|, |colorbar|. The option |scatter| has a slightly misleading name as
we already had a scatter plot before we added that option. It activates scatter
plots with individual appearance: without further options, it chooses
individual colors for every marker. The ``individual colors'' are based on
something which is called ``|point meta|'' in \PGFPlots{}. The |point meta| is
typically a scalar value for every input coordinate. In the default
configuration, it is interpreted as ``color data'' for the coordinate in
question. This also explains the other option: |point meta=...| tells
\PGFPlots{} which values are to be used to determine colors. Note that the
default value of |point meta| is to use the $y$~coordinate. In our case, we
have a complicated math expression which is related to our input file: it
contains small quantities in column |f(x)| which are based shown on a
logarithmic scale as their differ over a huge range. Since a logarithm must not
have a non-positive argument, we have $10^{-6} + \text{abs}(\dotsb)$ as
expression which ensures that the argument is never smaller than |10^{-6}| and
that is is positive. The divider |/ln(10)| means that we have logarithms
base~$10$. But the key point of the whole complicated expression can be
summarized as follows:
%
\begin{enumerate}
    \item We can use |\thisrow|\marg{column name} to refer to table columns.
        Here, ``this row'' means to evaluate the table for the ``data point
        which is being read from the current row''.
    \item We can combine |\thisrow| with any complicated math expression.
\end{enumerate}
%
The third new option |colorbar| activates the color bar on the right hand side
(as you guessed correctly). We see that the smallest value is $-6$ which
corresponds to our value |1e-6| in the math expression.

You might wonder how a scalar value (the number stored in the |f(x)| column)
results in a color. \PGFPlots{} computes the minimum and maximum value of all
such numbers. Then, it maps every number into a |colormap|. A |colormap|
defines a couple of colors and interpolates linearly between such colors. That
means that the smallest value of |point meta| is mapped to the first color in a
|colormap| whereas the largest value of |point meta| is mapped to the last
color in the |colormap|. All others are mapped to something in-between.

More information about |colormap| and |point meta| can be found in
Section~\ref{sec:colormap:input:format} and in
Section~\ref{sec:pgfplots:point:meta}.


\subsection{Scatter Plot Use Case B}

As already mentioned, there are various use cases for scatter plots. The
default configuration of the |scatter| key is to read numeric values of
|point meta| and choose colors by mapping that value into the current |colormap|.

A different application would be to expect symbolic input (some string) and
choose different markers depending on that input symbol.

Suppose that you are given a sequence of input coordinates of the form $(x,y)$
\meta{class label} and that you want to choose marker options depending on the
\meta{class label}. A \PGFPlots{} solution could be
%
\begin{codeexample}[]
\begin{tikzpicture}
\begin{axis}
    \addplot [
        scatter,
        only marks,
        point meta=explicit symbolic,
        scatter/classes={
            a={mark=square*,blue},
            b={mark=triangle*,red},
            c={mark=o,draw=black}% <-- don't add comma
        },
    ] table [meta=label] {
        x      y        label
        0.1    0.15     a
        0.45   0.27     c
        0.02   0.17     a
        0.06   0.1      a
        0.9    0.5      b
        0.5    0.3      c
        0.85   0.52     b
        0.12   0.05     a
        0.73   0.45     b
        0.53   0.25     c
        0.76   0.5      b
        0.55   0.32     c
    };
\end{axis}
\end{tikzpicture}
\end{codeexample}
%
As in our previous use case in Section~\ref{sec:tut3:usecaseA}, we have the
options |scatter|, |only marks|, and a configuration how to retrieve the
|point meta| values by means of the |meta| key. One new key is
|point meta=explicit symbolic|: it tells \PGFPlots{} that any encountered values
of |point meta| are to be interpreted as string symbols. Furthermore, it tells
\PGFPlots{} that the every input coordinate comes with an explicit value (as
opposed to a common math expression, for example). The other different option
is |scatter/classes|. As you guessed from the listing, it is a map from string
symbol to marker option list. This allows to address such use cases in a simple
way.

This example has actually been replicated from the reference manual section for
|scatter/classes|.


\subsection{Scatter Plot Use Case C}

Finally, this tutorial sketches a further use case for scatter plots: given a
sequence of coordinates $(x,y)$ with individual string labels, we want to draw
the string label at the designated positions.

This can be implemented by means of the |nodes near coords| feature of
\PGFPlots{}, which is actually based on |scatter|:
%
\begin{codeexample}[]
\begin{tikzpicture}
\begin{axis}[
    enlargelimits=0.2,
]
    \addplot+ [nodes near coords,only marks,
       point meta=explicit symbolic]
    table [meta=label] {
        x    y   label
        0.5  0.2 1
        0.2  0.1 t2
        0.7  0.6 3
        0.35 0.4 Y4
        0.65 0.1 5
    };
\end{axis}
\end{tikzpicture}
\end{codeexample}
%
In this case, we have |point meta=explicit symbolic| in order to express the
fact that our labels are of textual form (see the reference manual section for
|nodes near coords| for applications of \emph{numeric} labels). The remaining
stuff is done by the implementation of |nodes near coords|. Note that enlarged
the axis limits somewhat in order to include the text nodes in the visible
area.

There is much more to say about |scatter| plots, and about |nodes near coords|.
Please consider this subsection as a brief pointer to
Section~\ref{sec:pgfplots:scatter:2d} in the reference manual.


\subsection{Summary}

We learned how to generate scatter plots with single color using |only marks|,
scatter plots with individually colored markers using the |scatter| key,
scatter plots with specific marker styles depending on some class label using
|scatter/classes| and text nodes using |nodes near coords|.

Furthermore, we introduced the concept of ``|point meta| data'': once as
(scalar valued) color data, once as symbolic class label and once as text
label.

There is much more to say, especially about |point meta| which is introduced
and explained in all depth in Section~\ref{sec:pgfplots:point:meta}.

There is also more to say about |scatter| plots, for example how to generate
scatter plots with individually sized markers and/or colors (by relying on
|\pgfplotspointmetatransformed|, see the reference manual section for
|visualization depends on|). In addition, |scatter| plots can be customized to
a high degree which is explained in Section~\ref{sec:pgfplots:scatter:2d}.
