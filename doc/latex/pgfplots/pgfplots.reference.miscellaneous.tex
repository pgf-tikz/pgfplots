

\subsection{Miscellaneous Options}
\label{pgfplots:misc}

\begin{pgfplotskey}{disablelogfilter=\mchoice{true,false} (initally false, default true)}
Disables numerical evaluation of $\log(x)$ in \TeX. If you specify this option, any plot coordinates and tick positions must be provided as $\log(x)$ instead of $x$. This may be faster and -- possibly -- more accurate than the numerical log. The current implementation of $\log(x)$ normalizes~$x$ to $m\cdot 10^e$ and computes
\[ \log(x) = \log(m) + e \log(10) \]
where $y = \log(m)$ is computed with a newton method applied to $\exp(y) - m$. The normalization involves string parsing without \TeX-registers. You can savely evaluate $\log(1\cdot 10^{-7})$ although \TeX-registers would produce an underflow for such small numbers. 
\end{pgfplotskey}

\label{sec:disabledatascaling}%
\begin{pgfplotskey}{disabledatascaling=\mchoice{true,false} (initally false, default true)}
\index{Accuracy!Data Transformation}%
\index{Errors!dimension too large}%
Disables internal re-scaling of input data. Normally, every input data like plot coordinates, tick positions or whatever, are parsed without using \TeX's limited number precision. Then, a transformation like 
	\[ T(x) = 10^{q-m} \cdot x - a \]
is applied to every input coordinate/position where $m$ is ``the order of $x$'' base~$10$. Example: $x=1234 = 1.234\cdot 10^3$ has order~$m=4$ while $x=0.001234 = 1.234\cdot 10^{-3}$ has order $m=-2$. The parameter~$q$ is the order of the axis' width/height.

The \textbf{effect} of the transformation is that your plot coordinates can be of \emph{arbitrary magnitude} like $0.0000001$ and $0.0000004$. For these two coordinates, \PGFPlots\ will use 100pt and 400pt internally. The transformation is quit fast since it relies only on period shifts. This scaling allows precision beyond \TeX's capabilities.
%\footnote{Please note that while plot coordinates can be of quite large magnitude like $10^12$ or $10^{-9}$, \PGFPlots\ still uses \TeX-registers internally (the math parser of \PGF). If your axis interval is $[1234567.8, 1234567.9]$ or something like that, }.

The option ``|disabledatascaling|'' disables this data transformation. This has two consequences: first, coordinate expressions like \parg{{\normalfont\texttt{axis cs:}}x,y} have the same effect like \parg{x,y}, no re-scaling is applied. Second, coordinates are restricted to what \TeX\ can handle\footnote{Please note that the axis' scaling requires to compute $1/( x_\text{max} - x_{\text{min}} )$. The option \protect\pgfmanualpdfref{disabledatascaling}{\texttt{disabledatascaling}} may lead to overflow or underflow in this context, so use it with care! Normally, the data scale transformation avoids this problem.}.

So far, the data scale transformation applies only to normal axis (logarithmic scales do not need it). 
\end{pgfplotskey}


\begin{pgfplotskey}{execute at begin plot=\marg{commands}}
This axis option allows to invoke \marg{commands} at the beginning of each |\addplot| command. The argument \marg{commands} can be any \TeX\ content.

You may use this in conjunction with |x filter=...| to reset any counters or whatever. An example would be to change every $4$th coordinate.
\end{pgfplotskey}

\begin{pgfplotskey}{execute at end plot=\marg{commands}}
This axis option allows to invoke \marg{commands} after each |\addplot| command. The argument \marg{commands} can be any \TeX\ content.
\end{pgfplotskey}

\begin{pgfplotskey}{forget plot=\marg{true,false} (initially false)}
\label{pgfplots:forgetplot}
	Allows to include plots which are not remembered for legend entries, which do not increase the number of plots and which are not considered for cycle lists.

	A forgotten plot can be some sort of decoration which has a separate style and does not influence the axis state, although it is processed as any other plot.
	Provide this option to |\addplot| as in the following example.
\begin{codeexample}[]
\begin{tikzpicture}
	\begin{loglogaxis}[
		table/x=Basis,
		table/y={L2/r},
		xlabel=Degrees of Freedom,
		ylabel=relative Error,
		title=New Experiments (old in gray),
		legend entries={$e_1$,$e_2$,$e_3$}
	]
	\addplot[black!15,forget plot] 
		table {plotdata/oldexperiment1.dat};
	\addplot[black!15,forget plot] 
		table {plotdata/oldexperiment2.dat};
	\addplot[black!15,forget plot] 
		table {plotdata/oldexperiment3.dat};
	\addplot table {plotdata/newexperiment1.dat};
	\addplot table {plotdata/newexperiment2.dat};
	\addplot table {plotdata/newexperiment3.dat};
	\end{loglogaxis}
\end{tikzpicture}
\end{codeexample}
	Since forgotten plots won't increase the plot index, they will use the same |cycle list| entry as following plots. This can be used to ``interrupt'' plots as is described in section~\ref{pgfplots:interrupt}.
\index{Interrupted Plots}

	The style |every forget plot| can be used to configure styles for each such plot. Please note that |every plot no |\meta{index} styles are not applicable here.

	A forgotten plot will be stacked normally if |stack plots| is enabled!
\end{pgfplotskey}

\begin{pgfplotscodekey}{before end axis}
Allows to insert \marg{commands} just before the axis is ended. This option takes effect inside of the clipped area.
\begin{codeexample}[]
\pgfplotsset{every axis/.append style={
	before end axis/.code={
		\fill[red] (axis cs:1,10) circle(5pt);
		\node at (axis cs:-4,10) 
			{\large This text has been inserted 
			 using \texttt{before end axis}.};
	}}}
\begin{tikzpicture}
	\begin{axis}
	\addplot {x^2};
	\end{axis}
\end{tikzpicture}
\end{codeexample}
\end{pgfplotscodekey}

\begin{pgfplotscodekey}{after end axis}
Allows to insert \marg{commands} right after the end of the clipped drawing commands. While |befor end axis| has the same effect as if \marg{commands} had been placed inside of your axis, |after end axis| allows to access axis coordinates without being clipped.
\begin{codeexample}[]
\pgfplotsset{every axis/.append style={
	after end axis/.code={
		\fill[red] (axis cs:1,10) circle(5pt);
		\node at (axis cs:-4,10) 
			{\large This text has been inserted using \texttt{after end axis}.};
	}}}
\begin{tikzpicture}
	\begin{axis}
	\addplot {x^2};
	\end{axis}
\end{tikzpicture}
\end{codeexample}
\end{pgfplotscodekey}

\begin{pgfplotskey}{nodes near coords=\marg{content} (default \textbackslash pgfmathprintnumber\textbackslash pgfplotspointmeta)}
	A style which places text nodes near every coordinate.

\begin{codeexample}[]
\begin{tikzpicture}
\begin{axis}[nodes near coords]
	\addplot+[only marks] coordinates {
		(0.5,0.2) (0.2,0.1) (0.7,0.6) 
		(0.35,0.4) (0.65,0.1)};
\end{axis}
\end{tikzpicture}
\end{codeexample}
	The \marg{content} is, if nothing else has been specified, the content of the ``point meta'', displayed using the default \meta{content}=|\pgfmathprintnumber{\pgfplotspointmeta}|. The macro |\pgfplotspointmeta| contains whatever has been selected by the |point meta| key, it defaults to the $y$ coordinate for two dimensional plots and the $z$ coordinate for three dimensional plots.

	Since |point meta=explicit symbolic| allows to treat string data, you can provide textual descriptions which will be shown inside of the generated nodes\footnote{In this case, the |\textbackslash pgfmathprintnumber| will be skipped automatically.}:

\begin{codeexample}[]
\begin{tikzpicture}
\begin{axis}[nodes near coords,enlargelimits=0.2]
	\addplot+[only marks,
		point meta=explicit symbolic] 
	coordinates {
		(0.5,0.2) [(1)]
		(0.2,0.1) [(2)]
		(0.7,0.6) [(3)]
		(0.35,0.4) [(4)]
		(0.65,0.1) [(5)]
	};
\end{axis}
\end{tikzpicture}
\end{codeexample}
	The square brackets are the way to provide explicit |point meta| for |plot coordinates|. Please refer to the documentation of |plot file| and |plot table| for how to get point meta from files.

	The style |nodes near coords| might be useful for bar plots, see |ybar| for an example of |nodes near coords|.

	\paragraph{Remarks and Details:}
	\begin{itemize}
		\item |nodes near coords| uses the same options for line styles and colors as the current plot. This may be changed using the style |every node near coord|, see below.

		\item |nodes near coords| is actually one of the |scatter| plot styles. It redefines |scatter/@pre marker code| to generate several \Tikz\ |\node| commands.

		In order to use |nodes near coords| together with other |scatter| plot styles (like |scatter/use mapped color| or |scatter/classes|), you may append a star to each of these keys. The variant \declareandlabel{nodes near coords*} will \emph{append} code to |scatter/@pre marker code| without overwriting the previous value.
		\item Consider using |enlargelimits| together with |nodes near coords| if text is clipped away.
		\item Currently |nodes near coords| does not work satisfactory for |ybar interval| or |xbar interval|, sorry.

	\end{itemize}
\end{pgfplotskey}

\begin{stylekey}{/pgfplots/every node near coord}
	A style used for every node generated by |nodes near coords|. It is initially empty.
\end{stylekey}

\begin{pgfplotskey}{nodes near coords align=\marg{alignment method} (initially auto)}
	Specifies how to align nodes generated by |nodes near coords|. 

	Possible choices for \marg{alignment} are

	\begin{description}
		\item[]\declare{auto} Uses |horizontal| if the $x$ coordinates are shown or |vertical| in all other cases. This checks the current value of |point meta|.
		\item[]\declare{horizontal} uses |left| if |\pgfplotspointmeta| $<0$ and |right| otherwise.
		\item[]\declare{vertical}   uses |below| if |\pgfplotspointmeta| $<0$ and |above| otherwise.
		\item[] Its also possible to provide any \Tikz\ alignment option such as |anchor=north east|, |below| or something like that. It is also allowed if multiple options are provided.
	\end{description}
\end{pgfplotskey}

\begin{pgfplotskey}{clip marker paths=\mchoice{true,false} (initially false)}
	The initial choice |clip marker paths=false| causes markers to be drawn \emph{after} the clipped region. Only their positions will be clipped. As a consequence, markers will be drawn completely, or not at all. The value |clip marker paths=true| is here for backwards compatibility: it does not introduce special marker treatment, so markers may be drawn partially if they are close to the clipping boundary\footnote{Please note that clipped marker paths may be slightly faster during \TeX\ compilation.}.
\end{pgfplotskey}

\begin{pgfplotskey}{clip=\mchoice{true,false} (initially true)}
	Whether any paths inside an axis shall be clipped.
\end{pgfplotskey}

\begin{pgfplotskey}{axis on top=\mchoice{true,false} (initially false)}
	If set to |true|, axis lines, ticks, tick labels and grid lines will be drawn on top of plot graphics.
\begin{codeexample}[]
\begin{tikzpicture}
    \begin{axis}[
		axis on top=true,
		axis x line=middle,
		axis y line=middle]
    \addplot+[fill] {x^3} \closedcycle;
    \end{axis}
\end{tikzpicture}
\end{codeexample}

\begin{codeexample}[]
\begin{tikzpicture}
    \begin{axis}[
		axis on top=false,
		axis x line=middle,
		axis y line=middle]
    \addplot+[fill] {x^3} \closedcycle;
    \end{axis}
\end{tikzpicture}
\end{codeexample}
Please note that this feature does not affect plot marks. I think it looks unfamiliar if plot marks are crossed by axis descriptions.
\end{pgfplotskey}

\begin{key}{/pgf/fpu=\marg{true,false} (initially true)}
\index{Precision}
	This key activates or deactivates the floating point unit. If it is disabled (|false|), the core \PGF\ math engine written by Mark Wibrow and Till Tantau will be used for |plot expression|.
	However, this engine has been written to produce graphics and is not suitable for scientific computing. It is limited to fixed point numbers in the range $\pm 16384.00000$.

	If the |fpu| is enabled (|true|, the initial configuration) the high-precision floating point library of \PGF\ written by Christian Feuers\"anger will be used. It offers the full range of IEEE double precision computing in \TeX. This FPU is also part of \PGFPlotstable, and it is activated by default for |create col/expr| and all other predefined mathematical methods.

	Use
\begin{codeexample}[code only]
\pgfkeys{/pgf/fpu=false}
\end{codeexample}
	\noindent in order to de-activate the extended precision. If you prefer using the |fp| (fixed point) package, possibly combined with Mark Wibrows corresponding \PGF\ library, the |fpu| will be deactivated automatically. Please note, however, that |fp| has a smaller data range (about $\pm 10^{17}$) and may be slower.
\end{key}

\subsection{Miscellaneous Commands}
\begin{command}{\autoplotspeclist}
This command should no longer be used, although it will be kept as technical implementation detail. Please use the `|cycle list|' option, section~\ref{sec:cycle:list}.
\end{command}

\begin{command}{\pgfmathlogtologten\meta{number}}
Assigns the result of $\meta{number}/\log(10)$ to |\pgfmathresult|.
\end{command}

\begin{command}{\logten}
Expands to the constant $\log(10)$. Useful for logplots because $\log(10^i) = i\log(10)$. This command is only available inside of an \Tikz-picture.
\end{command}

\begin{command}{\pgfmathprintnumber\marg{number}}
Generates pretty--printed output\footnote{This method was previously \texttt{\textbackslash prettyprintnumber}. It's functionality has been included into \PGF\ and the old command is now deprecated.} for \marg{number}. This method is used for every tick label.

The number is printed using the current number printing options, see section~\ref{sec:number:printing} for the different number styles, rounding precision and rounding methods.
\end{command}

\begin{command}{\plotnum}
	Inside of |\addplot| or any associated style, option or command, |\plotnum| expands to the current plot's number, starting with~$0$.
\end{command}

\begin{command}{\numplots}
	Inside of any of the axis environments, associated style, option or command, |\numplots| expands to the total number of  plots.
\end{command}

\begin{command}{\coordindex}
	Inside of an |\addplot| command, this macro expands to the number of the actual coordinate (starting with~$0$).

	It is useful together with |x filter| or |y filter| to (de-)select coordinates.
\end{command}

\begin{command}{\pgfplotstableread\marg{file}}
	Please refer to the manual of \PGFPlotstable, |pgfplotstable.pdf|, which is part of the \PGFPlots-bundle.
\end{command}
\begin{command}{\pgfplotstabletypeset\marg{\textbackslash macro}}
	Please refer to the manual of \PGFPlotstable, |pgfplotstable.pdf|, which is part of the \PGFPlots-bundle.
\end{command}

\subsubsection{Some Commands Of PGF}
\begin{command}{\pgfmathparse\marg{expression}}
	Invokes the \pgfname\ math parser for \meta{expression} and defines \declareandlabel{\pgfmathresult} to be the result.
\begin{codeexample}[]
\pgfmathparse{1+41}

The result is `\pgfmathresult'.
\end{codeexample}
	Please refer to \cite{tikz} for more details.
\end{command}

\begin{commandlist}{\path,\draw,\fill,\node,\matrix}
	These commands are \Tikz\ drawing commands all of which are documented in~\cite{tikz}. They are used to draw or fill paths, generate text nodes or aligned text matrizes. They are equivalent to 
	\pgfmanualpdflabel{/tikz/draw}{}|\path[draw]|, 
	\pgfmanualpdflabel{/tikz/fill}{}|\path[fill]|, 
	\pgfmanualpdflabel{/tikz/node}{}|\path[node]|, 
	\pgfmanualpdflabel{/tikz/matrix}{}|\path[matrix]|, 
	respectively.
\end{commandlist}

\begin{commandlist}{\pgfkeys,\pgfeov,\pgfkeysvalueof,\pgfkeysgetvalue}
	These commands are part of the \Tikz\ way of specifying options, its sub-package |pgfkeys|. The |\pgfplotsset| command is actually nothing but a wrapper around |\pgfkeys|.

	A short introduction into |\pgfkeys| can be found in~\cite{keyvalintro} whereas the complete reference is, of course, the \Tikz\ manual~\cite{tikz}.

	The key |\pgfkeysvalueof|\marg{key name} expands to the value of a key; |\pgfkeysgetvalue|\marg{key name}\marg{\textbackslash macro} stores the value of \meta{key name} into \meta{\textbackslash macro}. The |\pgfeov| macro is used to delimit arguments for code keys in |\pgfkeys|, please refer to the documentation above.
\end{commandlist}

\begin{keylist}{/tikz/xshift=\marg{dimension},/tikz/yshift=\marg{dimension}}
	These \Tikz\ keys allow to shift something by \marg{dimension} which is any \TeX\ size (or expression).
\end{keylist}

\begin{command}{\foreach \meta{variables} |in| \meta{list} \meta{commands}}
	A powerful loop command provided by \Tikz, see~\cite[Section Utilities]{tikz}.
\begin{codeexample}[]
\foreach \x in {1,2,...,4} {Iterating \x. }%
\end{codeexample}

	A \PGFPlots\ related example could be
\begin{codeexample}[code only]
\foreach \i in {1,2,...,10} {\addplot table {datafile\i}; }%
\end{codeexample}
\end{command}

\begin{command}{\pgfplotsforeachungrouped \meta{variable} |in| \meta{list} \meta{command}}
	A specialised variant of |\foreach| which can do two things: it does not introduce extra groups while executing \meta{command} and it allows to invoke the math parser for (simple!) \meta{$x_0$}|,|\meta{$x_1$}|,...,|\meta{$x_n$} expressions.

\begin{codeexample}[]
\def\allcollected{}
\pgfplotsforeachungrouped \x in {1,2,...,4} {Iterating \x. \edef\allcollected{\allcollected, \x}}%
All collected = \allcollected.
\end{codeexample}

	A more useful example might be to work with tables. The following example is taken from \PGFPlotstable:

\begin{codeexample}[code only]
\pgfplotsforeachungrouped \i in {1,2,...,10} {%
	\pgfplotstablevertcat{\output}{datafile\i} % appends `datafile\i' -> `\output'
}%
\end{codeexample}

	\paragraph{Remark: } If \meta{list} doesn't have the form \meta{$x_0$}|,|\meta{$x_1$}|,...,|\meta{$x_n$},  the implementation of |\pgfplotsforeachungrouped| invokes |\foreach| and provides the results without \TeX\ groups. If the special form of \meta{list} is given, it does also employ the math parser for its task, allowing larger number ranges than |\foreach|.
	
\end{command}
