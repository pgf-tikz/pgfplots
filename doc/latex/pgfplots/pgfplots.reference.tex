% main=manual.tex

\section{Command Reference}
\subsection{The axis--environments}
There are four axis environments,
\begin{enumerate}
\item The axis environment for normal plots with linear axis scaling,
\begin{lstlisting}
\begin{axis}
...
\end{axis}
\end{lstlisting}
You can use
\begin{lstlisting}
\tikzstyle{every linear axis}+=[...]
\end{lstlisting}
to install styles specifically for linear axes. These styles can contain both \Tikz- and \PGFPlots\ options.

\item The axis environment for logarithmic scaling of~$x$ and normal scaling of~$y$,
\begin{lstlisting}
\begin{semilogxaxis}
...
\end{semilogxaxis}
\end{lstlisting}
You can use
\begin{lstlisting}
\tikzstyle{every semilogx axis}+=[...]
\end{lstlisting}
to install styles specifically for the case with \texttt{xmode=log}, \texttt{ymode=normal}.

\item The axis environment for normal scaling of~$x$ and logarithmic scaling of~$y$,
\begin{lstlisting}
\begin{semilogyaxis}
...
\end{semilogyaxis}
\end{lstlisting}
The style `\texttt{every semilogy axis}' will be installed for each such plot, so
\begin{lstlisting}
\tikzstyle{every semilogy axis}+=[...]
\end{lstlisting}
sets options.

\item The axis environment for logarithmic scaling of both, $x$~and~$y$ axes,
\begin{lstlisting}
\begin{loglogaxis}
...
\end{loglogaxis}
\end{lstlisting}
As for the other axis possibilities, there is a style `\texttt{every loglog axis}' which is installed at the environment's beginning.
\end{enumerate}

They are all equivalent to
\begin{lstlisting}
\begin{axis}[
	xmode=log|normal,
	ymode=log|normal]
...
\end{axis}
\end{lstlisting}
with properly set variables `\texttt{xmode}' and `\texttt{ymode}' (see below).

\subsection{\texttt{\textbackslash addplot[OPTIONS] PATH}}
\label{sec:addplot}%
This is the main plotting command. It is used as
\begin{lstlisting}
\addplot coordinates {
		(0,0)
		(1,1)
	};
\end{lstlisting}
or
\begin{lstlisting}
\addplot[color=blue,mark=*]
	coordinates {
		(0,0)
		(1,1)
	};
\end{lstlisting}
or
\begin{lstlisting}
\addplot file{datafile.dat};
\end{lstlisting}
or
\begin{lstlisting}
\addplot (\x,{exp(\x)});
\end{lstlisting}
or
\begin{lstlisting}
\addplot plot[id=sin] function{sin(x)};
\end{lstlisting}
or
\begin{lstlisting}
\addplot table[x=columnA,y=columnD]{datafile.dat};
\end{lstlisting}
The `\texttt{plot coordinates}', `\texttt{plot file}', plot expression and `\texttt{plot function}' variants are well--known \Tikz\ plotting methods. The `\texttt{plot table}' command is introduced in \PGFPlots\ and allows to draw specific columns of a text file against each other (see below for more details).

The optional argument to \lstinline!\addplot! assigns line/marker specifications for the plot. These specifications will also be used for the legend. If you omit them, line/marker specifications will be determined using the \texttt{cycle list}  option (see section~\ref{sec:cycle:list}).

\noindent
Some more details:
\begin{itemize}
	\item You can modify \texttt{OPTIONS} with
	\begin{lstlisting}[gobble=1]
	\tikzstyle{every axis plot}=[...]
	\end{lstlisting}
	or, even better, with
	\begin{lstlisting}[gobble=1]
	\tikzstyle{every axis plot}+=[...]
	\end{lstlisting}
	If you have more than one plot inside of an axis, you can also use
	\begin{lstlisting}[gobble=1]
	\tikzstyle{every axis plot no 3}+=[...]
	\end{lstlisting}
	to modify options for the 3rd plot only. The first plot has number~$1$.
	\item The \texttt{OPTIONS} are remembered for the legend. Furthermore, they are available as `\texttt{current plot style}' as long as the path is not yet finished or in associated error bars.
	\item See subsection~\ref{sec:markers} for a list of available markers and line styles.
	\item For log plots, \PGFPlots\ will compute the natural logarithm $\log(\cdot)$ numerically. This works with normal fixed point numbers or in scientific notation. For example, the following numbers are valid input to \lstinline!\addplot!.
\begin{lstlisting}
\begin{tikzpicture}
\begin{loglogaxis}
\addplot coordinates {
	(769,	1.6227e-04)
	(1793,	4.4425e-05)
	(4097,	1.2071e-05)
	(9217,	3.2610e-06)
	(1e6,	0.00003341)
	(2.3e7,	0.00131415)
;
\end{loglogaxis}
\end{tikzpicture}
\end{lstlisting}
	You can represent arbirtrarily small or very large numbers as long as its logarithm can be represented as a \TeX-length (up to about~$16384$). Of course, any coordinate~$x\le 0$ is not possible since the logarithm of a non-positive number is not defined. Such coordinates will be skipped automatically.

	\item For normal plots, \PGFPlots\ understands arbirtrarily large or small numbers like 0.00000001234 or $1.234\cdot 10^{24}$. \PGFPlots\ has its own number parsing tools which are complete text based and do not rely on \TeX's limited numerical number representation. Before the plots are actually drawn, any input coordinate is transformed into \TeX-precision using transformations
		\[ T_x(x) = 10^{s_x} \cdot x \text{ and } T_y(y) = 10^{s_y} \cdot y \]
	with properly chosen integers $s_x, s_y \in \Z$. This ensures invariance of axis ranges. See section~\ref{sec:disabledatascaling} for more details.

	\item As a consequence of the coordinate parsing routines, you can't use the mathematical expression parsing method of \PGF\ as coordinates (that means: you will need to provide coordinates without suffixes like ``cm'' or ``pt'' and you can't invoke mathematical functions).
	
	\item If you did not specify axis limits for $x$ and $y$ manually, \lstinline!\addplot! will compute them automatically. 

	The automatic computation of axis limits works as follows:
		\begin{enumerate}
			\item Every coordinate will be checked. Care has been taken to avoid \TeX's limited numerical capabilities.
			\item Since more than one \lstinline!\addplot! command may be used inside an \lstinline!\begin{axis}...\end{axis}!, all drawing commands will be postponed until \lstinline!\end{axis}!.
		\end{enumerate}

	\item You can omit the `\texttt{plot}' prefix of `\texttt{plot coordinates}', `\texttt{plot file}' and `\texttt{plot table}'.

	\item You can specify trailing path commands after the plotting command. The plot command is processed by \PGFPlots\ and any trailing path parts are forwarded to \Tikz. Examples are shown in figure~\ref{fig:after:paths}.
\end{itemize}
\begin{figure}
	\centering
	\tikzstyle{every picture}+=[baseline]%
	\tikzstyle{every axis}+=[width=6cm,anchor=north west,y label style={text width=0pt}]%
	\lstset{basicstyle=\ttfamily\footnotesize,aboveskip=0pt,belowskip=0pt}%
	\begin{tikzpicture}%
		\begin{pgfinterruptboundingbox}
		\begin{axis}[ymin=0,ymax=1,enlargelimits=false,name=an axis]
		\addplot[blue!80!black,fill=blue,fill opacity=0.5] coordinates 
			{(0,0.1) (0.1,0.15) (0.2,0.5) (0.3,0.62) (0.4,0.56) (0.5,0.58) (0.6,0.65) (0.7,0.6) (0.8,0.58) (0.9,0.55) (1,0.52)} 
		|- (axis cs:0,0) -- cycle;
		\addplot[red,fill=red!90!black,opacity=0.5] coordinates 
			{(0,0.25) (0.1,0.27) (0.2,0.24) (0.4,0.26) (0.5,0.3) (0.6,0.23) (0.7,0.2) (0.8,0.15) (0.9,0.1) (1,0.1)}
		 |- (axis cs:0,0) -- cycle;
		\addplot[green!20!black] coordinates
			{(0,0.4) (0.2,0.75) (1,0.75)};
		\end{axis}

		\begin{axis}[name=a second axis,at={(0,-4.5cm)}]
		\addplot plot[id=parable,domain=-5:5] function{4*x**2 - 5} node[pin=180:{$4x^2-5$}]{};
		\end{axis}
		\end{pgfinterruptboundingbox}
		\useasboundingbox (a second axis.below south west) (an axis.above north east);
	\end{tikzpicture}%
	\nobreak
	\hspace{10pt}%
	\nobreak
	\begin{minipage}[t]{8cm}%
	\vspace{0pt}%
\begin{lstlisting}
\begin{axis}[ymin=0,ymax=1,enlargelimits=false]
\addplot
	[blue!80!black,fill=blue,fill opacity=0.5] 
coordinates
{(0,0.1)	(0.1,0.15)	(0.2,0.5)	(0.3,0.62)
 (0.4,0.56)	(0.5,0.58)	(0.6,0.65)	(0.7,0.6)
 (0.8,0.58) (0.9,0.55)	(1,0.52)} 
|- (axis cs:0,0) -- cycle;
\addplot
	[red,fill=red!90!black,opacity=0.5]
coordinates 
{(0,0.25)	(0.1,0.27)	(0.2,0.24)	(0.3,0.24)
 (0.4,0.26)	(0.5,0.3)	(0.6,0.23)	(0.7,0.2)
 (0.8,0.15)	(0.9,0.1)	(1,0.1)}
|- (axis cs:0,0) -- cycle;
\addplot[green!20!black] coordinates
	{(0,0.4) (0.2,0.75) (1,0.75)};
\end{axis}

\begin{axis}
\addplot plot
	[id=parable,domain=-5:5] 
	function{4*x**2 - 5} 
	node[pin=180:{$4x^2-5$}]{};
\end{axis}
\end{lstlisting}
	\end{minipage}%
	\caption{Two examples of using \Tikz-paths after plotting commands.}
	\label{fig:after:paths}
\end{figure}

\subsubsection{The `\texttt{plot coordinates}' command}
The `\texttt{plot coordinates}' command is provided by \Tikz\ and takes a sequence of point coordinates as input. Example:
\begin{lstlisting}
\addplot plot coordinates {
	(0,0)
	(0.5,1)
	(1,2)
};
\end{lstlisting}
See the remarks above for features and limitations.

You can also supply error coordinates (reliability bounds) if you are interested in error bars. Simply append the error coordinates with `\texttt{+- (ex,ey)}' to the associated coordinate:
\begin{lstlisting}
\addplot plot coordinates {
	(0,0) +- (0.1,0)
	(0.5,1) +- (0.4,0.2)
	(1,2)
	(2,5) +- (1,0.1)
};
\end{lstlisting}
or 
\begin{lstlisting}
\addplot plot coordinates {
	(1300,1e-6) +- (0.1,0.2)
	(2600,5e-7) +- (0.2,0.5)
	(4000,1e-7) +- (0.1,0.01)
};
\end{lstlisting}
These error coordinates are only used in case of error bars, see section~\ref{sec:errorbars}. You will also need to configure whether these values denote absolute or relative errors.


\subsubsection{The `\texttt{plot file}' command}
\PGFPlots\ supports two ways to read plot coordinates of external files, and one of them is the \Tikz-command `\texttt{plot file}'. It is to be used like
\begin{lstlisting}
\addplot plot file {datafile.dat};
\end{lstlisting}
where \texttt{datafile.dat} looks like
\begin{lstlisting}
#Curve 0, 20 points
#x y type
0.00000 0.00000 i
0.52632 0.50235 i
1.05263 0.86873 i
1.57895 0.99997 i
...
9.47368 -0.04889 i
10.00000 -0.54402 i
\end{lstlisting}
This listing has been copied from~\cite[section~16.4]{tikz}. The format is produced by \textsc{gnuplot}. The mentioned section also contains some more details about using \texttt{plot file}.

\subsubsection{The `\texttt{plot function}' command}
The plot function command uses the external program \texttt{gnuplot} to compute coordinates. The resulting coordinates are written to a text file which will be plotted with \texttt{plot file}. \PGF\ checks whether coordinates need to be re-generated and calls \texttt{gnuplot} whenever necessary (this is usually the case if you change the number of samples, the argument to \texttt{plot function} or the plotted domain).

`\texttt{plot function}' is used as follows:
\begin{lstlisting}
\addplot plot[<options>] function{<gnuplot code>};
\end{lstlisting}
or
\begin{lstlisting}
\addplot[<line/marker-specifications>] 
	plot[<options>] function{<gnuplot code>};
\end{lstlisting}
The \lstinline!<line/marker-specifications>! determine the appearance of the plotted function; these parameters also affect the legend. The \lstinline!<options>! are specific to the gnuplot interface. These options are described in all detail in \cite[section~18.6]{tikz}. Summary:
\begin{description}
\item[\texttt{/tikz/domain=[start:end]}] Determines the plotted range. This is not necessarily the same as the axis limits (which are configured with the \texttt{xmin}/\texttt{xmax} options). 
\item[\texttt{/tikz/id=<id>}] A unique identifier for the current plot. It is used to generate temporary filenames for gnuplot output.
\item[\texttt{/tikz/prefix=<prefix>}] A common path prefix for temporary filenames (see \cite[section~18.6]{tikz} for details).
\item[\texttt{/tikz/samples=<number>}] Sets the number of sample points.
\item[\texttt{/tikz/raw gnuplot}] Disables the use of samples and domains.
\end{description}
Please refer to \cite[section~18.6]{tikz} for more details.

\subsubsection{The `\texttt{plot table}' command}
\PGFPlots\ comes with a new plotting command, the `\texttt{plot table}' macro. It's usage is similar in spirit to `\texttt{plot file}', but its flexibility is higher. Given a data file like
\begin{lstlisting}[tabsize=8]
dof	L2		Lmax		maxlevel
5	8.31160034e-02	1.80007647e-01	2
17	2.54685628e-02	3.75580565e-02	3
49	7.40715288e-03	1.49212716e-02	4
129	2.10192154e-03	4.23330523e-03	5
321	5.87352989e-04	1.30668515e-03	6
769	1.62269942e-04	3.88658098e-04	7
1793	4.44248889e-05	1.12651668e-04	8
4097	1.20714122e-05	3.20339285e-05	9
9217	3.26101452e-06	8.97617707e-06	10
\end{lstlisting}
one may want to plot `\texttt{dof}' versus `\texttt{L2}' or `\texttt{dof}' versus `\texttt{Lmax}'. This can be done by
\begin{lstlisting}
\begin{tikzpicture}
	\begin{loglogaxis}[
		xlabel=Dof,
		ylabel=$L_2$ error$]
	\addplot table[x=dof,y=L2] {datafile.dat};
	\end{loglogaxis}
\end{tikzpicture}
\end{lstlisting}
or
\begin{lstlisting}
\begin{tikzpicture}
	\begin{loglogaxis}[
		xlabel=Dof,
		ylabel=$L_infty$ error$]
	\addplot table[x=dof,y=Lmax] {datafile.dat};
	\end{loglogaxis}
\end{tikzpicture}
\end{lstlisting}
Alternatively, you can load the table \emph{once} and use it \emph{multiple} times:
\begin{lstlisting}
\pgfnumtableread{datafile.dat} to \table
...
\addplot table[x=dof,y=L2] from \table;
...
\addplot table[x=dof,y=Lmax] from \table;
...
\end{lstlisting}
I am not really sure how much time can be saved, but it works anyway.

If you do prefer to access columns by column indices instead of column names (or your tables do not have table names), you can also use
\begin{lstlisting}
\addplot table[x index=2,y index=3] {datafile.dat};
\addplot table[x=dof,y index=2] {datafile.dat};
\end{lstlisting}

Summary and remarks:
\begin{itemize}
	\item Use \texttt{plot table[x=COLNAME,y=COLNAME]} to access column names. Those names are case sensitive and need to exist.
	\item Use \texttt{plot table[x index=INTEGER,y index=INTEGER]} to access column indices. Indexing starts with~$0$. You may also use an index for~$x$ and a column name for~$y$.
	\item Use \lstinline!plot table[header=false] {FILE}! if your input file has no column names. Otherwise, the first non-comment line is checked for column names: if all entries are numbers, they are treated as numerical data; if one of them is not a number, all are treated as column names.
	\item It is possible to read error coordinates from tables as well. Simply add options `\texttt{x error}', `\texttt{y error}' or `\texttt{x error index}'/`\texttt{y error index}' to denote the columns containing coordinates. See section~\ref{sec:errorbars} for details about error bars.
	\item Use \lstinline!plot table {FILE};! to read the file, plot the selected columns, and throw its contents away afterwards.
	\item Use \lstinline!plot table[OPTIONS] from \macro! to use a pre--read table. Tables can be read using
\begin{lstlisting}
\pgfnumtableread{datafile.dat} to \macroname.
\end{lstlisting}
	\item The accepted input format is as follows:
		\begin{itemize}
			\item Columns are separated by white spaces.
			\item Any line starting with `\#' or `\%' is ignored.
			\item The first line will be checked if it contains numerical data. If there is a column in the first line which is \emph{no} number, the complete line is considered to be a header which contains column names. Otherwise it belongs to the numerical data and you need to access column indices instead of names.

			\item There is future support for a second header line which must start with `\texttt{\$flags}'. Currently, such a line is ignored. It may be used to provide number formatting hints like precision and number format if those tables shall be typeset inside of \TeX.
			\item The accepted number format is the same as for `\texttt{plot coordinates}', see above.
			\item If you omit column selectors, the default is to plot the first column against the second. That means \texttt{plot table} does exactly the same job as \texttt{plot file} for this case.
		\end{itemize}
\end{itemize}

\subsection{\texttt{\textbackslash addplot+[OPTIONS] ...}}
Does the same like \lstinline!\addplot[OPTIONS] ...! except that \texttt{OPTIONS} is \emph{appended} to the arguments which would have been taken for \lstinline!\addplot ...! (the element of the default list).

Example:
\begin{lstlisting}
\addplot+[only marks] coordinates {...};
\end{lstlisting}
will use the same line style, markers, colors as
\begin{lstlisting}
\addplot coordinates {...};
\end{lstlisting}
but it will only draw markers, no lines.

\subsection{Accessing axis coordinates with \texttt{axis cs}}
\label{sec:axis:coords}%
\PGFPlots\ provides a new coordinate system for use inside of an axis, the ``axis coordinate system'', \texttt{axis cs}.

It can be used to draw any \Tikz-graphics at axis coordinates. It is used like
\begin{lstlisting}
\draw 
	(axis cs:18943,2.873391e-05) 
|-	(axis cs:47103,8.437499e-06);
\end{lstlisting}
see section~\ref{sec:annot:plot} for example graphics.

\paragraph{Attention:} Whenever you draw additional graphics, consider using \texttt{axis cs}! It applies any logarithms, data scaling transformations or whatever \PGFPlots\ usually does!

\subsection{\texttt{\textbackslash addlegendentry\{name\}}}
Adds a single legend entry to the legend list. Example:
\begin{lstlisting}
\begin{tikzpicture}
\begin{axis}
\addplot[smooth,mark=*,blue] coordinates {
	(0,2)
	(2,3)
	(3,1)
};
\addlegendentry{Case 1}

\addplot[smooth,color=red,mark=x]
	coordinates {
		(0,0)
		(1,1)
		(2,1)
		(3,2)
	};
\addlegendentry{Case 2}
\end{axis}
\end{tikzpicture}
\end{lstlisting}
The outcome of this listing is shown in figure~\ref{fig:addlegendentry}.
\begin{figure}
	\centering
	\begin{tikzpicture}
	\begin{axis}
	\addplot[smooth,mark=*,blue] coordinates {
		(0,2)
		(2,3)
		(3,1)
	};
	\addlegendentry{Case 1}

	\addplot[smooth,color=red,mark=x]
		coordinates {
			(0,0)
			(1,1)
			(2,1)
			(3,2)
		};
	\addlegendentry{Case 2}
	\end{axis}
	\end{tikzpicture}

	\caption{An example for the \texttt{\textbackslash addlegendentry} command}%
	\label{fig:addlegendentry}%
\end{figure}%
It does not matter where \lstinline[breaklines=false]!\addlegendentry! commands are placed, only the sequence matters. You will need one \lstinline[breaklines=false]!\addlegendentry! for every \lstinline!\addplot! command.

There is limited support to change options for single legend entries using 
\begin{lstlisting}
\addlegendentry[key-value-list]{...}
\end{lstlisting}



\subsection{\texttt{\textbackslash legend\{LIST\}}}
\label{sec:legenddef}%
You can use
\begin{lstlisting}
\begin{tikzpicture}
\begin{axis}
...
\legend{$d=2$,$d=3$,$d=4$,$d=5$,$d=6$}
...
\end{axis}
\end{tikzpicture}
\end{lstlisting}
to assign a complete legend. The argument of \lstinline[breaklines=false]!\legend! is a comma--separated list of entries, one for each plot. It is processed using the \PGF-foreach command\footnote{Older versions of \PGFPlots\ used \texttt{\textbackslash legend\{first\textbackslash\textbackslash second\textbackslash\textbackslash third\textbackslash\textbackslash\}} instead of comma--separated lists. This syntax is still accepted.}.
The short marker/line combination shown in legends is acquired from the argument to \lstinline[breaklines=false]!\addplot[...]!.

\subsubsection{Legend appearance}
The style ``\texttt{every axis legend}'' determines the legend's position and outer appearance:
\begin{lstlisting}
\tikzstyle{every axis legend}+=[
		at={(0,0)},
		anchor=south west]
\end{lstlisting}
will draw it at the lower left corner of the axis while
\begin{lstlisting}
\tikzstyle{every axis legend}+=[
		at={(1,1)},
		anchor=north east]
\end{lstlisting}
means the upper right corner. The `\texttt{anchor}' option determines which point \emph{of the legend} will be placed at $(0,0)$ or $(1,1)$ (see below for more examples).

The legend is a \Tikz-matrix, so you can use any \Tikz\ option which affects
nodes and matrizes (see~\cite[section 13~and~14]{tikz}). The matrix is created by something like
\begin{lstlisting}
\matrix[style=every axis legend] {
	draw plot specification 1 & \node{legend 1}\\
	draw plot specification 2 & \node{legend 2}\\
	...
};
\end{lstlisting}
You can configure the number of horizontal legend entries with the axis-option ``\texttt{legend columns=NUMBER}''. For example,
\begin{lstlisting}
\begin{tikzpicture}
\begin{axis}[legend columns=4]
...
\legend{legend 1,legend 2,legend 3}
\end{axis}
\end{tikzpicture}
\end{lstlisting}
would use (up to)~$4$ adjacent legend entries.

Examples:
\begin{itemize}
\item 
\begin{lstlisting}
\tikzstyle{every axis legend}+=[
		at={(1.02,1)},
		anchor=north west]
\end{lstlisting}
draws the legend OUTSIDE TOP RIGHT.

\item
\begin{lstlisting}
\tikzstyle{every axis legend}+=[
		at={(1,0.5)},
		anchor=east,outer sep=0.5cm]
\end{lstlisting}
draws the legend INSIDE MIDDLE RIGHT, separated by 0.5cm from the axis.
\end{itemize}

\noindent
The default is
\begin{lstlisting}
\tikzstyle{every axis legend}=[%
	cells={anchor=center},
	inner xsep=3pt,inner ysep=2pt,nodes={inner sep=2pt,text depth=0.15em},
	anchor=north east,%
	shape=rectangle,%
	fill=white,%
	draw=black,
	at={(0.98,0.98)}
]
\end{lstlisting}
\paragraph{Attention:} you should \emph{not reset} the default style to stay compatible with future versions. If possible, use
\begin{lstlisting}
\tikzstyle{every axis legend}+=[...]
\end{lstlisting}


\subsection{\texttt{\textbackslash autoplotspeclist}}
This command should no longer be used, although it will be kept as technical implementation detail. Please use the `\texttt{cycle list}' option, section~\ref{sec:cycle:list}.

\subsection{\texttt{\textbackslash pgfmathlogtologten\{ARG\}}}
Assigns the result of $\texttt{ARG}/\log(10)$ to \lstinline!\pgfmathresult!.

\subsection{\texttt{\textbackslash logten}}
Expands to the constant $\log(10)$. Useful for logplots because $\log(10^i) = i\log(10)$. This command is only available inside of an \Tikz-picture.

\subsection{\texttt{\textbackslash pgfmathprintnumber\{NUM\}}}
Generates pretty--printed output\footnote{This method was previously \texttt{\textbackslash prettyprintnumber}. It's functionality has been included into \PGF\ and \texttt{\textbackslash prettyprintnumber} is now deprecated.} for the number \texttt{NUM}. This method is used for every tick label.

The number is printed using the current number printing options, see section~\ref{sec:number:printing} for the different number styles, rounding precision and rounding methods.

%--------------------------------------------------
% DEPRECATED:
% \subsection{\texttt{\textbackslash axispreset\{key=value,key=value\}}}
% Allows to define default options for any axis. For example,
% \begin{lstlisting}
% ...
% \axispreset{width=\textwidth}%
% ...
% \begin{tikzpicture}
% \begin{axis}
% ...
% \end{axis}
% \end{tikzpicture}
% \end{lstlisting}
% will produce a width of \lstinline!\textwidth! for any following axis. You can preset any of the \texttt{axis}-options described below.
% 
%-------------------------------------------------- 
\section{Option Reference}
There are several required and even more optional arguments to modify axes. They are used like
\begin{lstlisting}
\begin{tikzpicture}
\begin{axis}[key=value,key2=value2]
...
\end{axis}
\end{tikzpicture}
\end{lstlisting}
The overall appeareance can be changed with
\begin{lstlisting}
\tikzstyle{every axis}+=[line width=1pt]
\end{lstlisting}
for example. There are several other styles predefined to modify the appearance, see section~\ref{sec:styles}.

All \PGFPlots-options have the {\PGF}keys \emph{full key prefix}
\begin{lstlisting}
/pgfplots/...
\end{lstlisting}
Only the `\texttt{every ...}' styles have the full key prefix `\texttt{/tikz/every ...}' to maintain compatibility with `\lstinline!\tikzstyle!'.


\subsection{Axis options and \Tikz\ options}
You can mix \Tikz\ options and axis options inside of the axis arguments and in any of the axis--styles. For example,
\begin{lstlisting}
\tikzstyle{every axis legend}+=[
	legend columns=3,font=\Large]
\end{lstlisting}
assigns the `\texttt{legend columns}' option (an axis option) and will use `\texttt{font}' for drawing the legend (a \Tikz\ option).

The axis environments will process any known axis options, and all `\texttt{every}'--styles will be parsed for axis options. Every unknown option is supposed to be a \Tikz\ option and will be forward to the associated \Tikz\ drawing commands. For example, the `\lstinline{font=\Large}' above will be used as argument to the legend matrix, and the `\lstinline{font=\Large}' argument in 
\begin{lstlisting}
\tikzstyle{every axis label}+=[
	ylabel=Error,xlabel=Dof,font=\Large}
\end{lstlisting}
will be used in the nodes for axis labels (but not the axis title, for example).

It is an error if you assign incompatible options to axis labels, for example `\texttt{xmin}' and `\texttt{xmax}' can't be set inside of `\texttt{every axis label}'.

\paragraph{Remark:} \PGFPlots\ separates its own options from those of \Tikz. Misspelled options lead to \Tikz\ error messages. This has the unfortunate side effect that you need the \PGF\ 2.0-key syntax
\begin{lstlisting}
\pgfkeys{/pgfplots/<own style name>/.style=
	{<style variables>}}
\end{lstlisting}
instead of
\begin{lstlisting}
\tikzstyle{<own style name>}=[<style variables>].
\end{lstlisting}
Please refer to section~\ref{sec:styles:own} for more details.



\subsection{Plot Types}
\PGFPlots\ supports several two-dimensional line-plots like piecewise linear line plots, piecewise constant plots, smoothed plots, bar plots and comb plots. Most of them use the \PGF\ plot handler library directly, see \cite[section 18.8]{tikz}.

Plot types are part of the plot style, so they are set with options. The following list contains a short summary of \cite[section 18.8]{tikz}.

\subsubsection{Linear plots}
Linear (`sharp') plots are the default. Point coordinates are simply connected by straight lines. Use the \Tikz-option `\texttt{sharp plot}' in
\begin{lstlisting}
\addplot+[sharp plot] ...
\end{lstlisting}
to explicitly choose linear plot type. The `\texttt{+}' here means to use the normal plot cycle list and append `\texttt{sharp plot}' to its option list.

\subsubsection{Smooth plots}
Smooth plots interpolate smoothly between successive points as in figure~\ref{fig:addlegendentry} on page~\pageref{fig:addlegendentry}. They are set using the \Tikz-option `\texttt{smooth}',
\begin{lstlisting}
\addplot+[smooth] ...
\end{lstlisting}

\subsubsection{Constant plots}
Constant plots draw lines parallel to the $x$-axis to connect coordinates. The discontinuos edges may be drawn or not, and marks may be placed on left or right ends.

\begin{description}
\item[\texttt{const plot}] Connects all points with horizontal and vertical lines. Marks are placed left-handed on horizontal line segments, causing the plot to be right-sided continuous at all data points, see figure~\ref{fig:const:plots} top.
\item[\texttt{const plot mark left}] An alias for `\texttt{const plot}'.
\item[\texttt{const plot mark right}] A variant which places marks on the right of each line segment, causing plots to be left-sided continuous at coordinates.
\item[\texttt{jump mark left}] A variant of `\texttt{const plot mark left}' which does not draw vertical lines, see figure~\ref{fig:const:plots} bottom.
\item[\texttt{jump mark right}]A variant of `\texttt{const plot mark right}' which does not draw vertical lines, see figure~\ref{fig:const:plots} bottom.
\end{description}

\subsubsection{Bar plots}
Bar plots place horizontal (`\texttt{xbar}') or vertical (`\texttt{ybar}') bars at coordinates. They are enabled with
\begin{lstlisting}
\addplot+[xbar] ...
\end{lstlisting}
or
\begin{lstlisting}
\addplot+[ybar] ...
\end{lstlisting}
where the bar width is set by `\lstinline!bar width=10pt!'. You can access the current bar width with \lstinline!\pgfplotbarwidth!, see \cite[section~18.8]{tikz} for more details.

You can place multiple bar plots in one axis using horizontal/vertical shifts (maybe in multiples of \verb|\pgfplotbarwidth|)\footnote{Please note that up to now, axis limits are not adjusted correctly to reflect horizontal/vertical shifts. This will have to be fixed in a later version, sorry.}.

Figure~\ref{fig:bar:and:comb} top shows an example for `\texttt{ybar}' and `\texttt{xbar}' with (fictional) numbers. Users can place more than one bar plot into the same axis using horizontal/vertical shifts. The shifts in figure~\ref{fig:bar:and:comb} are set with
\label{xybar:styles}
\begin{lstlisting}[basicstyle=\footnotesize\ttfamily]
\pgfkeys{/pgfplots/ybar style/.style={
		/pgfplots/legend image code/.code={%
			\draw[##1,yshift=-0.2em] plot coordinates {(0cm,0.8em)};},
		legend style={
			at={(0.5,-0.15)},anchor=north,
			legend columns=-1,font=\footnotesize},
		/tikz/bar width=6pt,
		/tikz/every axis plot no 2/.append style=
			{xshift=0.6*\pgfplotbarwidth},
		/tikz/every axis plot no 3/.append style=
			{xshift=2*0.6*\pgfplotbarwidth},
		/tikz/every axis plot no 4/.append style=
			{xshift=3*0.6*\pgfplotbarwidth},
		/tikz/every axis plot no 5/.append style=
			{xshift=4*0.6*\pgfplotbarwidth},
	}
}
\pgfkeys{/pgfplots/xbar style/.style={
		/tikz/bar width=6pt,
		/tikz/every axis plot no 2/.append style=
			{yshift=\pgfplotbarwidth+1pt},
		/tikz/every axis plot no 3/.append style=
			{yshift=2*\pgfplotbarwidth+1pt},
		/tikz/every axis plot no 4/.append style=
			{yshift=3*\pgfplotbarwidth+1pt},
		/tikz/every axis plot no 5/.append style=
			{yshift=4*\pgfplotbarwidth+1pt},
	}
}
\end{lstlisting}
The `\texttt{ybar style}' is configured to generate bar-legends with the `\texttt{legend image code}' option (see~\ref{opt:legend:image:code}).

\subsubsection{Comb plots}
Comb plots are very similar to bar plots except that they emplot single horizontal/vertical lines instead of rectangles. They are used with
\begin{lstlisting}
\addplot+[xcomb] ...
\end{lstlisting}
or 
\begin{lstlisting}
\addplot+[ycomb] ...
\end{lstlisting}
see figure~\ref{fig:bar:and:comb} bottom.

\begin{figure}
	\centering
	\tikzstyle{every picture}+=[baseline]%
	\tikzstyle{every axis}+=[width=6cm,anchor=north west]%
	\lstset{basicstyle=\ttfamily\footnotesize,aboveskip=0pt,belowskip=0pt}%
	\begin{tikzpicture}%
		\begin{pgfinterruptboundingbox}
		\begin{axis}[ymin=0,ymax=1,enlargelimits=false,name=an axis]
		\addplot[const plot,fill=blue,draw=black] coordinates 
			{(0,0.1) (0.1,0.15) (0.2,0.5) (0.3,0.62) (0.4,0.56) (0.5,0.58) (0.6,0.65) (0.7,0.6) (0.8,0.58) (0.9,0.55) (1,0.52)} 
		|- (axis cs:0,0) -- cycle;
		\end{axis}
		\end{pgfinterruptboundingbox}
		\useasboundingbox (an axis.below south west) (an axis.above north east);
	\end{tikzpicture}%
	\nobreak
	\hspace{10pt}%
	\nobreak
	\begin{minipage}[t]{8cm}%
	\vspace{0pt}%
\begin{lstlisting}
\begin{tikzpicture}
\begin{axis}[ymin=0,ymax=1,enlargelimits=false]
\addplot
	[const plot,fill=blue,draw=black] 
coordinates
{(0,0.1)	(0.1,0.15)	(0.2,0.5)	(0.3,0.62)
 (0.4,0.56)	(0.5,0.58)	(0.6,0.65)	(0.7,0.6)
 (0.8,0.58) (0.9,0.55)	(1,0.52)} 
|- (axis cs:0,0) -- cycle;
\end{axis}
\end{tikzpicture}
\end{lstlisting}
	\end{minipage}%

	
	\begin{tikzpicture}[samples=8]
		\begin{pgfinterruptboundingbox}
		\begin{axis}[name=a second axis
		%	,at={(0,-4.5cm)}
		]
		\addplot+[jump mark left] plot[id=parablex,domain=-5:0] function{4*x**2 - 5};
		\addplot+[jump mark right] plot[id=cubic,domain=-5:0] function{0.7*x**3 + 50};
		\end{axis}
		\end{pgfinterruptboundingbox}
		\useasboundingbox (a second axis.below south west) (a second axis.above north east);
	\end{tikzpicture}%
	\nobreak
	\hspace{10pt}%
	\nobreak
	\begin{minipage}[t]{8cm}%
	\vspace{0pt}%
\begin{lstlisting}
\begin{tikzpicture}[samples=8]
\begin{axis}
\addplot+[jump mark left] 
	plot[id=parablex,domain=-5:0] 
	function{4*x**2 - 5};
\addplot+[jump mark right] 
	plot[id=cubic,domain=-5:0] 
	function{0.7*x**3 + 50};
\end{axis}
\end{tikzpicture}
\end{lstlisting}
	\end{minipage}%
	\caption{Examples for connected constant plots (top image, using `\texttt{const plot}') and plots with jumps (bottom image, using `\texttt{jump mark left}' and `\texttt{jump mark right}').}
	\label{fig:const:plots}
\end{figure}

\begin{figure}
	\centering
	\tikzstyle{every picture}+=[baseline]%
	\tikzstyle{every axis}+=[width=6cm,anchor=north west]%
	\lstset{basicstyle=\ttfamily\footnotesize,aboveskip=0pt,belowskip=0pt}%
\pgfkeys{/pgfplots/ybar style/.style={%
		/pgfplots/legend image code/.code={\draw[##1,yshift=-0.2em] plot coordinates {(0cm,0.8em)};},
		legend style={at={(0.5,-0.15)},anchor=north,legend columns=-1,font=\footnotesize},
		/tikz/bar width=6pt,
		/tikz/every axis plot no 2/.append style={xshift=0.6*\pgfplotbarwidth},%
		/tikz/every axis plot no 3/.append style={xshift=2*0.6*\pgfplotbarwidth},%
		/tikz/every axis plot no 4/.append style={xshift=3*0.6*\pgfplotbarwidth},%
		/tikz/every axis plot no 5/.append style={xshift=4*0.6*\pgfplotbarwidth},%
	}
}%
\pgfkeys{/pgfplots/xbar style/.style={%
		/tikz/bar width=6pt,
		/tikz/every axis plot no 2/.append style={yshift=\pgfplotbarwidth+1pt},%
		/tikz/every axis plot no 3/.append style={yshift=2*\pgfplotbarwidth+1pt},%
		/tikz/every axis plot no 4/.append style={yshift=3*\pgfplotbarwidth+1pt},%
		/tikz/every axis plot no 5/.append style={yshift=4*\pgfplotbarwidth+1pt},%
	}
}%
\begin{tikzpicture}%
	\begin{pgfinterruptboundingbox}
	\begin{axis}[
		xtick={1930,1950,1970},
		x tick label style={/pgf/number format/set thousands separator=},
		ylabel=Population,
		tick align=outside,
		ybar style,
		ybar,
		name=an axis,
	]
	\addplot[draw=blue,fill=blue!80!black] 
		coordinates {(1930,50e6) (1940,33e6) (1950,40e6) (1960,50e6) (1970,70e6)};

	\addplot[red,fill=red!80!black] 
		coordinates {(1930,38e6) (1940,42e6) (1950,43e6) (1960,45e6) (1970,65e6)};

	\addplot[brown,fill=brown!80!black] 
		coordinates {(1930,15e6) (1940,12e6) (1950,13e6) (1960,25e6) (1970,35e6)};
	\legend{FarFarAway,NotSoFar,Near}
	\end{axis}
	\end{pgfinterruptboundingbox}
	\useasboundingbox (an axis.below south west) (an axis.above north east);
\end{tikzpicture}%
	\nobreak
	\hspace{10pt}%
	\nobreak
	\begin{minipage}[t]{8cm}%
	\vspace{0pt}%
\begin{lstlisting}
\begin{axis}[xtick={1930,1950,1970},
	x tick label style={/pgf/number format/set thousands separator=},
	ylabel=Population,ybar style,ybar,tick align=outside]

\addplot[draw=blue,fill=blue!80!black] 
	coordinates {(1930,50e6) (1940,33e6)
		 (1950,40e6) (1960,50e6) (1970,70e6)};

\addplot[red,fill=red!80!black] 
	coordinates {(1930,38e6) (1940,42e6) 
		(1950,43e6) (1960,45e6) (1970,65e6)};

\addplot[brown,fill=brown!80!black] 
	coordinates {(1930,15e6) (1940,12e6) 
		(1950,13e6) (1960,25e6) (1970,35e6)};
\legend{FarFarAway,NotSoFar,Near}
\end{axis}
\end{lstlisting}
	\end{minipage}%

	
\begin{tikzpicture}%
	\begin{pgfinterruptboundingbox}
	\begin{axis}[
		xbar style,
		xbar,
		name=an axis,
		tick align=outside,
	]
	\addplot[draw=blue,pattern=horizontal lines light blue] 
		coordinates {
			(10,5) (15,10) (5,15) (24,20) (30,25)
		};
	\addplot[draw=black,pattern=horizontal lines dark blue] 
		coordinates {
			(3,5) (5,10) (15,15) (20,20) (35,25)
		};
	\end{axis}
	\end{pgfinterruptboundingbox}
	\useasboundingbox (an axis.below south west) (an axis.above north east);
\end{tikzpicture}%
	\nobreak
	\hspace{10pt}%
	\nobreak
	\begin{minipage}[t]{8cm}%
	\vspace{0pt}%
\begin{lstlisting}
\begin{axis}[xbar style,xbar,tick align=outside]
\addplot
[draw=blue,pattern=horizontal lines light blue] 
coordinates {(10,5) (15,10) (5,15) (24,20) (30,25)};

\addplot
[draw=black,pattern=horizontal lines dark blue] 
coordinates {(3,5) (5,10) (15,15) (20,20) (35,25)};
\end{axis}
\end{lstlisting}
	\end{minipage}%


\begin{tikzpicture}%
	\begin{pgfinterruptboundingbox}
	\begin{axis}[
		xtick={1930,1950,1970},
		x tick label style={/pgf/number format/set thousands separator=},
		ylabel=Population,
		ybar style,
		ycomb,
		name=an axis,
	]
	\addplot
		coordinates {(1930,50e6) (1940,33e6) (1950,40e6) (1960,50e6) (1970,70e6)};

	\addplot
		coordinates {(1930,38e6) (1940,42e6) (1950,43e6) (1960,45e6) (1970,65e6)};

	\addplot
		coordinates {(1930,15e6) (1940,12e6) (1950,13e6) (1960,25e6) (1970,35e6)};
	\legend{FarFarAway,NotSoFar,Near}
	\end{axis}
	\end{pgfinterruptboundingbox}
	\useasboundingbox (an axis.below south west) (an axis.above north east);
\end{tikzpicture}%
	\nobreak
	\hspace{10pt}%
	\nobreak
	\begin{minipage}[t]{8cm}%
	\vspace{0pt}%
\begin{lstlisting}
\begin{axis}[xtick={1930,1950,1970},
	x tick label style={/pgf/number format/set thousands separator=},
	ylabel=Population,ybar style,ybar]

\addplot coordinates {(1930,50e6) ... (1970,70e6)};
\addplot coordinates {(1930,38e6) ... (1970,65e6)};
\addplot coordinates {(1930,15e6) ... (1970,35e6)};

\legend{FarFarAway,NotSoFar,Near}
\end{axis}
\end{lstlisting}
	\end{minipage}%

	
\begin{tikzpicture}%
	\begin{pgfinterruptboundingbox}
	\begin{axis}[
		xbar style,
		xcomb,
		name=an axis,
	]
	\addplot
		coordinates {
			(10,5) (15,10) (5,15) (24,20) (30,25)
		};
	\addplot
		coordinates {
			(3,5) (5,10) (15,15) (20,20) (35,25)
		};
	\end{axis}
	\end{pgfinterruptboundingbox}
	\useasboundingbox (an axis.below south west) (an axis.above north east);
\end{tikzpicture}%
	\nobreak
	\hspace{10pt}%
	\nobreak
	\begin{minipage}[t]{8cm}%
	\vspace{0pt}%
\begin{lstlisting}
\begin{axis}[xbar style,xbar]
\addplot coordinates {(10,5) (15,10) (5,15) (24,20) (30,25)};

\addplot coordinates {(3,5) (5,10) (15,15) (20,20) (35,25)};
\end{axis}
\end{lstlisting}
	\end{minipage}%
	\caption{Examples for `\texttt{ybar}' (first image), `\texttt{xbar}' (second image), `\texttt{ycomb}' (third image) and `\texttt{xcomb}' (bottom image). Bar width and shifts are set in `\texttt{xbar style}' and `\texttt{ybar style}', see definitions on page~\pageref{xybar:styles}.}
	\label{fig:bar:and:comb}
\end{figure}


\subsection{Available markers}
\label{sec:markers}%
The following options of \Tikz\ may be useful for plots.
\subsubsection{Markers}
This list is copied from~\cite[section~29]{tikz}:
\begingroup
\newenvironment{longdescription}[0]{%
	\begin{list}{}{%
		\leftmargin=4.3cm
		\setlength{\labelwidth}{4.3cm}%
		\renewcommand{\makelabel}[1]{\hfill\textbf{\texttt{##1}}}%
	}%
}{%
	\end{list}%
}%
\def\showit#1{%
	\tikz\draw[%
		gray,
		thin,
		mark options={fill=yellow!80!black,draw=black,scale=2},
		x=0.8cm,y=0.3cm,
		#1]
	plot coordinates {(0,0) (1,1) (2,0) (3,1)};%
}%
\begin{longdescription}
	\item[mark=*] \showit{mark=*}
	\item[mark=x] \showit{mark=x}
	\item[mark=+] \showit{mark=+}
	\item[mark=ball] \showit{mark=ball}
\end{longdescription}
And with \lstinline!\usetikzlibrary{plotmarks}!:
\begin{longdescription}
	\item[mark=$-$] \showit{mark=-}
	\item[mark=$|$] \showit{mark=|}
	\item[mark=o] \showit{mark=o}
	\item[mark=asterisk] \showit{mark=asterisk}
	\item[mark=star] \showit{mark=star}
	\item[mark=oplus] \showit{mark=oplus}
	\item[mark=oplus*] \showit{mark=oplus*}
	\item[mark=otimes] \showit{mark=otimes}
	\item[mark=otimes*] \showit{mark=otimes*}
	\item[mark=square] \showit{mark=square}
	\item[mark=square*] \showit{mark=square*}
	\item[mark=triangle] \showit{mark=triangle}
	\item[mark=triangle*] \showit{mark=triangle*}
	\item[mark=diamond] \showit{mark=diamond}
	\item[mark=diamond*] \showit{mark=diamond*}
	\item[mark=pentagon] \showit{mark=pentagon}
	\item[mark=pentagon*] \showit{mark=pentagon*}
\end{longdescription}
All these options have been drawn with the additional options
\begin{lstlisting}
\draw[
	gray,
	thin,
	mark options={%
		scale=2,fill=yellow!80!black,draw=black
	}
]
\end{lstlisting}

\subsubsection{Line styles}
\def\showit#1{%
	\tikz\draw[%
		black,
		x=0.8cm,y=0.3cm,
		#1]
	plot coordinates {(0,0) (1,1) (2,0) (3,1)};%
}%
The following line styles are predefined in \Tikz:
\begin{longdescription}
	\item[style=solid] \showit{style=solid}
	\item[style=dotted] \showit{style=dotted}
	\item[style=densely dotted] \showit{style=densely dotted}
	\item[style=loosely dotted] \showit{style=loosely dotted}
	\item[style=dashed] \showit{style=dashed}
	\item[style=densely dashed] \showit{style=densely dashed}
	\item[style=loosely dashed] \showit{style=loosely dashed}
\end{longdescription}
You may need the option \lstinline!mark options={solid}! to avoid dotted or dashed marker boundaries. The string ``\texttt{style=}'' can be omitted.
\endgroup





\subsection{Axis descriptions}
\subsubsection{\texttt{[xy]label=TEXT}}
The options \texttt{xlabel} and \texttt{ylabel} change axis labels to `\texttt{TEXT}' which is any \TeX\ text. Use `\lstinline!{TEXT}!' if you need grouping.

Labels are \Tikz-Nodes which are placed with
\begin{lstlisting}
\node 
	[style=every axis label,
	style=every axis x label]
\node 
	[style=every axis label,
	style=every axis y label] 
\end{lstlisting}
so you can reconfigure their position and appearance. As for legends, the coordinate \lstinline!(0,0)! denotes the lower left axis corner and \lstinline!(1,1)! the upper right. 

The default styles are
\begin{lstlisting}
\tikzstyle{every axis label}=[]
\tikzstyle{every axis x label}=[
	at={(0.5,0)},
	below,
	yshift=-15pt]
\tikzstyle{every axis y label}=[
	at={(0,0.5)},
	xshift=-35pt,
	rotate=90]
\end{lstlisting}
You should use
\begin{lstlisting}
\tikzstyle{every axis label}+=[...]
\tikzstyle{every axis x label}+=[...]
\tikzstyle{every axis y label}+=[...]
\end{lstlisting}
to modify options to ensure compatibility with future versions.

\subsubsection{\texttt{title=TEXT}}
Adds a caption to the plot. This will place a \Tikz-Node with
\begin{lstlisting}
\node[style=every axis title] {TEXT};
\end{lstlisting}
to the current axis. An example is shown in figure~\ref{fig:titleexample}. The title is placed in the middle of the axis (the placing does not incorporate any axis descriptions). You can reconfigure the appearance and/or placing of the title for example with
\begin{lstlisting}
\tikzstyle{every axis title}+=[at={(0.75,1)}]
\end{lstlisting}
This will place the title at~75\% of the $x$-axis. The coordinate~$(0,0)$ is the lower left corner and~$(1,1)$ the upper right one.

\begin{figure}
	\centering
	\begin{tikzpicture}
	\begin{loglogaxis}[
		width=0.48\textwidth,
		xlabel=Dof,ylabel=Error,
		title={$\mu=0.1$, $\sigma=0.2$}]

		\addplot coordinates {
			(5,		8.312e-02)
			(17,	2.547e-02)
			(49,	7.407e-03)
			(129,	2.102e-03)
			(321,	5.874e-04)
			(769,	1.623e-04)
			(1793,	4.442e-05)
			(4097,	1.207e-05)
			(9217,	3.261e-06)
		};
	\end{loglogaxis}
	\end{tikzpicture}%
	\hfill
	\begin{tikzpicture}
	\begin{loglogaxis}[
		width=0.48\textwidth,
		xlabel=Dof,ylabel=Error,
		title={$\mu=1$, $\sigma=\frac{1}{2}$}]

		\addplot[color=red,mark=*] coordinates {
			(7,		8.472e-02)
			(31,	3.044e-02)
			(111,	1.022e-02)
			(351,	3.303e-03)
			(1023,	1.039e-03)
			(2815,	3.196e-04)
			(7423,	9.658e-05)
			(18943,	2.873e-05)
			(47103,	8.437e-06)
		};
	\end{loglogaxis}
	\end{tikzpicture}

	\begin{lstlisting}[basicstyle=\small\ttfamily]
	\begin{tikzpicture}
	\begin{loglogaxis}[
		width=0.48\textwidth,
		xlabel=Dof,ylabel=Error,
		title={$\mu=0.1$, $\sigma=0.2$}]

		\addplot coordinates {
			(5,		8.312e-02)
			(17,	2.547e-02)
			...
			(4097,	1.207e-05)
			(9217,	3.261e-06)
		};
	\end{loglogaxis}
	\end{tikzpicture}%
	\hfill
	\begin{tikzpicture}
	\begin{loglogaxis}[
		width=0.48\textwidth,
		xlabel=Dof,ylabel=Error,
		title={$\mu=1$, $\sigma=\frac{1}{2}$}]

		\addplot[color=red,mark=*] coordinates {
			(7,		8.472e-02)
			(31,	3.044e-02)
			...
			(18943,	2.873e-05)
			(47103,	8.437e-06)
		};
	\end{loglogaxis}
	\end{tikzpicture}
	\end{lstlisting}
	\caption{An example for the `\texttt{title}' option. Some data points have not been listed, the `\texttt{...}' is not part of the plot.}
	\label{fig:titleexample}
\end{figure}

\subsubsection{\texttt{legend columns=NUMBER}}
Allows to configure the maximum number of adjacent legend entries. The default is \texttt{legend columns=1}, so legends are placed vertically below each other. Examples can be found in section~\ref{sec:legendexamples:cols}.

\subsubsection{\texttt{legend plot pos=left$|$right$|$none}}
Configures where the small line specifications will be drawn: left of the description, right of the description or not at all. See an example in section~\ref{sec:legendexamples:plotpos}.

\subsubsection{\texttt{legend image code/.code={...}}}
\label{opt:legend:image:code}
Allows to replace the default images which are drawn inside of legends. The first argument to this option is the plot specification, a key-value list which has been determined by \lstinline!\addplot!.

The default is
\begin{lstlisting}
/pgfplots/legend image code/.code={%
	\draw[#1,mark repeat=2,mark phase=2] 
		plot coordinates {
			(0cm,0cm) 
			(0.3cm,0cm)
			(0.6cm,0cm)%
		};%
}
\end{lstlisting}
This default is inappropriate for bar or comb plots, see figure~\ref{fig:bar:and:comb} on page~\pageref{fig:bar:and:comb} for an example.


\subsection{Scaling Options}

\subsubsection{\texttt{width=DIMEN}, \texttt{height=DIMEN}}
The axis is always scaled such that its dimension is 

{\centering\lstinline!(\axisdefaultwidth,\axisdefaultheight)!.

}%
\noindent
The options \lstinline!width=DIMEN! and \lstinline!height=DIMEN! override the default scaling. If just one of them is specified, the other one is scaled such that the aspect ratio stays the same.

\noindent\underline{Remarks:} 
\begin{itemize}
	\item The scaling only affects the width of one unit in $x$-direction or the height for one unit in $y$-direction. Axis labels and tick labels won't be resized, but their size is used to determine the axis scaling.

	\item You can use the \Tikz-\lstinline!scale=NUMBER! option,
	\begin{lstlisting}
		\begin{tikzpicture}[scale=2]
		\begin{axis}
		...
		\end{axis}
		\end{tikzpicture}
	\end{lstlisting}
	to scale the complete picture.

	\item The \Tikz-options \lstinline!x! and \lstinline!y! which set the unit dimensions in $x$ and $y$ directions can be specified as arguments to \lstinline!\begin{axis}[x=1.5cm,y=2cm]! if needed (see below). These settings override the \lstinline!width! and \lstinline!height! options.

	\item You can also force a fixed width/height of the axis (without looking at labels) with
	\begin{lstlisting}
\begin{tikzpicture}
\begin{axis}[width=5cm,scale only axis]
	...
\end{axis}
\end{tikzpicture}
	\end{lstlisting}

	\item Use
	\begin{lstlisting}
\renewcommand{\axisdefaultwidth}{3cm}
\renewcommand{\axisdefaultheight}{6cm}
\begin{axis}
...
\end{axis}
	\end{lstlisting}
	to replace the default dimension.

	\item Please note that up to the writing of this manual, \PGFPlots\ only estimates the size needed for axis- and tick labels. It does not include legends which have been placed outside of the axis\footnote{I.e. the `\texttt{width}' option will not work as expected, but the bounding box is still ok.}. This may be fixed in future versions.

	Use the \lstinline!x=DIMEN!, \lstinline!y=DIMEN! and \lstinline!scale only axis! options if the scaling happens to be wrong.
\end{itemize}

\subsubsection{\texttt{scale only axis=[true$|$false]}}
This boolean option allows to apply \texttt{width} and \texttt{height} only to the axis rectangle. If `\texttt{scale only axis}' is enabled, label, tick and legend dimensions won't influence the size of the axis rectangle.

If \texttt{scale only axis=false} (the default), \PGFPlots\ will try to produce the desired width \emph{including} labels, titles and ticks.

\subsubsection{\texttt{x=DIMEN}, \texttt{y=DIMEN}}
Use
	\begin{lstlisting}
		\begin{tikzpicture}
		\begin{axis}[x=1.5cm,...]
		...
		\end{axis}
		\end{tikzpicture}
	\end{lstlisting}
	to set the unit size for one $x$-coordinate to 1.5cm. The same is possible for the $y$-coordinate.

	Setting $x$ explicitly overrides the \lstinline!width! option. Setting $y$ explicitly overrides the \lstinline!height! option.

	Please note that it is \emph{not} possible to specify \lstinline!x! as argument to \lstinline!tikzpicture!. The option 
	\begin{lstlisting}
		\begin{tikzpicture}[x=1.5cm]
		\begin{axis}
			...
		\end{axis}
		\end{tikzpicture}
	\end{lstlisting}
	won't have any effect because an axis is always scale to the final size 
	
	{\centering\lstinline!(\axisdefaultwidth,\axisdefaultheight)!
	
	}%
	\noindent
	(see the \lstinline!width! option).


\subsection{Error bars}
\label{sec:errorbars}
{%
\def\pgfplotserror#1{\ensuremath{\epsilon_{#1}}}%
\PGFPlots\ supports error bars for normal and logarithmic plots. 

Error bars are enabled for each plot separately, using options after \lstinline!\addplot!:
\begin{lstlisting}
\addplot plot[/pgfplots/error bars/.cd,x dir=both,y dir=both] ...
\end{lstlisting}
Error bars inherit all drawing options of the associated plot.

\begin{figure}
	\centering
	\tikzstyle{every picture}+=[baseline]%
	\tikzstyle{every axis}+=[width=6cm,anchor=north west]%
	\lstset{basicstyle=\ttfamily\footnotesize,aboveskip=0pt,belowskip=0pt}%
	\begin{tikzpicture}%
		\begin{pgfinterruptboundingbox}
		\begin{axis}[name=an axis]
		\addplot plot[
			/pgfplots/error bars/.cd,
			y dir=plus,y explicit,
		]
			coordinates
			{(0,0) +- (0.5,0.1) 
			(0.1,0.1)  +- (0.05,0.2)
			(0.2,0.2) 	+- (0,0.05)
			(0.5,0.5) +- (0.1,0.2)
			(1,1) +- (0.3,0.1)
			};
		\end{axis}
		\end{pgfinterruptboundingbox}
		\useasboundingbox (an axis.below south west) (an axis.above north east);
	\end{tikzpicture}%
	\nobreak
	\hspace{10pt}%
	\nobreak
	\begin{minipage}[t]{8cm}%
	\vspace{0pt}%
\begin{lstlisting}
\begin{axis}
\addplot plot[/pgfplots/error bars/.cd,
	y dir=plus,y explicit]
coordinates {(0,0) +- (0.5,0.1) 
	(0.1,0.1)  +- (0.05,0.2)
	(0.2,0.2) 	+- (0,0.05)
	(0.5,0.5) +- (0.1,0.2)
	(1,1) +- (0.3,0.1)};
\end{axis}
\end{lstlisting}
	\end{minipage}%

	\begin{tikzpicture}%
		\begin{pgfinterruptboundingbox}
		\begin{axis}[name=an axis]
		\addplot plot[
			/pgfplots/error bars/.cd,
			y dir=both,y explicit,
			x dir=both,x fixed=0.05,
			error mark=diamond*,
		]
			coordinates
			{(0,0) +- (0.5,0.1) 
			(0.1,0.1)  +- (0.05,0.2)
			(0.2,0.2) 	+- (0,0.05)
			(0.5,0.5) +- (0.1,0.2)
			(1,1) +- (0.3,0.1)
			};
		\end{axis}
		\end{pgfinterruptboundingbox}
		\useasboundingbox (an axis.below south west) (an axis.above north east);
	\end{tikzpicture}%
	\nobreak
	\hspace{10pt}%
	\nobreak
	\begin{minipage}[t]{8cm}%
	\vspace{0pt}%
\begin{lstlisting}
\begin{axis}
\addplot plot[/pgfplots/error bars/.cd,
	y dir=both,y explicit,
	x dir=both,x fixed=0.05,
	error mark=diamond*]
coordinates {(0,0) +- (0.5,0.1) 
	(0.1,0.1)  +- (0.05,0.2)
	(0.2,0.2) 	+- (0,0.05)
	(0.5,0.5) +- (0.1,0.2)
	(1,1) +- (0.3,0.1)};
\end{axis}
\end{lstlisting}
	\end{minipage}%

	\begin{tikzpicture}%
		\begin{pgfinterruptboundingbox}
		\begin{loglogaxis}[name=an axis]
		\addplot plot[
			/pgfplots/error bars/.cd,
			x dir=both,x fixed relative=0.5,
			y dir=both,y explicit relative,
			error mark=triangle*,
		]
			coordinates {
				(32,32)
				(64,64)
				(128,128) +- (0,0.3)
				(1024,1024) +- (0,0.2)
				(32068,32068)  +- (0,0.6)
				(64000,64000) +- (0,0.6)
				(128000,128000) +- (0,0.6)
			};
		\end{loglogaxis}
		\end{pgfinterruptboundingbox}
		\useasboundingbox (an axis.below south west) (an axis.above north east);
	\end{tikzpicture}%
	\nobreak
	\hspace{10pt}%
	\nobreak
	\begin{minipage}[t]{8cm}%
	\vspace{0pt}%
\begin{lstlisting}
\begin{loglogaxis}
\addplot plot[/pgfplots/error bars/.cd,
	x dir=both,x fixed relative=0.5,
	y dir=both,y explicit relative,
	error mark=triangle*]
table[x=XCol,y=YCol,
	x error=XErrCol,y error=YErrCol]
	{<file>};
\end{loglogaxis}
\end{lstlisting}
	\end{minipage}%

	\begin{tikzpicture}%
		\begin{pgfinterruptboundingbox}
		\begin{axis}[enlargelimits=false]
		\addplot[red,mark=*] plot[
			/pgfplots/error bars/.cd,
			y dir=minus,y fixed relative=1,
			x dir=minus,x fixed relative=1,
			error mark=none,
			error bar style={dotted},
		]
			coordinates
			{(0,0) 
			(0.1,0.1)  
			(0.2,0.2) 	
			(0.5,0.5) 
			(1,1) 
			};
		\end{axis}
		\end{pgfinterruptboundingbox}
		\useasboundingbox (an axis.below south west) (an axis.above north east);
	\end{tikzpicture}%
	\nobreak
	\hspace{10pt}%
	\nobreak
	\begin{minipage}[t]{8cm}%
	\vspace{0pt}%
\begin{lstlisting}
\begin{axis}[enlargelimits=false]
\addplot[red,mark=*] 
	plot[/pgfplots/error bars/.cd,
	y dir=minus,y fixed relative=1,
	x dir=minus,x fixed relative=1,
	error mark=none,
	error bar style={dotted}]
coordinates
	{(0,0) (0.1,0.1)  (0.2,0.2) 	
	(0.5,0.5) (1,1)};
\end{axis}
\end{lstlisting}
	\end{minipage}%
	\caption{Error bars with different styles for linear and log-scaled axis.}
	\label{fig:errorbars}
\end{figure}

\subsubsection{\texttt{error bars/x dir=[none$|$plus$|$minus$|$both]}}
Draws either no error bars at all, only marks at $x+\pgfplotserror x$, only marks at $x-\pgfplotserror x$ or marks at both, $x+\pgfplotserror x$ and $x-\pgfplotserror x$. The $x$-error $\pgfplotserror x$ is acquired using one of the following options.

\subsubsection{\texttt{error bars/x fixed=VAL}}
Provides a common, absolute error $\pgfplotserror x=\text{\texttt{VAL}}$ for all input coordinates.

\paragraph{Attention:} You cannot use absolute errors for logarithmic axis yet, sorry!

\subsubsection{\texttt{error bars/x fixed relative=PERCENT}}
Provides a common, relative error $\pgfplotserror x = \text{\texttt{VAL}} \cdot x$ for all input coordinates. \texttt{VAL} is thus given relatively to input $x$ coordinates such that $\text{\texttt{VAL}} = 1$ means $100\%$.

\subsubsection{\texttt{error bars/x explicit}}
Configures the error bar algorithm to draw $x$-error bars at any input coordinate for which user-specified errors are available.
 Each error is interpreted as absolute error.

The different input formats of errors are described in section~\ref{sec:errorbar:input}.

\paragraph{Attention:} You cannot use absolute errors for logarithmic axis yet, sorry!

\subsubsection{\texttt{error bars/x explicit relative}}
Configures the error bar algorithm to draw $x$-error bars at any input coordinate for which user-specified errors are available.
 Each error is interpreted as relative error, that means error marks are placed at $x (1 \pm \text{\texttt{VAL}})$ (works as for \texttt{error bars/x fixed relative}).

The different input formats of errors are described in section~\ref{sec:errorbar:input}.

\subsubsection{\texttt{error bars/y dir=[none$|$plus$|$minus$|$both]}}
This option (and the following ones) works in the same way as the \texttt{error bars/x ...} ones.
\subsubsection{\texttt{error bars/y fixed=VAL}}
\subsubsection{\texttt{error bars/y fixed relative=PERCENT}}
\subsubsection{\texttt{error bars/y explicit}}
\subsubsection{\texttt{error bars/y explicit relative}}

\subsubsection{\texttt{error bars/error mark=MARKER}}
Sets an error marker for any error bar. \texttt{MARKER} is expected to be a valid plot mark, see section~\ref{sec:markers}.
\subsubsection{\texttt{error bars/error mark options=\{\}}}
Sets a key-value list of options for any error mark. This option works similary to the \Tikz\ `\texttt{mark options}' key.

\subsubsection{\texttt{error bars/error bar style=\{\}}}
Appends the argument to `\texttt{/tikz/every error bar}' which is installed at the beginning of every error bar.
\subsubsection{\texttt{error bars/draw error bar/.code 2 args=\{\}}}
Allows to change the default drawing commands for error bars. The two arguments are
\begin{itemize} 
\item the source point, $(x,y)$ and
\item the target point, $(\tilde x,\tilde y)$.
\end{itemize}
Both are determined by \PGFPlots\ according to the options described above. The default code is
\begin{lstlisting}[basicstyle=\footnotesize\ttfamily]
/pgfplots/error bars/draw error bar/.code 2 args={%
	\pgfkeysgetvalue{/pgfplots/error bars/error mark}%
		{\pgfplotserrorbarsmark}%
	\pgfkeysgetvalue{/pgfplots/error bars/error mark options}%
		{\pgfplotserrorbarsmarkopts}%
	\draw #1 -- #2 node[pos=1,sloped,allow upside down] {%
		\expandafter\tikz\expandafter[\pgfplotserrorbarsmarkopts]{%
			\expandafter\pgfuseplotmark\expandafter{\pgfplotserrorbarsmark}%
			\pgfusepath{stroke}}%
	};
}
\end{lstlisting}

\subsubsection{Input formats of error coordinates}
\label{sec:errorbar:input}%
Error bars with explicit error estimations for single data points require some sort of input format. This applies to `\texttt{error bars/[xy] explizit}' and `\texttt{error bars/[xy] explizit relative}'.

Error bar coordinates can be read from `\texttt{plot coordinates}' or from `\texttt{plot table}'. The inline plot coordinates format is
\begin{lstlisting}
\addplot coordinates {
	(1,2) +- (0.4,0.2)
	(2,4) +- (1,0)
	(3,5)
	(4,6) +- (0.3,0.001)
}
\end{lstlisting}
where $(1,2) \pm (0.4,0.2)$ is the first coordinate, $(2,4) \pm (1,0)$ the second and so forth. The point $(3,5)$ has no error coordinate.

The `\texttt{plot table}' format is
\begin{lstlisting}
\addplot table[x error=COLNAME,y error=COLNAME]
\end{lstlisting}
or
\begin{lstlisting}
\addplot table[x error index=COLINDEX,y error index=COLINDEX]
\end{lstlisting}
These options are used as the `\texttt{x}' and `\texttt{x index}' options.

You can supply error coordinates even if they are not used at all; they will be ignored silently in this case.

}%

\subsection{Number formatting options}
\begingroup
\def\examplenumbers{%
	4.568,5e-04,0.1,24415.98123,123456.12345%
}%
\def\showexamplenumberswith#1{%
	\expandafter\showexamplenumberswithXX\expandafter{\examplenumbers}{#1}{#1}%
}%
\def\showexamplenumberswithFORMATTED#1#2{%
	\expandafter\showexamplenumberswithXX\expandafter{\examplenumbers}{#1}{#2}%
}%
\def\showexamplenumberswithXX#1#2#3{{%
	\small
	\noindent Example:

	\noindent
	{\ttfamily\textbackslash pgfkeys\{/pgf/number format/.cd,#3\}\par\noindent
	\foreach \x in {#1} {%
		\textbackslash pgfmathprintnumber\{\x\}\par\noindent
	}%
	}%
	 leads to 
	\pgfkeys{/pgf/number format/.cd,#2}%
	\foreach \x in {#1} {\pgfmathprintnumber{\x}\hspace{1em}}\unskip.
}}%
\label{sec:number:printing}%
Any tick label is computed numerically with \TeX's limited precision which means it needs to be rounded properly. Number rounding, the precision and various display styles can be configured with a set of options. These options have been included into \PGF, so the \PGF-manual is the latest reference. The following list is an extract of the \PGF-manual.

\begin{description}
\item[\texttt{/pgf/number format/fixed}]
Rounds the number to a fixed number of digits after the period, discarding any trailing zeros.

\showexamplenumberswith{fixed,precision=2} 

See section~\ref{sec:number:styles} for how to change the appearance.

\item[\texttt{/pgf/number format/fixed zerofill}]
Rounds the number to a fixed number of digits after the period, filling missing digits after the period with zeros.

\showexamplenumberswith{fixed zerofill,precision=2} 

See section~\ref{sec:number:styles} for how to change the appearance.

\item[\texttt{/pgf/number format/sci}]
Displays the number in scientific format, that means sign, mantisse and exponent (basis~$10$). The mantisse is rounded to the desired precision.

\showexamplenumberswith{sci,precision=2} 

See section~\ref{sec:number:styles} for how to change the exponential display style.

\item[\texttt{/pgf/number format/sci zerofill}]
As above, but the mantisse is rounded using `\texttt{fixed zerofill}'.

\showexamplenumberswith{sci zerofill,precision=2} 

See section~\ref{sec:number:styles} for how to change the exponential display style.


\item[\texttt{/pgf/number format/std}]
The default number format chooses either \texttt{fixed} or \texttt{sci}, depending on the order of magnitude. Let $n=s \cdot m \cdot 10^e$ be the input number and $p$ the current precision. If $-p/2 \le e \le 4$, the number is displayed using the fixed format. Otherwise, it is displayed using the scientific format. 

\showexamplenumberswith{std,precision=2} 

\item[\texttt{/pgf/number format/int detect}]
If the input number is an integer, no period is displayed at all. If not, the scientific format is choosen.

\begingroup
\def\examplenumbers{15,20,20.4,0.01,0}%
\showexamplenumberswith{int detect,precision=2}%
\endgroup

\item[\texttt{/pgf/number format/int trunc}]
Truncates every number to integers (discards any digit after the period).

\showexamplenumberswith{int trunc}%

\item[\texttt{/pgf/number format/precision=<INT>}]
Sets the desired rounding precision for any display operation. For scientific format, this affects the mantisse.
\end{description}

\subsubsection{Changing display styles}%
\label{sec:number:styles}%
You can change the way how numbers are displayed. For example, if you use the `\texttt{fixed}' style, the input number is rounded to the desired precision and the current fixed point display style is used to print the number. The same is applied to any other routine: first, rounding routines are used to get the correct digits, afterwards a display style generates proper \TeX-code.

\begin{description}
\def\examplenumbers{12.346}%
\item[\texttt{set decimal separator=TEXT}]
Assigns \texttt{TEXT} as decimal separator for any fixed point numbers (including the mantisse in sci format).

\item[\texttt{set thousands separator=TEXT}]
Assigns \texttt{TEXT} as thousands separator for any fixed point numbers (including the mantisse in sci format).

{%
\def\examplenumbers{1234.56,1234567890}%

\showexamplenumberswith{fixed zerofill,precision=2,set thousands separator={}}

\showexamplenumberswith{fixed zerofill,precision=2,set thousands separator={.}}

\showexamplenumberswith{fixed zerofill,precision=2,set thousands separator={,}}

\showexamplenumberswithFORMATTED
	{fixed zerofill,precision=2,set thousands separator={\relax{,}}}%
	{fixed zerofill,precision=2,set thousands separator=\{\textbackslash relax\{,\}\}}%
}%

The last example employs commas and disables the default comma-spacing. The \texttt{\textbackslash relax} command is a technical thing which enables \PGF keys to recognize the braces.

\item[\texttt{use period}] 
A predefined style which installs periods `\texttt{.}' as decimal separators and commas `\texttt{,}' as thousands separators. This style is the default.

\showexamplenumberswith{fixed,precision=2,use period}

\item[\texttt{use comma}] 
A predefined style which installs commas `\texttt{,}' as decimal separators and periods `\texttt{.}' as thousands separators.

\showexamplenumberswith{fixed,precision=2,use comma}

\item[\texttt{skip 0.=true$|$false}]
	Configures whether numbers like $0.1$ shall be typeset as $.1$ or not.
{\def\examplenumbers{0.56,0.01}%

\showexamplenumberswith{fixed zerofill,precision=2,skip 0.}

\showexamplenumberswith{fixed zerofill,precision=2,skip 0.=false}
}

\item[\texttt{sci 10e}] Uses $m \cdot 10^e$ for any number displayed in scientific format.

\showexamplenumberswith{sci,sci 10e}

\item[\texttt{sci 10\textasciicircum e}] The same as `\texttt{sci 10e}'.

\item[\texttt{sci e}] Uses the `$1e{+}0$' format which is generated by common scientific tools for any number displayed in scientific format.

\showexamplenumberswith{sci,sci e}

\item[\texttt{sci E}] The same with an uppercase `\texttt{E}'.

\showexamplenumberswith{sci,sci E}

\item[\texttt{sci subscript}] Typesets the exponent as subscript for any number displayed in scientific format. This style requires very few space.

\showexamplenumberswith{sci,sci subscript}
\end{description}

\subsubsection[\texttt{log identify minor tick positions}]{\texttt{log identify minor tick positions=[true$|$false]}}
\label{sec:identify:minor:log}%
Set this to \texttt{true} if you want to identify log--plot tick labels at positions 
\[ i \cdot 10^j \]
with $i \in \{2,3,4,5,6,7,8,9\},\, j \in \Z$. This may be valueable in conjunction with the `\texttt{extra x ticks}' and `\texttt{extra y ticks}' options. An example is shown in figure~\ref{fig:identify:minor:log}: The axis range is so small that only one tick label $10^j$ is inside of it. Extra tick labels can be placed on top of the normal ticks, and placing ticks at $i \cdot 10^j $ may be appropriate.
\begin{figure}
	\centering
	\tikzstyle{every axis}+=[%
		width=6cm,
		xmin=7e-3,xmax=7e-2,
		extra x ticks={3e-2,6e-2},
		extra x tick style={major tick length=0pt,font=\footnotesize}
	]%
	\begin{tikzpicture}%
		\begin{loglogaxis}
		\addplot coordinates {
			(1e-2,10)
			(3e-2,100)
			(6e-2,200)
		};
		\end{loglogaxis}
	\end{tikzpicture}%
	\hspace{0.2cm}
	\begin{tikzpicture}%
		\begin{loglogaxis}[log identify minor tick positions=false]
		\addplot coordinates {
			(1e-2,10)
			(3e-2,100)
			(6e-2,200)
		};
		\end{loglogaxis}%
	\end{tikzpicture}%

	\begin{lstlisting}
		\tikzstyle{every axis}+=[
			xmin=7e-3,xmax=7e-2,
			extra x ticks={3e-2,6e-2},
			extra x tick style={major tick length=0pt,
				font=\footnotesize}]
	\end{lstlisting}
	\caption{Demonstration of `\texttt{log identify minor tick positions}': it is enabled for the left image and disabled for the right one.}
	\label{fig:identify:minor:log}
\end{figure}

\subsubsection[\texttt{log number format code/.code=\{\}}]{\texttt{log number format code/.code=\{\TeX-code\}}}
Provides \TeX-code to generate log plot tick labels. Argument `\texttt{\#1}' is the (natural) logarithm of the tick position.
The default implementation changes the log basis to~$10$ and invokes `\texttt{log base 10 number format code}' with the result. It also checks the other log plot options.


\subsubsection[\texttt{log base 10 number format code/.code=\{\}}]{\texttt{log base 10 number format code/.code=\{\TeX-code\}}}
Allows to change the overall appearance of base 10 log plot tick labels. The default is
\begin{lstlisting}
log base 10 number format code/.code={%
	$10^{\pgfmathprintnumber{#1}}$}
\end{lstlisting}
where the `\texttt{log plot exponent style}' allows to change number formatting options.

\subsubsection{\texttt{log plot exponent style=\{key-value pairs\}}}
Allows to configure the number format of log plot exponents. This style is installed just before `\texttt{log base 10 number format code}' will be invoked. Please note that this style will be installed within the default code for `\texttt{log number format code}'. Figure~\ref{fig:log:exponent:style} shows two examples.
\begin{figure}
	\tikzset{samples=15}%
	\tikzstyle{every picture}+=[baseline]%
	\tikzstyle{every axis}+=[
			width=7cm,
			xlabel=$x$,
			ylabel=$f(x)$,
			extra y ticks={45},
			legend style={at={(0.03,0.97)},anchor=north west}]%
	\begin{tabular}{*{2}{p{7cm}}}%
	\begin{tikzpicture}%
		\begin{semilogyaxis}[
			log plot exponent style/.style={/pgf/number format/fixed zerofill,/pgf/number format/precision=1}
		]
			\addplot plot[id=gnuplot_exp,domain=-5:10] function{exp(x)};
			\addplot plot[id=gnuplot_expv,domain=-5:10] function{exp(2*x)};
			\legend{$e^x$,$e^{2x}$}
		\end{semilogyaxis}
	\end{tikzpicture}%
	&
	\begin{tikzpicture}%
		\begin{semilogyaxis}[
			log plot exponent style/.style={/pgf/number format/fixed,/pgf/number format/use comma,/pgf/number format/precision=2}
		]
			\addplot plot[id=gnuplot_exp,domain=-5:10] function{exp(x)};
			\addplot plot[id=gnuplot_expv,domain=-5:10] function{exp(2*x)};
			\legend{$e^x$,$e^{2x}$}
		\end{semilogyaxis}
	\end{tikzpicture}%
	\\
{\small
\begin{lstlisting}[tabsize=2]
log plot exponent style/.style={
	/pgf/number format/fixed zerofill,
	/pgf/number format/precision=1}
\end{lstlisting}
}%
	&
{\small
\begin{lstlisting}[tabsize=2]
log plot exponent style/.style={
	/pgf/number format/fixed,
	/pgf/number format/use comma,
	/pgf/number format/precision=2}
\end{lstlisting}
}%
\\
	\end{tabular}
	\caption{Example of using `\texttt{log plot exponent style}'.}
	\label{fig:log:exponent:style}
\end{figure}


%--------------------------------------------------
% \subsubsection{Defining own display styles}
% You can define own display styles, although this may require some insight into \TeX-programming. Here are two examples:
% \begin{enumerate}
% 	\item A new fixed point display style: The following code defines a new style named `\texttt{my own fixed point style}' which uses $1{\cdot}00$ instead of $1.00$.
% \begin{lstlisting}
% \def\myfixedpointstyleimpl#1.#2\relax{%
% 	#1{\cdot}#2%
% }%
% \def\myfixedpointstyle#1{%
% 	\pgfutilensuremath{%
% 	\ifpgfmathfloatroundhasperiod
% 		\expandafter\myfixedpointstyleimpl#1\relax
% 	\else
% 		#1%
% 	\fi
% 	}%
% }
% \pgfkeys{/my own fixed point style/.code={%
% 	\let\pgfmathprintnumber@fixed@style=\myfixedpointstyle}
% }%
% \end{lstlisting}
% 	You only need to overwrite the macro \lstinline!\pgfmathprintnumber@fixed@style!. This macro takes one argument (the result of any numerical computations). The \TeX-boolean \lstinline!\ifpgfmathfloatroundhasperiod! is true if and only if the input number contains a period.
% 
% 	\item An example for a new scientific display style:
% \begin{lstlisting}
% % #1:
% % 		0 == '0' (the number is +- 0.0),
% % 		1 == '+', 
% % 		2 == '-',
% % 		3 == 'not a number'
% % 		4 == '+ infinity'
% % 		5 == '- infinity'
% % #2: the mantisse
% % #3: the exponent
% \def\myscistyle#1#2e#3\relax{%
% 	...
% }
% \pgfkeys{/my own sci style/.code={%
% 	\let\pgfmathfloatrounddisplaystyle=\myscistyle},
% }%
% \end{lstlisting}
% \end{enumerate}
%-------------------------------------------------- 
\endgroup




\subsection{Specifying the plotted range}

\subsubsection{\texttt{[xy]min=COORD}, \texttt{[xy]max=COORD}}
The options \texttt{xmin}, \texttt{xmax} and \texttt{ymin}, \texttt{ymax} allow to define the axis limits, i.e. the lower left and the upper right corner. Some remarks:
\begin{itemize}
	\item The axis limits determine the plotted range. Everything else will be clipped away.
	\item The width of every unit $x$-coordinate will be scaled such that the plot has width \lstinline!\axisdefaultwidth! (including the tick- and axis labels). The height of every unit $y$-coordinate will be scaled such that the plot has height \lstinline!\axisdefaultheight!.

	You can override this default behavior with the options \lstinline!width=DIMEN!, \lstinline!height=DIMEN!, \lstinline!x=DIMEN! and \lstinline!y=DIMEN!, see below.
	\item If one of \lstinline!xmin! or \lstinline!xmax! is missing, the $x$-interval will be determined automatically (see \lstinline!\addplot!). The same holds true if one of \lstinline!ymin! or \lstinline!ymax! is missing: in this case, the $y$-interval will be determined automatically.

	If $x$-limits have been specified explicitly and $y$-limits are computed automatically, the automatic computation of $y$-limits will only considers points which fall into the specified $x$-range (and vice--versa). See option \texttt{clip limits} for details.

	\item The option \lstinline!enlargelimits! will automatically increase the plotted range.
\end{itemize}

\subsubsection{\texttt{[xy]mode=normal$|$log}}
Allows to choose between normal plots or logplots for each $x,y$-combination. Default is \lstinline!xmode=normal!, \lstinline!ymode=normal!.


\subsubsection{\texttt{clip limits=[true$|$false]}}

\subsubsection{\texttt{enlargelimits=[true$|$false$|$auto$|$VAL]}}
Enlarges the axis size somewhat if enabled.

You can set \texttt{xmin}, \texttt{xmax} and \texttt{ymin}, \texttt{ymax} to the minimum/maximum values of your data and \texttt{enlargelimits} will enlarge the canvas such that the axis doesn't touch the plots.

\begin{itemize}
	\item The value \texttt{true} enlarges all axes.
	\item The value \texttt{false} uses tight axis limits as specified by the user (or read from input coordinates).
	\item The value \texttt{auto} will enlarge limits only for axis for which axis limits have been determined automatically.
	\item All other values like `\texttt{enlargelimits=0.1}' will enlarge all axis limits relatively (in this example, 10\% of the axis limits will be added at all sides).
	\item If no relative threshold is provided, axis enlargement is done relatively to $x_\text{max}-x_\text{min}$ for normal plots and absolutely for log--plots up to now.
\end{itemize}
A small value of \texttt{enlargelimits} may avoid problems with large markers near the boundary.





\subsection{Tick Options}

\subsubsection{\texttt{[xy]tick=\{LIST\}}}
The options \texttt{xtick} and \texttt{ytick} assigns a list of \emph{Positions} where ticks shall be placed. The argument \texttt{LIST} will be used inside of a \lstinline!\foreach \x in {LIST}! statement, and \texttt{LIST} contains one of the following formats:
\begin{itemize}
	\item \lstinline!{0,1,2,5,8}! (a series of coordinates),
	\item \lstinline!{0,...,5}! (the same as \lstinline!{0,1,2,3,4,5}!),
	\item \lstinline!{0,2,...,10}! (the same as \lstinline!{0,2,4,6,8,10}!),
	\item \lstinline!{9,...,3.5}! (the same as \lstinline!{9, 8, 7, 6, 5, 4}!),
	\item See \cite[Section~34]{tikz} for a more detailed definition of the options.
	\item Use `\texttt{xtick=}' to use the default tick placement rules.
	\item Use `\lstinline!xtick=\empty!' to disable any ticks.
\end{itemize}
For logplots, \PGFPlots\ will apply $\log(\cdot)$ to each element in `\texttt{LIST}'. 

\paragraph{Attention:} You can't use the `\texttt{...}' syntax if the elements are too large for \TeX! For example, `\texttt{xtick=1.5e5,2e7,3e8}' will work (because the elements are interpreted as strings, not as numbers), but `\texttt{xtick=1.5,3e5,...,1e10}' will fail because it involves real number arithmetics beyond \TeX's capacities.
\vspace*{0.3cm}

\noindent
The default choice for tick \emph{positions} in normal plots is to place a tick at each coordinate~$i\cdot h$. The step size~$h$ depends on the axis scaling and the axis limits. It is chosen from a list a ``feasable'' step sizes such that neither too much nor too few ticks will be generated. The default for logplots it to place ticks at positions $10^i$ in the axis' range. Which positions depends on the axis scaling and the dimensions of the picture. The default tick positions can be reconfigured with
\begin{itemize}
	\item `\lstinline!max space between ticks=NUMBER!' where the integer argument denotes the maximum space between adjacent ticks in full points. The suffix ``\texttt{pt}'' has to be omitted and fractional numbers are not supported. The default is~\axisdefaulttickwidth.
	\item `\lstinline!try min ticks=NUMBER!' configures a loose lower bound on the number of ticks. It should be considered as a suggestion, not a tight limit. The default is~\axisdefaulttryminticks. This number will increase the number of ticks if `\texttt{max space between ticks}' produces too few of them.
	\item `\lstinline!try min ticks log=NUMBER!' The same for logarithmic axis.
\end{itemize}
The total number of ticks may still vary because not all fractional numbers in the axis' range are valid tick positions.


\noindent
The tick \emph{appearance} can be (re-)configured with
\begin{lstlisting}
\tikzstyle{every tick}=[very thin,gray]
\tikzstyle{every minor tick}=[]
\end{lstlisting}
or
\begin{lstlisting}
\tikzstyle{every tick}+=[very thin,gray]
\tikzstyle{every minor tick}+=[black]
\end{lstlisting}
Please prefer the `\texttt{+=}' versions to ensure compatibility with future versions.

This style commands can be used at any time. The tick line width can be configured with the \lstinline!axis!-options `\texttt{tickwidth}' and `\texttt{subtickwidth}'.

\subsubsection{\texttt{extra [xy] ticks=\{LIST\}}}
Adds additional tick positions and tick labels to the $x$~or~$y$ axis. `Additional' tick positions do not affect the normal tick placement algorithms, they are drawn after the normal ticks. This has two benefits: first, you can add single, important tick positions without disabling the default tick label generation and second, you can draw tick labels `on top' of others, possibly using different style flags.

Section~\ref{sec:identify:minor:log} shows an application of `\texttt{extra x ticks}' for log--plots: there, only one tick label has been generated due to the limited axis range, and extra $x$ ticks allows to compensate this effect. There also an example for linear axis in section~\ref{sec:examples:extra:ticks}.

Remarks:
\begin{itemize} 
\item Use \texttt{extra x ticks} to highlight special tick positions. The use of \texttt{extra x ticks} does not affect minor tick/grid line generation, so you can place extra ticks at positions $j\cdot 10^i$ in log--plots. 
\item Extra ticks are always typeset as major ticks, that means they employ the options `\texttt{major tick length}' and respect options like `\texttt{grid=major}'.
\item Use the style `\texttt{every extra x tick}' (`\texttt{every extra y tick}') to configure the appearance.
\item You can also use `\texttt{extra x tick style=\{...\}}' which has the same effect.
\end{itemize}

\subsubsection{\texttt{space between ticks=NUMBER}}
see Options \texttt{xtick} and \texttt{ytick} for a description.

\subsubsection{\texttt{try min ticks=NUMBER}}
see Options \texttt{xtick} and \texttt{ytick} for a description.

\subsubsection{\texttt{try min ticks log=NUMBER}}
see Options \texttt{xtick} and \texttt{ytick} for a description.

\subsubsection{\texttt{[xy]tickten=\{LIST\}}}
The options \texttt{xtickten} and \texttt{ytickten} allow to place ticks at selected positions $10^k, k \in \text{\texttt{LIST}}$. They are only used for logplots. The syntax for `\texttt{LIST}' is the same as above for `\texttt{xtick=LIST}' or `\texttt{ytick=LIST}'.

Using `\texttt{xtickten=\{1,2,3,4\}}' is equivalent to `\texttt{xtick=\{1e1,1e2,1e3,1e4\}}', but it requires fewer computational time and it allows to use `\texttt{xtickten=\{1,...,4\}}'.

\subsubsection{\texttt{[xy]ticklabels=\{LIST\}}}
Assigns a \emph{list} of tick \emph{labels} to each tick position. Tick \emph{positions} are assigned using the \texttt{xtick} and \texttt{ytick}-options.

This is one of two options to assign tick labels directly. The other option is `\texttt{xticklabel=\{COMMAND\}}' (or \texttt{yticklabel=\{COMMAND\}}).
Option `\texttt{\*ticklabel}' offers higher flexibility while `\texttt{\*ticklabels}' is easier to use.

The argument \texttt{LIST} has the same format as for ticks, that means
\begin{lstlisting}
xticklabels={$\frac{1}{2}$,$e$}
\end{lstlisting}
Denotes the two--element--list $\{\frac 12, e\}$. The list indices match the indices of the tick positions. If you need commas inside of list elements, use 
\begin{lstlisting}
xticklabels={{0,5}, $e$}.
\end{lstlisting}


Example:
\begin{lstlisting}
\begin{tikzpicture}
\begin{axis}[
	xtick={-1.5,-1,...,1.5},
	xticklabels={%
		$-1\frac 12$,
		$-1$,
		$-\frac 12$,
		$0$,
		$\frac 12$,
		$1$}
]
\addplot[smooth,blue,mark=*] coordinates {
	(-1,	1)
	(-0.75,	0.5625)
	(-0.5,	0.25)
	(-0.25,	0.0625)
	(0,		0)
	(0.25,	0.0625)
	(0.5,	0.25)
	(0.75,	0.5625)
	(1,		1)
};
\end{axis}
\end{tikzpicture}
\end{lstlisting}
yields figure~\ref{fig:ticklistexample}.
\begin{figure}
	\centering
	\begin{tikzpicture}[baseline]
	\begin{axis}[
		anchor=north east,
		width=6cm,
		xtick={-1.5,-1,...,1.5},
		xticklabels={%
			$-1\frac 12$,
			$-1$,
			$-\frac 12$,
			$0$,
			$\frac 12$,
			$1$}
	]
	\addplot[smooth,blue,mark=*] coordinates {
		(-1,	1)
		(-0.75,	0.5625)
		(-0.5,	0.25)
		(-0.25,	0.0625)
		(0,		0)
		(0.25,	0.0625)
		(0.5,	0.25)
		(0.75,	0.5625)
		(1,		1)
	};
	\end{axis}
	\end{tikzpicture}
	\hspace{10pt}%
	\begin{minipage}[t]{6cm}%
	\vspace{0pt}%
	\begin{lstlisting}
xtick={-1.5,-1,...,1.5},
xticklabels={%
	$-1\frac 12$,
	$-1$,
	$-\frac 12$,
	$0$,
	$\frac 12$,
	$1$}
	\end{lstlisting}
	\end{minipage}
	\caption{An example for the \texttt{xticklabels} option.}
	\label{fig:ticklistexample}
\end{figure}

\subsubsection{\texttt{[xy]ticklabel=\{COMMAND\}}}
Use \texttt{xticklabel} or \texttt{yticklabel} to change the \TeX-command which creates the tick \emph{labels} assigned to each tick position (see options \texttt{xtick} and \texttt{ytick}). 

This is one of two options to assign tick labels directly. The other option is `\texttt{xticklabels=\{LIST\}}' (or \texttt{yticklabels=\{LIST\}}). Option `\texttt{\*ticklabel}' offers higher flexibility while `\texttt{\*ticklabels}' is easier to use.

The argument `\texttt{COMMAND}' can be any \TeX-text. The following commands are valid inside of \texttt{COMMAND}:
\begin{description}
	\item[\textbackslash tick] The current element of option \lstinline!xtick! (or \lstinline!ytick!).
	\item[\textbackslash ticknum] The current tick number, starting with~0 (a counter).
\end{description}
The default argument is 
\begin{itemize}
	\item \lstinline!\axisdefaultticklabel! for normal plots and 
	\item \lstinline!\axisdefaultticklabellog! for logplots, see below.
\end{itemize}
(the same holds for \lstinline!yticklabel!). The defaults are set to
\begin{lstlisting}
\def\axisdefaultticklabel{%
	$\pgfmathprintnumber{\tick}$%
}

\def\axisdefaultticklabellog{%
	\pgfkeysgetvalue{/pgfplots/log number format code/.@cmd}\pgfplots@log@label@style
	\expandafter\pgfplots@log@label@style\tick\pgfeov
}
\end{lstlisting}
that means you can configure the appearance of linear axis with the number formatting options described in section~\ref{sec:number:printing} and logarithmic axis with \texttt{log number format code}, see below.

You can change the appearance of tick labels with
\begin{lstlisting}
\tikzstyle{every tick label}+=[
	font=\tiny,
	/pgf/number format/sci]
\end{lstlisting}
and/or
\begin{lstlisting}
\tikzstyle{every x tick label}+=[
	above,
	/pgf/number format/fixed zerofill]
\end{lstlisting}
and
\begin{lstlisting}
\tikzstyle{every y tick label}+=[font=\bfseries]
\end{lstlisting}
Another possibility is to use 
\begin{lstlisting}
\begin{axis}[y tick label style={above,
	/pgf/number format/fixed zerofill}
]
...
\end{axis}
\end{lstlisting}
which has the same effect as the `\texttt{every x tick label}' statement above. This is possible for all \PGFPlots-\texttt{every}-styles, see section~\ref{sec:styles}.

\subsubsection{\texttt{scaled ticks=[true$|$false]}}
\label{sec:scaled:ticks}%
Allows to factor out common exponents in tick labels. For example, if you have tick labels $20000,40000$ and $60000$, you may want to save some space and write $2,4,6$ with a separate factor `$\cdot 10^4$'. Use `\texttt{scaled ticks=true}' to enable this feature (default is \texttt{true}). An example is shown in figure~\ref{fig:scaled:ticks}.

\begin{figure}
	\begin{center}
	\tikzstyle{every axis}+=[width=6cm]%
	\tikzstyle{every picture}+=[baseline]%
	\begin{tikzpicture}
		\begin{axis}
			\addplot coordinates {
				(20000,0.0005)
				(40000,0.0010)
				(60000,0.0020)
			};
		\end{axis}
	\end{tikzpicture}%
	\hspace{0.5cm}%
	\begin{tikzpicture}
		\begin{axis}[scaled ticks=false]
			\addplot coordinates {
				(20000,0.0005)
				(40000,0.0010)
				(60000,0.0020)
			};
		\end{axis}
	\end{tikzpicture}
	\end{center}
	\caption{Left figure: `\texttt{scaled ticks=true}', right figure: `\texttt{scaled ticks=false}'.}
	\label{fig:scaled:ticks}
\end{figure}

\subsubsection{\texttt{tick scale label code/.code=\{\TeX-code\}}}
Allows to change the default code for scaled tick labels. The default is
\begin{lstlisting}
tick scale label code/.code={$\cdot 10^{#1}$}.
\end{lstlisting}

\subsubsection{\texttt{scale ticks below=EXPONENT}}
Allows fine tuning of the '\texttt{scaled ticks}' algorithm: if the axis limits are of magnitude $10^e$ and $e<$\texttt{EXPONENT}, the common prefactor~$10^e$ will be factored out. The default is 
\makeatletter
\pgfplots@scale@ticks@below@exponent
\makeatother.

\subsubsection{\texttt{scale ticks above=EXPONENT}}
Allows fine tuning of the '\texttt{scaled ticks}' algorithm: if the axis limits are of magnitude $10^e$ and $e>$\texttt{EXPONENT}, the common prefactor~$10^e$ will be factored out. The default is
\makeatletter
\pgfplots@scale@ticks@above@exponent
\makeatother.


\subsubsection{\texttt{tickpos=left$|$right$|$both}}
Allows to choose where to place the small tick lines. Default is ``\texttt{both}''. This setting applies to both $x$~and~$y$ axis where ``left'' and ``right'' mean ``bottom'' and ``top'' for~$y$.

\subsubsection{\texttt{[xy]minorticks=true$|$false}}
\subsubsection{\texttt{[xy]majorticks=true$|$false}}
\subsubsection{\texttt{ticks=minor$|$major$|$both$|$none}}
Enables/disables the small tick lines either for single axis or for all of them. Major ticks are those placed at the tick positions and minor ticks are only used in log plots. Please note that minor ticks are automatically disabled if \texttt{xtick} is not a uniform range\footnote{A uniform list means the difference between all elements is~$1$ for normal plots or, for logplots, $\log(10)$.}. Default is to use minor and major ticks.

You can configure the length of the tick line with `\texttt{minor tick length=DIMEN}' and `\texttt{major tick length=DIMEN}' (where DIMEN is a \TeX-length like 1cm) and its appearance using the following styles:
\begin{lstlisting}
\tikzstyle{every tick}+=[color=black] % applies to major and minor ticks,
\tikzstyle{every minor tick}+=[thin]  % applies only to minor ticks,
\tikzstyle{every major tick}+=[thick] % applies only to major ticks.
\end{lstlisting}
There is also the style ``\texttt{every tick}'' which applies to both, major and minor ticks.

\subsubsection{\texttt{major tick length=DIMEN}}
\subsubsection{\texttt{minor tick length=DIMEN}}
Allows to configure the length of the small tick lines. Minor ticks are only displayed for logarithmic plots. See option ``\texttt{ticks}'' for appearance styles for ticks.

\subsubsection{\texttt{[xy]minorgrids=true$|$false}}
\subsubsection{\texttt{[xy]majorgrids=true$|$false}}
\subsubsection{\texttt{grid=minor$|$major$|$both$|$none}}
Enables/disables different grid lines. Major grid lines are placed at the normal tick positions (see \texttt{xmajorticks}) while minor grid lines are placed at minor ticks (see \texttt{xminorticks}). 

Grid lines will be drawn before tick lines are processed, so ticks will be drawn on top of grid lines. You can configure the appearance of grid lines with the styles
\begin{lstlisting}
\tikzstyle{every axis grid}=[style=help lines] % applies to major and minor grids,
\tikzstyle{every minor grid}+=[color=blue]     % applies only to minor grid lines,
\tikzstyle{every major grid}+=[thick]          % applies only to major grid lines.
\end{lstlisting}
An example for grid lines can be found in section~\ref{sec:gridlines}.






\subsection{Style options}
\subsubsection{All supported styles}
\PGFPlots\ knows the following styles which are used with \lstinline!\tikzstyle{...}=[...]! or \lstinline!\tikzstyle{...}+=[...]!.
\label{sec:styles}%
\begin{description}
\item[\texttt{every axis}] Installed at the beginning of every axis. \Tikz\ options inside of it will be used for the axis rectangle, and any axis descriptions.
\item[\texttt{every semilogx axis}] Installed at the beginning of every plot with linear $x$~axis and logarithmic $y$~axis, but after `\texttt{every axis}'.
\item[\texttt{every semilogy axis}] Likewise, but with interchanged roles for $x$~and~$y$.
\item[\texttt{every loglog axis}] Installed at the beginning of every double--logarithmic plot.
\item[\texttt{every linear axis}] Installed at the beginning of every plot with normal axis scaling.
\item[\texttt{every axis plot}] Used for the \Tikz-drawing command in any `\lstinline!\addplot!' command. May only contain \Tikz\ options.
\item[\texttt{every axis plot no \#}] Used for every \#th plot where $\#=1,2,3,4,\dotsc$.
\item[\texttt{every axis label}] Used for $x$~and~$y$ axis label. You can use `\texttt{at=\{(x,y)\}}' to set its position where $(0,0)$ refers to the lower left corner and $(1,1)$ to the upper right one.
\item[\texttt{every axis x label}] Used for $x$~labels, installed after `\texttt{every axis label}'.
\item[\texttt{every axis y label}] Like `\texttt{every axis x label}', just for~$y$.
\item[\texttt{every axis title}] Used for any axis title. The `\texttt{at=\{(x,y)\}}' command works as for `\texttt{every axis label}'.
\item[\texttt{every tick}] Installed for each of the small tick lines.
\item[\texttt{every minor tick}] Used for each minor tick line, installed after `\texttt{every tick}'.
\item[\texttt{every major tick}] Used for each major tick line, installed after `\texttt{every tick}'.
\item[\texttt{every x tick}]
\item[\texttt{every minor x tick}]
\item[\texttt{every major x tick}]
\item[\texttt{every y tick}]
\item[\texttt{every minor y tick}]
\item[\texttt{every major y tick}]
\item[\texttt{every axis grid}] Used for each grid line.
\item[\texttt{every minor grid}] Used for each minor grid line, installed after `\texttt{every axis grid}'.
\item[\texttt{every major grid}] Likewise, for major grid lines.
\item[\texttt{every axis x grid}]
\item[\texttt{every minor x grid}]
\item[\texttt{every major x grid}]
\item[\texttt{every axis y grid}]
\item[\texttt{every minor y grid}]
\item[\texttt{every major y grid}]
\item[\texttt{every tick label}] Used for each $x$~and~$y$ tick labels.
\item[\texttt{every x tick label}] Used for each $x$~tick label, installed after `\texttt{every tick label}'.
\item[\texttt{every y tick label}] Likewise, for $y$~tick labels.
\item[\texttt{every x tick scale label}] Configures placement and display of the nodes containing the order of magnitude of $x$~tick labels, see~\ref{sec:scaled:ticks} for more information about \texttt{scaled ticks}.
\item[\texttt{every y tick scale label}] Likewise, but for $y$-tick scale labels.
\item[\texttt{every extra x tick}] Allows to configure the appearance of `\texttt{extra x ticks}'. This style is installed before touching the first extra $x$~tick, so you can set any option which affects tick generation, for example
\item[\texttt{every error bar}] Installed for every error bar.
\begin{lstlisting}
\tikzstyle{every extra x tick}+=[grid=major]
\tikzstyle{every extra x tick}+=[major tick length=0pt]
\tikzstyle{every extra x tick}+=[/pgf/number format=sci subscript]
\end{lstlisting}
or something like that.
\item[\texttt{every extra y tick}] Likewise, but for extra $y$-ticks.
\item[\texttt{every axis legend}] Installed for each legend. The legend's position can be placed in the same way as for `\texttt{every axis label}', see above.
\end{description}

\subsubsection{\texttt{x tick label style=\{...\}}}
\subsubsection{\texttt{y tick label style=\{...\}}}
\subsubsection{\texttt{x tick scale label style=\{...\}}}
\subsubsection{\texttt{y tick scale label style=\{...\}}}
\subsubsection{\texttt{label style=\{...\}}}
\subsubsection{\texttt{x label style=\{...\}}}
\subsubsection{\texttt{y label style=\{...\}}}
\subsubsection{\texttt{title style=\{...\}}}
\subsubsection{\texttt{tick style=\{...\}}}
\subsubsection{\texttt{minor tick style=\{...\}}}
\subsubsection{\texttt{major tick style=\{...\}}}
\subsubsection{\texttt{x tick style=\{...\}}}
\subsubsection{\texttt{minor x tick style=\{...\}}}
\subsubsection{\texttt{major x tick style=\{...\}}}
\subsubsection{\texttt{y tick style=\{...\}}}
\subsubsection{\texttt{minor y tick style=\{...\}}}
\subsubsection{\texttt{major y tick style=\{...\}}}
\subsubsection{\texttt{grid style=\{...\}}}
\subsubsection{\texttt{minor grid style=\{...\}}}
\subsubsection{\texttt{major grid style=\{...\}}}
\subsubsection{\texttt{x grid style=\{...\}}}
\subsubsection{\texttt{minor x grid style=\{...\}}}
\subsubsection{\texttt{major x grid style=\{...\}}}
\subsubsection{\texttt{y grid style=\{...\}}}
\subsubsection{\texttt{minor y grid style=\{...\}}}
\subsubsection{\texttt{major y grid style=\{...\}}}
\subsubsection{\texttt{extra x tick style=\{...\}}}
\subsubsection{\texttt{extra y tick style=\{...\}}}
All these options are equivalent to the corresponding `\texttt{every ...}'--styles. For example, `\texttt{label style=\{...\}}' has the same effect as 
\begin{lstlisting}
\tikzstyle{every axis label}+=[...]
\end{lstlisting}
but can be provided as an option (or as part of a user defined style).
See section~\ref{sec:styles} for more information about the available styles.

\subsubsection{Assigning own styles}
\label{sec:styles:own}%
Use 
\begin{lstlisting}
\pgfkeys{/pgfplots/<style name>/.style={key-value-list}}
\end{lstlisting}
to create own styles. You \emph{can't use} \lstinline!\tikzstyle{<style name>}=[]!. For example,
\begin{lstlisting}
\pgfkeys{/pgfplots/my personal style/.style=
	{grid=major,font=\large}
}%
\begin{tikzpicture}
\begin{axis}[my personal style]
	\addplot coordinates {(0,0) (1,1)};	
\end{axis}
\end{tikzpicture}
\end{lstlisting}
results in figure~\ref{fig:personal:style}.

\begin{figure}
{%
\centering
\pgfkeys{/pgfplots/my personal style/.style={grid=major,font=\large}}%
\begin{tikzpicture}
\begin{axis}[my personal style]
	\addplot coordinates {(0,0) (1,1)};	
\end{axis}
\end{tikzpicture}

\begin{lstlisting}
		\pgfkeys{/pgfplots/my personal style/.style=
			{grid=major,font=\large}
		}%
		...
		\begin{axis}[my personal style]
		...
\end{lstlisting}
}%
\caption{An example of using a personal style.}%
\label{fig:personal:style}
\end{figure}




\subsection{General purpose Options}

\subsubsection{\texttt{cycle list=LIST}}
\label{sec:cycle:list}%
\subsubsection{\texttt{cycle list name=MACRO NAME}}
Allows to specify a list of plot specifications which will be used for each \hbox{\lstinline!\addplot!}-command without explicit plot specification.

There are several possiblities to change it:
\begin{enumerate}
	\item Use one of the predefined lists,
\begin{lstlisting}
\begin{tikzpicture}
	\begin{axis}[
		cycle list name=\coloredplotspeclist]
		...
	\end{axis}
\end{tikzpicture}
\end{lstlisting}
or
\begin{lstlisting}
\begin{tikzpicture}
	\begin{axis}[
		cycle list name=\blackwhiteplotspeclist]
		...
	\end{axis}
\end{tikzpicture}
\end{lstlisting}
	\item Provide the list explicitly,
\begin{lstlisting}
\begin{axis}[cycle list={%
	{blue,mark=*},
	{red,mark=square},
	{dashed,mark=o},
	{loosely dotted,mark=+},
	{brown!60!black,
		mark options={fill=brown!40},
		mark=otimes*}}
]
...
\end{axis}
\end{lstlisting}
	(This example list requires \lstinline!\usetikzlibrary{plotmarks}!).
	\item Define macro names and use them with `\texttt{cycle list name}':
\begin{lstlisting}
\pgfcreateplotcyclelist{\mylist}{%
	{blue,mark=*},
	{red,mark=square},
	{dashed,mark=o},
	{loosely dotted,mark=+},
	{brown!60!black,mark options={fill=brown!40},mark=otimes*}}
}
...
\begin{axis}[cycle list name=\mylist]
	...
\end{axis}
\end{lstlisting}
\end{enumerate}

\paragraph{Remark:} You can also terminate single entries with `\lstinline!\\!' as in
\begin{lstlisting}
\begin{axis}[cycle list={%
	blue,mark=*\\%
	red,mark=square\\%
	dashed,mark=o\\%
	loosely dotted,mark=+\\%
	brown!60!black,
		mark options={fill=brown!40},
		mark=otimes*\\}
]
...
\end{axis}
\end{lstlisting}
In this case, the \emph{last} entry also needs a terminating `\lstinline!\\!', but you can omit braces around the single entries.

\subsubsection{\texttt{anchor=NAME}}
\label{option:anchor}%
This option shifts the axis horizontally and vertically such that the axis anchor (a point on the axis) is placed at coordinate $(0,0)$.

Anchors are useful in conjunction with horizontal or vertical alignment of plots, see the examples in section~\ref{sec:align} and section~\ref{sec:halign}.

There are three sets of anchors available: anchors positioned on the axis rectangle, anchors on the outer bounding box and anchors which have one coordinate on the outer bounding box and the other one at a position of the axis rectangle.

{%
\tikzstyle{every picture}+=[background rectangle/.style={help lines},show background rectangle]%
\pgfnumtableread{pgfplots.testplot} to \plottable
\def\plot{%
	\begin{axis}[
		width=5cm,
		name=test plot,
		xlabel=$x$,
		ylabel={$y$},% = \frac 12 \cdot x^3 - 4 x^2 -16 x$},
		y label style={yshift=-15pt},
		legend style={at={(1.03,1)},anchor=north west},
		title=A test plot.
	]
		\addplot table from{\plottable};
		%\addplot coordinates {(0,0) (1,1)};
		\addlegendentry{$f(x)$}
		\addplot[red] plot[id=gnuplot_ppp,domain=-40:40,samples=120] function{10000*sin(x/3)};
		\addlegendentry{$g(x)$}
	\end{axis}
}%
\def\showit#1#2{%
	%\node[show them,#2] at (test plot.#1) {(s.#1)};
	\node[pin=#2:(s.#1),fill=black,circle,scale=0.3] at (test plot.#1) {};
}%
In more detail, we have
\tikzstyle{every pin}=[opacity=0.5,fill=yellow,rectangle,rounded corners=3pt,font=\tiny]
Anchors on the axis rectangle,
		\begin{center}
			\begin{tikzpicture}
				\plot
				\showit{north}{90}
				\showit{north west}{135}
				\showit{west}{180}
				\showit{south west}{225}
				\showit{south}{270}
				\showit{south east}{305}
				\showit{east}{0}
				\showit{north east}{45}
				\showit{center}{90}
			\end{tikzpicture}
		\end{center}
Anchors on the outer bounding box,
		\begin{center}
			\begin{tikzpicture}
				\plot
				\showit{outer north}{90}
				\showit{outer north west}{135}
				\showit{outer west}{180}
				\showit{outer south west}{225}
				\showit{outer south}{270}
				\showit{outer south east}{305}
				\showit{outer east}{0}
				\showit{outer north east}{45}
				\showit{outer center}{90}
			\end{tikzpicture}
		\end{center}
Finally there are anchors which have one coordinate on the outer bounding box, and one on the axis rectangle,
		\begin{center}
			\begin{tikzpicture}[show background rectangle]
				\plot
				{\tikzstyle{every pin}+=[pin distance=1cm]%
				\showit{above north}{90}
				}%
				\showit{above north east}{90}
				\showit{right of north east}{0}
				\showit{right of east}{0}
				\showit{right of south east}{0}
				\showit{below south east}{-90}
				{\tikzstyle{every pin}+=[pin distance=1cm]%
				\showit{below south}{-90}
				}%
				\showit{below south west}{-90}
				\showit{left of south west}{180}
				\showit{left of west}{180}
				\showit{left of north west}{180}
				\showit{above north west}{90}
			\end{tikzpicture}
		\end{center}
The default value is \texttt{anchor=south west}. You can use anchors in conjunction with \lstinline!\begin{pgfinterruptboundingbox}! to reduce the bounding box, see section~\ref{sec:bounding:box:example} for an example.
}

\subsubsection{\texttt{at=<Coordinate expression>}}
Assigns a position for the complete axis image. This option works similarly to the \texttt{at}-option of \lstinline!\node[at=<Coordinate expression>]!.

\subsubsection{\texttt{hide axis=[true$|$false]}}
Allows to hide the axis. No outer rectangle, no tick marks and no labels will be drawn. Only titles and legends will be processed as usual.

Axis scaling and clipping will be done as if you did not use \texttt{hide axis}.




\subsection{Miscellaneous options}

\subsubsection{\texttt{disablelogfilter}}
Disables numerical evaluation of $\log(x)$ in \TeX. If you specify this option, any plot coordinates and tick positions must be provided as $\log(x)$ instead of $x$. This may be faster and -- possibly -- more accurate than the numerical log. The current implementation of $\log(x)$ normalizes~$x$ to $m\cdot 10^e$ and computes
\[ \log(x) = \log(m) + e \log(10) \]
where $y = \log(m)$ is computed with a newton method applied to $\exp(y) - m$. The normalization involves string parsing without \TeX-registers. You can savely evaluate $\log(1\cdot 10^{-7})$ although \TeX-registers would produce an underflow for such small numbers. 

The exponential function is implemented in pgf using a truncated Euler McLaurin--series.

The current implementation of $\log(\cdot)$ is done by the macro \lstinline!\pgfmathlog! which is provided by package \lstinline!pgfmathlog.sty! (contained in the \PGFPlots-bundle).

\subsubsection{\texttt{disabledatascaling}}
\label{sec:disabledatascaling}%
Disables internal re-scaling of input data. Normally, every input data like plot coordinates, tick positions or whatever, are parsed without using \TeX's limited number precision. Then, a transformation like 
	\[ T(x) = 10^{q-m} \cdot x \]
is applied to every input coordinate/position where $m$ is ``the order of $x$'' base~$10$. Example: $x=1234 = 1.234\cdot 10^3$ has order~$m=4$ while $x=0.001234 = 1.234\cdot 10^{-3}$ has order $m=-2$. The parameter~$q$ is the order of the axis' width/height.

The \textbf{effect} is that your plot coordinates can be of \emph{arbitrary precision} like $0.0000001$ and $0.0000004$. For these two coordinates, \PGFPlots\ will use 100pt and 400pt internally. The transformation is quit fast since it relies only on period shifts.

The option ``\texttt{disabledatascaling}'' disables this data transformation. That means you are restricted to coordinates which \TeX\ can handle\footnote{Please note that the axis' scaling requires to compute $1/( x_\text{max} - x_{\text{min}} )$. The option \texttt{disabledatascaling} may lead to overflow or underflow in this context, so use it with care! Normally, the data scale transformation avoids this problem.}.

So far, the data scale transformation applies only to normal axis (logarithmic scales do not need it). 


\subsubsection{\texttt{[xy]filter=CMD}}
The option \texttt{xfilter} and \texttt{yfilter} allow coordinate filtering. A coordinate filter maps an input coordinate to an output coordinate (or discards it completely).

Coordinate filters are useful in automatic processing system, where \PGFPlots\ is used to display automatically generated plots. You may not want to filter your coordinates by hand, so these options provide a tool to do this automatically.

Be warned: `\texttt{xfilter}' and `\texttt{yfilter}' are advanced options and may require advanced knowledge about \TeX.

They are used as in the following example. The code
\begin{lstlisting}
\def\myOwnXfilter#1\to#2{%
	\def#2{0.5}%
}%
\begin{tikzpicture}
\begin{axis}[xfilter={\myOwnXfilter}]
\addplot coordinates {
	(4,0)
	(6,1)
};
\end{axis}
\end{tikzpicture}
\end{lstlisting}
will result in
\begin{center}
{%
\def\myOwnXfilter#1\to#2{%
	\def#2{0.5}%
}%
\begin{tikzpicture}
\begin{axis}[xfilter={\myOwnXfilter}]
\addplot coordinates {
	(4,0)
	(6,1)
};
\end{axis}
\end{tikzpicture}
}%
\end{center}
because all $x$-coordinates are replaced by~$0.5$.

The Argument `\texttt{CMD}' is the name of a \TeX-macro which takes exactly two arguments which are separated by the string `\texttt{\string\to}'. Such a macro is defined as
\begin{lstlisting}
\def\exampleFilter#1\to#2{%
	\def#2{#1}%
}%
\end{lstlisting}
This example uses the \TeX-command \lstinline!\def! to define variables and commands. The arguments are used as follows:
\begin{itemize}
	\item \PGFPlots\ invokes the filter with argument \texttt{\#1} set to the input coordinate. For $x$-filters, this is the $x$-coordinate as it is specified to \lstinline!\addplot!, for $y$-filters it is the $y$-coordinate.
	\item If the corresponding axis is logarithmic, \texttt{\#1} is the \emph{logarithm} of the coordinate as a real number, for example \texttt{\#1=4.2341}.
	\item The arguments to coordinate filters are not transformed. You may need to call coordinate parsing routines.
	\item Argument \texttt{\#2} is the name of a \TeX\ command. The filter should assign this command. The first filter above assigned the constant~$0.5$ and the second filter did not filter anything because it is the identity.

	The replacement text of \texttt{\#2} is expected to be \emph{either} empty \emph{or} a real number (without any length-suffix like `cm' or `pt'). If it is empty, the coordinate won't be drawn at all, it will be thrown away.
\end{itemize}

\subsubsection{\texttt{execute at begin plot=COMMANDS}}
This axis option allows to invoke `\texttt{COMMANDS}' at the beginning of each \lstinline!\addplot! command. The argument `\texttt{COMMANDS}' can be any \TeX\ content.

You may use this in conjunction with \texttt{xfilter=...} to reset any counters or whatever. An example would be to change every 4th coordinate.

\subsubsection{\texttt{execute at end plot=COMMANDS}}
This axis option allows to invoke `\texttt{COMMANDS}' after each \lstinline!\addplot! command. The argument `\texttt{COMMANDS}' can be any \TeX\ content.
