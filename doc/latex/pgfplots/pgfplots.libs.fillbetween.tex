\section{Fill between}
\begin{pgfplotslibrary}{fillbetween}
	The |fillbetween| library allows to fill the area between two arbitrary named plots.
	It can also identify segments of the intersections and fill the segments individually.
\end{pgfplotslibrary}


\begin{addplotoperation}[]{fill between}{[\marg{source options}]}
	A special plotting operation which takes two named paths on input and generates one or more paths resembling the filled area between the input paths.

	A |fill between| plot is different from other plotting operations with respect to the following items:
	\begin{enumerate}
		\item It has no own markers.
		\item It supports no |pos| nodes.
		\item It supports no error bars.
		\item It cannot be stacked.
	\end{enumerate}
\end{addplotoperation}

\begin{pgfplotskey}{fill between/of=\meta{A} and \meta{B}}
	This key is mandatory. It defines which paths should be combined.

	The arguments are names which have been assigned to paths or plots in advance using |name path|. Paths with these two names are expected in the same tikz picture.

	The |fillbetween| library supports a variety of input paths, namely
	\begin{itemize}
		\item plots of functions, i.e.\ each $x$ coordinate has at most one $y$ coordinate,
		\item plots with interruptions,
		\item \Tikz\ paths which meet the same restrictions and are labelled by |name path|,
		\item |smooth| curves or |curveto| paths,
		\item mixed smooth / non-smooth parts.
	\end{itemize}
	However, it has at most restricted support (or none at all) for paths which
	\begin{itemize}
		\item have self-intersections (i.e.\ parametric plots might pose a problem),
		\item have coordinates which are given in a strange input sequence,
		\item consist of lots of individually separated sub-paths (like |mesh| or |surf| plots),
	\end{itemize}

	Note that the input paths do not necessarily need to be given in the same sequence, see |fill between/reverse|.
\end{pgfplotskey}

\begin{pgfplotskey}{/tikz/name path=\marg{name}}
	A \Tikz\ instruction which assigns a name to a path or plot.

	This is mandatory to define input arguments for |fill between/of|.
\end{pgfplotskey}

\begin{pgfplotskey}{fill between/split=\mchoice{true,false} (initially false)}
	Activates the generation of more than one output segment. 

	The initial choice |split=false| is quite fast and robust, it simply concatenates the input paths and generates exactly \emph{one} output segment.

	The choice |split=true| results in a computation of every intersection of the two curves (by means of the |tikz| library |intersections|). Then, each resulting segment results in a separate drawing instruction.
\end{pgfplotskey}

\begin{stylekey}{/pgfplots/fill between/every segment}
	A style which installed for every segment generated by |fill between|.

	The sequence of styles which are being installed is 

	|every segment|, any \meta{options} provided after |\addplot[|\meta{options}|] fill between|, then the appropriate |every segment no |\meta{index}, then one of |every odd segment| or |every even segment|.
\end{stylekey}

\begin{stylekey}{/pgfplots/fill between/every odd segment}
	A style which is installed for every odd segment generated by |fill between|.

	\emph{Attention:} this style makes only sense if |split| is active.

	See |every segment| for the sequence in which the styles will be invoked.
\end{stylekey}

\begin{stylekey}{/pgfplots/fill between/every even segment}
	A style which is installed for every even segment generated by |fill between|.

	\emph{Attention:} this style makes only sense if |split| is active.

	See |every segment| for the sequence in which the styles will be invoked.
\end{stylekey}
\begin{stylekey}{/pgfplots/fill between/every segment no \meta{index}}
	A style which is installed for every segment with the designated index \meta{index} generated by |fill between|.

	An index is a number starting with $0$ (which is used for the first segment).

	\emph{Attention:} this style makes only sense if |split| is active.

	See |every segment| for the sequence in which the styles will be invoked.
\end{stylekey}

\begin{pgfplotskey}{fill between/reverse=\mchoice{auto,true,false} (initially auto)}
	Configures whether the input paths specified by |of| need to be reversed in order to arrive at a suitable path concatenation.

	The initial choice \declaretext{auto} will handle this automatically.

	The choice \declaretext{true} will always reverse one of the involved paths. This is suitable if both paths have the same direction (for example, both are specified in increasing $x$ order).

	The choice \declaretext{false} will not reverse the involved paths. This is suitable if one path has, for example, coordinates in increasing $x$ order whereas the other path has coordinates in decreasing $x$ order.
\end{pgfplotskey}
