\subsection{Ternary Diagrams}

\begin{pgfplotslibrary}{ternary}
	A library to draw ternary diagrams.

	A ternary diagram visualizes three--component systems such that the sum of them yields $100\%$. Ternary phase diagrams are visualized as triangular axes.
\end{pgfplotslibrary}


\begin{environment}{{ternaryaxis}\oarg{options}}
	The axis environment for ternary axes.


\begin{codeexample}[]
\begin{tikzpicture}
\begin{ternaryaxis}
	\addplot3 coordinates {
	    (0.81,	0.19,	0.00)
	    (0.76,	0.17,	0.07)
	    (0.66,	0.16,	0.16)
	    (0.76,	0.07,	0.17)
	    (0.81,	0.00,	0.19)
	};

	\addplot3 coordinates {
	    (0.85,	0.15,	0.00)
	    (0.82,	0.13,	0.05)
	    (0.73,	0.14,	0.13)
	    (0.82,	0.06,	0.13)
	    (0.84,	0.00,	0.16)
	};
	\legend{$10$\textdegree, $20$\textdegree}
\end{ternaryaxis}
\end{tikzpicture}
\end{codeexample}

	Each of the three expected input coordinates is mapped linearly into the range $[0,1]$, then $(\tilde x,\tilde y,\tilde z)$ is interpreted as barycentric coordinate. In other words, a coordinate is placed at
	\[ 
		\begin{bmatrix}
			X(x,y,z)\\
			Y(x,y,z)
		\end{bmatrix}
		=
		\tilde x A + \tilde y B + \tilde z C
		= 
		\begin{bmatrix}
			\frac12 \frac{x+2z}{x+y+z}\\
			\frac{\sqrt 3}{2} \frac{x}{x+y+z}
		\end{bmatrix}
	\]
	where $A=(\nicefrac12,\nicefrac{\sqrt3}{2})$ is top corner of the triangle, $B=(0,0)$ the lower left and $C=(1,0)$ the lower right one. The axis range for $A$ is on the \emph{opposite} site of $A$; the ticks for $B$ are on the \emph{opposite} site of $B$ and similarly for $C$.

	The input coordinate $(100\%,0\%,0\%)$ is mapped to $A$, the input coordinate $(0\%,100\%,0\%)$ to $B$ and $(0\%,0\%,100\%)$ to $C$ (acrobat reader: click into the axis to verify it).

\begin{codeexample}[]
\begin{tikzpicture}
\begin{ternaryaxis}[xlabel=A,ylabel=B,zlabel=C]
	\addplot3 coordinates {
        (0.81,  0.19,  0.00)
        (0.76,  0.17,  0.07)
        (0.66,  0.16,  0.16)
        (0.76,  0.07,  0.17)
        (0.81,  0.00,  0.19)
	};

	\addplot3 coordinates {
        (0.85,  0.15,  0.00)
        (0.82,  0.13,  0.05)
        (0.73,  0.14,  0.13)
        (0.82,  0.06,  0.13)
        (0.84,  0.00,  0.16)
	};
	\legend{$10$\textdegree, $20$\textdegree}
\end{ternaryaxis}
\end{tikzpicture}
\end{codeexample}
\end{environment}

	A |ternaryaxis| can contain zero, one or more |\addplot3| commands, just as a usual |axis| (however, it really needs |\addplot3|, |\addplot| is not supported). The |\addplot3| command can use any of the accepted input formats, for example using |coordinates|, |table|, |expression| or whatever -- but the input is always interpreted as barycentric coordinates (three components summing up to $100\%$).

\begin{codeexample}[]
\begin{tikzpicture}
\begin{ternaryaxis}[xlabel=A,ylabel=B,zlabel=C,xmin=0.5]
	\addplot3 coordinates {
        (0.81,  0.19,  0.00)
        (0.76,  0.17,  0.07)
        (0.66,  0.16,  0.16)
        (0.76,  0.07,  0.17)
        (0.81,  0.00,  0.19)
	};

	\addplot3 coordinates {
        (0.85,  0.15,  0.00)
        (0.82,  0.13,  0.05)
        (0.73,  0.14,  0.13)
        (0.82,  0.06,  0.13)
        (0.84,  0.00,  0.16)
	};
	\legend{$10$\textdegree, $20$\textdegree}
\end{ternaryaxis}
\end{tikzpicture}
\end{codeexample}


\begin{codeexample}[]
\begin{tikzpicture}
\begin{ternaryaxis}[
	title=Sloped labels and minor ticks,
	xlabel=Water,
	ylabel=D--Threonine,
	zlabel=L--Threonine,
	label style={sloped},
	minor tick num=2,
]
	\addplot3 coordinates {
        (0.82,  0.18,  0.00)
        (0.75,  0.17,  0.08)
        (0.77,  0.12,  0.11)
        (0.75,  0.08,  0.17)
        (0.81,  0.00,  0.19)
	};
	\addplot3 coordinates {
        (0.75,  0.25,  0.00)
        (0.69,  0.25,  0.06)
        (0.64,  0.24,  0.12)
        (0.655, 0.23,  0.115)
        (0.67,  0.17,  0.16)
        (0.66,  0.12,  0.22)
        (0.64,  0.11,  0.25)
        (0.69,  0.05,  0.26)
        (0.76,  0.01,  0.23)
	};
\end{ternaryaxis}
\end{tikzpicture}
\end{codeexample}

\begin{codeexample}[]
\begin{tikzpicture}
\begin{ternaryaxis}[
	title=Sloped labels and minor grids,
	xlabel=Water,
	ylabel=D--Threonine,
	zlabel=L--Threonine,
	label style={sloped},
	minor tick num=2,
	grid=both,
]
	\addplot3 coordinates {
        (0.82,  0.18,  0.00)
        (0.75,  0.17,  0.08)
        (0.77,  0.12,  0.11)
        (0.75,  0.08,  0.17)
        (0.81,  0.00,  0.19)
	};
	\addplot3 coordinates {
        (0.75,  0.25,  0.00)
        (0.69,  0.25,  0.06)
        (0.64,  0.24,  0.12)
        (0.655, 0.23,  0.115)
        (0.67,  0.17,  0.16)
        (0.66,  0.12,  0.22)
        (0.64,  0.11,  0.25)
        (0.69,  0.05,  0.26)
        (0.76,  0.01,  0.23)
	};
\end{ternaryaxis}
\end{tikzpicture}
\end{codeexample}

A |ternaryaxis| supports (most of) the \PGFPlots\ axis interface, among them the |grid| option, the |xtick=|\marg{positions} way to provide ticks, including |extra x ticks| and its variants. Of course, tt can also contain any of the |mark|, |color| and |cycle list| options of a normal axis.

The following example is a (rude) copy of an example of 

\url{http://www.sv.vt.edu/classes/MSE2094_NoteBook/96ClassProj/experimental/ternary2.html}.

\begin{codeexample}[]
\begin{tikzpicture}
\begin{ternaryaxis}[
	title=Want--be--Stainless Steel,
	xlabel=Weight Percent Chromium,
	ylabel=Weight Percent Iron,
	zlabel=Weight Percent Nickel,
	label style=sloped,
	area style,
	%clickable coords,
]
	\addplot3 table[row sep=\\] {
	A B C\\
	1 0 0\\
	0.5 0.4 0.1\\
	0.45 0.52 0.03\\
	0.36 0.6 0.04\\
	0.1 0.9 0\\
	};
	\addlegendentry{Cr}

	\addplot3 table[row sep=\\] {
	A B C\\
	1 0 0\\
	0.5 0.4 0.1\\
	0.28 0.35 0.37\\
	0.4 0 0.6\\
	};
	\addlegendentry{Cr+$\gamma$FeNi}

	\addplot3 table[row sep=\\] {
	0.4 0 0.6\\
	0.28 0.35 0.37\\
	0.25 0.6 0.15\\
	0.1 0.9 0\\
	0 1 0\\
	0 0 1\\
	};
	\addlegendentry{$\gamma$FeNi}

	\addplot3 table[row sep=\\] {
	0.1 0.9 0\\
	0.36 0.6 0.04\\
	0.25 0.6 0.15\\
	};
	\addlegendentry{Cr+$\gamma$FeNi}

	\addplot3 table[row sep=\\] {
	0.5 0.4 0.1\\
	0.45 0.52 0.03\\
	0.36 0.6 0.04\\
	0.25 0.6 0.15\\
	0.28 0.35 0.37\\
	};
	\addlegendentry{$\sigma$+$\gamma$FeNi}

	\node[inner sep=0.5pt,circle,draw,fill=white,pin=-15:\footnotesize Stainless Steel] at (axis cs:0.18,0.74,0.08) {};
	
\end{ternaryaxis}
\end{tikzpicture}
\end{codeexample}

\begin{tikzpicture}
\begin{ternaryaxis}[
	title=Want--be--Stainless Steel,
	xlabel=Weight Percent Chromium,
	ylabel=Weight Percent Iron,
	zlabel=Weight Percent Nickel,
	label style=sloped,
	area style,
	%clickable coords,
]

\addplot3 graphics[points={(1,0,0) (0,1,0) (0,0,1)},
	includegraphics={trim=22 30 17 11,clip}]
	{/home/ludewich/P.png};

\end{ternaryaxis}
\end{tikzpicture}

\begin{codeexample}[]
\begin{tikzpicture}
\begin{ternaryaxis}[
	title=Want--be--Stainless Steel,
	xlabel=Weight Percent Chromium,
	ylabel=Weight Percent Iron,
	zlabel=Weight Percent Nickel,
	label style=sloped,
	clickable coords,
]

\addplot3[contour prepared={labels over line},point meta=explicit]
	coordinates {
	(0.90, 0.02, 0.07) [1700]
	(0.85, 0.14, 0.00) [1700]

	(0.85, 0.03, 0.12) [1600]
	(0.78, 0.14, 0.08) [1600]
	(0.71, 0.28, 0.01) [1600]

	(0.77, 0.03, 0.20) [1500]
	(0.69, 0.21, 0.10) [1500]
	(0.57, 0.38, 0.05) [1500]
	(0.41, 0.57, 0.02) [1500]
	(0.30, 0.68, 0.01) [1500]
	(0.19, 0.79, 0.02) [1500]
	(0.06, 0.89, 0.05) [1500]
	(0.00, 0.90, 0.10) [1500]

	(0.70, 0.02, 0.28) [1400]
	(0.59, 0.22, 0.19) [1400]
	(0.48, 0.39, 0.13) [1400]
	(0.39, 0.50, 0.11) [1400]
	(0.35, 0.53, 0.12) [1400]
	(0.33, 0.54, 0.13) [1400]

	(0.60, 0.02, 0.38) [1400]
	(0.56, 0.11, 0.33) [1400]
	(0.51, 0.19, 0.30) [1400]
	(0.49, 0.21, 0.30) [1400]

	(0.00, 0.96, 0.04) [1500]
	(0.07, 0.93, 0.01) [1500]

	(0.00, 0.96, 0.04) [1500]
	(0.01, 0.92, 0.07) [1500]
	(0.07, 0.86, 0.08) [1500]
	(0.12, 0.81, 0.07) [1500]
	(0.15, 0.78, 0.07) [1500]
	(0.19, 0.72, 0.09) [1500]
	(0.23, 0.66, 0.11) [1500]
	(0.29, 0.59, 0.12) [1500]
	(0.35, 0.51, 0.14) [1500]
	(0.40, 0.43, 0.17) [1500]
	(0.45, 0.36, 0.19) [1500]
	(0.49, 0.28, 0.23) [1500]
	(0.50, 0.20, 0.30) [1500]
	(0.50, 0.14, 0.36) [1500]
	(0.50, 0.10, 0.40) [1500]
	(0.53, 0.06, 0.41) [1500]
	(0.56, 0.03, 0.40) [1500]
	(0.59, 0.02, 0.39) [1500]

	(0.00, 0.94, 0.06) [1600]
	(0.01, 0.90, 0.08) [1600]
	(0.04, 0.86, 0.10) [1600]
	(0.10, 0.81, 0.09) [1600]
	(0.14, 0.75, 0.11) [1600]
	(0.18, 0.69, 0.14) [1600]
	(0.21, 0.63, 0.16) [1600]
	(0.25, 0.57, 0.18) [1600]
	(0.30, 0.49, 0.21) [1600]
	(0.34, 0.43, 0.23) [1600]
	(0.38, 0.35, 0.27) [1600]
	(0.42, 0.27, 0.31) [1600]
	(0.44, 0.20, 0.35) [1600]
	(0.46, 0.16, 0.39) [1600]
	(0.48, 0.09, 0.43) [1600]
	(0.48, 0.04, 0.48) [1600]
	(0.47, 0.01, 0.51) [1600]

	(0.43, 0.02, 0.55) [1350]
	(0.38, 0.07, 0.55) [1350]
	(0.35, 0.14, 0.51) [1350]
	(0.34, 0.23, 0.43) [1350]
	(0.36, 0.26, 0.38) [1350]
	(0.41, 0.28, 0.31) [1350]

	(0.38, 0.02, 0.61) [1360]
	(0.31, 0.07, 0.62) [1360]
	(0.24, 0.13, 0.63) [1360]
	(0.19, 0.17, 0.63) [1360]
	(0.17, 0.25, 0.58) [1360]
	(0.18, 0.31, 0.51) [1360]
	(0.20, 0.37, 0.43) [1360]
	(0.24, 0.42, 0.34) [1360]
	(0.28, 0.45, 0.27) [1360]
	(0.33, 0.45, 0.22) [1360]

	(0.30, 0.03, 0.68) [1370]
	(0.23, 0.07, 0.70) [1370]
	(0.18, 0.12, 0.70) [1370]
	(0.15, 0.17, 0.68) [1370]
	(0.15, 0.23, 0.62) [1370]
	(0.15, 0.30, 0.55) [1370]
	(0.16, 0.36, 0.48) [1370]
	(0.19, 0.41, 0.40) [1370]
	(0.22, 0.47, 0.31) [1370]
	(0.26, 0.49, 0.25) [1370]
	(0.28, 0.51, 0.21) [1370]

	(0.19, 0.02, 0.79) [1400]
	(0.12, 0.11, 0.77) [1400]
	(0.09, 0.17, 0.74) [1400]
	(0.08, 0.23, 0.68) [1400]
	(0.08, 0.31, 0.61) [1400]
	(0.08, 0.40, 0.52) [1400]
	(0.10, 0.46, 0.44) [1400]
	(0.12, 0.52, 0.36) [1400]
	(0.15, 0.56, 0.29) [1400]
	(0.18, 0.58, 0.24) [1400]
	(0.24, 0.59, 0.17) [1400]

	(0.09, 0.02, 0.89) [1420]
	(0.04, 0.14, 0.82) [1420]
	(0.01, 0.25, 0.74) [1420]
	(0.01, 0.33, 0.66) [1420]
	(0.01, 0.47, 0.52) [1420]
	(0.02, 0.55, 0.42) [1420]
	(0.04, 0.62, 0.34) [1420]
	(0.07, 0.67, 0.26) [1420]
	(0.11, 0.70, 0.19) [1420]
	(0.14, 0.70, 0.16) [1420]
	(0.17, 0.70, 0.13) [1420]

	(0.00, 0.78, 0.22) [1490]
	(0.04, 0.81, 0.15) [1490]
	(0.10, 0.80, 0.10) [1490]

};
\end{ternaryaxis}
\end{tikzpicture}
\end{codeexample}

\begin{pgfplotskey}{ternary limits relative=\mchoice{true,false} (initially true)}
	Allows to switch tick labels between relative numbers in the range $[0,100]$ or absolute numbers.
	
	Absolute coordinates use the coordinate range as lower and upper limits.
\end{pgfplotskey}

\begin{coordinatesystem}{cartesian cs}
	A coordinate system which allows cartesian coordinates. The lower left point has coordinate $(0,0)$, the lower right point has $(1,0)$ and the upper point of the triangle is at $(\nicefrac12, \nicefrac{\sqrt3}{2})$.

	If you use the standard point syntax $(x,y)$ in path commands inside of the axis, you'll get cartesian coordinates. If you want to use it for axis descriptions (like |xlabel|), you'll have to write |cartesian cs:0,0| explicitly (axis labels have the default coordinate system |axis description cs|).
\begin{codeexample}[]
\begin{tikzpicture}
	\begin{ternaryaxis}[
		title=Cartesian Annotations,
		clip=false]

	\addplot3 coordinates {
		(0.1,0.5,0.4)
		(0.2,0.5,0.3)
		(0.3,0.6,0.1)
	};

	\node[fill=white,draw] at (0,0) {$y (0,0)$};
	\node[fill=white,draw] at (1,0) {$z (1,0)$};
	\node[fill=white,draw] at (0.5,{sqrt(3)/2}) 
		{$x (\frac12,\frac{\sqrt3}{2})$};
	
	\draw[red,-stealth] (0.5,0) -- (0.5,0.7);
	\end{ternaryaxis}
\end{tikzpicture}
\end{codeexample}
\end{coordinatesystem}

\begin{stylekey}{/pgfplots/every ternary axis}
	A style which is installed at the beginning of every ternary axis. It is used to adjust some of the \PGFPlots\ keys to fit the triangular shape.

	The initial configuration is
\begin{codeexample}[code only]
\pgfplotsset{
	every ternary axis/.style={
		/pgfplots/tick align=outside,
		grid=major,
		xticklabel style={anchor=west},
		every 3d description/.style={},
		every axis x label/.style={at={(ticklabel cs:0.5)},anchor=near ticklabel},
		every axis y label/.style={at={(ticklabel cs:0.5)},anchor=near ticklabel},
		every axis z label/.style={at={(ticklabel cs:0.5)},anchor=near ticklabel},
		every x tick scale label/.style={at={(xticklabel cs:0.95,5pt)},anchor=near xticklabel,inner sep=0pt},
		every y tick scale label/.style={at={(yticklabel cs:0.95,5pt)},anchor=near yticklabel,inner sep=0pt},
		every z tick scale label/.style={at={(yticklabel cs:0.95,5pt)},anchor=near yticklabel,inner sep=0pt},
		every axis title shift=15pt,
		every axis legend/.style={
			cells={anchor=center},
			inner xsep=3pt,inner ysep=2pt,nodes={inner sep=2pt,text depth=0.15em},
			shape=rectangle,
			fill=white,
			draw=black,
			at={(1.03,1.03)},
			anchor=north west,
		},
	},
}
\end{codeexample}
\end{stylekey}
