\section{Related Libraries}
This section describes some libraries which come with \PGFPlots, but they are more or less special and need to be activated separately.

\subsection{Dates as Input Coordinates}
\begin{tikzlibrary}{dateplot}
	A library which allows to use dates like |2008-01-01| as input coordinates in plots. The library converts dates to numbers and tick labels will be pretty-printed dates.

	This library is documented in section~\ref{pgfplots:sec:symbolic:coords} on page~\pageref{pgfplots:sec:date:coords}.
\end{tikzlibrary}

\subsection{Clickable Plots}
\begin{tikzlibrary}{pgfplotsclickable}
	A library which generates small popups whenever one clicks into a plot. The popup displays the coordinate under the mouse pointer. Furthermore, the library allows to display slopes if holds the mouse pressed and drags it to another point in the plot.

	It is completely sufficient to write
\begin{codeexample}[code only]
\usetikzlibrary{pgfplotsclickable}
\end{codeexample}
	\noindent in the document preamble. This will automatically prepare every plot.

	The library works with Acrobat Javascript and \pdf\ forms: every plot becomes a push--button. 

	\includegraphics[height=6cm]{figures/pgfplotsclickable-fig1.png}
	\includegraphics[height=6cm]{figures/pgfplotsclickable-fig2.png}

	\includegraphics[height=6cm]{figures/pgfplotsclickable-fig3.png}
	\includegraphics[height=6cm]{figures/pgfplotsclickable-fig4.png}

	\nobreak
	These screenshots show the result of clicking into the axis range (left column) and of dragging from one point to another (right column). The second case shows start- and end points and the slope of the line segment in--between.

	This document has been produces with |pgfplotsclickable|, so it is possible to load it into Acrobat Reader and simply click into a plot.

	\paragraph{Requirements:}
	\begin{itemize}
		\item The library relies on the \LaTeX\ packages |insdljs| (``Insert document level Javascript'') and |eforms| which are both part of the freely available |AcroTeX| bundle~\cite{acrotex}\footnote{These packages rely on \LaTeX, so the library is only available for \LaTeX, not for plain \TeX\ or Con\TeX t.}.
		
		\item At the time of this writing, only Adobe Acrobat Reader interpretes Javascript and Forms properly. The library doesn't have any effect if the resulting document is used in other viewers (as far as I know).
	\end{itemize}
\end{tikzlibrary}

It is possible to customize |pgfplotsclickable| with several options.

\begin{pgfplotskey}{clickable=\mchoice{true,false} (initially true)}
	Allows to disable the library for single plots.
\end{pgfplotskey}

\begin{pgfplotskey}{annot/js fillColor=\marg{javascript color} (initially ["RGB",1,1,.855])}
	Sets the background (fill) color of the short popup annotations. 
	
	Possible choices are |transparent|, gray, RGB or CMYK color specified as four--element--arrays of the form
	|["RGB", |\meta{red}|,|\meta{green}|,|\meta{blue}|]|.

	Again: this option is for Javascript. It is \emph{not} possible to use colors as in \pgfname.
\end{pgfplotskey}

\begin{pgfplotskey}{annot/point format=\marg{sprintf-format} (initially {(\%.1f,\%.1f)})}
	% mention the every-styles
\end{pgfplotskey}

\begin{pgfplotskey}{annot/slope format=\marg{sprintf-format} (initially \%.1f*x + \%.1f)}
\end{pgfplotskey}

\begin{pgfplotskey}{annot/printable=\mchoice{true,false} (initially false)}
	
\end{pgfplotskey}

\begin{pgfplotskey}{annot/font=\marg{javascript font name} (initially font.Times)}
	
\end{pgfplotskey}

\begin{pgfplotskey}{annot/textSize=\marg{Size in Point} (initially 11)}
	
\end{pgfplotskey}
