\subsection{Layers}
{
\tikzset{external/figure name/.add={}{layers_}}%

It is important that several parts of an axis are drawn ``on top'' of others. Usually, \PGFPlots\ ensures this by drawing them in a suitable sequence (usually background followed by grid lines, followed by tick lines and tick labels, followed by plots and finally axis descriptions). While this works reasonable in most cases, there are cases where more control is desired. One common use-case is if multiple axes shall be drawn into the same picture: here, the sequence from above should be applied to all involved axes simultaneously. 

The main key to control layered graphics with \PGFPlots\ is |set layers|:

\pgfkeys{
	/pgfmanual/gray key prefixes/.add={/pgfplots/layers/,}{},
}

\begin{pgfplotskey}{set layers=\mchoice{none,layer configuration name} (initially none)}
   This key enables layered graphics for either the current axis or for all following axes. 

   The invocation |set layers=none| disables layered graphics.

   The invocation |set layers| (without equal sign and without arguments) is the same as |set layers=||default|.

   In all other cases, |set layers| expects a \meta{layer configuration name}. There are two predefined configurations available (the prefix |/pgfplots/layers/| is optional):

   \begin{pgfplotskey}{layers/standard}
   	A layer configuration which defines the layers \texttt{axis background}, \texttt{axis grid}, \texttt{axis ticks}, \texttt{axis lines}, \texttt{axis tick labels}, \texttt{main}, \texttt{axis descriptions}, \texttt{axis foreground}. They are drawn in the order of appearance.
   \end{pgfplotskey}

   \begin{pgfplotskey}{layers/axis on top}
   	A layer configuration which uses the same layer names as |layers/standard|, but with a different sequence: \texttt{axis background}, \texttt{main}, \texttt{axis grid}, \texttt{axis ticks}, \texttt{axis lines}, \texttt{axis tick labels}, \texttt{axis descriptions}, \texttt{axis foreground}.

	This layer is automatically used if the key |axis on top| is used together with |set layers=|\meta{any layer configuration name}.
   \end{pgfplotskey}
  	
	As soon as the key |set layers=|\meta{layer configuration name} is encountered, \PGFPlots\ calls |\pgfsetlayers|\marg{layer names} with the layer names of the respective configuration. Usually, this \emph{replaces} the current layer configuration of the embedding |tikzpicture| (see |/.define layer set| for how to merge layer names). Furthermore, |set layers| stores the name of \meta{layer configuration name} such that every following |axis| knows how to map graphical elements to layer names.
	
\end{pgfplotskey}



\begin{handler}{{.define layer set}=\marg{ordered list}\marg{style definitions}}
	Allows to define a new layer set named \meta{key}. Afterwards, \meta{key} is a valid argument for |set layers=|\meta{key}.

	The first argument \meta{ordered list} is a comma-separated list of layer names. The names are arbitrary, and |\pgfdeclarelayer| will be called for every encountered argument\footnote{To be more precise: \texttt{set layers} calls \texttt{\textbackslash pgfdeclarelayer} when it uses a \meta{ordered list}.}. There is just one ``magic'' name: the layer |main| should be part of every \meta{ordered list} as it will contain every graphical element which is not associated with a specific layer.

	The second argument \meta{style definitions} contains options -- just as if you would have written \meta{key}|/.style=|\marg{style definitions}. The \meta{style definitions} are supposed to contain \PGFPlots\ style redefinitions which make use of each encountered element of \meta{ordered list}. Note that if you have an element in \meta{ordered list} which is never referenced inside of \meta{style definitions}, this layer will always be empty. In other words: the \emph{only} reference to the names in \meta{ordered list} is \meta{style definitions}, \PGFPlots\ has no hard-coded magic layer names (except for |main| as explained above).

	For example, the |default| layer configuration is defined by
\begin{codeexample}[code only]
\pgfplotsset{
    layers/standard/.define layer set=
    {axis background,axis grid,axis ticks,axis lines,axis tick labels,main,%
        axis descriptions,axis foreground}
    {
        grid style=         {/pgfplots/on layer=axis grid},
        tick style=         {/pgfplots/on layer=axis ticks},
        axis line style=    {/pgfplots/on layer=axis lines},
        label style=        {/pgfplots/on layer=axis descriptions},
        legend style=       {/pgfplots/on layer=axis descriptions},
        title style=        {/pgfplots/on layer=axis descriptions},
        colorbar style=     {/pgfplots/on layer=axis descriptions},
        ticklabel style=    {/pgfplots/on layer=axis tick labels},
        axis background@ style={/pgfplots/on layer=axis background},
        3d box foreground style={/pgfplots/on layer=axis foreground},
    },
}
\end{codeexample}
	\noindent This definition declares a couple of layers, and it adjusts \PGFPlots\ styles by adding an |on layer| commands. The arguments for |on layer| are the elements of \meta{ordered list}.
\end{handler}
}
