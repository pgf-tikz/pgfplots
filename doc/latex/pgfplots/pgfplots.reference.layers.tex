\subsection{Layers}
{
\tikzset{external/figure name/.add={}{layers_}}%

It is important that several parts of an axis are drawn ``on top'' of others. Usually, \PGFPlots\ ensures this by drawing them in a suitable sequence (usually background followed by grid lines, followed by tick lines and tick labels, followed by plots and finally axis descriptions). While this works reasonable in most cases, there are cases where more control is desired. One common use-case is if multiple axes shall be drawn into the same picture: here, the sequence from above should be applied to all involved axes simultaneously. 

\subsubsection{Using Predefined Layers}
The main key to control layered graphics with \PGFPlots\ is |set layers|:

\pgfkeys{
	/pgfmanual/gray key prefixes/.add={/pgfplots/layers/,}{},
}

\begin{pgfplotskey}{set layers=\mchoice{none,\normalfont\meta{layer configuration name}} (initially none)}
   This key enables layered graphics for either the current axis or for all following axes. 

   Enabling layered graphics has the effect that the order in which graphical elements are given is unrelated to the ordering in which they will be drawn. The main benefit is if you have multiple axes in the same figure: the axes can share the same layers.

   The invocation |set layers=none| disables layered graphics.

   The invocation |set layers| (without equal sign and without arguments) is the same as |set layers=||default|.

   In all other cases, |set layers| expects a \meta{layer configuration name}. There are two predefined configurations available (the prefix |/pgfplots/layers/| is optional):

   \begin{pgfplotskey}{layers/standard}
   	A layer configuration which defines the layers \texttt{axis background}, \texttt{axis grid}, \texttt{axis ticks}, \texttt{axis lines}, \texttt{axis tick labels}, \texttt{main}, \texttt{axis descriptions}, \texttt{axis foreground}. They are drawn in the order of appearance.
   \end{pgfplotskey}

   \begin{pgfplotskey}{layers/axis on top}
   	A layer configuration which uses the same layer names as |layers/standard|, but with a different sequence: \texttt{axis background}, \texttt{main}, \texttt{axis grid}, \texttt{axis ticks}, \texttt{axis lines}, \texttt{axis tick labels}, \texttt{axis descriptions}, \texttt{axis foreground}.

	This layer is automatically used if the key |axis on top| is used together with |set layers=|\meta{any layer configuration name}.
   \end{pgfplotskey}
  	
	As soon as the key |set layers=|\meta{layer configuration name} is encountered, \PGFPlots\ calls |\pgfsetlayers|\marg{layer names} with the layer names of the respective configuration. Usually, this \emph{replaces} the current layer configuration of the embedding |tikzpicture|. Furthermore, |set layers| stores the name of \meta{layer configuration name} such that every following |axis| knows how to map graphical elements to layer names.
	
	
	The fact that layer configurations are properties of a complete |tikzpicture| (whereas an axis is only one of multiple graphical elements of that picture), they need to be given at one of several supported positions:
    \begin{enumerate}
        \item Directly within the picture:
\begin{codeexample}[code only]
\begin{tikzpicture}
	\pgfplotsset{set layers=default}
	\begin{axis}
		...
	\end{axis}
\end{tikzpicture}
\end{codeexample}
	\noindent This option explicitly tells the reader of your source code that a significant portion of your picture has been changed: the complete picture has and uses a \meta{layer configuration name} (in this case |default|).

	 \item As option for one or more axes which is/are directly within the picture:
\begin{codeexample}[code only]
\begin{tikzpicture}
	\begin{axis}[set layers]
		...
	\end{axis}
\end{tikzpicture}
\end{codeexample}
		Here, \PGFPlots\ implicitly communicates its layer configuration to the enclosing |tikzpicture|. Thus, the effect of |set layers| is \emph{not local to an axis}; it survives until |\end{tikzpicture}|. Any other option only survives until |\end{axis}|.

		In this case, only the \emph{last} activated layer configuration will apply to the picture.

    \paragraph{Limitation: no environments or local \TeX\ groups allowed.} Standard usages as within the examples of this manual will always work. But since the layer name configuration is essentially part of a \PGF\ picture (at a low level), one cannot arbitrarily set them; \PGF\ will complain if they are changed within some nested \TeX\ groups or \LaTeX\ environments. Typically, you will never need to worry about this. 

	In short, the following examples are \emph{forbidden} because the axis is within locally nested groups. 
\begin{codeexample}[code only]
\begin{tikzpicture}
	{% FORBIDDEN! Consider using case (1) above!
		\begin{axis}[set layers]
			...
		\end{axis}
	}
\end{tikzpicture}
\end{codeexample}
\begin{codeexample}[code only]
\begin{tikzpicture}
	\begin{scope} % FORBIDDEN! Consider using case (1) above!
		\begin{axis}[set layers]
			...
		\end{axis}
	\end{scope}
\end{tikzpicture}
\end{codeexample}
	\noindent These examples are forbidden because the layer configuration will be cleared by the `|}|' of the first forbidden example and by the `|\end{scope}|' of the second example. A solution would be one of the different placement options (i.e. choice (1.) or (3.)).

	\item outside of any picture:
\begin{codeexample}[code only]
\pgfplotsset{set layers=default}
\begin{tikzpicture}
	\begin{axis}
		...
	\end{axis}
\end{tikzpicture}
\end{codeexample}
	\noindent This choice configures the layer configuration for \emph{every} following |tikzpicture|. 

    \end{enumerate}


    \paragraph{Limitation: axis alignment restricted to inner anchors.} This applies only if you changed the default value of |anchor| (which is |anchor=south west|).
    Any axis which uses layered graphics should use one of the following values of |anchor|: |north|, |north west|, |west|, |south west|, |south|, |south east|, |east|, |north east|, |north|, |center|, |origin|, |above origin|, |left of origin|, |right of origin|, |below origin|. In case you really need another anchor, \PGFPlots\ requires the use of |cell picture=true|, causing the layers to be local for that specific axis.
	
	The technical background for this limitation is a hen-and-egg problem: outer anchors (like |outer south west|) are only available \emph{after} the complete axis has been generated -- and layers can only be drawn after each drawing instruction has been issued. The technical keys for further reading are |cell picture=false| or |cell picture=if necessary| (one of them is active for layered graphics).
\end{pgfplotskey}

\begin{command}{\pgfplotssetlayers}
    An alias for |\pgfplotsset{set layers}|. It activates the |layers/default| layer configuration.
     
\end{command}
\begin{command}{\pgfplotssetlayers\marg{layer configuration name}}
    An alias for |\pgfplotsset{set layers=|\marg{layer configuration name}|}|.
\end{command}

\begin{handler}{{.define layer set}=\marg{ordered list}\marg{style definitions}}
	Allows to define a new layer set named \meta{key}. Afterwards, \meta{key} is a valid argument for |set layers=|\meta{key}.

	The first argument \meta{ordered list} is a comma-separated list of layer names. The names are arbitrary, and |\pgfdeclarelayer| will be called for every encountered argument\footnote{To be more precise: \texttt{set layers} calls \texttt{\textbackslash pgfdeclarelayer} when it uses a \meta{ordered list}.}. There is just one ``magic'' name: the layer |main| should be part of every \meta{ordered list} as it will contain every graphical element which is not associated with a specific layer.

	The second argument \meta{style definitions} contains options -- just as if you would have written \meta{key}|/.style=|\marg{style definitions}. The \meta{style definitions} are supposed to contain \PGFPlots\ style redefinitions which make use of each encountered element of \meta{ordered list}. Note that if you have an element in \meta{ordered list} which is never referenced inside of \meta{style definitions}, this layer will always be empty. In other words: the \emph{only} reference to the names in \meta{ordered list} is \meta{style definitions}, \PGFPlots\ has no hard-coded magic layer names (except for |main| as explained above).

	For example, the |default| layer configuration is defined by
\begin{codeexample}[code only]
\pgfplotsset{
    layers/standard/.define layer set=
    {axis background,axis grid,axis ticks,axis lines,axis tick labels,main,%
        axis descriptions,axis foreground}
    {
        grid style=         {/pgfplots/on layer=axis grid},
        tick style=         {/pgfplots/on layer=axis ticks},
        axis line style=    {/pgfplots/on layer=axis lines},
        label style=        {/pgfplots/on layer=axis descriptions},
        legend style=       {/pgfplots/on layer=axis descriptions},
        title style=        {/pgfplots/on layer=axis descriptions},
        colorbar style=     {/pgfplots/on layer=axis descriptions},
        ticklabel style=    {/pgfplots/on layer=axis tick labels},
        axis background@ style={/pgfplots/on layer=axis background},
        3d box foreground style={/pgfplots/on layer=axis foreground},
    },
}
\end{codeexample}
	\noindent This definition declares a couple of layers, and it adjusts \PGFPlots\ styles by adding an |on layer| commands. The arguments for |on layer| are the elements of \meta{ordered list}.

    Since the second argument \meta{style definitions} defines \meta{key} to be a normal style key, one can simply use \meta{key} in order to set \meta{style definitions}. This allows to inherit them. For example, the |layers/axis on top| layer configuration is defined by means of
\begin{codeexample}[code only]
\pgfplotsset{
	/pgfplots/layers/axis on top/.define layer set=
        {axis background,main,axis grid,axis ticks,axis lines,axis tick labels,%
            axis descriptions,axis foreground}
        {/pgfplots/layers/standard}
}
\end{codeexample}
    \noindent i.e.\ it only redefines the \emph{sequence} of the layers and re-uses the style definitions of |layers/standard|.

    Any number of layer configurations can be defined. 
\end{handler}


\subsubsection{Changing the Layer of Graphic Elements}
There are a couple of keys which change the layer of a graphical element. 

\begin{pgfplotskey}{on layer=\marg{layer name}}
    Providing this key somewhere in a \PGFPlots\ style or inside of a \PGFPlots\ axis will change the layer for all graphical elements for which the style applies.

    For example,
\begin{codeexample}[code only]
...
\begin{axis}[set layers,grid style={/pgfplots/on layer=axis foreground}]
...
\end{codeexample}
    \noindent will change the layer for any grid lines.

    The argument \meta{layer name} is expected to be part of the current layer configuration, i.e.\ the argument of |set layers| should contain it.
\end{pgfplotskey}

\begin{pgfplotskey}{mark layer=\mchoice{auto,like plot,\meta{layer name}} (initially auto)}
    This key can be used to define the layer for plot |mark|s. If you write |\addplot[/pgfplots/on layer=\meta{name}]|, the
    layer will be used for the complete plot. Plot marks are treated with special care, so you can define an own layer for plot marks. 

    The initial choice \declaretext{auto} will automatically define a ``suitable'' choice, leaving the responsability with \PGFPlots.

    The choice \declaretext{like plot} will pack the marks onto the same layer as the plot they belong to.

    Finally, one can provide any \meta{layer name}, just as for |on layer|.
\end{pgfplotskey}

\begin{command}{\pgfplotsonlayer\marg{layer name}}
    A low-level command which will check if the current axis has layer support activated and, if so, calls |\pgfonlayer|\marg{layer name}.

    There must be a |\endpgfplotsonlayer| to delimit the environment.
\end{command}
\begin{command}{\endpgfplotsonlayer}
    The end of |\pgfplotsonlayer|.
\end{command}

\begin{command}{\pgfonlayer\marg{layer name}}
    A low-level command of \PGF\ which will collect everything until the matching |\endpgfonlayer| into layer \meta{layer name}.

    The \meta{layer name} must be active, i.e.\ it must be part of the layer names of |set layers|. 
    
    The only special case is if you call |\pgfdeclarelayer{discard}| somewhere and use |\pgfonlayer{discard}| -- this layer has a ``magical name'' in that it servers as |/dev/null|: it does not need to be active and everything assigned to this layer will be thrown away if it is not part of the layer name configuration.

    There must be a |\endpgfonlayer| to delimit the environment.
\end{command}
\begin{command}{\endpgfonlayer}
    The end of |\pgfonlayer|.
\end{command}


\begin{command}{\pgfsetlayers\marg{layer list}}
	This is a low-level command of \PGF. At the time of this writing, it is the only way to tell \PGF\ which layers it shall use for the current / next picture. It is used implicitly by |set layers|.
\end{command}
}
