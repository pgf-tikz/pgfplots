\subsection{Smith Charts}

\begin{pgfplotslibrary}{smithchart}
	A library to draw Smith Charts.

	A Smith Chart maps complex half plane with positive real parts to the unit circle. The |smithchart| library allows \PGFPlots\ to visualize smith charts: it visualizes two--dimensional input coordinates $z \in \C $ of the form $z = x+ j y \in \C$ ($j$ being the imaginary unit, $j^2=-1$) with $x \ge 0$ using the map
	\[ r\colon [0,\infty] \times [-\infty,\infty] \to \{ a+j b \;\vert\;  a^2 + b^2 = 1 \}, \quad r(z) = \frac{z-1}{z+1} \]
	using complex number division. The result is always in the unit circle.

	The main application for Smith Charts is in the area of electrical and electronics engineers specializing in radio frequency: to show the reflection coefficient $r(z)$ for normalised impedance $z$. It is beyond the scope of this manual to delve into the radio frequency techniques; for us, it is important to note that the |smithchart| library supports
	\begin{itemize}
		\item the data map $r(z)$ shown above,
		\item an axis class which interpretes $x$ as the real components and $y$ as the imaginary components,
		\item a visualization of grid lines as arcs,
		\item the possibility to stop grid lines to allow uniform spacing in Smith Charts,
		\item a large set of the \PGFPlots\ axis fine tuning parameters,
		\item input of already mapped coordinates $r(z)$ (i.e.\ cartesian coordinates in the unit circle),
		\item many of the \PGFPlots\ plot handlers.
	\end{itemize}
\end{pgfplotslibrary}

\subsection{Smith Chart Axes}

\begin{environment}{{smithchart}\oarg{options}}
	The |\begin{smithchart}| environment draws smith charts. It accepts the same \meta{options} as |\begin{axis}|. In fact, it is equivalent to |\begin{axis}[|\meta{options}|,axis type=smithchart]|.
\begin{codeexample}[]
\begin{tikzpicture}
	\begin{smithchart}[title=Default Smith Chart]
	\addplot coordinates {(0.5,0.2) (1,0.8) (2,2)};
	\end{smithchart}
\end{tikzpicture}
\end{codeexample}
	The example above visualizes three data points using the initial configuration of Smith Charts; the data points are interpreted as complex numbers $z = x + j y$ and are mapped using $r(z)$.

\end{environment}

\subsection{Size Control}
A Smith Chart can be resized by providing either |width| or |height| as argument to the axis. If you provide both, the Chart is drawn as an ellipsis.

The tick- and grid positions for |smithchart| axes are realized by means of three manually tuned sets of grid lines: one for small sizes plots, one for medium sized plots and one for huge plots. The actual parameters for |width| or |height| are considered to select one of the following sets:

\begin{stylekey}{/pgfplots/few smithchart ticks}%
	This produces the output of the example above -- it constitutes the initial configuration for Smith Chart which has a width of less than |14cm|.
	
	The |few smithchart ticks| style is defined by:
\begin{codeexample}[code only]
\pgfplotsset{
	few smithchart ticks/.style={
		xtick={0.2,0.5,1,2,5},
		ytick={%
			0,%
			 0.2, 0.5, 1, 2, 5,%
			-0.2,-0.5,-1,-2,-5},
		xgrid stop each nth at y={2},
		ygrid stop each nth at x={2},
	},
}
\end{codeexample}
	
	For fine tuning of the scaling descisions, see the |smith chart ticks by size| key.

\end{stylekey}

\begin{stylekey}{/pgfplots/many smithchart ticks}%
	The |many smithchart ticks| style is used for every Smith Chart whose width exceeds |14cm| although it is less than |20cm|:
\begin{codeexample}[]
\begin{tikzpicture}
	\begin{smithchart}[
		title=Medium--Sized Smith Chart,
		width=14cm]
	\addplot coordinates {(0.5,0.2) (1,0.8) (2,2)};
	\end{smithchart}
\end{tikzpicture}
\end{codeexample}
	
	We see that |many smithchart ticks| has different placement- and alignment options than |few smithchart ticks|: it uses sloped tick labels inside of the unit circle for the $y$ descriptions (imaginary axis).

	The initial configuration is realized by means of \emph{two} separate styles: one which defines only the tick positions (the \declareandlabel{many smithchart ticks*} style) and one which also changes placement- and alignment options. The initial configuration can be changed individually (see the end of this section for examples). The initial configuration is:
\begin{codeexample}[code only]
\pgfplotsset{
	many smithchart ticks*/.style={
		xtick={
			0.1,0.2,0.3,0.4,0.5,1,1.5,2,3,4,5,10,20%
		},
		minor xtick={0.6,0.7,0.8,0.9,1.1,1.2,1.3,1.4,1.6,1.7,1.8,1.9,2.2,2.4,2.6,2.8,3.2,3.4,3.6,3.8,4.5,6,7,8,9,50},
		ytick={%
			0,%
			0.1,0.2,...,1,1.5,2,3,4,5,10,20,%
			-0.1,-0.2,...,-1,-1.5,-2,-3,-4,-5,-10,-20%
		},
		minor ytick={%
			1.1,1.2,1.3,1.4,1.6,1.7,1.8,1.9,2.2,2.4,2.6,2.8,3.2,3.4,3.6,3.8,4.5,6,7,8,9,50,%
			-1.1,-1.2,-1.3,-1.4,-1.6,-1.7,-1.8,-1.9,-2.2,-2.4,-2.6,-2.8,-3.2,-3.4,-3.6,-3.8,-4.5,-6,-7,-8,-9,-50%
		},
		xgrid stop each nth at y={1,2,4,5,10,20},
		ygrid stop each nth at x={1,2,3,5,10:3,20:3},
	},
	/pgfplots/many smithchart ticks/.style={
		many smithchart ticks*,
		yticklabel in circle,
		show origin=true,
	},
}
\end{codeexample}
\end{stylekey}

\begin{stylekey}{/pgfplots/dense smithchart ticks}%
	The |dense smithchart ticks| style assigns the set of tick positions for every Smith Chart whose width is at least |20cm|:
\begin{codeexample}[]
\begin{tikzpicture}
	\begin{smithchart}[
		title=Huge Smith Chart,
		width=20cm]
	\addplot coordinates {(0.5,0.2) (1,0.8) (2,2)};
	\end{smithchart}
\end{tikzpicture}
\end{codeexample}

	Similarly to |many smithchart ticks| (see above), the initial configuration is realized by means of \emph{two} separate styles: one which defines only the tick positions (the \declareandlabel{many smithchart ticks*} style) and one which also changes placement- and alignment options:
\begin{codeexample}[code only]
\pgfplotsset{
	dense smithchart ticks*/.style={
		ygrid stop each nth at x start=1,
		xtick={
			0.1,0.2,0.3,0.4,0.5,0.6,0.7,0.8,0.9,1,1.2,1.4,1.6,1.8,2,3,4,5,10,20%
		},
		minor xtick={%
			0.01,0.02,0.03,0.04,0.05,0.06,0.07,0.08,0.09,0.11,0.12,0.13,0.14,0.15,0.16,0.17,0.18,0.19,%
			0.22,0.24,0.26,0.28,0.32,0.34,0.36,0.38,0.42,0.44,0.46,0.48,%
			0.55,0.65,0.75,0.85,0.95,%
			1.1,1.3,1.5,1.7,1.9,%
			2.2,2.4,2.6,2.8,3.2,3.4,3.6,3.8,4.5,6,7,8,9,50},
		ytick={%
			0,%
			0.1,0.2,...,1,1.2,1.4,1.6,1.8,2,3,4,5,10,20,%
			-0.1,-0.2,...,-1,-1.2,-1.4,-1.6,-1.8,-2,-3,-4,-5,-10,-20%
		},
		minor ytick={%
			0.01,0.02,0.03,0.04,0.05,0.06,0.07,0.08,0.09,0.11,0.12,0.13,0.14,0.15,0.16,0.17,0.18,0.19,%
			0.22,0.24,0.26,0.28,0.32,0.34,0.36,0.38,0.42,0.44,0.46,0.48,%
			0.55,0.65,0.75,0.85,0.95,%
			1.1,1.3,1.5,1.7,1.9,2.2,2.4,2.6,2.8,3.2,3.4,3.6,3.8,4.5,6,7,8,9,50,%
			-0.01,-0.02,-0.03,-0.04,-0.05,-0.06,-0.07,-0.08,-0.09,-0.11,-0.12,-0.13,-0.14,-0.15,-0.16,-0.17,-0.18,-0.19,%
			-0.22,-0.24,-0.26,-0.28,-0.32,-0.34,-0.36,-0.38,-0.42,-0.44,-0.46,-0.48,%
			-0.55,-0.65,-0.75,-0.85,-0.95,%
			-1.1,-1.3,-1.5,-1.7,-1.9,-2.2,-2.4,-2.6,-2.8,-3.2,-3.4,-3.6,-3.8,-4.5,-6,-7,-8,-9,-50%
		},
		xgrid stop each nth at y={0.2 if < 0.2,0.5 if < 0.5,1 if < 1,2,4,5,10,20},
		ygrid stop each nth at x={0.2 if < 0.2,0.5 if < 0.5,1 if < 1,2,3,5,10:3,20:3},
	},
	dense smithchart ticks/.style={
		yticklabel in circle,
		dense smithchart ticks*,
		show origin=true,
	},
}
\end{codeexample}
\end{stylekey}
