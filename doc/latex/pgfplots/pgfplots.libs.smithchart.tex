\subsection{Smith Charts}

\begin{pgfplotslibrary}{smithchart}
	A library to draw Smith Charts.

	A Smith Chart maps complex half plane with positive real parts to the unit circle. The |smithchart| library allows \PGFPlots\ to visualize smith charts: it visualizes two--dimensional input coordinates $z \in \C $ of the form $z = x+ j y \in \C$ with $x \ge 0$ using the map
	\[ r\colon [0,\infty] \times [-\infty,\infty] \to \{ a+j b \vert  a^2 + b^2 = 1 \}, \quad r(z) = \frac{z-1}{z+1} \]
	using complex number division. The result is always in the unit circle.

	The main application for Smith Charts is in the area of electrical and electronics engineers specializing in radio frequency: to show the reflection coefficient $r(z)$ for normalised impedance $z$. It is beyond the scope of this manual to delve into the radio frequency techniques; for us, it is important to note that the |smithchart| library supports
	\begin{itemize}
		\item the data map $r(z)$ shown above,
		\item an axis class which interpretes $x$ as the real components and $y$ as the imaginary components,
		\item a visualization of grid lines as arcs,
		\item the possibility to stop grid lines to allow uniform spacing in Smith Charts,
		\item a large set of the \PGFPlots\ axis fine tuning parameters,
		\item input of already mapped coordinates $r(z)$ (i.e.\ cartesian coordinates in the unit circle),
		\item many of the \PGFPlots\ plot handlers.
	\end{itemize}
\end{pgfplotslibrary}

\begin{environment}{{smithchart}\oarg{options}}
	The |\begin{smithchart}| environment draws smith charts. It accepts the same \meta{options} as |\begin{axis}|. In fact, it is equivalent to |\begin{axis}[|\meta{options}|,axis type=smithchart]|.
\begin{codeexample}[]
\begin{tikzpicture}
	\begin{smithchart}[title=Default Smith Chart]
	\end{smithchart}
\end{tikzpicture}
\end{codeexample}
\end{environment}
